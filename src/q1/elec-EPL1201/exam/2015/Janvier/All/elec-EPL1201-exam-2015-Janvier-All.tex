\documentclass[fr,license=none]{../../../../../../eplexam}

\usepackage{enumerate}
\usepackage{circuitikz}
\usepackage{../../../../../../eplelec}
\usepackage{../../../../../../eplunits} 
\sisetup{
    list-final-separator = { et },
    list-pair-separator = { et }
}

\hypertitle[']{Électricité}{1}{EPL}{1201}{2015}{Janvier}{All}
{Mattéo Couplet}
{Jean-Didier Legat}

\section{}
Soit une capacité cylindrique de longueur $L$, de rayon interne $r_a$ et de rayon externe $r_b$.
\begin{enumerate}
    \item Déduire l'expression du champ électrique pour $r_a < r < r_b$.
    \item Déduire l'expression de $C$.
\end{enumerate}

\begin{solution}
\begin{enumerate}
    \item
    On applique la loi de Gauss sur le cylindre de rayon variable $r$ :
    \[ \oint \vec{E} \cdot \dif \vec{l} = \frac{Q_\mathrm{encl}}{\perm_0} \]
    En tout point d'une surface cylindrique, le champ électrique est uniforme et dirigé radialement vers l'extérieur (si $\lambda > 0$). On peut donc simplifier
    \[ E(r) \cdot A(r) = \frac{\lambda L}{\perm_0} \ \Rightarrow \ E(r) = \frac{\lambda}{2\pi\perm_0} \cdot \frac{1}{r} \]

\item
    On cherche d'abord la différence de potentiel entre les deux plaques
    \[ V_{ab} = \int_a^b \vec{E} \cdot \dif\vec{l} \]
    Le champ ayant toujours la même orientation, on peut simplifier
    \[ V_{ab} = \int_{r_a}^{r_b} E(r) \dif r = \frac{Q}{2\pi\perm_0 L} \int_{r_a}^{r_b} \frac{1}{r} \dif r = \frac{Q}{2\pi\perm_0 L} \ln\left(\frac{r_b}{r_a}\right) \]
    En remplaçant dans l'équation connue de la capacité $Q = CV$, on trouve
    \[ C = \frac{2\pi\perm_0 L}{\ln\left(\frac{r_b}{r_a}\right)} \]
\end{enumerate}
\end{solution}

\section{}
Étant donné le circuit ci-dessous
\begin{center}
\begin{circuitikz}[american]
    \draw (0, 0) 
    to[V=\SI{10}{\volt}] (0, 3)
    to[R=\SI{10}{\kilo\ohm}] (3, 3)
    to[V=$2V_x$] (6, 3)
    to[R=\SI{10}{\kilo\ohm}] (6, 0)
    to[short] (0, 0);
    \draw (3, 0) to[R=\SI{6}{\kilo\ohm}, v_>=$V_x$] (3, 3);

    \draw (0, 3)
    to[short] (0, 5)
    to[R=\SI{10}{\kilo\ohm}] (3, 5)
    to[R=\SI{3}{\kilo\ohm}] (8, 5)
    to[short] (8, 0)
    to[short] (6, 0);
    \draw (3, 5) to[short] (3, 3);
\end{circuitikz}
\end{center}
\begin{enumerate}
    \item Calculer $V_x$.
    \item Donner la puissance des deux sources de tension.
\end{enumerate}

\begin{solution}
    Avant toute chose, on simplifie le circuit : les deux résistances de \SI{10}{\kilo\ohm} en haut à gauche sont en parallèle, et les résistances de \SIlist{3;6}{\kilo\ohm} sont en parallèle également.
\begin{center}
\begin{circuitikz}[american]
    \draw (0, 0) 
    to[V=\SI{10}{\volt}] (0, 3)
    to[R=\SI{5}{\kilo\ohm}] (3, 3) node[above, label={$V_x$}] {}
    to[V=$2V_x$] (6, 3)
    to[R=\SI{10}{\kilo\ohm}] (6, 0)
    to[short] (0, 0);
    \draw (3, 0) to[R=\SI{2}{\kilo\ohm}, v_>=$V_x$] (3, 3);
\end{circuitikz}
\end{center}

\begin{enumerate}
    \item 
On applique la méthode des n\oe{}uds sur le n\oe{}ud $V_x$ :
\[ \frac{10 - V_x}{5} = \frac{V_x - 0}{2} + \frac{(V_x + 2V_x) - 0}{10} \ \Rightarrow \ V_x = \SI{2}{\volt} \]
    \item 
        Il s'agit de trouver les courants qui passent dans les deux sources, puis de multiplier par la tension
        \begin{alignat*}{4}
            &I_{\SI{10}{\volt}} &&= \frac{10-V_x}{5} &&= \SI{1.6}{\milli\ampere} \ &&\Rightarrow \ P = VI = -\SI{16}{\milli\watt} \\
            &I_{\SI{4}{\volt}} &&= \frac{3V_x}{10} &&= \SI{0.6}{\milli\ampere} \ &&\Rightarrow \ P = VI = -\SI{2.4}{\milli\watt} 
        \end{alignat*}
\end{enumerate}
\end{solution}

\section{}
Calculer l'équivalent de Thévenin du circuit ci-dessous.
\begin{center}
\begin{circuitikz}[american]
    \draw (0, 0) to[short] (13, 0) node[below, label={B}] {}
    to[open, v^>=$V_\mathrm{th}$] (13, 3);
    \draw (2, 0) to[I=\SI{10}{\milli\ampere}] (2, 3);
    \draw (4, 0) to[R=\SI{3}{\kilo\ohm}] (4, 3);
    \draw (7, 0) to[V=\SI{10}{\volt}] (7, 3);
    \draw (10, 0) to[R=\SI{30}{\kilo\ohm}] (10, 3);
    \draw (2, 3) 
    to[short] (4, 3)
    to[R=\SI{5}{\kilo\ohm}] (7, 3)
    to[R=\SI{10}{\kilo\ohm}] (10, 3)
    to[R=\SI{4}{\kilo\ohm}] (13, 3) node[above, label={A}] {};
    \draw (0, 0) to[V=\SI{30}{\volt}] (0, 5)
    to[R=\SI{30}{\kilo\ohm}] (10, 5)
    to[short] (10, 3);
\end{circuitikz}
\end{center}

\begin{solution}
On sait qu'aucun courant ne traverse la résistance de \SI{4}{\kilo\ohm}, donc la tension de Thévenin équivaut à la tension du points juste à gauche. On applique donc en ce point la méthode des n\oe{}uds
\[ \frac{10 - V_\mathrm{th}}{10} + \frac{30 - V_\mathrm{th}}{30} = \frac{V_\mathrm{th} - 0}{30} \ \Rightarrow \ V_\mathrm{th} = \SI{12}{\volt} \]

Il reste à calculer la résistance équivalente $R_\mathrm{éq}$. Pour cela, on annule toutes les sources (en s'assurant qu'aucune ne soit commandée). Les résistances de \SIlist{3;5}{\kilo\ohm} s'annulent car la différence de potentiel à leurs bornes est nulle ; on met le reste ensemble, ce qui nous donne
\[ R_\mathrm{éq} = (30 \parallel 10 \parallel 30) + 4 = \SI{10}{\kilo\ohm} \]
\end{solution}

\section{}
Dans le circuit ci-dessous
\begin{center}
    \begin{circuitikz}[american]
        \draw (0, 0) to[short] (12, 0);
        \draw (0, 0) to[V=\SI{50}{\volt}] (0, 4);
        \draw (3, 0) to[R=\SI{10}{\kilo\ohm}] (3, 4);
        \draw (5, 0) to[R=\SI{2}{\kilo\ohm}] (5, 2) to[L=\SI{12}{\milli\henry}] (5, 4) node[above, label={$V_x$}] {};
        \draw (7, 0) to[R=\SI{20}{\kilo\ohm}] (7, 4);
        \draw (9, 0) to[I=\SI{10}{\milli\ampere}] (9, 4);
        \draw (12, 0) to[R=\SI{4}{\kilo\ohm}] (12, 4);
        \draw (0, 4) to[R=\SI{10}{\kilo\ohm}] (3, 4)
        to[short] (9, 4)
        to[cspst, l=${\text{se ferme en }t=0}$] (12, 4);
    \end{circuitikz}
\end{center}

\begin{enumerate}
    \item Pour l'inductance, déterminer l'expression de $I_L(t)$.
    \item Déterminer la valeur de $V_x$ pour $t=0^+$.
\end{enumerate}

\begin{solution}
    \begin{enumerate}
        \item 
            L'expression du courant en fonction du temps dans l'inductance a pour forme générale
            \[ I_L(t) = A+B e^{-t/\tau} \]
            Commençons par calculer $I_L(0^-) = A+B$ ; l'interrupteur est ouvert et l'inductance se comporte comme un court-circuit. On applique tout d'abord la méthode des n\oe{}uds en $V_x$ :
            \[ \frac{50 - V_x}{10} + 10 = \frac{V_x - 0}{10} + \frac{V_x - 0}{2} + \frac{V_x - 0}{20} \ \Rightarrow \ V_x = \SI{20}{\volt} \]
            Ce qui nous donne immédiatement
            \[ I_L(0^-) = \SI{10}{\milli\ampere} \]
            Ensuite, on cherche $I_L(\infty) = A$ ; l'interrupteur est fermé et l'inductance se comporte comme un court-circuit. Encore une fois, méthode des n\oe{}uds en $V_x$ :
            \[ \frac{50 - V_x}{10} + 10 = \frac{V_x - 0}{10} + \frac{V_x - 0}{2} + \frac{V_x - 0}{20} + \frac{V_x - 0}{4} \ \Rightarrow \ V_x = \SI{15}{\volt} \]
            Et donc
            \[ I_L(\infty) = \SI{7.5}{\milli\ampere} \]
            Reste à calculer $\tau = \frac{L}{R_\mathrm{éq}}$. $R_\mathrm{éq}$ est la résistance équivalente que voit l'inductance si on annule les sources. En faisant ainsi, on obtient
            \[ R_\mathrm{éq} = (10 \parallel 10 \parallel 20 \parallel 4) + 2 = \SI{4}{\kilo\ohm} \]
            Pour résumer, on a $A = \SI{7.5}{\milli\ampere}$, $B = \SI{2.5}{\milli\ampere}$, $\tau = \SI{3}{\micro\second}$, ce qui nous permet d'écrire la fonction
            \[ I_L(t) = \num{7.5e-3} + \num{2.5e-3} e^{-t/\num{3e-6}} \]

        \item 
            Pour obtenir la tension aux bornes de l'inductance, on dérive la fonction précédemment trouvée
            \[ V_L(t) = L\fdif{I_L(t)}{t} = -10 e^{-t/\num{3e-6}} \]
            La tension $V_x$ s'obtient donc aisément
            \[ V_x(0^+) = 2 I(0) + V_L(0) = \SI{10}{\volt} \]
            Notons le signe négatif de la tension aux bornes de l'inductance : en effet, lorsque l'interrupteur se ferme, le courant passant à travers l'inductance diminue ; elle s'oppose à cette variation, et crée donc une tension dans le sens du courant.
    \end{enumerate}
\end{solution}

\section{}

En vous basant sur des ampli op idéaux (courant d'entrée nul et gain infini),
\begin{enumerate}
    \item Donnez le schéma du montage inverseur et déduisez l'expression de la tension de sortie $V_\mathrm{out}$ en fonction de la tension d'entrée $V_\mathrm{in}$.
    \item Donnez le schéma d'un circuit qui réalise la fonction suivante : $V_\mathrm{out} = 2V_\mathrm{in, 1} - 4V_\mathrm{in, 2}$. Précisez les valeurs des éléments que vous seriez amenés à utiliser.
\end{enumerate}

\begin{solution}
    \begin{enumerate}
        \item Voir\cite[p.~13]{synthese}
        \item Un circuit adéquat est le montage sommateur suivant
            \begin{center}
                % I HATE CircuiTikZ !!!
                \begin{circuitikz}[american]
                    \draw (0, 0) node[op amp] (opamp1) {}
                    (opamp1.+) to[short] ++(0, -1) node[ground]{}
                    (opamp1.-) to[R=$R_1$] ++(-2, 0)
                    ++(0, -2) node[ground]{} to[V=$V_\mathrm{in, 1}$] ++(0, 2)
                    (opamp1.-) -- ++(0, 1) coordinate (bs1) 
                    to[R=$R_a$] (bs1 -| opamp1.out)
                    -- (opamp1.out)
                    -- ++(1, 0)
                    to[R=$R_b$] ++(2, 0) coordinate (bs2)
                    -- ++(0, -1)
                    to[R=$R_2$] ++(-2, 0)
                    ++(-2, 0) to[V, l_=$V_\mathrm{in, 2}$] ++(2, 0)
                    ++(-2, 0) -- ++(0, -.5) node[ground]{}
                    (bs2) ++(1.5, -0.5) node[op amp] (opamp2) {}
                    (bs2) -- (opamp2.-)
                    (opamp2.+) -- ++(0, -.5) node[ground]{}
                    (opamp2.-) -- ++(0, 1) coordinate (bs3) 
                    to[R=$R_c$] (bs3 -| opamp2.out) -- (opamp2.out)
                    to[short,-*] ++(1, 0) node[above, label={$V_\mathrm{out}$}] {}
                    ;
                \end{circuitikz}
            \end{center}
            Si $R_a = R_b$, la tension de sortie vaudra
            \[ V_\mathrm{out} = R_c\left( \frac{V_\mathrm{in, 1}}{R_1} - \frac{V_\mathrm{in, 2}}{R_2} \right) \]
            Pour avoir les bons rapports, il faut que l'égalité suivante soit respectée
            \[ R_c = 2R_1 = 4R_2 \]
            On pourrait prendre p. ex. n'importe quelle valeur pour $R_a$ et $R_b$ (pour autant que $R_a = R_b$ !) et pour les autres résistances
            \begin{align*}
                R_c &= \SI{4}{\kilo\ohm} \\
                R_1 &= \SI{2}{\kilo\ohm} \\
                R_2 &= \SI{1}{\kilo\ohm} \\
            \end{align*}
    \end{enumerate}
\end{solution}


\begin{thebibliography}{9}
    \bibitem{synthese}
        Nicolas Cognaux, Mattéo Couplet, Guillaume François et Benoît Legat,
        \emph{Synthèse d'Électricité},
        \texttt{https://github.com/Gp2mv3/Syntheses}
\end{thebibliography}

\end{document}
