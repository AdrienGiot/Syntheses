\documentclass[fr]{../../../../../../eplexam}
\usepackage{../../../../../../eplcode}

\hypertitle{info-FSAB1401}{1}{FSAB}{1401}{2009}{Juin}{All}
{Jean-Martin Vlaeminck}
{Olivier Bonaventure et Charles Pecheur}[
\paragraph{Remarque de l'auteur}
Ce document ne contient pas l'énoncé détaillé de l'examen ; il peut cependant être retrouvé à l'adresse suivante
\begin{center}
	\url{https://drive.google.com/drive/folders/0B7aBBTDXcgqpZnZMdWYwS09lSHc}
\end{center}
Vous pouvez aussi consulter le dossier
\begin{center}
	\url{https://drive.google.com/a/student.uclouvain.be/folderview?id=0B5WS2yn5sWqDZFFibHM2d0lQVEE&usp=sharing}
\end{center}
et chercher le Drive Q1 dans BACHELIER > Tronc commun > Q1 (pas celui de Benoît Legat) > Informatique > Examens et Interro.\newline
Le répertoire Github contient également un code complet et fonctionnel de chacune des classes de l'examen, permettant également de jouer au jeu puissance4-epl.
]

\lstset{
	language={Java},
	tabsize=4,
	escapechar=\$,
	breakatwhitespace
}

\newcommand{\code}[1]{\lstinline{#1}}

% Question 1
\section{}
Écrivez toute la classe \code{Move}, y compris :
\begin{itemize}
	\item un constructeur qui crée un coup à partir d'un jeton et d'une colonne donnée,
	\item deux méthodes \code{getToken} et \code{getColumn} qui retournent respectivement le jeton et la colonne,
	\item les variables d'instance et/ou de classe nécessaires.
\end{itemize}
Il n'est pas demandé de fournit les spécifications.
\lstinputlisting[linerange={1-7}]{src/Move.java}

\begin{solution}
	% Sans commentaire
	\lstinputlisting[linerange={11-55}]{src/Move.java}
\end{solution}

% Question 2
\section{}
Dans la classe SimpleGame, complétez la signature et le corps du constructeur.
\lstinputlisting[linerange={32-37}]{src/SimpleGame.java}

\begin{solution}
	Le constructeur est classique, initialisant chaque attribut de manière correcte. Mettre la valeur $0$ dans chacune des cases du tableau \code{firstFreeCell} est inutile en (qui initialise toutes les variables à $0$ ou à \code{null}), mais a l'avantage d'être clair.
	\lstinputlisting[linerange={38-49}]{src/SimpleGame.java}
\end{solution}

% Question 3
\section{}
Dans la classe \code{SimpleGame}, écrivez la signature et le code de la méthode \lstinline|add(Token p, int c)|. N'oubliez pas les exceptions.
\lstinputlisting[linerange={87-92}]{src/SimpleGame.java}

\begin{solution}
	\begin{sloppypar}
		On commence par vérifier que la colonne n'est pas déjà remplie, auquel cas on jette une \code{IllegalMoveException}. Sinon, on ajoute le jeton \code{p} dans la première case libre de la colonne, et on incrémente \code{firstFreeCell[c]} pour indiquer la nouvelle case libre. Il faut bien faire attention qu'il faut renvoyer la case libre précédente, donc avant l'incrémentation.% Il est aussi possible de raccourcir le code en utilisant la propriété de la post-incrémentation (la valeur précédente est renvoyée, puis la variable est incrémentée).
	\end{sloppypar}
	\newpage
	\lstinputlisting[linerange={93-108}]{src/SimpleGame.java}
\end{solution}

% Question 4
\section{}
Dans la classe \code{SimpleGame}, complétez le corps de la méthode \code{isFull}.
\lstinputlisting[linerange={193-199}]{src/SimpleGame.java}

\begin{solution}
	Il suffit de vérifier que chaque colonne contienne \code{height} éléments. Sinon, le plateau n'est pas plein.
	\lstinputlisting[linerange={201-205}]{src/SimpleGame.java}
\end{solution}

% Question 5
\section{}
Dans la classe \code{UndoableGame}, écrivez la signature et le corps de la méthode \code{add}. N'oubliez pas les exceptions.
\lstinputlisting[linerange={34-39}]{src/UndoableGame.java}

\begin{solution}
	% Sans commentaire
	\lstinputlisting[linerange={40-44}]{src/UndoableGame.java}
\end{solution}

% Question 6
\section{}
Dans la classe \code{MoveStack}, complétez le corps de la méthode \code{push}.
\lstinputlisting[linerange={30-36}]{src/MoveStack.java}

\begin{solution}
	% Sans commentaire
	\lstinputlisting[linerange={38-41}]{src/MoveStack.java}
\end{solution}

% Question 7
\section{}
Dans la classe \code{MoveStack}, complétez le corps de la méthode \code{getAll}. Attention, les coups sont retournés dans l'ordre chronologique (sommet de la pile en dernier).
\lstinputlisting[linerange={75-81}]{src/MoveStack.java}

\newpage
\begin{solution}
	Dans cette solution, une petite particularité syntaxique de la boucle \code{for} a été utilisé : la partie \og incrémentation \fg{} de la boucle \code{for} peut contenir plus d'une instruction, et pas nécessairement des opérations arithmétiques. Cela permet notamment d'y insérer le code d'avancement du pointeur \code{runner}, et évite de l'oublier à la fin de la boucle \code{for} (erreur fréquente).
	\lstinputlisting[linerange={83-95}]{src/MoveStack.java}
\end{solution}

% Question 8
\section{}
Dans la classe \code{GameController}, complétez le corps de la méthode \code{save}. Remarquez que cette méthode ne propage pas d'\lstinline|IOException|.
\lstinputlisting[linerange={35-47}]{src/GameController.java}

\begin{solution}
	Pour rappel, à l'exception du constructeur et de la méthode \code{close}, les méthodes de la classe \code{PrintWriter} ne jette pas d'exceptions lors des opérations d'écriture. Pour vérifier si l'écriture s'est faite sans erreur, il faut utiliser la méthode \code{checkError}.
	\newpage
	\lstinputlisting[linerange={49-76}]{src/GameController.java}
\end{solution}

\end{document}
