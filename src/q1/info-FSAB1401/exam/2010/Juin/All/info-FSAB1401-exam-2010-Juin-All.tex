\documentclass[fr]{../../../../../../eplexam}

\usepackage{../../../../../../eplcode}

%\begin{document}

\hypertitle{Informatique}{1}{FSAB}{1401}{2010}{Juin}
{Mattéo Couplet}
{Olivier Bonaventure et Charles Pecheur}[
    \paragraph{Remarque de l'auteur} Ce document ne contient pas l'énoncé détaillé de l'examen ; il peut cependant être retrouvé à l'adresse en bas de page \footnote{\url{https://drive.google.com/a/student.uclouvain.be/folderview?id=0B-fKQCKgzACFZTFfY1B0UzFwUVk&usp=sharing_eid&tid=0B7aBBTDXcgqpYjVuZUxycV9JN0k}}.
]

\lstset{
    language={Java},
    tabsize=4
}
\let\code\lstinline

% Question 1
\section{}
Spécifiez et écrivez une méthode \code{toString()} pour la classe \code{Mesure}. Cette méthode doit redéfinir de manière appropriée celle de \code{Object}.

\begin{solution}
    \lstinputlisting{src/q1.java}
\end{solution}

% Question 2
\section{}
Écrivez complètement le constructeur de la classe \code{MesureEtendue}. Utilisez à bon escient les expressions booléennes.
\lstinputlisting{src/q2_spec.java}

\begin{solution}
    On fait appel au constructeur de la classe-mère, et on initialise la validité de la mesure.
    \lstinputlisting{src/q2.java}
\end{solution}
\newpage

% Question 3
\section{}
Complétez la méthode \code{moyenneTemperature} de la classe \code{StationMeteoSimple}. Pour rappel, la moyenne (arithmétique) d'un ensemble de mesures est définie comme étant le rapport entre la somme de ces mesures et le nombre de mesures. Attention, seules les mesures valides doivent être prises en compte.
\lstinputlisting{src/q3_spec.java}

\begin{solution}
    On parcoure le tableau des mesures en ne prenant que les mesures valides.
    \lstinputlisting{src/q3.java}
\end{solution}

% Question 4
\section{}
Dans la classe \code{StationMeteoSimple}, complétez la méthode \code{addMesure}.
\lstinputlisting{src/q4_spec.java}

\begin{solution}
    On parcoure tout le tableau en sens inverse, en affectant à chaque entrée l'entrée précédente. Enfin, on affecte la première entrée à \code{m}.
    \lstinputlisting{src/q4.java}
\end{solution}
\newpage

% Question 5
\section{}
Dans la classe \code{StationMeteoAvancee}, complétez la méthode \code{addMesure}.
\lstinputlisting{src/q5_spec.java}

\begin{solution}
    On fait appel à la méthode \code{addMesure} de la classe-mère, puis si la mesure est valide, on met à jour la pression maximale.
    \lstinputlisting{src/q5.java}
\end{solution}

% Question 6
\section{}
Dans la classe \code{StationMeteoAvancee}, complétez la méthode \code{sauvegarde}. Un exemple du texte généré dans le fichier est donné ci-dessous.
\begin{lstlisting}
102415:-4.2
101180:11.2
102320:10.2
101325:12.2
102445:19.2
\end{lstlisting}
\lstinputlisting{src/q6_spec.java}

\begin{solution}
    On ouvre le fichier dans un bloc \code{try catch} afin de pouvoir attraper une éventuelle \code{IOException}. Dans ce cas on afficher une erreur et on quitte le programme. Dans le bloc \code{finally}, on ferme le \code{PrintWriter} s'il est défini.
    \lstinputlisting{src/q6.java}
    % Sauf que finally n'est pas exécuté si System.exit, et que PrintWriter n'envoie pas d'exception quand il écrit, juste quand il est créé.
\end{solution}

% Question 7
\section{}
Cette question nécessite la connaissance des \code{Listener}, qui ne font plus partie de la matière ; elle ne sera donc pas traitée.

% Question 8
\section{}
Complétez la méthode \code{remove} de la classe \code{ListeMesure}. Remarquez que le paramètre est un nœud de la liste et non une mesure.
\lstinputlisting{src/q8_spec.java}

\begin{solution}
    On repère trois cas particuliers : la liste est composée d'un seul élément (qu'on supprime), on veut supprimer le premier élément, on veut supprimer le dernier élément, et enfin on veut supprimer l'élément (qui n'est ni au début ni à la fin). On doit traiter chaque cas séparément afin d'éviter des \code{NullPointerException}.
    \lstinputlisting{src/q8.java}

    Une autre manière de faire, si la liste est bien construite, est présenté dans le code suivant. Notons que le cas où \code{n} est le seul élément de la liste correspond à l'exécution des deux \code{if} (\code{n} est au début \emph{et} à la fin), et que le cas où \code{n} n'est ni au début ni à la fin correspond à l'exécution des deux \code{else}.
    \lstinputlisting[linerange={1-18}]{src/q8_alter.java}
\end{solution}

\end{document}
