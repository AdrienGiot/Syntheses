\documentclass[fr]{../../../../../../eplexam}

\usepackage{../../../../../../eplcode}

\hypertitle{info-FSAB1401}{1}{FSAB}{1401}{2011}{Juin}{All}
{Sébastien Binard \and Jean-Martin Vlaeminck}
{Olivier Bonaventure et Charles Pecheur}[
	\paragraph{Remarque de l'auteur}
	Ce document ne contient pas l'énoncé détaillé de l'examen ; il peut cependant être retrouvé à l'adresse suivante
	\begin{center}
		\url{https://drive.google.com/drive/folders/0B7aBBTDXcgqpZnZMdWYwS09lSHc}
	\end{center}
	Vous pouvez aussi consulter le dossier
	\begin{center}
		\url{https://drive.google.com/a/student.uclouvain.be/folderview?id=0B5WS2yn5sWqDZFFibHM2d0lQVEE&usp=sharing}
	\end{center}
	et chercher le Drive Q1 dans BACHELIER>Tronc commun>Q1.\newline
	Le répertoire Github contient également un code complet et fonctionnel de chacune des classes de l'examen.
]

\lstset{
	language={Java},
	tabsize=4,
	escapechar=\$,
	breakatwhitespace
}

\newcommand{\code}[1]{\lstinline{#1}}

% Question 1
\section{}
Écrivez le code de la méthode \code{removeProcess} de la classe \code{BasicComputer}.
\lstinputlisting[linerange={46-46, 48-51}]{src/BasicComputer.java}

\begin{solution}
	% Sans commentaire
	\lstinputlisting[linerange={51-59}]{src/BasicComputer.java}
\end{solution}

% Question 2
\section{}
Écrivez complètement le constructeur de la classe \code{FullComputer}
\lstinputlisting[linerange={20-24}]{src/FullComputer.java}

\begin{solution}
	% Sanc commentaire
	\lstinputlisting[linerange={25-31}]{src/FullComputer.java}
\end{solution}

% Question 3
\section{}
Écrivez complètement (y compris les spécifications) la classe \code{Process} qui représente un processus. Chaque processus a un nom (\code{String}), une capacité de stockage requise (\code{int}) et un identifiant de processus ou PID (\code{int}). \strong{Les PID sont attribués séquentiellement à la création de chaque nouveau processus} (1 pour le premier, 2 pour le deuxième, etc.) La classe doit définir :
\begin{itemize}
	\item Un constructeur avec le nom et la capacité comme paramètres.
	\item Des méthodes \code{getName}, \code{getRequiredStorage} et \code{getPID} retournant respectivement le nom, la capacité et le PID.
	\item Une méthode \code{getDescr} retournant une chaine de caractères comprenant le nom du processus et la capacité de stockage nécessaire, séparés par une espace.
	\item Toutes les variables et méthodes complémentaires nécessaires à votre implémentation.
\end{itemize}

\begin{solution}
	Le code suivant présente la classe \code{Process}. Quelques remarques :
	\begin{itemize}
		\item On retient le prochain PID libre dans la variable statique \code{nextPid}, initialisée à $1$ au chargement du programme. À chaque création d'un processus, on récupère la valeur de \code{nextPid}, qu'on stocke dans la propriété \code{pid} du nouveau processus, et on incrémente \code{nextPid} pour le prochain.
		\item Comme \code{nextPid} est statique, on doit l'initialiser à sa déclaration, et non dans un constructeur !
		\item \begin{sloppypar}La ligne 22 combine deux opérations : on récupère la valeur de \code{nextPid} que l'on met dans \code{pid}, puis on augmente \code{nextPid} (grâce à la post-incrémentation \code{++}). A noter que si on avait écrit \code{this.pid=++nextPid}, \code{nextPid} aurait été incrémenté d'abord, et le premier processus aurait un PID de 2.\end{sloppypar}
	\end{itemize}
	\lstinputlisting[linerange={3-57}]{src/Process.java}
\end{solution}

% Question 4
\section{}
Écrivez le corps de la méthode \code{addProcess} de la classe \code{FullComputer}. Pensez à utiliser les méthodes de la classe \code{Process} définies à la question 3.
\lstinputlisting[linerange={35-35, 37-43}]{src/FullComputer.java}

\begin{solution}
	Si l'une des deux conditions n'est pas respectée, le processus n'est pas ajouté et on retourne \code{false}.

	Sinon, on met le processus dans la première case libre de \code{procs}, à savoir \code{procs[count]}, on incrémente \code{count}, et on diminue la capacité de stockage restante.

	(À noter qu'on aurait pu écrire les lignes 5 et 6 comme \code{procs[count++]=p}, grâce à la post-incré\-men\-tation.)
	\lstinputlisting[linerange={43-51}]{src/FullComputer.java}
\end{solution}

% Question 5
\section{}
Écrivez complètement la méthode \code{addProcess} de la classe \code{Cluster}. Aidez-vouz de l'exemple présenté en page 2. Indice : vou pouvez utiliser la valeur de \code{count}.
\lstinputlisting[linerange={33-36}]{src/Cluster.java}

\begin{solution}
	On commence par vérifier s'il y a de la place sur l'ordinateur \code{current}. Sinon, on parcourt les différents ordinateurs à la suite. Si l'on revient ainsi à \code{current}, cela signifie qu'il n'y a pas de place. Sinon, on a trouvé notre ordinateur, à \code{runner}, et il ne reste plus qu'à déplacer \code{current} correctement.

	L'indice avec \code{count} nous aurait permis de ne pas avoir à vérifier que \code{runner!=current} dans la boucle \code{while}, et de ne pas avoir à séparer le cas \code{runner=current} des autres cas.%TODO éventuellement coder cette variante
	\lstinputlisting[linerange={37-54}]{src/Cluster.java}
\end{solution}

% Question 6
\section{}
Écrivez complètement la méthode \code{removeComputer} de la classe \code{Cluster}. Indice: vous pouvez utiliser la valeur de \code{count}.
\lstinputlisting[linerange={105-108}]{src/Cluster.java}

\begin{solution}
	On traite les cas où il n'y a aucun ordi et où il n'y a qu'un seul ordi à part, en utilisant \code{count}.

	Dans les autres cas, on parcourt chaque n\oe{}ud (grâce à \code{runner.next}), en partant de \code{current.next}, et on s'arrête dès que l'on trouve le n\oe{}ud contenant l'ordinateur, ou que l'on arrive à \code{current}. (Si le n\oe{}ud contenant l'ordinateur est \code{current}, le code fonctionne quand même, la boucle ayant deux raisons de s'arrêter.)

	Lorsqu'on a trouvé l'ordinateur, on vérifie s'il s'agit de \code{current}, auquel cas on déplace la tête de liste. Si on n'a pas trouvé l'ordinateur, on renvoie \code{false}. Et si on a trouvé l'ordinateur, il ne faut pas oublier de décrémenter \code{count}.
	\lstinputlisting[linerange={109-138}]{src/Cluster.java}
\end{solution}

% Question 7
\section{}
Écrivez le corps de la méthode \code{loadState} de la classe \code{Cluster}. Notez que la précondition garantit le format du contenu du fichier; pas besoin donc de traiter les erreurs de format de texte.
\lstinputlisting[linerange={175-179}]{src/Cluster.java}

\begin{solution}
	On commence par retirer tous les processus. Puis on crée un \code{BufferedReader} afin de lire le fichier ligne par ligne. Chaque ligne est ensuite découpée à l'aide de \code{split()}, et un processus est créé à partir des informations de la ligne. On tente d'ajouter le processus au cluster. Si ça réussit, on lit la ligne suivante. On termine en fermant le fichier.

	Deux types d'erreurs peuvent survenir. Tout d'abord, l'ajout de processus peut échouer (signalé par \code{UnavailableException}). Dans ce cas, le premier \code{catch} intercepte l'exception, on affiche un message d'erreur approprié, on ferme le fichier et on termine le programme. Ensuite, on peut avoir une erreur d'I/O, et l'\code{IOException} est alors captée par le deuxième \code{catch}, qui affiche un message d'erreur, ferme le fichier (si possible) et termine le programme.

	Il est possible que lors de la fermeture d'un fichier, un \code{IOException} soit lancée. Il faut donc mettre un \code{try catch} autour du \code{close()}, mais ce n'est pas demandé à l'examen.
	\lstinputlisting[linerange={179-208}]{src/Cluster.java}
\end{solution}

\end{document}
