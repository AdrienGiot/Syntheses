\documentclass[fr]{../../../../../../eplexam}

\usepackage{../../../../../../eplcode}
\usepackage{tikz}
\usepackage{array}

\hypertitle{Informatique}{1}{FSAB}{1401}{2012}{Juin}
{Mattéo Couplet \and Jean-Martin Vlaeminck}
{Olivier Bonaventure et Charles Pecheur}[
    \paragraph{Remarque de l'auteur}
    Ce document ne contient pas l'énoncé détaillé de l'examen ; il peut cependant être retrouvé à l'adresse suivante
    \begin{center}
        \url{https://drive.google.com/a/student.uclouvain.be}
    \end{center}
]

%\begin{document}

\newcommand{\Song}[3]{
\begingroup
\renewcommand{\arraystretch}{2}
\begin{tabular}{|>{\sffamily}c|}
	\hline
	\textbf{Song} \\
	\hline
	\begin{tabular}{|c|c|c|}
		\hline
		#1 & #2 & #3 \\
		\hline
	\end{tabular} \\[0.4em]
	\hline
\end{tabular}
\endgroup
}
\newcommand{\tikzmark}[1]{\tikz[remember picture] \node[baseline] (#1) {};}

\lstset{
	language={Java},
	tabsize=4
}
\let\code\lstinline % gros flemmard

%Question 1
\section{}
Écrivez la spécification, la signature et le corps du constructeur de la classe \code{Song} qui permet de représenter une chanson utilisée par le juke-box. Chaque chanson est caractérisée par un nom d'auteur (de type \code{String}), un titre (de type \code{String}) et une durée en secondes (de type \code{int}).

\begin{solution}
    \lstinputlisting{src/q1.java} 
\end{solution}

%\newpage
% Question 2
\section{}
Dans la classe \code{Catalog}, écrivez le corps du constructeur dont la spécification est reprise ci-dessous :
\lstinputlisting{src/q2_spec.java}

\begin{solution}
    On crée un \code{BufferedReader} qui va lire le contenu du fichier. Après lecture de la première ligne qui contient le nombre de chansons, on instancie le tableau \code{contents}. On parcourt ensuite le fichier ligne par ligne en remplissant le tableau. Tout cela est englobé dans un \code{try catch} afin d'attraper une éventuelle \code{IOException}.

    Il ne faut pas oublier de fermer le \code{BufferedReader} à la fin, ce qui implique de le fermer dans le \code{try} (si tout se passe bien) et le \code{catch} (s'il y a une erreur) ; un bloc \code{finally} ne s'exécuterait pas, en raison du \code{System.exit(-1)}.%FIXME utile pour l'examen, ou les profs s'en fichent ?
    \lstinputlisting{src/q2.java}
    % de 1 à 29 ; le catch amélioré : 20-28, le while 11-18
    Il est aussi possible de réaliser ces opérations avec une boucle \code{while}, en lisant tant qu'il y a des lignes. Le code suivant montre une solution alternative. De même, comme \code{br.close()} peut jeter une exception, il est mis dans un \code{try catch} (non requis à l'examen, mais requis pour un vrai projet).
    \lstinputlisting[linerange={1-29}]{src/q2_alternative.java}
\end{solution}

\newpage
% Question 3
\section{}
La classe \code{PlayList} est une structure chainée permettant de stocker la liste des chansons qui doivent être jouées par le juke-box. Complétez le diagramme d'objets représentant la \code{Playlist plist} après l'exécution du programme suivant :
\lstinputlisting{src/q3_spec.java}
Représentez sur votre dessin la \code{PlayList plist} et ses différents n\oe{}uds (les trois instances de \code{Song} sont déjà dessinées) et les références entre ces objets.

\begin{solution}
    La solution est présentée dans la figure suivante.
    % tentez d'enlever le \sffamily, et observez l'espacement inférieur wtf du sous-tableau : http://tex.stackexchange.com/questions/177074/possible-to-create-table-inside-a-table
    % Tenter d'insérer la 'figure' suivante dans un flottant résulte en flottant perdu car l'environnement solution est une minipage (je pense). Chercher "float lost".
  \begin{center}
    \begin{tikzpicture}[remember picture, every node/.style={outer sep=0}]
    % Si quelqu'un a une manière plus simple de réaliser un diagramme de classes ou d'objets, je suis preneur. Et n'hésitez pas à relooker les flèches ^^

    \node (song1) at (6, 5) { \Song{"Sheller"}{"Nicolas"}{192} };
    \node (song2) at (6, 2.5) { \Song{"Sheller"}{"Genève"}{192} };
    \node (song3) at (6, 0) { \Song{"Lennon"}{"Imagine"}{192} };

    \node (noeud1) at (0, 6) {
        \renewcommand{\arraystretch}{1.2}
        \begin{tabular}{|>{\sffamily}c|}
        \hline
        \textbf{Node} \\ \hline
        data \tikzmark{n1d} \\
        next \tikzmark{n1n} \\ \hline
        \end{tabular}
    };
    \node (noeud2) at (0, 3.5) {
        \renewcommand{\arraystretch}{1.2}
        \begin{tabular}{|>{\sffamily}c|}
        \hline
        \textbf{Node} \\ \hline
        data \tikzmark{n2d} \\
        next \tikzmark{n2n} \\ \hline
        \end{tabular}
    };
    \node (noeud3) at (0, 1) {
        \renewcommand{\arraystretch}{1.2}
        \begin{tabular}{|>{\sffamily}c|}
        \hline
        \textbf{Node} \\ \hline
        data \tikzmark{n3d} \\
        next \tikzmark{n3n} \\ \hline
        \end{tabular}
    };
    \node (playlist) at (4, 7.5) {
        \renewcommand{\arraystretch}{1.2}
        \begin{tabular}{|>{\sffamily}c|}
        \hline
        \textbf{PlayList} \\ \hline
        \tikzmark{plh} head \\
        \tikzmark{plt} tail \\ \hline
        \end{tabular}
    };
    \node[draw] (plist) at (7.5, 8.5) {\textsf{plist}};

    \draw[*->] (n1d) -- (song1.170);
    \draw[*->] (n2d) -- (song2.170);
    \draw[*->] (n3d) -- (song3.170);

    \draw[*->] (n1n) to[bend left] (noeud2.70);
    \draw[*->] (n2n) to[bend left] (noeud3.70);
    \node (null1) at (2,-0.5) {\textsf{null}};%sale hack
    \draw[*->] (n3n) to[bend right] (null1);

    \draw[*->] (plh) to[bend right=15] (noeud1.north east);
    \draw[*->] (plt) to[out=225, in=45] (noeud3.north east);
    \draw[->] (plist.west) to[bend left=10] (playlist.east);
    \end{tikzpicture}
    %\label{fig:q3}
    %\caption{Diagramme d'objets de l'objet \code{plist}}
  \end{center}
\end{solution}

\newpage
% Question 4
\section{}
Dans la classe \code{PlayList}, écrivez le corps de la méthode \code{addBefore} dont la spécification est la suivante.

\lstinputlisting{src/q4_spec.java}

\begin{solution}
    On cherche le premier n\oe{}ud tel que son \code{.next()} contient \code{songin} (en traitant le \code{head} comme un cas particulier).
    \lstinputlisting[linerange={1-17}]{src/q4.java}
\end{solution}

%\newpage
% Question 5
\section{}
Dans la classe \code{JukeBox}, écrivez complètement la classe privée \code{JukeBoxListener} qui implémente l'interface \code{PlayerListener}, selon la spécification suivante. Inutile de recopier cette spécification.
\lstinputlisting{src/q5_spec.java}
\newpage
\begin{solution}
    On définit un constructeur vide qui nous sera utile par la suite. Tant qu'une chanson est en cours, on ne fait rien. Puis, si la playlist n'est pas vide, on la \code{dequeue} et on place la nouvelle chanson dans le \code{player}.
    \lstinputlisting[linerange={1-21}]{src/q5.java}
    % Juste, comme la méthode n'est appelée que quand la chanson est finie, est-ce que c'est utile de mettre ce while ?
\end{solution}

%\newpage
% Question 6
\section{}
Dans la classe \code{JukeBox}, écrivez le corps du constructeur dont la spécification est reprise ci-dessous. N'oubliez pas d'associer au lecteur le \code{listener} défini à la question précédente.
\lstinputlisting{src/q6_spec.java}

\begin{solution}
    \lstinputlisting{src/q6.java}
\end{solution}

\newpage
% Question 7
\section{}
Dans la classe \code{Catalog}, écrivez le corps de la méthode \code{NAuthors} dont la spécification est la suivante.
\lstinputlisting{src/q7_spec.java}

\begin{solution}
    Il s'agit de compter le nombre de chansons ayant des auteurs distincts ; pour cela, on stocke les auteurs déjà visités dans une \code{LinkedList}. On parcourt le tableau \code{contents}, et pour chaque chanson on parcourt la \code{LinkedList} pour voir si l'auteur est déjà présent ; sinon, on l'ajoute à la fin de la liste. Ainsi, le nombre d'auteurs distincts égale la taille de la liste.
    \lstinputlisting{src/q7.java}

    Néanmoins, comme la classe \code{LinkedList} n'est pas importée, il vaut mieux utiliser une solution avec un tableau intermédiaire, \code{auteurs}, qui contient les différents auteurs rencontrés jusqu'à présent. Pour chaque chanson, on regarde si son auteur est présent dans ce tableau. Si ce n'est pas le cas, on ajoute l'auteur à la fin du tableau (\code{compteur} est l'index de la première case libre) et on \og augmente \fg{} virtuellement la taille du tableau.
    \lstinputlisting{src/q7_alternative.java}
\end{solution}
%FIXME java.util.ArrayList ou LinkedList n'est pas importé, ce qui fait que cette solution est potentiellement fausse.

\end{document}
