\section{Atomistique}

\subsection{}
Répondez aux questions suivantes par un nombre.
\begin{enumerate}[label = (\alph*)]
	\item Combien de neutrons comporte l'isotope 81 du brome, \ce{Br}?
	\item Combien d'électrons de coeur comporte un atome d'arsenic, \ce{As}?
	\item Combien y a-t-il d'orbitales dans une sous-couche dont $l=2$ ?
	\item Combien d'électonrs non-appariés comporte un atome de manganèse, \ce{Mn}?
	\item Combien d'électrons peut contenir une couche dont n = 4?
	\item Dans un atome d'étain, \ce{Sn} quel est le nombre d'électrons de valence?
\end{enumerate}

\begin{solution}
	\begin{enumerate}[label = (\alph*)]
		\item $Nb_{\text{neutrons}} = M_{\ce{Br}} - Z_{\ce{Br}} = 81 - 35 = 46$ neutrons
		\item $Nb_{\text{électrons de coeur}} = Z_{\ce{As}}  - e^-_{\text{valence}}= 33 - 5 = 28$ électrons
		\item $m = 0, \pm1, ..., \pm l = \{-2,-1,0,1,2\} \rightarrow$ 5 orbitales
		\item $Z_{\ce{Mn}} = 25$\\
			\begin{tikzpicture}
				\OAboxs{(0,0)}{updown}{1s}
				\OAboxs{(0,1.5)}{updown}{2s}
				\OAboxs{(0,3)}{updown}{3s}
				\OAboxs{(0,4.5)}{updown}{4s}
				\OAboxp{(1.5,2.25)}{updown}{updown}{updown}{2p}
				\OAboxp{(1.5,3.75)}{updown}{updown}{updown}{3p}
				\OAboxd{(5,5.25)}{up}{up}{up}{up}{up}{3d}
			\end{tikzpicture}
			
			$e^-_{\text{non-apparié}} = 5$
			\item $e^-_{n = 4} = e^-_{4s}+e^-_{4p}+e^-_{4d}+e^-_{4f} = 2 +6+10+14 = 32$
			\item $Z_{\ce{Sn}} = 5$
			
			\configElectronique{50}
			
			$e^-_{\text{valence}} = e^-_{5s}+e^-_{5p}=4$électrons
 	\end{enumerate}
\end{solution}

\subsection{}
Dans quelle orbitale se trouve un électron ayant comme nombre quantiques $n=3$, $l=2$ et $m_l = 0$?
\begin{solution}
	$l=2$ indique une sous-couche d donc l'électron se trouve dans l'orbitale 3d
\end{solution}
\subsection{}
Ecrivezla configuration électronique des ions \ce{F^-} et \ce{Ga^3+} à l'état fondamental et des atomes N et Ga dans leur premier état excité.
\begin{solution}
	\ce{F^-}: $\big[\ce{He}\big] 2\text{s}^2 2\text{p}^6$\\
	\ce{Ga^3+}: $\big[\ce{Ar}\big] 3\text{d}^{10}$\\
	\ce{N^*}: $1\text{s}^2 2\text{s}^1 2\text{p}^4$ \\
	\ce{Ga^*}: $\big[\ce{Ar}\big]4\text{s}^1 3\text{d}^{10} 4\text{p}^2$
\end{solution}
\subsection{}
Donner le nom de l'atome dont les électrons de valence se trouve dans l'orbitale $4\text{s}^23\text{d}^3$.
\begin{solution}
	\configElectronique{23}
	
	Z = 23 $\rightarrow$ Vanadium
\end{solution}
\subsection{}
Quel est l'atome dont les électrons de valence adoptent la configuration $ 2\text{s}^1 2\text{p}^3$ dans un des états excités?

\begin{solution}
	\ce{C}: $1\text{s}^2 2\text{s}^2 2\text{p}^2$\\
	\ce{C^*}: $1\text{s}^2 2\text{s}^1 2\text{p}^3$
\end{solution}
\subsection{}

Pourquoi l'énergie de première ionisation de l'atome de soufre est-elle plus faible que celle de l'atome de phosphore ?
\begin{solution}
	\ce{P}:\configElectronique{15}\\
	\ce{S}:\configElectronique{16}\\
	L'énergie de 1\ere ionisation est inférieure pour le soufre car l'électron possédant un spin down est repoussé par celui possédant un spin up sur la sous-couche 3p.
\end{solution}

\subsection{}
Classez par ordre croissant des énergies d'ionisation les atomes suivants : \ce{B}, \ce{O},\ce{F} et \ce{Al}. Justifiez votre réponse.
\begin{solution}
	\ce{Al}: \configElectronique{13}\\
	\ce{B}: \configElectronique{5}\\
	\ce{O}: \configElectronique{8}\\
	\ce{F}: \configElectronique{9}\\
	
	Les énergies de première ionisation des éléments augmentent de gauche à droite par période (ligne) et diminuent en descendant les colonnes (sauf exceptions qui ne sont pas d'application ici).
\end{solution}

\subsection{}
Les énergies de première ionisation de \ce{Mg}, \ce{Al} et \ce{Si} sont de 735, 580 et 780 \SI{}{\kilo\joule\per\mole}, respectivement. Justifiez ces valeurs à l'aide des configurations électroniques.
\begin{solution}
	
\end{solution}

\subsection{}
Quelle est l'énergie absorbée ou dégagée par les réactions suivantes:\\
\ce{Si(g)} + $e^-$ \ce{->} \ce{Si^-(g)}\\
\ce{S^2+(g)} + $e^-$ \ce{->} \ce{Si^+(g)}
\begin{solution}
	La première équation est une capture d'électron et donc la formation d'un anion. Pour connaître l'énergie émise ou absorbée, on se réfère au tableau des électroaffinités. L'électroaffinité du silicium est de $\SI{-134}{\kilo\joule\per\mole}$.
	La deuxième équation représente la neutralisation partielle d'un cation. On commence avec un cation chargé $++$ et on lui donne un électron pour qu'il passe en $+$. Cette réaction est la réaction inverse de la deuxième ionisation, et l'énergie en jeu est donc l'opposée de celle-ci. On trouve donc que $E = \SI{-2260}{\kilo\joule\per\mole}$.
\end{solution}

\subsection{}
Déterminez le nombre d'électrons dans les orbitales s, p et d situées au-delà du coeur du gaz noble précédant les éléments suivants: argent, titane et germanium. Ces électrons sont-ils tous des électrons de valence?
\begin{solution}
	
\end{solution}
\subsection{}
