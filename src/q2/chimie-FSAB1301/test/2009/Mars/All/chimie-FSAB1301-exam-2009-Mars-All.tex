\documentclass[fr]{../../../../../../eplexam}
\usepackage{chemfig}

\hypertitle{Chimie et chimie physique}{2}{FSAB}{1301}{2009}{Mars}
{Martin Braquet}
{Bernard Nysten, Sophie Demoustier et Alain Jonas}

\section{(4 point)}

\begin{enumerate}
    \item Soit la formule linéaire de formule brute $C_2NOH$, l'hydrogène étant lié à un des atomes de carbone. Établir sa formule de Lewis.
    \item Expliciter à partir de quelles orbitales atomiques s'est formée la liaison carbone dans la molécule $C_2NOH$.
    \item La molécule présente-t-elle un moment dipolaire et si oui, dans quelle direction? Justifiez.
    \item Quel est le modèle proposé à partir de la mécanique quantique pour décrire l'orbitale moléculaire? Quels sont les apports nouveaux de cette approche par rapport au modèle du lien de valence?
\end{enumerate}

\begin{solution}

\begin{enumerate}
    \item 
    \chemfig{O=C(-[1]C(-[1]H))(-[7]N(=[2]))}       \qquad         
    \chemfig{N~C-C(-[1]H)(=[7]O)}
    
    Ces deux formules sont possibles mais celle requise est probablement la deuxième.
    
    \item La liaison $C-C$ possède une liaison $\sigma$ formée par recouvrement axial d'un orbitale $sp$ du $C$ et d'un orbitale $sp^2$ du $C$. 
    \item En ce qui concerne l'électronégativité:
    \begin{itemize}
        \item $C\rightarrow H$
        \item $N\rightarrow C$
        \item $O\rightarrow C$
    \end{itemize}
    On a ainsi un vecteur moment dipolaire qui pointe dans la direction $\uparrow $ par rapport à la molécule 2 ci-dessus. (cette réponse est à prendre avec des pincettes)
    \item Une orbitale moléculaire est, selon la théorie LCAO, une combinaison linéaire de deux orbitales atomiques.
    
Cette notion a permis de comprendre la plus grande densité électronique (plus que la simple additivité) en deux atomes liés par une liaison covalente (également, la réactivité et le paramagnétisme de l’oxygène).
\end{enumerate}

\end{solution}

\section{(5 points)}

\begin{enumerate}
    
\item Le composé $MnO_2$ est supposé purement ionique

\begin{itemize}
    \item Donner la configuration électronique de l’ion $Mn$ de ce composé.
    \item Citer un cation isoélectronique de l’ion oxygène de ce composé.
\end{itemize}

\item Comment justifier une énergie de liaison entre 2 atomes de chlore différente de celle entre 2 atomes de brome (valeurs données en annexe) ?

\item Les énergies de première ionisation et les affinités électroniques du $Na$ et du $Cl$ sont données en annexe. Comment expliquer les différences importantes de ces énergies entre ces deux atomes?

\item L’enthalpie de vaporisation du $Na$ est nettement supérieure à son enthalpie de fusion (voir annexe). Comment expliquer cette observation ?

\end{enumerate}

\begin{solution}

\begin{enumerate}

\item 

\begin{itemize}
    \item $Mn^{4+}$: $1s^22s^22p^63s^23p^63d^3$.
    \item Un cation possédant 10 $e^-$ est le $Na^+$.
\end{itemize}

\item L'énergie de liaison diminue avec le nombre d'électrons puisque les électrons de valence sont plus éloignés du noyau. L'électroaffinité est donc réduite.

\item Le $Na$ a tendance à perdre un électron pour se stabiliser alors que le chlore a tendance à en capter un. Le $Na$ a donc une électroaffinité plus faible en valeur absolue (énergie libérée lors de la capture d'un électron) et une énergie d'ionisation plus faible (énergie à fournir pour enlever un électron).

\item Lors de la fusion on casse un peu de liaisons entre les atomes $Na$ qui passent d'un état ordonné (solide: atomes liés à des places bien fixes) à un état désordonné (liquide: atomes mobiles mais encore liés).

Lors de la vaporisation on coupe toutes les liaisons entre les atomes $Na$ qui les maintenaient dans le liquide pour avoir $Na$ gazeux avec des atomes isolés.

\end{enumerate}

\end{solution}

\section{(4 points)}

Soit la réaction hypothétique suivante effectuée à température constante :
$$Na^{+}_{(g)} + Cl^{-}_{(g)} \Rightarrow Na_{(s)}+1/2\:{Cl_2}_{(g)}.$$
Déterminez en détaillant le calcul la variation d’enthalpie de la réaction. Les données nécessaires à ce
calcul se trouvent dans l’annexe.

\begin{solution}

Cette réaction consiste en une série de réaction élémentaires:

\begin{itemize}
    \item $Na^+_{(g)} + 1 e^-\Rightarrow Na_{(g)} \qquad \Delta E=-E_{ionis,Na}=-494\:[kJ/mol]$
    \item $Na_{(g)}\Rightarrow Na_{(s)}\qquad \Delta E=-\Delta H_{vap,Na}-\Delta H_{fus,Na}=-100,6\:[kJ/mol]$
    \item $Cl^-_{(g)}\Rightarrow Cl_{(g)} + 1 e^-\qquad \Delta E=-E_{aff,Cl}=349\:[kJ/mol]$
    \item $Cl_{(g)}\Rightarrow 1/2\:{Cl_2}_{(g)}\qquad \Delta E=-1/2\:E_{liaison,Cl_2}=-122\:[kJ/mol]$
\end{itemize}

$$ \Rightarrow \Delta_r H=-367,6\:[kJ/mol]$$

\end{solution}

\section{(3 points)}

Il existe plusieurs oxydes de cobalt. Afin d’obtenir du cobalt ($Co$) à l’état pur, on place dans un four 35,0 grammes d’un oxyde de cobalt et on le fait réagir à température élevée sous hydrogène. On récolte 9,4 grammes d’eau, la précision de la balance est de 0,1 gramme et le rendement de la réaction de 90\%. Déterminez en explicitant le calcul la formule chimique de l’oxyde.

\begin{solution}

Soit $Co_xO_y$, l'oxyde de cobalt recherché, la réaction s'écrit:
$$ Co_xO_y + y\:H_2 \Rightarrow y\:H_2O + x\:Co $$
On a au départ n moles d'oxyde,
$$(x\:Mm_{Co}+y\:Mm_{O})\:n=35\:g$$
\begin{equation}
    (58,93x+16y)\:n=35
\end{equation}
Au vu de l'équation réactionnelle, on a $0,9\:y\:n$ moles d'$H_2O$ formées.
$$0,9yn\:Mm{H_2O}=9,4\:g$$
\begin{equation}
    yn=\frac{9,4}{0,9*18}=0,58
\end{equation}
En divisant ensuite (1) par (2):
$$58,93\frac{x}{y}+16=\frac{35}{0,58}$$
$$\Rightarrow \frac{x}{y}=\frac{3}{4}$$
L'oxyde de cobalt est donc du
$$  Co_3O_4.$$

\end{solution}

\section{(1 point + 3 points pour le rapport de l'APP)}

Dans l’APP comment la quantité de fer perdue par affinage de la fonte a-t-elle été calculée?

\begin{solution}

Cette question est propre aux étudiants ayant suivi ce cours en 2009.

\end{solution}

\end{document}

