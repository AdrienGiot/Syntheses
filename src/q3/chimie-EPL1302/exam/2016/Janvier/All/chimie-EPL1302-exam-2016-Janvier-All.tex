\documentclass[fr]{../../../../../../eplexam}

\usepackage{../../../../../../eplchem}
\usepackage{../../../../../../eplunits}

\newcommand{\potstd}[1]{E\std_{\text{éq},\text{#1}}}
\newcommand{\potstdce}{\potstd{cell}}
\newcommand{\potstdca}{\potstd{cathode}}
\newcommand{\potstdan}{\potstd{anode}}
\newcommand{\cv}{c_\textnormal{v}}
\newcommand{\cp}{c_\textnormal{p}}
\newcommand{\pf}{P_\textnormal{f}}
\renewcommand{\pi}{P_\textnormal{i}}
\newcommand{\tf}{T_\textnormal{f}}
\newcommand{\ti}{T_\textnormal{i}}

% TODO add other questions

\hypertitle{Chimie et chimie physique}{3}{FSAB}{1302}{2016}{Janvier}{All}
{Louis Devillez\and Gilles Peiffer}
{Hervé Jeanmart et Joris Proost}

\section{Question 1}
Afin de maximiser le rendement de certaines installations thermiques,
on valorise au maximum l'énergie à basse température
avant de rejeter le reste à la source froide.
C'est le cas des turbines à gaz régénératives
dont un équivalent en cycle fermé est composé des transformations suivantes:

\begin{itemize}
	\item $1 \to 2$: compression adiabatique;
	\item $2 \to 3$: réchauffement isobare;
	\item $3 \to 4$: apport de chaleur isobare;
	\item $4 \to 5$: détente adiabatique;
	\item $5 \to 6$: refroidissement isobare;
	\item $6 \to 1$: échange isobare avec la source froide.
\end{itemize}

Le gaz qui parcourt le cycle est assimilé à de l'air (gaz parfait)
dont les propriétés thermiques sont telles que $\gamma = 1.3$.
Dans ce cycle, la chaleur nécessaire au réchauffement du fluide de 2 à 3
est fournie par le refroidissement du fluide de 5 à 6.
Cet échange se fait dans un échangeur parfait.
Cela veut dire que $T_6 = T_2$ et qu'il n'y a pas de pertes
vers l'extérieur du cycle.
Toute la chaleur perdue de 5 à 6 est récupérée de 2 à 3.
L'apport de chaleur de la combustion est de ce fait réduit
et équivaut à $\SI{1000}{\kilo\joule\per\kilogram}$.

Sur base de ces informations
et des valeurs déjà données dans le tableau ci-dessous, on vous demande de
\begin{enumerate}
	\item compléter le tableau (en justifiant très succinctement);
	\item calculer le travail net du cycle;
	\item calculer le rendement du cycle.
\end{enumerate}

\begin{center}
	\begin{tabular}{|c|c|c|c|c|}
		\hline
		&p$[\si{\bar}]$&$T[\si{\kelvin}]$&$V[\si{\meter\cubed}]$&$S-S_1 [\SI{}{\joule\per\kelvin}]$\\
		\hline
		1&1&300&0.002&\\
		\hline
		2&20&&&\\
		\hline
		3&&&&\\
		\hline
		4&&&&\\
		\hline
		5&&&&\\
		\hline
		6&&&&\\
		\hline
	\end{tabular}
\end{center}

\begin{solution}
	\begin{enumerate}
	\item Commençons par remplir le tableau avec ce que l'on connaît.
	\begin{center}
		\begin{tabular}{|c|c|c|c|c|}
			\hline
			&p$[\si{\bar}]$&$T[\si{\kelvin}]$&$V[\si{\meter\cubed}]$&$S-S_1 [\SI{}{\joule\per\kelvin}]$\\
			\hline
			1&1&300&$0.002$&0\\
			\hline
			2&20&$T_2$&&0\\
			\hline
			3&20&&&\\
			\hline
			4&20&&&\\
			\hline
			5&1&&&\\
			\hline
			6&1&$T_2$&&\\
			\hline
		\end{tabular}
	\end{center}
	Comme le gaz est assimilé à l'air, on a $R^* = \SI{287.1}{\joule\per\kelvin\per\kilogram}$.
	Ensuite, la masse se calcule facilement, et on trouve
	\[
	m = \frac{P V}{R^* T} = \SI{0.002322}{\kilogram}\,.
	\]
	Étant donné que $\gamma$ nous est donné
	nous devons recalculer les $\cv$ et $\cp$:
	\begin{align*}
	\gamma &= \frac{\cp}{\cv} = \frac{\cv + R^*}{\cv}\,,\\
	\iff \cv &= \frac{R^*}{\gamma - 1} = \SI{957}{\joule\per\kelvin\per\kilogram}\,,\\
	\iff \cp &= \cv + R^* = \SI{1244.1}{\joule\per\kelvin\per\kilogram}\,.
	\end{align*}

	La différence d'entropie se calcule comme
	\[
	S_2 - S_1 = m\cp \ln\left(\frac{\tf}{\ti}\right) - mR^*\ln\left(\frac{\pf}{\pi}\right) = 0\,.
	\]
	On trouve la température suivante en $2$:
	\[
	T_2 = \exp\left(\frac{R^*\ln\left(\frac{\pf}{\pi}\right)}{\cp}\right)T_1 = \SI{599}{\kelvin}\,.
	\]

	On peut également calculer
	\[
	Q_{5 \to 6} = -Q_{2 \to 3}\,,
	\]
	ce qui donne
	\begin{align*}
		c_p (T_3 - T_2) &= - c_p (T_6 - T_5)\\
		\iff T_3 - T_2 &= - (T_6 - T_5)\\
		\iff T_3 &= T_5\,.\\
	\end{align*}

	On calcule $T_4$ comme suit:
	\[
	T_4 = T_3 + \delta T = T_3 + \frac{Q}{\cp} = T_3 + 803.794\,.
	\]

	En utilisant la transformation $4 \to 5$, qui est adiabatique, on trouve
	\[
	m\cp\ln\left(\frac{\tf}{\ti}\right) = mR^*\ln\left(\frac{\pf}{\pi}\right)\,.
	\]
	On applique ceci pour calculer
	\begin{align*}
		T_5 &= \exp\left(\frac{R^*\ln\left(\frac{\pf}{\pi}\right)}{\cp}\right)T_4 = 0.500913\,T_4 \\
		\iff 1.9963\,T_5 &= T_5 + 803.794\\
		\iff T_5 &= \SI{807}{\kelvin} = T_3\,,\\
		\iff T_4 &= \SI{1611}{\kelvin}\,.
	\end{align*}

	On peut ensuite finir le tableau.
	\begin{center}
		\begin{tabular}{|c|c|c|c|c|}
			\hline
			&p$[\si{\bar}]$&$T[\si{\kelvin}]$&$V[\si{\meter\cubed}]$&$S-S_1 [\si{\joule\per\kelvin}]$\\
			\hline
			1&1&300&$0.002$&0\\
			\hline
			2&20&599&$0.0002$&0\\
			\hline
			3&20&807&$0.00027$&$S_3$\\
			\hline
			4&20&1611&$0.00054$&$S_4$\\
			\hline
			5&1&807&$0.00538$&$S_4$\\
			\hline
			6&1&599&$0.00399$&$S_6$\\
			\hline
		\end{tabular}
	\end{center}
	Les entropies ont été obtenues comme suit:
	\begin{align*}
		S_3 - S_2 &= m\cp \ln\left(\frac{\tf}{\ti}\right) = \SI{0.86}{\joule\per\kelvin}\,,\\
		S_4 - S_3 &= m\cp \ln\left(\frac{\tf}{\ti}\right) = \SI{2.86}{\joule\per\kelvin}\,,\\
		S_6 - S_4 &= m\cp \ln\left(\frac{\tf}{\ti}\right) = \SI{2.00}{\joule\per\kelvin}\,.
	\end{align*}

	\item On calcule le travail total comme
	\[
	W_{\textnormal{tot}} = \sum m\cp \Delta T= 0.0023 \cdot 287 \big(208 + 804 + (- 208) + (-299)\big) = \SI{1459}{\joule}\,.
	\]

	\item Le rendement se trouve comme
	\[
	\eta = \frac{W_{\textnormal{tot}}}{Q} = \frac{1459}{m\cp(T_4-T_3)} = 0.628\,.
	\]
	\end{enumerate}
\end{solution}
\end{document}
