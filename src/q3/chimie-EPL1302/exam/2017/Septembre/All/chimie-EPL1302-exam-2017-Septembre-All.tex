\documentclass[fr]{../../../../../../eplexam}
\usepackage{../../../../../../eplchem}
\usepackage{../../../../../../eplunits}

\hypertitle{Chimie et chimie physique}{3}{FSAB}{1302}{2017}{Août}{All}
{Martin Braquet}
{Hervé Jeanmart et Joris Proost}

\section{Cycle}

On considère un système fermé parcourant un cycle réversible composé de 4 transformations:
\begin{itemize}
	\item $1 \to 2$: compression adiabatique
	\item $2 \to 3$: apport de chaleur isobare 
	\item $3 \to 4$: détente adiabatique
	\item $4 \to 1$: refroidissement isochore
\end{itemize}
Le gaz contenu dans le système est de l'air (gaz parfait) dont les propriétés sont constantes et telles que $\gamma=1,4$.
\begin{enumerate}
	\item Le travail total effectué au cours du cycle est de \SI{1000}{\joule} (en valeur absolue). La pression maximale est de 100 bar et la température maximale vaut $2000K$ (ou $2200K$ pour le second questionnaire).
	
	À partir de ces valeurs, on vous demande de compléter le
	tableau ci-dessous en justifiant succinctement vos résultats.
	\begin{center}
		\begin{tabular}{|c|cccc|}
			\hline
			& $p[\si{\bar}]$ & $V[m^3]$ & $T[\si{\kelvin}]$ & $S-S_1[J/K]$\\
			\hline
			1 & 2 &  & 300 & \\
			\hline
			2 &  &  &  & \\
			\hline
			3 &  &  &  & \\
			\hline
			4 &  &  &  & \\
			\hline
		\end{tabular}
	\end{center}
	\item Calculez le rendement de ce cycle.
	
\end{enumerate}

\begin{solution}
	
	On cherche les capacités calorifiques:
	$$ \frac{c_p}{c_v}=\gamma=1,4 \qquad c_p-c_v=R^*$$
	$$ \Rightarrow c_p=\frac{7}{2}R^*=1004,5 \: [J/(kg.K)] \qquad c_v=\frac{5}{2}R^*=717,5\: [J/(kg.K)]$$
	On rappelle aussi que $R^*=R/M_m=287,058\:[J/(kg*K)]$.
	
	La pression augmente de $1 \to 2$, est identique de $2 \to 3$ et diminue de $3 \to 4$, la pression maximale est donc en 2 et 3.\\
	La température augmente de $1 \to 2$ et de $2 \to 3$, et puis diminue de $3 \to 4$, la température maximale est donc en 3.\\
	On peut maintenant calculer $T_2$ par la loi de Poisson (adiabatique):
	$$ T_2=T_1(\frac{p_1}{p_2})^\frac{1-\gamma}{\gamma}=917,363 \:[K]  $$
	
	On a ainsi ce tableau-ci en complétant avec les données trouvées, aucune autre valeur ne peut être obtenue directement (hormis $\Delta S$):
	\begin{center}
	\begin{tabular}{|c|cccc|}
		\hline
		& $p[\si{\bar}]$ & $V[m^3]$ & $T[\si{\kelvin}]$ & $S-S_1[J/K]$\\
		\hline
		1 & 2 &  & 300 & \\
		\hline
		2 & 100 &  & 917,363 & \\
		\hline
		3 & 100 &  & $T_{MAX}$ & \\
		\hline
		4 &  &  &  & \\
		\hline
	\end{tabular}
	\end{center}
	A ce stade, une bonne technique est d'écrire toutes les équations \textbf{non redondantes}. Dans les informations citées dans le texte, on met en équation la donnée $|W_{TOT}|=1000\:[J]$:
	$$\pm 1000=-W_{TOT}=Q_{TOT}=mc_p(T_3-T_2)+mc_v(T_1-T_4)$$
	car 
	\begin{itemize}
		\item $Q_{1-2}=0$ (adiabatique)
		\item $Q_{2-3}=\Delta H_{2-3}=mc_p(T_3-T_2)$  (isobare)
		\item $Q_{3-4}=0$ (adiabatique)
		\item $Q_{4-1}=\Delta U_{4-1}=mc_v(T_1-T_4)$  (isochore)
	\end{itemize}
	et on ne connait pas le signe du travail total pour le moment (négatif si c'est un cycle moteur).
	
	Il faut encore utiliser les 4 équations de la loi des gaz parfaits et celles liées aux 4 transformations. Les transformations $1 \to 2$ et $2 \to 3$ ont déjà été utilisées pour trouver $T_2$ et $p_3$, il reste donc les 2 transformations suivantes qui n'ont pas été utilisées et qui mènent à $p_3V_3^\gamma=p_4V_4^\gamma$ et $V_4=V_1$. 
	
	On a ainsi un système de 7 équations à 7 inconnues ($m,V_1,V_2,V_3,V_4,p_4,T_4$):
	$$
	\left\{ 
	\begin{array}{rcl} 
	mR^*T_1 & = & p_1V_1 \\ 
	mR^*T_2 & = & p_2V_2  \\ 
	mR^*T_3 & = & p_3V_3 \\ 
	mR^*T_4 & = & p_4V_4 \\ 
	p_3V_3^\gamma & = & p_4V_4^\gamma \\ 
	V_4 & = & V_1 \\ 
	\pm 1000 & = & mc_p(T_3-T_2)+mc_v(T_1-T_4)
	\end{array} 
	\right.
	$$
	
	\subsection*{Questionnaire 1: $T_{MAX}=2000K$}
	On le résout:
	$$\frac{mR^*T_1}{mR^*T_4}=\frac{p_1V_1}{p_4V_4}\Longrightarrow \frac{T_1}{T_4}=\frac{p_1}{p_4}\Longrightarrow p_4=\frac{p_1T_4}{T_1}$$
	Par la loi de poisson,
	$$(\frac{T_4}{T_3})^{\gamma}=(\frac{p_3}{p_4})^{1-\gamma}=(\frac{p_3T_1}{p_1T_4})^{1-\gamma}$$
	$$\Longrightarrow T_4=T_3^\gamma(\frac{p_3T_1}{p_1})^{1-\gamma}=893,317\: [K]$$
	$$\Longrightarrow p_4=\frac{p_1T_4}{T_1}=5,95546\: [bar]$$
	La dernière équation ne possède maintenant qu'une inconnue,
	$$m=\frac{\pm 1000}{c_p(T_3-T_2)+c_v(T_1-T_4)}=1,511.10^{-3}\:[kg]$$
	et le signe + est associé à $Q_{TOT}$, le travail total est donc négatif et c'est un cycle moteur.\\
	Il suffit enfin de calculer les volumes,
	$$ V_1=\frac{mR^*T_1}{p_1}=6,505.10^{-4}\:[m^3]$$
	$$ V_2=\frac{mR^*T_2}{p_2}=3,9782.10^{-5}\:[m^3]$$
	$$ V_3=\frac{mR^*T_3}{p_3}=8,6732.10^{-5}\:[m^3]$$
	$$ V_4=\frac{mR^*T_4}{p_4}=6,505.10^{-4}\:[m^3]$$
	En ce qui concerne les variations d'entropie:
	$$S_2-S_1=0 \mbox{   (adiabatique)}$$
	$$S_3-S_2=\int_2^3\frac{\delta Q_{2-3}}{T}=\int_2^3\frac{\mbox{d}h}{T}=\int_2^3\frac{mc_p \mbox{d}T}{T}=mc_p \ln\frac{T_3}{T_2}=1,183\: [J/K]$$
	$$S_4-S_3=0  \mbox{   (adiabatique)}$$
	$$S_1-S_4=\int_4^1\frac{\delta Q_{4-1}}{T}=\int_4^1\frac{\mbox{d}u}{T}=\int_4^1\frac{mc_v \mbox{d}T}{T}=mc_v \ln\frac{T_1}{T_4}=-1,183\: [J/K]$$
	Le tableau final est donc:
	\begin{center}
		\begin{tabular}{|c|cccc|}
			\hline
			& $p[\si{\bar}]$ & $V[m^3]$ & $T[\si{\kelvin}]$ & $S-S_1[J/K]$\\
			\hline
			1 & 2 & $6,505.10^{-4}$ & 300 & 0\\
			\hline
			2 & 100 & $3,97827.10^{-5}$ & 917,363 & 0\\
			\hline
			3 & 100 & $8,67326.10^{-5}$ & 2000 & 1,183\\
			\hline
			4 & 5,95546 & $6,505.10^{-4}$ & 893,317 & 1,183\\
			\hline
		\end{tabular}
	\end{center}
	
	$$\eta=\frac{|W_{tot}|}{Q_{HOT}}=\frac{|W_{tot}|}{Q_{2-3}}=\frac{|W_{tot}|}{mc_p(T_3-T_2)}=\frac{1000}{1643,23}=0,609$$
	
	\subsection*{Questionnaire 2: $T_{MAX}=2200K$}
	On le résout:
	$$\frac{mR^*T_1}{mR^*T_4}=\frac{p_1V_1}{p_4V_4}\Longrightarrow \frac{T_1}{T_4}=\frac{p_1}{p_4}\Longrightarrow p_4=\frac{p_1T_4}{T_1}$$
	Par la loi de poisson,
	$$(\frac{T_4}{T_3})^{\gamma}=(\frac{p_3}{p_4})^{1-\gamma}=(\frac{p_3T_1}{p_1T_4})^{1-\gamma}$$
	$$\Longrightarrow T_4=T_3^\gamma(\frac{p_3T_1}{p_1})^{1-\gamma}=1020,8\: [K]$$
	$$\Longrightarrow p_4=\frac{p_1T_4}{T_1}=6,80556\: [bar]$$
	La dernière équation ne possède maintenant qu'une inconnue,
	$$m=\frac{\pm 1000}{c_p(T_3-T_2)+c_v(T_1-T_4)}=1,29666.10^{-3}\:[kg]$$
	et le signe + est associé à $Q_{TOT}$, le travail total est donc négatif et c'est un cycle moteur.\\
	Il suffit enfin de calculer les volumes,
	$$ V_1=\frac{mR^*T_1}{p_1}=5,58.10^{-4}\:[m^3]$$
	$$ V_2=\frac{mR^*T_2}{p_2}=3,414.10^{-5}\:[m^3]$$
	$$ V_3=\frac{mR^*T_3}{p_3}=8,187.10^{-5}\:[m^3]$$
	$$ V_4=\frac{mR^*T_4}{p_4}=5,58.10^{-4}\:[m^3]$$
	
	En ce qui concerne les variations d'entropie:
	$$S_2-S_1=0 \mbox{   (adiabatique)}$$
	$$S_3-S_2=\int_2^3\frac{\delta Q_{2-3}}{T}=\int_2^3\frac{\mbox{d}h}{T}=\int_2^3\frac{mc_p \mbox{d}T}{T}=mc_p \ln\frac{T_3}{T_2}=1,14\: [J/K]$$
	$$S_4-S_3=0  \mbox{   (adiabatique)}$$
	$$S_1-S_4=\int_4^1\frac{\delta Q_{4-1}}{T}=\int_4^1\frac{\mbox{d}u}{T}=\int_4^1\frac{mc_v \mbox{d}T}{T}=mc_v \ln\frac{T_1}{T_4}=-1,14\: [J/K]$$
	Le tableau final est donc:
	\begin{center}
		\begin{tabular}{|c|cccc|}
			\hline
			& $p[\si{\bar}]$ & $V[m^3]$ & $T[\si{\kelvin}]$ & $S-S_1[J/K]$\\
			\hline
			1 & 2 & $5,58.10^{-4}$ & 300 & 0\\
			\hline
			2 & 100 & $3,414.10^{-5}$ & 917,363 & 0\\
			\hline
			3 & 100 & $8,187.10^{-5}$ & 2200 & 1,14\\
			\hline
			4 & 6,806 & $5,58.10^{-4}$ & 1020,8 & 1,14\\
			\hline
		\end{tabular}
	\end{center}
	
	$$\eta=\frac{|W_{tot}|}{Q_{HOT}}=\frac{|W_{tot}|}{Q_{2-3}}=\frac{|W_{tot}|}{mc_p(T_3-T_2)}=\frac{1000}{1671}=0,6$$
	
	Je conseille cette méthodologie car on est ainsi certain d'utiliser toutes les données et puisqu'on arrive à un système possédant le même nombre d'équations que d'inconnues, on est déjà certain à ce stade d'avoir résolu le cycle. En résumé, 
	\begin{enumerate}
		\item Calculer toutes les données que l'on peut trouver directement
		\item Mettre en équation toutes les données dans l'énoncé
		\item Former un système avec toutes les équations non encore utilisées à ce stade et vérifier qu'il y a bien le même nombre d'équations que d'inconnues
		\item Résoudre le système
		\item Calculer les variation d'entropie (les calculer avant permettent parfois de simplifier les calculs mais les équations supplémentaires qui leur sont liées sont redondantes avec celles déjà énoncées dans le système)
		\item Calculer le rendement
	\end{enumerate}
	
\end{solution}

\end{document}
