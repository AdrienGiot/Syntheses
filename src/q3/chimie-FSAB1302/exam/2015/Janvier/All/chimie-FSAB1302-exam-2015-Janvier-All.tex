\documentclass[fr]{../../../../../../eplexam}

\usepackage{../../../../../../eplchem}
\usepackage{../../../../../../eplunits}

\newcommand{\potstd}[1]{E\std_{\text{éq},\text{#1}}}
\newcommand{\potstdce}{\potstd{cell}}
\newcommand{\potstdca}{\potstd{cathode}}
\newcommand{\potstdan}{\potstd{anode}}

\hypertitle{Chimie et chimie physique}{3}{FSAB}{1302}{2015}{Janvier}
{Benoît Legat}
{Hervé Jeanmart et Joris Proost}

\section{}
\begin{enumerate}
  \item Ecrivez, pour la corrosion en milieu aqueux du \ce{Cu} en son ion \ce{Cu++}
    \begin{itemize}
      \item toutes les demi-réactions cathodiques possibles;
      \item toutes les réactions rédox globales possibles;
    \end{itemize}
    (il ne faut pas se faire se soucis sur l'évolution éventuelle des produits de corrosion).
  \item Calculez en se basant sur l'échelle rédox ci-dessous,
    les forces électromotrices de ces réactions de corrosion en
    milieu acide sous conditions standart.
  \item Comment pourrait-on envisager de protéger des structures en \ce{Cu} contre la corrosion ?
  \item Sachant que la vitesse de corrosion peut très souvent
    être considérée comme une constante, trouvez une expression
    générale pour sa constante de vitesse en fonction du courant de corrosion.
\end{enumerate}
[tableau de Nernst fourni].

\begin{solution}
  \begin{enumerate}
    \item
      Le \ce{Cu} donne des électrons donc il fait une oxydation, il est à l'anode.
      La demi-réaction est
      \[ \ce{Cu} = \ce{Cu++} + 2e^- \]

      Il faut maintenant voir si on est dans le cas pile ou électrolyse.
      La corrosion est une pile donc
      \begin{equation}
        \label{eq:potstdce}
        \potstdce = \potstdca - \potstdan \geq 0.
      \end{equation}
      On a
      \[ \potstdan = E\std_{\ce{Cu++}/\ce{Cu}} = 0.34. \]
      Donc, pour avoir $\potstdce$, il faut
      \[ \potstdca \geq 0.34 \]

      Les demi-réactions cathodiques/de réduction possibles sont:
      \begin{align*}
        \ce{O2(g) + 4H+(aq)} + 4e^- & = 2\ce{H2O(l)}\\
        \ce{O2(g) + 2H2O(l)} + 4e^- & = 4\ce{OH^-(aq)}
     \end{align*}
      et les réactions globalent correspondantes sont
      \begin{align*}
        \ce{2Cu(s) + O2(g) + 4H+(aq)} & = \ce{2Cu++(aq) + 2H2O(l)}\\
        \ce{2Cu(g) + O2(g) + 2H2O(l)} & = \ce{2Cu++(aq) + 4OH^-(l)}.
      \end{align*}

      Je n'ai mis ici uniquement ceux avec l'eau.
      On peut aussi s'amuser à le faire avec tous les autres qui sont au dessus du Cu dans le tableau des potentiels standards.
      Par exemple avec $\ce{I2(s)} + 2e^- = 2\ce{I^-(aq)}$.
    \item En utilisant l'équation~\eqref{eq:potstdce}, on a respectivement
      \begin{align*}
        \potstdce & = 1.23 - 0.34 = 0.89\\
        \potstdce & = 0.40 - 0.34 = 0.06.
      \end{align*}
    \item
      La réaction qui va se faire préférablement est celle avec le plus grand $\potstdce$.
      En effet, le plus $\potstdca$ est grand, le plus le réducteur est un réducteur fort et le plus $\potstdan$ est petit le plus l'oxydant est un réducteur faible et donc un oxydant fort.

      Si on ajoute du zinc par exemple, $E\std_{\ce{Zn++}/\ce{Zn}} < E\std_{\ce{Cu++}/\ce{Cu}}$ donc il réagira à la place du cuivre.
      De plus du carbonate du zinc se crée avec le \ce{CO2} qui crée une barrière supplémentaire,
      le \ce{Cu} se corrode donc encore moins.
    \item
      C'est la loi de Faraday.
      Soit $z$ le nombre d'électron échangés, ici $z = 4$.
      En utilisant la stochiométrie de l'équation, on voit que
      pour 4 électrons, on a 2 atomes de \ce{Cu++}.
      Si le courant de corrosion est $I$, après un temps $t$, on a $It$ coulombs d'électrons.
      Pour passer en mole on divise par la constante de Faraday car cette constante vaut la charge en coulomb d'un électron.
      On a donc $It/F$ moles d'électrons en un temps $t$ ce qui fait
      \[ \frac{2}{z}\frac{It}{F} = \frac{It}{2F} \]
      moles de \ce{Cu++}.
      La vitesse de corrosion en mole par seconde est donc
      \[ \frac{I}{2F}. \]
      Si on veut l'avoir en gramme par seconde il suffit de multiplier par la masse molaire du \ce{Cu}
      \[ \frac{IM_{\ce{Cu}}}{2F}. \]
  \end{enumerate}
\end{solution}

\section{}
\begin{enumerate}
  \item Exposez (citez les relations mathématiques) et expliquez la définition du second principe.
  \item Expliquez comment établir le lien entre l'évolution d'entropie associée à une transformation
    adiabatique irréversible et la définition fondamentale de l'entropie $\dif S = \frac{\delta Q}{T}$.
  \item Appliquez la démarche exposée au point 2 à un exemple pertinent de votre choix.
  \item Expliquez la seconde expérience de Joules.
\end{enumerate}

\nosolution
%\begin{solution}
% \begin{enumerate}
%   \item
%     Le second principe dit que la différence d'entropie d'un système et son environnement ne peut qu'augmenter. Ecrit mathématiquement:
%     \[ dS = dS_{interieur} + ds_{echange} > 0 \]
%   \item
%     Une transformation adiabatique n'échange pas de chaleur avec son environnement, il n'y a donc pas de $dS_{echange}$ avec l'environnement.
%     Puisqu'elle est irreversible, il y a création interne d'entropie.
%   \item
%     Je pense que l'experience ou on sépare 2 volumes, l'un contenant du gaz, l'autre du vide, par une plaque, puis qu'on enleve la plaque est adia-irreversible. Y'auras pas d'echange de chaleur, mais création d'entropie du a l'expansion du gaz.
% \end{enumerate}
%\end{solution}

\section{}
Limité à 1/2 face par réponse

Note: cette question semble adaptée de l'exercice 14.91 de la  5\ieme{} ou 6\ieme{} édition du Atkins et Jones.

Des observations expérimentales en labo ont montré que la vitesse de la réaction
\[ \ce{2NO(g) + 2H2(g)} = \ce{N2(g) + 2H2O(g)} \]
peut être raisonablement approximée comme $k[\ce{NO}]^2[\ce{H2}]$.
Le mécanisme réactionnel suivant est alors proposé:
\begin{itemize}
  \item étape 1: $\ce{NO + NO} = \ce{N2O2}$
  \item étape 2: $\ce{N2O2 + H2} = \ce{N2O + H2O}$
  \item étape 3: $\ce{N2O + H2} = \ce{N2 + H2O}$
\end{itemize}

\begin{enumerate}
  \item Quelle est vraissemblablement l'étape déterminante de vitesse ?
  \item Sur base de votre réponse ci-dessus, trouvez une expression pour la constante de vitesse
    en fonction des constantes thermodynamiques et/ou cinétiques d'une ou plusieurs de ces 3 étapes.
  \item Dessinez un schéma énergétique plausible pour la réaction globale dont on sait qu'elle est exothermique.
    Indiquez là-dessus également les énergies d'activation pour chaque étape avec les symboles $E_{a,1}$, $E_{a,2}$ et $E_{a,3}$.
\end{enumerate}
\begin{solution}
  \begin{enumerate}
    \item
      L'étape déterminante est la plus lente.
      Les autres sont supposés rapide par rapport à l'étape déterminante.
      Rappelons nous que les réactions élémentaires qui suivent l'étape déterminante de la vitesse ne contribuent pas à la loi de vitesse déduite du mécanisme.
      \begin{itemize}
        \item Si c'est l'étape 1, on a $k_1[\ce{NO}]^2$ comme vitesse. C'est pas ça.
        \item Si c'est l'étape 2, on a $k_2[\ce{N2O2}][\ce{H2}]$ comme vitesse.
          Hors comme la première est très rapide par rapport à la deuxième,
          on peut supposer qu'elle satisfait la condition de pré-équilibre
          \[ K_1 = \frac{k_1}{k_1'} = \frac{[\ce{N2O2}]}{[\ce{NO}]^2}. \]
          où $K_1$ est la constante d'équilibre, $k_1$ la constante de vitesse de $\ce{NO + NO} -> \ce{N2O2}$
          et $k_1'$ la constante de vitesse de $\ce{N2O2} -> \ce{NO + NO}$.
          On trouve donc que la vitesse est $k_2K_1[\ce{NO}]^2[\ce{H2}$.
        \item Si c'est l'étape 3, il faut être un peu plus prudent et pas utiliser les pré-équilibre mais l'hypothèse de quasi-stationnarité des radicaux.
          Après des calculs dégueux on ne trouve pas la bonne expression de la vitesse.
      \end{itemize}
      Hors comme la première est rapide par rapport à la deuxième
    \item En fonction d'uniquement les constantes cinétiques, on a $k = k_2\frac{k_1}{k_1'}$
      et en utilisant la constante thermodynamique $K_1$, on a $k = k_2K_1$.
    \item
      Les pics qui représentent la formation des intermédiaires \ce{N2O2} et \ce{N2O} ne sont pas à la même énergie,
      mais nous avons aucune information pour déterminer lequel sera plus bas.
  \end{enumerate}
\end{solution}

\section{}
On considère un système fermé parcourant un cycle moteur réversible composé de 5 transformations (cycle de Atkinson):
\begin{itemize}
  \item $1 \to 2$: compression adiabatique
  \item $2 \to 3$: échauffement isochore
  \item $3 \to 4$: détente adiabatique
  \item $4 \to 5$: refroidissement isochore
  \item $5 \to 1$: diminution de volume isobare
\end{itemize}
Le gaz contenu dans le système est de l'air (gaz parfait) dont le propriétés sont constantes
et évaluées à température ambiante.
\begin{enumerate}
  \item De ce système, on connait le rapport de compression,
    $V_1/V_2 = 10$ et le rapport de détente $V_4/V_3 = 13$.
    La chaleur apportée lors de l'échauffement est de \SI{2000}{\kilo\joule\per\kilogram}.
    À partir de ces valeurs, on vous demande de compléter le
    tableau ci-dessous en justifiant succinctement vos résultat.
    \[
      \begin{array}{|c|cccc|}
        \hline
        & p[\si{\bar}] & T[\si{\kelvin}] & V[\si{\meter\cubed}] & S-S_1[\si{\joule\per\kelvin}]\\
        \hline
        1 & 1 & 300 & 0.001 & 0\\
        \hline
        2 &  &  &  & \\
        \hline
        3 &  &  &  & \\
        \hline
        4 &  &  &  & \\
        \hline
        5 &  &  &  & \\
        \hline
      \end{array}
    \]
  \item Calculez le travail net de ce cycle et son rendement.
  \item Tout autre paramètre étant constant, quel serait le rendement
    dans ce cas ? Justifiez votre démarche.
\end{enumerate}

\nosolution

\section{QCM}
Notes: Il y avait 20 questions.

\begin{enumerate}
  \item Le(s) quel(s) des symboles ci-dessous est (sont) synonyme(s) du potentiel chimique standard d'un composant
    $i$ dans un mélange réactionnel ?

    $G_i\std$, $G_{i,m}\std$, $(\partial H/\partial n_i)_{S,p,n_{j \neq i}}$, ${G_m^i}\std$, $\dif G / \dif n_i$
  \item W.H. Nernst est né avant J.W. Gibbs.

    Vrai, Faux.

  \item Une machine frigorifique transmet selon un cycle de Carnot inverse entre une température extérieure de \SI{25}{\celsius}
    et intérieure de $-\SI{10}{\celsius}$.
    Quel est le coefficient de performance de cette machine ?

  \item
    À \SI{300}{\kelvin} et à pression atmosphérique,
    quelle est la vitesse quadratique moyenne
    des molécules de dioxyde de carbone ?

    \SI{11}{\meter\per\second}, \SI{337}{\meter\per\second}, \SI{412}{\meter\per\second}, \SI{300}{\meter\per\second}, \SI{505}{\meter\per\second}.
  \item
    Quelle est la définition générale de l'activité d'un composant gazeux $i$ ?

    $p_i\std/1$, $p/p_i\std$, $p_i/1$, $p_i/p_i\std$, $p_i/p\std$, Aucun.
  \item
    Dans un système ouvert, une transformation d'un gaz parfait s'effectue de manière réversible et
    sans échange de chaleur entre un état 1 et un état 2.
    On néglige les termes liés à l'énergie cinétique et l'énergie potentielle.
    Quelle(s) expression(s) est (sont) correcte(s) pour exprimer le travail ? Échange ?

    $w_m = \frac{\gamma - 1}{\gamma} p_1v_1 \left(\frac{p_2}{p_1} \frac{\gamma - 1}{\gamma}\right)$,
    $w_m = \frac{\gamma}{\gamma - 1} p_1v_1 \left(\frac{p_2}{p_1} \frac{\gamma - 1}{\gamma}\right)$,
    $w_m = c_p T_2 \left(\left(\frac{p_2}{p_1}\right)^{\frac{\gamma - n}{\gamma}} - 1\right)$,
    $w_m = \frac{p_1v_1}{\gamma - 1} \left(\left(\frac{V_1}{V_2}\right)^{\gamma - 1} - 1\right)$,
    $w_m = c_p (T_2 - T_2)$,
    aucune.
  \item
    Dans un système fermé, on souhaite augmenter la température avec volume constant.
    Quelle transformation est ok ?

    Chaleur à volume constant, Augmenter le volume à pression constante puis compression isotherme,
    Détente isotherme puis compression adiabatique,
    Compression isotherme puis détente adiabatique,
    Apporter travail à volume constant,
    Aucune correcte.
  \item
    Mesure d'un débit dans un conduit de diamètre \SI{100}{\milli\meter},
    on réalise localement une contraction régulière avec un rapport de diamètre 4.
    La chute de pression locale est \SI{2000}{\pascal}.
    Le débit approximatif est:

    \SI{1000}{\liter\per\second}, \SI{1}{\kilogram\per\second}, \SI{200}{\liter\per\second},
    \SI{2}{\kilogram\per\second}, \SI{16}{\kilogram\per\second}.
  \item
    Transformer sulfures en oxyde (grillage):
    \[ \ce{2ZnS(s) + 3O2(g)} = \ce{2ZnO(s) + 2SO2(g)}. \]
    Quelle est à une température et pp d'\ce{O2} ($p_{\ce{O2}}$) donnée,
    la condition à respecter pour diriger le procédé dans le bon sens ?

    $p_{\ce{SO2}} < p_{\ce{SO2},\text{éq}}$,
    $p_{\ce{SO2}} > p_{\ce{SO2},\text{éq}}$,
    $\log(p_{\ce{SO2}}) < \log(p_{\ce{SO2},\text{éq}})$,
    $\log(p_{\ce{SO2}}) > \log(p_{\ce{SO2},\text{éq}})$,
    $p_{\ce{SO2}} > p_{\ce{O2}}$,
    $p_{\ce{SO2}} < p_{\ce{O2}}$.
\end{enumerate}

\begin{solution}
  \begin{enumerate}
    \item $G_{i,m}\std$ et $(\partial H/\partial n_i)_{S,P,n_{j \neq i}}$.
      Pas ${G_m^i}\std$ car c'est pour une substance $i$ pure et un est dans un mélange réactionnel.
    \item Faux.
    \item
      \[ COP = \frac{1}{\frac{T_I}{T_{II}} - 1} = 7.5 \]
    \item
      We have
      \begin{align*}
        p & = \frac{1}{3}\rho \overline{c^2}\\
        pV & = \frac{1}{3}m \overline{c^2}\\
        3mR^*T & = m \overline{c^2}\\
        3R^*T & =  \overline{c^2}\\
        \frac{3RT}{M_{\ce{CO2}}} & =  \overline{c^2}\\
        \frac{3\cdot(\SI{8.3145}{\joule\per\mole\per\kelvin})\cdot(\SI{300}{\kelvin})}{\SI{0.044}{\kilogram\per\mole}} & =  \overline{c^2}\\
        \SI{412}{\meter\per\second} & =  \sqrt{\overline{c^2}}.
      \end{align*}
    \item $p_i/p_i\std$.
    \item $w_m = c_p(T_2-T_1)$
    \item
      \begin{itemize}
        \item Chaleur à volume constant: OK.

          $\dif V = 0$ donc $W = 0$ et
          \[ \delta Q = \dif U = C_V \dif T \]
          donc $\delta Q > 0$ implique que $\dif T > 0$.
        \item Augmenter le volume à pression constante puis compression isotherme: OK.

          $pV = nRT$ donc si $V$ augmente a pression constante, $T$ augmente.
          Ensuite on remet le volume à sa valeur d'origine par compression isotherme.
        \item
          Détente isotherme puis compression adiabatique: OK.

          Si compression, $V_2 < V_1$. Si adiabatique,
          \begin{align*}
            p_2V_2 & = \frac{p_2V_2^\gamma}{V_2^{\gamma-1}}\\
                   & = \frac{p_1V_1^\gamma}{V_2^{\gamma-1}}\\
                   & > \frac{p_1V_1^\gamma}{V_1^{\gamma-1}}\\
                   & = p_1V_1.
          \end{align*}
          donc $nRT$ augmente.
        \item
          Compression isotherme puis détente adiabatique: NON.

          Si détente, $V_2 > V_1$. Si adiabatique,
          \begin{align*}
            p_2V_2 & = \frac{p_2V_2^\gamma}{V_2^{\gamma-1}}\\
                   & = \frac{p_1V_1^\gamma}{V_2^{\gamma-1}}\\
                   & < \frac{p_1V_1^\gamma}{V_1^{\gamma-1}}\\
                   & = p_1V_1.
          \end{align*}
          donc $nRT$ diminue.
        \item
          Apporter travail à volume constant: NON.

          Le travail c'est $-p \dif V$ donc si $\dif V = 0$, il ne peut pas y avoir de travail.
      \end{itemize}
    \item
      \SI{16}{\kilogram\per\second}
    \item $p_{\ce{SO2}} < p_{\ce{SO2},\text{éq}}$, $\log(p_{\ce{SO2}}) < \log(p_{\ce{SO2},\text{éq}})$
  \end{enumerate}
\end{solution}

\end{document}
