\documentclass[fr]{../../../../../../eplexam}

\usepackage{../../../../../../eplchem}
\usepackage{../../../../../../eplunits}

\newcommand{\potstd}[1]{E\std_{\text{éq},\text{#1}}}
\newcommand{\potstdce}{\potstd{cell}}
\newcommand{\potstdca}{\potstd{cathode}}
\newcommand{\potstdan}{\potstd{anode}}
\usepackage{physics}
\newcommand{\rom}[1]{%
  \textup{\uppercase\expandafter{\romannumeral#1}}%
}

\hypertitle{Chimie et chimie physique}{3}{FSAB}{1302}{2015}{Janvier}
{Benoît Legat\and Gilles Peiffer}
{Hervé Jeanmart et Joris Proost}

\section{}
\begin{enumerate}
  \item Écrivez, pour la corrosion en milieu aqueux du \ce{Cu} en son ion \ce{Cu++}
    \begin{itemize}
      \item toutes les demi-réactions cathodiques possibles;
      \item toutes les réactions rédox globales possibles;
    \end{itemize}
    (il ne faut pas se faire de soucis sur l'évolution éventuelle des produits de corrosion).
  \item Calculez en se basant sur l'échelle rédox ci-dessous,
    les forces électromotrices de ces réactions de corrosion en
    milieu acide sous conditions standard.
  \item Comment pourrait-on envisager de protéger des structures en \ce{Cu} contre la corrosion ?
  \item Sachant que la vitesse de corrosion peut très souvent
    être considérée comme une constante, trouvez une expression
    générale pour sa constante de vitesse en fonction du courant de corrosion.
\end{enumerate}
[tableau de Nernst fourni].

\begin{solution}
  \begin{enumerate}
    \item
      Le \ce{Cu} donne des électrons donc il fait une oxydation, il est à l'anode.
      La demi-réaction est
      \[ \ce{Cu -> Cu++ + 2e-}. \]

      Il faut maintenant voir si on est dans le cas pile ou électrolyse.
      La corrosion est considérée comme une pile donc
      \begin{equation}
        \label{eq:potstdce}
        \potstdce = \potstdca - \potstdan \geq 0.
      \end{equation}
      On a
      \[ \potstdan = E\std_{\ce{Cu++}/\ce{Cu}} = \SI{0.34}{\volt}. \]
      Donc, pour avoir $\potstdce$, il faut
      \[ \potstdca \geq \SI{0.34}{\volt}. \]

      Les demi-réactions cathodiques/de réduction possibles sont:
      \begin{align*}
        \ce{O2(g) + 4H+(aq) + 4e- & -> 2H2O(l)}\\
        \ce{O2(g) + 2H2O(l) + 4e- & -> 4OH-(aq)}
     \end{align*}
      et les réactions globales correspondantes sont
      \begin{align*}
        \ce{2Cu(s) + O2(g) + 4H+(aq) & -> 2Cu++(aq) + 2H2O(l)}\\
        \ce{2Cu(g) + O2(g) + 2H2O(l) & -> 2Cu++(aq) + 4OH-(l)}.
      \end{align*}

      J'ai mis ici uniquement celles avec l'eau.
      On peut aussi s'amuser à le faire avec tous les autres qui sont au dessus du \ce{Cu} dans le tableau des potentiels standards.
      Par exemple avec $\ce{I2(s) + 2e- -> 2I-(aq)}$.
    \item En utilisant l'équation~\eqref{eq:potstdce}, on a respectivement
      \begin{align*}
        \potstdce & = 1.23 - 0.34 = \SI{0.89}{\volt}\\
        \potstdce & = 0.40 - 0.34 = \SI{0.06}{\volt}.
      \end{align*}
    \item
      La réaction qui va se faire préférablement est celle avec le plus grand $\potstdce$.
      En effet, plus $\potstdca$ est grand, plus le réducteur est un réducteur fort et plus $\potstdan$ est petit plus l'oxydant est un réducteur faible et donc un oxydant fort.

      Si on ajoute du zinc par exemple, $E\std_{\ce{Zn++}/\ce{Zn}} < E\std_{\ce{Cu++}/\ce{Cu}}$ donc il réagira à la place du cuivre.
      De plus, du carbonate de zinc se crée avec le \ce{CO2} qui crée une barrière supplémentaire,
      le \ce{Cu} se corrode donc encore moins.
    \item
      Il faut ici utiliser la loi de Faraday.
      Soit $z$ le nombre d'électrons échangés. Ici, on a $z = 4$.
      En utilisant la st\oe chiométrie de l'équation, on voit que
      pour 4 électrons, on a 2 atomes de \ce{Cu++}.
      Si le courant de corrosion est $I$, après un temps $t$, on a $It$ coulombs d'électrons.
      Pour passer en moles on divise par la constante de Faraday car cette constante vaut la charge en coulombs d'un électron ($F = N_{\textnormal{A}} \ce{e}).$
      On a donc $It/F$ moles d'électrons en un temps $t$ ce qui fait
      \[ \frac{2}{z}\frac{It}{F} = \frac{It}{2F} \]
      moles de \ce{Cu++}.
      La vitesse de corrosion en moles par seconde est donc
      \[ \frac{I}{2F}. \]
      Si on veut l'avoir en grammes par seconde il suffit de multiplier par la masse molaire du \ce{Cu}:
      \[ \frac{IM_{\ce{Cu}}}{2F}. \]
  \end{enumerate}
\end{solution}

\section{}
\begin{enumerate}
  \item Exposez (citez les relations mathématiques) et expliquez la définition du second principe.
  \item Expliquez comment établir le lien entre l'évolution d'entropie associée à une transformation
    adiabatique irréversible et la définition fondamentale de l'entropie $\dif S = \frac{\delta Q}{T}$.
  \item Appliquez la démarche exposée au point 2 à un exemple pertinent de votre choix.
  \item Expliquez la seconde expérience de Joule.
\end{enumerate}

%\nosolution
%\begin{solution}
% \begin{enumerate}
%   \item
%     Le second principe dit que la différence d'entropie d'un système et son environnement ne peut qu'augmenter. Ecrit mathématiquement:
%     \[ dS = dS_{interieur} + ds_{echange} > 0 \]
%   \item
%     Une transformation adiabatique n'échange pas de chaleur avec son environnement, il n'y a donc pas de $dS_{echange}$ avec l'environnement.
%     Puisqu'elle est irreversible, il y a création interne d'entropie.
%   \item
%     Je pense que l'experience ou on sépare 2 volumes, l'un contenant du gaz, l'autre du vide, par une plaque, puis qu'on enleve la plaque est adia-irreversible. Y'auras pas d'echange de chaleur, mais création d'entropie du a l'expansion du gaz.
% \end{enumerate}
%\end{solution}

\begin{solution}
     \begin{enumerate}
          \item On peut scinder l'évolution d'entropie lors d'une transformation
          en deux : d'une part, on a les échanges de chaleur réversibles et
          d'autre part les irréversibilités. On écrit donc mathématiquement

          \[
          \dif S = \dif_{\textnormal{i}} S + \dif_{\textnormal{e}} S.
          \]

          Le second principe dit simplement que le premier terme est toujours
          positif :

          \[
          \int \dif_{\textnormal{i}} S \ge 0.
          \]

          L'égalité est obtenue dans le cas réversible. L'interprétation
          physique de ceci est que la nature nous impose en toutes
          circonstances que la génération d'entropie interne à un système soit
          positive.

          \item Comme la transformation est adiabatique, elle n'échange pas de
          chaleur avec son environnement, on a donc $\dif_{\textnormal{e}} S =
          0$.
          Comme elle est irréversible, on a par le second principe que
          l'entropie interne augmente, ou bien $\dif_{\textnormal{i}} S > 0$.

          \item Un exemple où on peut appliquer cette démarche est dans le cas
          d'un cycle de Carnot avec irréversibilités. Dans le cas du cycle
          réversible, on trouve la relation suivante

          \[
          \frac{Q_{\rom{1}}}{T_1} + \frac{Q_{\rom{2}}}{T_2} = 0.
          \]

          Dans le cas du cycle irréversible, on trouve

          \[
          \frac{Q_{\rom{1}}^{\textnormal{irr}}}{T_1} +
          \frac{Q_{\rom{2}}^{\textnormal{irr}}}{T_2} < 0.
          \]

          Cette équation indique que la perte à la source froide est plus grande
          dans le cycle irréversible. C'est dû à la génération d'entropie
          interne qui doit être évacuée.

          \item La seconde expérience de Joule visait à mesurer le terme
          $\pi_T$. Il est défini comme

          \[
          \pi_T = \left( \fpart{U}{V} \right)_T.
          \]

          Il a rempli un réservoir d’un gaz sous pression à température
          ambiante. Il a connecté ce réservoir à un second vidé préalablement
          de tout gaz. Ces deux réservoirs ont été plongés dans un bain d’eau
          dont la température a été mesurée. Ce bac rempli d’eau était lui-même
          isolé thermiquement de l’atmosphère ambiante. Joule a enregistré
          l’évolution de la température de l’eau après mise en communication
          des deux réservoirs. Il a constaté, avec la précision de l’époque,
          que la température de l’eau n’évoluait pas. Ce constat indique
          qu’aucun transfert de chaleur n’a eu lieu entre l’eau et les
          réservoirs contenant le gaz. Le système composé du gaz peut donc être
          considéré comme adiabatique. De plus, lors de la mise en
          communication, aucun travail n’a été accompli par le gaz puisqu’il se
          détendait (de manière irréversible) contre le vide. L’énergie interne
          du système gazeux est donc conservée lors de cette expérience.
          Puisque le volume a changé mais que l’énergie est restée constante,
          on en déduit que le coefficient $\pi_T$ doit être nul. S’il est nul,
          l’énergie interne d’un gaz ne dépend que de la température et l’on
          peut écrire:

          \[
          U = U(T) \dif U = C_V \dif T,
          \]

          pour toute transformation à volume constant ou non. Cela renforce
          l’importance de la capacité calorifique dans le calcul de l’évolution
          de l’énergie interne.


          Il faut cependant garder à l’esprit que le résultat obtenu par Joule
          avec les outils expérimentaux de l’époque n’est pas exact. Il y a en
          réalité un écart de température lié à l’expansion du gaz. Le résultat
          obtenu par Joule n’est valable que pour un gaz parfait. Pour les gaz
          réels, le terme est non nul.
     \end{enumerate}
\end{solution}

\section{}
Limité à 1/2 face par réponse.


\paragraph{Note}
Cette question semble adaptée de l'exercice 14.91 de la  5\ieme{} ou 6\ieme{} édition du Atkins et Jones.

Des observations expérimentales en labo ont montré que la vitesse de la réaction
\[ \ce{2NO(g) + 2H2(g) -> N2(g) + 2H2O(g)} \]
peut être raisonablement approximée comme $k[\ce{NO}]^2[\ce{H2}]$.
Le mécanisme réactionnel suivant est alors proposé:
\begin{itemize}
  \item étape 1: $\ce{NO + NO -> N2O2}$
  \item étape 2: $\ce{N2O2 + H2 -> N2O + H2O}$
  \item étape 3: $\ce{N2O + H2 -> N2 + H2O}$
\end{itemize}

\begin{enumerate}
  \item Quelle est vraisemblablement l'étape déterminante de vitesse ?
  \item Sur base de votre réponse ci-dessus, trouvez une expression pour la constante de vitesse
    en fonction des constantes thermodynamiques et/ou cinétiques d'une ou plusieurs de ces 3 étapes.
  \item Dessinez un schéma énergétique plausible pour la réaction globale dont on sait qu'elle est exothermique.
    Indiquez là-dessus également les énergies d'activation pour chaque étape avec les symboles $E_{\textnormal{a,}1}$, $E_{\textnormal{a,}2}$ et $E_{\textnormal{a,}3}$.
\end{enumerate}
\begin{solution}
  \begin{enumerate}
    \item
      L'étape déterminante est la plus lente.
      Les autres sont supposées rapides par rapport à l'étape déterminante.
      Rappelons-nous que les réactions élémentaires qui suivent l'étape déterminante de la vitesse ne contribuent pas à la loi de vitesse déduite du mécanisme.
      \begin{itemize}
        \item Si c'est l'étape 1, on a $k_1[\ce{NO}]^2$ comme vitesse. Ce n'est pas la loi de vitesse observée.
        \item Si c'est l'étape 2, on a $k_2[\ce{N2O2}][\ce{H2}]$ comme vitesse.
          Hors, comme la première est très rapide par rapport à la deuxième,
          on peut supposer qu'elle satisfait la condition de pré-équilibre
          \[\phantom{,} K_1 = \frac{k_1}{k_1'} = \frac{[\ce{N2O2}]}{[\ce{NO}]^2}, \]
          où $K_1$ est la constante d'équilibre, $k_1$ la constante de vitesse de $\ce{NO + NO -> N2O2}$
          et $k_1'$ la constante de vitesse de $\ce{N2O2 -> NO + NO}$.
          On trouve donc que la vitesse est $k_2K_1[\ce{NO}]^2[\ce{H2}$.
        \item Si c'est l'étape 3, il faut être un peu plus prudent et pas utiliser les pré-équilibres mais l'hypothèse de quasi-stationnarité des radicaux.
          Après des calculs dégueux on ne trouve pas la bonne expression de la vitesse.
      \end{itemize}
    \item En fonction uniquement des constantes cinétiques, on a $k = k_2\frac{k_1}{k_1'}$
      et en utilisant la constante thermodynamique $K_1 = \frac{k_1}{k_1'}$, on a $k = k_2 K_1$.
    \item
      Les pics qui représentent la formation des intermédiaires \ce{N2O2} et \ce{N2O} ne sont pas à la même énergie,
      mais nous avons aucune information pour déterminer lequel sera plus bas. Comme la question demande un schéma plausible, il n'y a pas d'unique bonne réponse, à condition de justifier les choix que l'on fait.

      % TODO : add graph example
  \end{enumerate}
\end{solution}

\section{}
On considère un système fermé parcourant un cycle moteur réversible composé de 5 transformations (cycle d'Atkinson):
\begin{itemize}
  \item $1 \to 2$: compression adiabatique
  \item $2 \to 3$: échauffement isochore
  \item $3 \to 4$: détente adiabatique
  \item $4 \to 5$: refroidissement isochore
  \item $5 \to 1$: diminution de volume isobare
\end{itemize}
Le gaz contenu dans le système est de l'air (gaz parfait) dont les propriétés sont constantes
et évaluées à température ambiante.
\begin{enumerate}
  \item De ce système, on connait le rapport de compression,
    $V_1/V_2 = 10$ et le rapport de détente $V_4/V_3 = 13$.
    La chaleur apportée lors de l'échauffement est de \SI{2000}{\kilo\joule\per\kilogram}.
    À partir de ces valeurs, on vous demande de compléter le
    tableau ci-dessous en justifiant succinctement vos résultats.
    \[
      \begin{array}{|c|cccc|}
        \hline
        & p[\si{\bar}] & T[\si{\kelvin}] & V[\si{\meter\cubed}] & S-S_1[\si{\joule\per\kelvin}]\\
        \hline
        1 & 1 & 300 & 0.001 & 0\\
        \hline
        2 &  &  &  & \\
        \hline
        3 &  &  &  & \\
        \hline
        4 &  &  &  & \\
        \hline
        5 &  &  &  & \\
        \hline
      \end{array}
    \]
  \item Calculez le travail net de ce cycle et son rendement.
  \item Tout autre paramètre étant constant, quel rapport de détente, $V_4/V_3$ permettrait de maximiser le rendement du cycle? Quel serait le rendement dans ce cas ? Justifiez votre démarche.
\end{enumerate}

% \nosolution

\begin{solution}
 \begin{enumerate}
    \item Quelques valeurs dans le tableau s'obtiennent immédiatement :
        \begin{itemize}
            \item $S_2 - S_1 = 0$ car la première transformation est adiabatique;
            \item $p_5 = p_1 = \SI{1}{\bar}$ car la dernière transformation est isobare;
            \item $V_2 = 0.1 \cdot V_1 = \SI{0.0001}{\cubic\metre}$, car le rapport de compression vaut $\num{10}$.
        \end{itemize}

        Ensuite, on peut attaquer la première transformation.
        On utilise la relation $p_1V^{\gamma}_1 = p_2V^{\gamma}_2$, valable pour les transformations adiabatiques, où $\gamma = c_p / c_V = \num{1.4}$ avec la dernière égalité valable pour un gaz parfait. En remplaçant les valeurs qu'on connaît déjà, et en réarrangeant un peu, on obtient que $p_2 \approx \SI{25.1}{\bar}$. Par la loi des gaz parfaits, on obtient alors que $T_2 \approx \SI{754}{\kelvin}$.

        Comme la troisième transformation est isochore, on peut déjà dire que $V_3 = V_2 = \SI{0.0001}{\cubic\metre}$. Ensuite, le rapport de détente nous dit que $V_4 = 13 V_3 = \SI{0.0013}{\cubic\metre}$. En utilisant le fait que la transformation $4 \to 5$ est isochore, on trouve alors que $V_5 = V_4 = \SI{0.0013}{\cubic\metre}$.

        Le tableau ressemble maintenant à ceci:
        \[
      \begin{array}{|c|cccc|}
        \hline
        & p[\si{\bar}] & T[\si{\kelvin}] & V[\si{\meter\cubed}] & S-S_1[\si{\joule\per\kelvin}]\\
        \hline
        1 & 1 & 300 & 0.001 & 0\\
        \hline
        2 & 25.1 & 754 & 0.0001 & 0 \\
        \hline
        3 &  &  & 0.0001 & \\
        \hline
        4 &  &  & 0.0013 & \\
        \hline
        5 & 1 &  & 0.0013 & \\
        \hline
      \end{array}
    \]

    Attaquons maintenant la ligne suivante. Il y a encore une donnée que nous n'avons pas utilisée: $Q_{2 \to 3} = \SI{2000}{\kilo\joule\per\kilo\gram}$. Comme la transformation est isochore, la chaleur échangée est donnée par la formule $Q_{2 \to 3} = c_V (T_3 - T_2) = \frac{5}{2} m R^*_{\textnormal{air}} (T_3 - T_2)$, où $R^*_{\textnormal{air}}$ est la constante spécifique du gaz, égale à $\SI{287.058}{\joule \per \kilo \gram \per \kelvin}$ pour l'air. La masse $m$ est ici (en utilisant les données pour éviter de propager des erreurs) $m = \frac{p_1 V_1}{R^*_{\textnormal{air}}T_1} \approx \SI{1.161}{\gram}$. En remplaçant ces valeurs dans la formule de $Q_{2 \to 3}$ et en isolant $T_3$, on trouve $T_3 \approx \SI{3540}{\kelvin}$.

    Par la loi des gaz parfaits, $p_3$ vaut alors environ $\SI{118}{\bar}$. Pour terminer de remplir la troisième ligne, on calcule alors le changement d'entropie $\Delta S_{1 \to 3} = \Delta S_{1 \to 2} + \Delta S_{2 \to 3} = 0 + c_V \ln\left(\frac{T_3}{T_2}\right) \approx \SI{1.29}{\joule\per\kelvin}$. On peut immédiatement voir que comme la transformation $3 \to 4$ est adiabatique, $S_4 - S_1 = S_3 - S_1 \approx \SI{1.29}{\joule\per\kelvin}$.

    Le tableau commence alors à se remplir :

     \[
      \begin{array}{|c|cccc|}
        \hline
        & p[\si{\bar}] & T[\si{\kelvin}] & V[\si{\meter\cubed}] & S-S_1[\si{\joule\per\kelvin}]\\
        \hline
        1 & 1 & 300 & 0.001 & 0\\
        \hline
        2 & 25.1 & 754 & 0.0001 & 0 \\
        \hline
        3 & 118 & 3540 & 0.0001 & 1.29 \\
        \hline
        4 &  &  & 0.0013 & 1.29\\
        \hline
        5 & 1 &  & 0.0013 & \\
        \hline
      \end{array}
    \]

    En utilisant le fait que $pV^\gamma$ est constant pour les transformations adiabatiques, on a la relation $p_3 V^{\gamma}_3 = p_4 V^{\gamma}_4$, d'où on tire facilement que $p_4 \approx \SI{3.25}{\bar}$. En utilisant la loi des gaz parfaits, on obtient également $T_4 \approx \SI{1269}{\kelvin}$.

    On peut encore une fois utiliser la loi des gaz parfaits pour la dernière ligne, et ainsi obtenir que $T_5 \approx \SI{390}{\kelvin}$. Il ne manque alors plus qu'une valeur, $S_5 - S_1$, que l'on retrouve facilement avec la relation $\Delta S = c_V \ln\left(\frac{T_\textnormal{f}}{T_\textnormal{i}}\right)$, valable dans le cas d'une transformation isochore. On obtient que $S_5 - S_4 \approx \SI{-0.98}{\joule \per \kelvin}$ et donc que $S_5 - S_1 \approx \SI{0.31}{\joule \per \kelvin}$. Cette dernière valeur peut également se retrouver d'une autre façon; en effet, si on calcule $c_p \ln\left(\frac{T_1}{T_5}\right)$, on trouve également $S_5 - S_1 = \SI{0.31}{\joule\per\kelvin}$, ce qui termine cette partie de l'exercice.

    En guise de vérification, on peut voir que les transformations sont bien telles qu'annoncées, c'est-à-dire que le refroidissement correspond bien à une baisse de température, alors que la compression correspond bien à une augmentation de pression, etc\ldots

    Voici le tableau des résultats finaux:

    \[
      \begin{array}{|c|cccc|}
        \hline
        & p[\si{\bar}] & T[\si{\kelvin}] & V[\si{\meter\cubed}] & S-S_1[\si{\joule\per\kelvin}]\\
        \hline
        1 & 1 & 300 & 0.001 & 0\\
        \hline
        2 & 25.1 & 754 & 0.0001 & 0 \\
        \hline
        3 & 118 & 3540 & 0.0001 & 1.29 \\
        \hline
        4 & 3.25 & 1269 & 0.0013 & 1.29 \\
        \hline
        5 & 1 & 390 & 0.0013 & 0.31 \\
        \hline
      \end{array}
    \]

    \item Le calcul du travail est relativement simple, car les transformations isochores n'interviennent pas:

    \begin{equation*} \label{w}
        \begin{split}
        W & = W_{1 \to 2} + W_{2 \to 3} + W_{3 \to 4} + W_{4 \to 5} + W_{5 \to 1} \\
        & = c_V (T_2 - T_1) + 0 + c_V (T_4 - T_3) + 0 - p_5 (V_1 - V_5) \\
        & \approx \SI{-1485}{\joule}.
        \end{split}
    \end{equation*}

    Le travail est bien négatif et le cycle est donc moteur (heureusement, car il s'agit d'une adaptation d'un cycle utilisé dans le moteur de certaines voitures modernes :-)!

    Pour obtenir le rendement, il faut évaluer l'expression \begin{equation} \label{eta}
        \phantom{,} \eta = \frac{-W}{Q_{\textnormal{H}}},
    \end{equation}

    où $Q_{\textnormal{H}}$ est la chaleur échangée à la source chaude. On connaît cette chaleur; elle fait partie des données de l'exercice, à condition de regarder les unités! On a donc que $Q_{\textnormal{H}} = \num{2000000} \cdot m \approx \SI{2322}{\joule}$.

    Il ne reste alors plus qu'à remplacer dans l'expression du rendement (eqn. \ref{eta}), ce qui donne

    \begin{equation*}
        \phantom{.} \eta \approx \frac{\num{1485}}{\num{2322}} \approx \num{0.64}.
    \end{equation*}

    \item

    La troisième question nécessite une petite astuce. Commençons par poser $\zeta = \frac{V_4}{V_3}$. On va considérer le rendement comme une fonction de $\zeta$ et on va essayer de maximiser celui-ci en annulant sa dérivée première par rapport à $\zeta$. Pour obtenir cette fonction, il faut exprimer les différents paramètres du système en fonction du rapport de détente $\zeta$. On obtient que:

    \begin{itemize}
        \item $\frac{T_{\textnormal{f}}}{T_{\textnormal{i}}} = \left( \frac{V_{\textnormal{i}}}{V_{\textnormal{f}}} \right)^{\gamma-1} \implies T_4 = \zeta^{1-\gamma}T_3 \implies T_4-T_3 = T_3 (\zeta^{1-\gamma} - 1)$;
        \item $V_1-V_5 = V_1 - V_4 = V_1 - \zeta V_3 = \num{e-4} (\num{10}-\zeta)$.
    \end{itemize}

    La fonction $\eta(\zeta)$ est donnée (approximativement) par

    \begin{equation*}
        \phantom{.} \eta(\zeta) \approx \frac{\num{-1}}{\num{2322}} \left( c_V (T_2 - T_1) + c_V T_3 (\zeta^{1 - \gamma} - 1) - p_5 \num{e-4} (10-\zeta) \right).
    \end{equation*}

    La dérivée première par rapport à $\zeta$ de $\eta(\zeta)$ est alors égale à

    \begin{equation*}
        \phantom{.} \dv{\eta(\zeta)}{\zeta} \approx \frac{\num{-1}}{\num{2322}} \left( c_V T_3 (1-\gamma) \zeta^{-\gamma} + p_5 \num{e-4} \right).
    \end{equation*}

    On veut annuler cette dérivée, donc le facteur constant n'intervient pas. Il suffit donc de résoudre l'équation en $\zeta$ suivante:
    \begin{equation*}
        \phantom{.} c_V T_3 (1 - \gamma) \zeta^{-\gamma} + p_5 \num{e-4} = 0.
    \end{equation*}

    En la résolvant, on trouve que $\zeta = \frac{V_4}{V_3} \approx 30$. Une dernière étape consiste à calculer la dérivée seconde et de vérifier que $\zeta \approx 30$ est bien un maximum (et non pas un minimum!).

    Voici la dérivée seconde:

    \begin{equation*}
        \phantom{.} \dv[2]{\eta(\zeta)}{\zeta} \approx \frac{\num{-1}}{\num{2322}}c_V T_3 (\gamma - 1) \gamma \zeta^{-\gamma-1} < 0 \quad \forall \zeta \in \R.
    \end{equation*}

    Comme indiqué, la dérivée seconde est donc négative (attention à ne pas oublier la constante cette fois-ci!), ce qui implique que la réponse que nous trouvons est bien un maximum.

    Finalement, pour calculer le rendement pour ce nouveau rapport de détente, il faut déterminer le travail pour ces nouveaux paramètres. On reprend le deuxième résultat de l'équation \ref{w}, et on trouve que $W \approx \SI{1626}{\joule}$ (pour arriver à ce résultat, il suffit de refaire la partie du tableau qui se basait sur un rapport de détente de $\num{13}$ avec un rapport de \num{30}). $Q$ est inchangé. On trouve alors le rendement suivant:

    \begin{equation*}
        \phantom{.} \eta \approx \frac{\num{1626}}{\num{2322}} \approx \num{0.7}.
    \end{equation*}

    En guise de vérification, on remarque évidemment que le rendement est meilleur que dans le cas du rapport de détente suboptimal (ouf!).

\end{enumerate}
\end{solution}

\section{QCM}
Notes: Il y avait 20 questions.

\begin{enumerate}
  \item Le(s) quel(s) des symboles ci-dessous est (sont) synonyme(s) du potentiel chimique standard d'un composant
    $i$ dans un mélange réactionnel ?

    $G_i\std$, $G_{i,m}\std$, $(\partial H/\partial n_i)_{S,p,n_{j \neq i}}$, ${G_m^i}\std$, $\dif G / \dif n_i$
  \item W.H. Nernst est né avant J.W. Gibbs.

    Vrai, Faux.

  \item Une machine frigorifique transmet selon un cycle de Carnot inverse entre une température extérieure de \SI{25}{\celsius}
    et intérieure de $-\SI{10}{\celsius}$.
    Quel est le coefficient de performance de cette machine ?

  \item
    À \SI{300}{\kelvin} et à pression atmosphérique,
    quelle est la vitesse quadratique moyenne
    des molécules de dioxyde de carbone ?

    \SI{11}{\meter\per\second}, \SI{337}{\meter\per\second}, \SI{412}{\meter\per\second}, \SI{300}{\meter\per\second}, \SI{505}{\meter\per\second}.
  \item
    Quelle est la définition générale de l'activité d'un composant gazeux $i$ ?

    $p_i\std/1$, $p/p_i\std$, $p_i/1$, $p_i/p_i\std$, $p_i/p\std$, Aucun.
  \item
    Dans un système ouvert, une transformation d'un gaz parfait s'effectue de manière réversible et
    sans échange de chaleur entre un état 1 et un état 2.
    On néglige les termes liés à l'énergie cinétique et l'énergie potentielle.
    Quelle(s) expression(s) est (sont) correcte(s) pour exprimer le travail ? Échange ?

    $w_m = \frac{\gamma - 1}{\gamma} p_1v_1 \left(\frac{p_2}{p_1} \frac{\gamma - 1}{\gamma}\right)$,
    $w_m = \frac{\gamma}{\gamma - 1} p_1v_1 \left(\frac{p_2}{p_1} \frac{\gamma - 1}{\gamma}\right)$,
    $w_m = c_p T_2 \left(\left(\frac{p_2}{p_1}\right)^{\frac{\gamma - n}{\gamma}} - 1\right)$,
    $w_m = \frac{p_1v_1}{\gamma - 1} \left(\left(\frac{V_1}{V_2}\right)^{\gamma - 1} - 1\right)$,
    $w_m = c_p (T_2 - T_1)$,
    aucune.
  \item
    Dans un système fermé, on souhaite augmenter la température avec volume constant.
    Quelle transformation est ok ?

    Chaleur à volume constant, Augmenter le volume à pression constante puis compression isotherme,
    Détente isotherme puis compression adiabatique,
    Compression isotherme puis détente adiabatique,
    Apporter travail à volume constant,
    Aucune correcte.
  \item
    Mesure d'un débit dans un conduit de diamètre \SI{100}{\milli\meter},
    on réalise localement une contraction régulière avec un rapport de diamètre 4.
    La chute de pression locale est \SI{2000}{\pascal}.
    Le débit approximatif est:

    \SI{1000}{\liter\per\second}, \SI{1}{\kilogram\per\second}, \SI{200}{\liter\per\second},
    \SI{2}{\kilogram\per\second}, \SI{16}{\kilogram\per\second}.
  \item
    Transformer sulfures en oxyde (grillage):
    \[ \ce{2ZnS(s) + 3O2(g)} = \ce{2ZnO(s) + 2SO2(g)}. \]
    Quelle est à une température et pp d'\ce{O2} ($p_{\ce{O2}}$) donnée,
    la condition à respecter pour diriger le procédé dans le bon sens ?

    $p_{\ce{SO2}} < p_{\ce{SO2},\text{éq}}$,
    $p_{\ce{SO2}} > p_{\ce{SO2},\text{éq}}$,
    $\log(p_{\ce{SO2}}) < \log(p_{\ce{SO2},\text{éq}})$,
    $\log(p_{\ce{SO2}}) > \log(p_{\ce{SO2},\text{éq}})$,
    $p_{\ce{SO2}} > p_{\ce{O2}}$,
    $p_{\ce{SO2}} < p_{\ce{O2}}$.
\end{enumerate}

\begin{solution}
  \begin{enumerate}
    \item $G_{i,m}\std$ et $(\partial H/\partial n_i)_{S,P,n_{j \neq i}}$.
      Pas ${G_m^i}\std$ car c'est pour une substance $i$ pure et un est dans un mélange réactionnel.
    \item Faux.
    \item
      \[ \phantom{.} COP = \frac{1}{\frac{T_I}{T_{II}} - 1} = 7.5. \]
    \item
      On a
      \begin{align*}
        p & = \frac{1}{3}\rho \overline{c^2}\\
        pV & = \frac{1}{3}m \overline{c^2}\\
        3mR^*T & = m \overline{c^2}\\
        3R^*T & =  \overline{c^2}\\
        \frac{3RT}{M_{\ce{CO2}}} & =  \overline{c^2}\\
        \frac{3\cdot(\SI{8.3145}{\joule\per\mole\per\kelvin})\cdot(\SI{300}{\kelvin})}{\SI{0.044}{\kilogram\per\mole}} & =  \overline{c^2}\\
        \SI{412}{\meter\per\second} & =  \sqrt{\overline{c^2}}.
      \end{align*}
    \item $p_i/p_i\std$.
    \item $w_m = c_p(T_2-T_1)$
    \item
      \begin{itemize}
        \item Chaleur à volume constant: OK.

          $\dif V = 0$ donc $W = 0$ et
          \[ \delta Q = \dif U = C_V \dif T \]
          donc $\delta Q > 0$ implique que $\dif T > 0$.
        \item Augmenter le volume à pression constante puis compression isotherme: OK.

          $pV = nRT$ donc si $V$ augmente a pression constante, $T$ augmente.
          Ensuite on remet le volume à sa valeur d'origine par compression isotherme.
        \item
          Détente isotherme puis compression adiabatique: OK.

          Si compression, $V_2 < V_1$. Si adiabatique,
          \begin{align*}
            p_2V_2 & = \frac{p_2V_2^\gamma}{V_2^{\gamma-1}}\\
                   & = \frac{p_1V_1^\gamma}{V_2^{\gamma-1}}\\
                   & > \frac{p_1V_1^\gamma}{V_1^{\gamma-1}}\\
                   & = p_1V_1.
          \end{align*}
          donc $nRT$ augmente.
        \item
          Compression isotherme puis détente adiabatique: NON.

          Si détente, $V_2 > V_1$. Si adiabatique,
          \begin{align*}
            p_2V_2 & = \frac{p_2V_2^\gamma}{V_2^{\gamma-1}}\\
                   & = \frac{p_1V_1^\gamma}{V_2^{\gamma-1}}\\
                   & < \frac{p_1V_1^\gamma}{V_1^{\gamma-1}}\\
                   & = p_1V_1.
          \end{align*}
          donc $nRT$ diminue.
        \item
          Apporter travail à volume constant: NON.

          Le travail c'est $-p \dif V$ donc si $\dif V = 0$, il ne peut pas y avoir de travail.
      \end{itemize}
    \item
      \SI{1}{\kilogram\per\second}
    \item $p_{\ce{SO2}} < p_{\ce{SO2},\text{éq}}$, $\log(p_{\ce{SO2}}) < \log(p_{\ce{SO2},\text{éq}})$
  \end{enumerate}
\end{solution}

\end{document}
