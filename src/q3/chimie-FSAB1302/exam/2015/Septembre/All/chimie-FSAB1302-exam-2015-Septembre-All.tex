\documentclass[fr]{../../../../../../eplexam}

\usepackage{../../../../../../eplchem}
\usepackage{../../../../../../eplunits}
\newcommand{\cp}{c_{\textnormal{p}}}
\newcommand{\cv}{c_{\textnormal{v}}}

\hypertitle{Chimie et chimie physique}{3}{FSAB}{1302}{2015}{Août}{All}
{Martin Braquet \and Gilles Peiffer}
{Hervé Jeanmart et Joris Proost}

\section{}

\begin{enumerate}[label=(\Alph*)]
     \item Trier les hydrocarbures ci-dessous selon la valeur absolue de leur
     chaleur standard de combustion:
     \[
     \ce{C2H4} \qquad \ce{C3H8} \qquad \ce{C2H6} \qquad \ce{CH4} \qquad
     \ce{C3H6}.
     \]
     \item Même question pour les composés suivants:
     \[
     \ce{C2H6} \qquad \ce{CH3COCH3} \qquad \ce{CH3CH2OH} \qquad \ce{CH4} \qquad
     \ce{CH3COOH} \qquad \ce{CH3OH}.
     \]
\end{enumerate}

\begin{solution}
     \begin{enumerate}[label=(\Alph*)]
     \item On a

     \[
     \ce{CH4} < \ce{C2H4} < \ce{C2H6} < \ce{C3H6} < \ce{C3H8},
     \]

     car les chaleurs standard de combustion sont les suivantes :

     \[
     \abs{\Delta_{\textnormal{c}} H\std} = 890 < 1411 < 1560 < 2058 < 2220
     \quad \si{\kilo \joule \per \mol}.
     \]

     \item On a

     \[
     \ce{CH3OH} < \ce{CH3COOH} < \ce{CH4} < \ce{CH3CH2OH} < \ce{C2H6} <
     \ce{CH3COCH3},
     \]

     car les chaleurs standard de combustion sont les suivantes :

     \[
     \abs{\Delta_{\textnormal{c}} H\std} = 726 < 875 < 890 < 1368 < 1560 <
     1790 \quad \si{\kilo \joule \per \mol}.
     \]
     \end{enumerate}
     \vspace*{\baselineskip{}}

     Les valeurs sont obtenues en connaissant les enthalpies de formation de
     ces composés (ainsi que celle de $\ce{H2O}$ et de $\ce{CO2}$). En effet,
     voici le raisonnement pour la combustion de $\ce{CH4}$:

     \[
     \ce{CH4(g) + 2O2(g) -> 2H2O(l) + CO2(g)}.
     \]

     avec

     \begin{alignat*}{2}
     &\Delta_{\textnormal{f}} H\std(\ce{CO2,g}) &&= \SI{-393}{\kilo \joule \per
     \mol}\\
     &\Delta_{\textnormal{f}} H\std(\ce{H2O,l}) &&= \SI{-286}{\kilo \joule \per
     \mol}\\
     &\Delta_{\textnormal{f}} H\std(\ce{CH4,g}) &&= \SI{-75}{\kilo \joule \per
     \mol}.
\end{alignat*}

     Cela donne ainsi

     \begin{align*}
     \Delta_{\textnormal{c}} H\std(\ce{CH4, g}) &= 2\Delta_{\textnormal{f}}
     H\std(\ce{H2O,l}) + \Delta_{\textnormal{f}} H\std(\ce{CO2,g}) -
     \Delta_{\textnormal{f}} H\std(\ce{CH4,g})\\
     &= \SI{-890}{\kilo \joule \per
     \mol}.
     \end{align*}

     puisque l'enthalpie de formation de $\ce{O2(g)}$ (composé de base) est
     nulle.
\end{solution}

\section{}
\begin{enumerate}[label=(\Alph*)]
     \item Écrivez pour la corrosion en milieu aqueux de \ce{Ag} en son ion
     $\ce{Ag+}$
     \begin{enumerate}
          \item toutes les demi-réactions cathodiques possibles;
          \item toutes les réactions globales possibles;
     \end{enumerate}
     \item Calculez en vous basant sur l'échelle rédox donnée, les forces
     électromotrices de ces réactions de corrosion en milieu acide sous des
     conditions standard.
\end{enumerate}

\begin{solution}
La réaction anodique de l'argent est

\[
\ce{Ag^+ + e^- -> Ag} \qquad E\std_{\textnormal{éq}} = \SI{0.8}{\volt}.
\]

Les réactions cathodiques possibles sont toutes celles avec un
$E\std_{\textnormal{éq}}$ supérieur à celui de l'argent.
Par exemple pour l'eau,


\[
\ce{O2 + 4H^+ + 4e^- -> 2H2O} \qquad E\std_{\textnormal{éq}} = \SI{1.23}{\volt}.
\]

La réaction est alors
\[
\ce{4Ag + O2 + 4H^+ -> 2H2O + 4Ag^+}.
\]

On obtient finalement le potentiel électrochimique de la cellule sous
conditions standard:

\begin{align*}
E\std_{\textnormal{éq,cell}} &= E\std_{\textnormal{éq,cath}} -
E\std_{\textnormal{éq,an}}\\
&= \SI{1.23}{\volt} - \SI{0.8}{\volt}\\
&= \SI{0.43}{\volt}.
\end{align*}
\end{solution}

\section{}
Des expériences en laboratoire ont permis de démontrer le mécanisme réactionnel
suivant en milieu aqueux:
\begin{enumerate}
    \item $\ce{ClO^- + H2O <=> HClO + OH^-}$ (équilibre rapide);
    \item $\ce{HClO + I^- -> HIO + Cl^-}$ (très lente);
    \item $\ce{HIO + OH^- <=> IO^- + H2O}$ (équilibre rapide);
\end{enumerate}

\begin{itemize}
    \item Quelle est la réaction globale?
    \item Écrivez la loi de vitesse de la réaction globale fondée sur ce
    mécanisme réactionnel
    \item Comment change la vitesse de réaction si on augmente le
    $\mathrm{pH}$? Vérifiez votre réponse avec le principe de Le Chatelier.
\end{itemize}

\begin{solution}
Réaction globale:
\[
\ce{ClO^- + I^- -> Cl^- + IO^-}.
\]

La vitesse globale est donnée par la vitesse de réaction la plus lente:
\begin{align*}
r &= r_2\\
&= k_2 [\ce{HClO}][\ce{I^-}].
\end{align*}

On a
\[
r_1 = k_1 [\ce{ClO^-}][\ce{H2O}] \quad \textnormal{et} \quad r'_1 =
k'_1[\ce{HClO}][\ce{OH^-}].
\]

$[\ce{H2O}]$ n'intervient pas reéllement dans le mécanisme réactionnel puisque
sa concentration est nettement supérieure à la concentration des autres
réactifs, on parle de dégénérescence d'ordre. On peut donc poser

\begin{align*}
k^*_1 &= k_1 [\ce{H2O}]\\
&= 55.5 k_1,
\end{align*}

avec $k_1$ la constante cinétique de la première réaction.


On a alors

\begin{align*}
     r_1 &= k_1[\ce{ClO^-}][\ce{H2O}]\\
     &= k^*_1[\ce{ClO^-}].
\end{align*}

Puisque $r_1 = r'_1$,

\[
[\ce{HClO}] = \frac{k^*_1}{k'_1} \frac{[\ce{ClO^-}]}{[\ce{OH^-}]}.
\]

La vitesse globale est donc

\[
r = \frac{k_2 k^*_1}{k'_1} \frac{[\ce{ClO^-}][\ce{I^-}]}{[\ce{OH^-}]}.
\]

Une augmentation du $\mathrm{pH}$ induit une augmentation des ions $\ce{OH^-}$,
et donc une diminution de la vitesse de réaction.


Une augmentation des produits dans la première réaction déplace l'équilibre
vers la gauche (le principe de le Chatelier stipule qu'un système s'oppose
toujours à une perturbation) et diminue donc la vitesse de la 2\ieme{} réaction
puisqu'il y aura moins de $\ce{HClO^-}$ produit.
\end{solution}

\section{}

Le rendement des moteurs à combustion interne croit avec la pression maximale
du cycle. Pour augmenter celle-ci, on envisage de réaliser la compression dans
2 cylindres différents, un à basse pression et un à haute pression. La détente
serait également en deux étapes, d'abord dans le cylindre haute-pression et
ensuite dans le cylindre basse-pression. Ce concept porte le nom de ``Split
cycle engine''. On vous propose de calculer le rendement thermique d'un tel
cycle qui, dans une forme simplifiée et idéalisée, est composée des
transformations suivantes:

\begin{align*}
1 \to 2& \quad \textnormal{Compression isentropique}\\
2 \to 3& \quad \textnormal{Refroidissement intermédiaire isobare}\\
3 \to 4& \quad \textnormal{Compression isentropique}\\
4 \to 5& \quad \textnormal{Apport de chaleur isobare}\\
5 \to 6& \quad \textnormal{Détente isentropique}\\
6 \to 1& \quad \textnormal{Refroidissement isochore}
\end{align*}

Le rapport volumétrique de compression du premier cylindre $V_1/V_2$ vaut 5.
Celui du deuxième cycle $V_3/V_4$ vaut 11. Entre les deux, le refroidissement
intermédiaire permet de ramener la température à \SI{350}{\kelvin} avant la
seconde compression. L'apport de chaleur de la source chaude est équivalent à
une combustion à charge diluée, soit \SI{2200}{\kilo\joule\per\kilo\gram}. La
détente $5 \to 6$ se fait dans les deux cylindres successivement. Le gaz qui
parcourt le cycle est assimilé à de l'air (se comportant comme un gaz parfait)
dont les propriétés thermiques sont telles que $\gamma$ est constant et vaut
$1.3$.

\begin{enumerate}
    \item Complétez le tableau suivant :
    \begin{center}
\begin{tabular}{|c|c|c|c|c|}
  \hline
  &$p/\si{\bar}$ & $T/\si{\kelvin}$ & $V/\si{\meter\cubed}$ &
  $(S-S_1)/(\si{\joule\per\kelvin})$\\
  \hline
  1 & 1 & 300 & $\num{e-3}$ & \\
  2 &  &  &  & \\
  3 &  &  & & \\
  4 & & & &\\
  5 & & & &\\
  6 & & & &\\
  \hline
\end{tabular}
\end{center}
    \item Donnez le travail net du cycle.
    \item Donnez le rendement du cycle.
\end{enumerate}

\begin{solution}
\begin{center}
\begin{tabular}{|c|c|c|c|c|}
  \hline
  &$p/\si{\bar}$ & $T/\si{\kelvin}$ & $V/\si{\meter\cubed}$ &
  $(S-S_1)/(\si{\joule\per\kelvin})$\\
  \hline
  1 & 1 & 300 & $\num{e-3}$ & 0\\
  2 & $8.1$ & $487.1$ & $\num{2e-4}$ & 0\\
  3 & $8.1$ & 350 & $\num{1.437e-4}$ & $-0.476$\\
  4 & 183 & $718.6$ & $\num{1.306e-5}$ & $-0.476$\\
  5 & 183 & 2487 & $\num{4.52e-5}$ & $1.31$\\
  6 & $3.267$ & $973.3$ & $\num{e-3}$ & $1.31$\\
  \hline
\end{tabular}
\end{center}
\vspace*{\baselineskip{}}

\noindent Le nombre de moles parcourant le cycle est $n = \SI{0.04}{\mol}$.

On cherche les capacités calorifiques:

\begin{align*}
\frac{\cp}{\cv} = \gamma = 1.3& \qquad \cp - \cv = R\\
\implies \cp = \SI{36}{\joule \per \mol \per \kelvin}& \qquad \cv =
\SI{27.69}{\joule \per \mol \per \kelvin}.
\end{align*}

La ligne 2 est obtenue par la loi de Poisson:

\[
p_2=p_1\left(\frac{V_1}{V_2}\right)^{\gamma}.
\]

La 3\ieme{} ligne est obtenue uniquement par la loi des gaz parfaits puisque
$T$ et $p$ sont donnés.


On connait $V_4$ par l'énoncé et on obtient $p_4$ par la loi de Poisson:

\[
p_4=p_3\left(\frac{V_3}{V_4}\right)^{\gamma}.
\]

On a

 \[
 Q = \num{2.2e6} M_{\textnormal{m}} n = \SI{2548.48}{\joule}.
 \]

Pour la ligne 5, la chaleur fournie à pression constante équivaut au $\Delta H$:

\begin{align*}
     \dif H &= \dif U + p \dif V + V \dif p\\
     &= \delta Q + \delta W + p \dif V + V \dif p\\
     &= \delta Q + V\dif p\\
     &= \delta Q.
\end{align*}

puisque $\dif p=0$.

Ainsi

\[
Q_{p=\mathrm{cst}} = \Delta H=n\cp \Delta T
\]

et

\[
\Delta T = T_5 - T_4 = \frac{Q}{n \cp}= \SI{1768.5}{\kelvin}.
\]

On a donc:

\[
T_5 = T_4 + 1768,5 = \SI{2487}{\kelvin}
\]

Enfin, on connait $V_6=V_1$, et $p_6=p_5\left(\frac{V_5}{V_6}\right)^{\gamma}$.
On calcule les variations d'entropie:

\[
\dif S = \frac{\delta Q}{T} = \frac{\dif U+p\dif V}{T} = n \cv \frac{\dif T}{T}
+ nR\frac{\dif V}{V}
\]

\[
\Delta S=n\cv\ln\frac{T_f}{T_i}+nR\frac{V_{\textnormal{f}}}{V_{\textnormal{i}}}.
\]

On obtient ainsi $S_3$ et $S_5$.
On calcule les travaux (les isentropiques sont des transformations réversibles
avec $Q=0$):

\begin{alignat*}{2}
     &W_{1 \to 2} = \Delta U = n\cv (T_2-T_1) &&= \SI{206.89}{\joule}\\
     &W_{2 \to 3} = -p\Delta V &&= \SI{45.6}{\joule}\\
     &W_{3 \to 4} = \Delta U = n\cv(T_4-T_3) &&= \SI{408.67}{\joule}\\
     &W_{4 \to 5} = -p\Delta V &&= \SI{-588}{\joule}\\
     &W_{5 \to 6} = \Delta U = n\cv(T_6-T_5) &&= \SI{-1668.3}{\joule}\\
     &W_{6 \to 1} &&= \SI{0}{\joule}.
\end{alignat*}

Donc,

\[
W_{\textnormal{tot}} = \SI{-1595}{\joule}.
\]

Finalement,

\[
\eta=\frac{\abs{W_{\textnormal{tot}}}}{Q_{4 \to 5}} = \frac{1595}{2548.5} =
\SI{62.8}{\percent}.
\]
\end{solution}

\section{}

Le travail effectué ou reçu par les fluides s'exprime de manière différente
suivant que le système soit fermé ou ouvert. Dans le cas d'un système fermé, on
a:

\[
W=-\int p\dif V.
\]

Tandis que pour les systèmes ouverts, l'expression est

\[
w_{\textnormal{m}} = \int^2_1 v \dif p,
\]

où il a été fait abstraction d'éventuels autres termes dans les deux
expressions ci-dessus.

\begin{enumerate}[label=(\Alph*)]
    \item Justifiez physiquement la forme prise par l'expression du travail
    dans le cas des systèmes fermés. Quelles sont les hypothèses nécessaires
    pour que la définition soit correcte?
    \item Réalisez le développement mathématique nécessaire pour établir
    l'expression du $w_{\textnormal{m}}$ pour un système ouvert à partir de
    celle développée pour le travail dans les systèmes fermés.
    \item Il n'y a pas de terme du type $p \dif v$ ou $v\dif p$ dans
    l'expression du travail moteur. Pourquoi? Justifiez votre réponse sur base
    physique et non mathématique.
\end{enumerate}

\begin{solution}
\begin{enumerate}[label=(\Alph*)]
     \item L'intégrale signale qu'une augmentation de volume induit un travail.
     Le signe $-$ signifie que le travail effectué sur le système est négatif
     s'il est en expansion. Cela veut donc dire que le système effectue
     lui-même un travail sur l'extérieur. On considère que la transformation
     est réversible et par conséquent, que la pression externe est identique à
     la pression interne.
     \item On sait que pour un système ouvert, la variation d'énergie cinétique
     du fluide au sein du système est proportionelle à la somme des travaux des
     forces exercées sur le système. Ces forces sont de trois types : forces
     internes, forces externes de contact et forces externes à distance. On
     écrira donc que

     \[
     \Delta k = w_{\textnormal{e}} + w_{\textnormal{d}} + w_{\textnormal{i}}.
     \]

     Développons chacun de ces termes séparément.

     \begin{enumerate}
          \item $\boxed{\Delta k}$

          On a

          \[
          \Delta k = \frac{c_2^2 - c_1^2}{2},
          \]

          terme qui représente la variation d'énergie cinétique d'un kilogramme
          de fluide parcourant le système.

          \item $\boxed{w_{\textnormal{i}}}$

          Lorsque le fluide parcourt un système, il rencontre des pressions
          différentes. S'il est compressible, il peut y avoir un travail de
          compression ou de détente. Ce travail s'écrit

          \[
          w_{\textnormal{in}} = \int_1^2 p \dif v.
          \]

          Cependant, ce terme ne correspond pas exactement au terme que l'on
          cherche. Il manque l'effet des frottements. Si on note le travail dû
          aux frottements $w_{\textnormal{if}}$, on trouve que

          \[
          w_{\textnormal{i}} = w_{\textnormal{in}} - w_{\textnormal{if}}.
          \]

          \item $\boxed{w_{\textnormal{e}}}$

          Ce travail est la résultante de plusieurs composantes. On a d'une part
          l'apport externe dtravail par un organe moteur, $w_{\textnormal{m}}$.
          D'autre part on a les travaux d'introduction et d'extraction du fluide
          du système, $w_{\textnormal{intro}}$ et $w_{\textnormal{extra}}$. Ils
          sont donnés respectivement par les formules suivantes :

          \begin{align*}
               w_{\textnormal{intro}} &= p_1 v_1\\
               w_{\textnormal{extra}} &= -p_2 v_2.
          \end{align*}

          La somme des trois termes devient donc

          \[
          w_{\textnormal{e}} = w_{\textnormal{m}} + p_1 v_1 - p_2 v_2.
          \]

          \item $\boxed{w_{\textnormal{d}}}$

          Dans notre cas, la seule force à distance est la gravité. On a donc

          \[
          w_{\textnormal{d}} = -g \Delta z.
          \]
     \end{enumerate}

     On développe ensuite

     \begin{align*}
          \Delta k + g \Delta z &= w_{\textnormal{e}} + w_{\textnormal{i}}\\
          \implies \Delta k + g \Delta z &=  w_{\textnormal{m}} + p_1 v_1 - p_2
          v_2 + \int_1^2 p \dif v - w_{\textnormal{f}}\\
          &= w_{\textnormal{m}} - \int_1^2 \dif(pv) + \int_1^2 p \dif v -
          w_{\textnormal{f}}\\
          &= w_{\textnormal{m}} - \int_1^2 v \dif p -
          w_{\textnormal{f}}
     \end{align*}

     On réorganise pour obtenir\footnote{Un développement plus complet est
     donné aux pages 28-30 du syllabus de thermodynamique.}

     \[
     w_{\textnormal{m}} = \int_1^2 v \dif p + \Delta k + g \Delta z +
     w_{\textnormal{f}}.
     \]

     \item Ces termes n'apparaissent pas car ils sont pris en compte dans le
     terme de variation d'enthalpie $\Delta H$. En plus, ces termes de travaux
     sont repris dans la chaleur échangée, $q$, qui affecte directement le
     contenu énergétique total du fluide.
\end{enumerate}
\end{solution}

\section{QCM}

\begin{enumerate}
    \item À \SI{300}{\kelvin} et à pression atmosphérique, quelle est la
    vitesse quadratique moyenne des molécules de dioxyde de carbone?

    \[
    \SI{505}{\meter\per\second} \qquad \SI{412}{\meter\per\second} \qquad
    \SI{337}{\meter\per\second} \qquad \SI{380}{\meter\per\second} \qquad
    \SI{11}{\meter\per\second}
    \]

    \item Une machine frigorifique travaille selon un cycle de Carnot inversé
    entre une température extérieure de \SI{25}{\celsius} et une température
    intérieure de \SI{-10}\celsius. Quel est le coefficient de performance de
    cette machine?

    \[
    0.12 \qquad 2.5 \qquad 7.5 \qquad 0.6 \qquad 8.5
    \]
    \item Selon le second principe, pour une transformation quelconque, il est
    impossible de convertir totalement de la chaleur en travail.

    \[
    \textnormal{vrai}  \qquad \textnormal{faux}
    \]

    \item Afin de mesurer le débit d'eau dans une conduite de
    \SI{100}{\milli\metre} de diamètre, on réalise localement une contraction
    régulière avec un rapport de diamètre de 4. La chute de pression est de
    \SI{2000}{\pascal}. Quel est le débit dans la conduite?

    \[
    \SI{1}{\kilo\gram\per\second} \qquad \SI{16}{\kilo\gram\per\second} \qquad
    \SI{2}{\kilo\gram\per\second} \qquad \SI{200}{\litre\per\second} \qquad
    \SI{1000}{\litre\per\second}
    \]

    \item Quelle est la définition du coefficient de Joule-Thomson?
\end{enumerate}

\begin{solution}
\begin{enumerate}
     \item \SI{412}{\meter\per\second}

     \[
     \overline{c^2}=\sqrt{\frac{3RT}{M_{\textnormal{m}}}} \qquad
     \textnormal{avec } M_{\textnormal{m}} = \SI{44e-3}{\kilo\gram\per\mol}.
     \]

     \item $7.5$

     \[
     \mathrm{COP}_{\textnormal{frigo}} =
     \frac{1}{\frac{T_{\textnormal{H}}}{T_{\textnormal{C}}} - 1}.
     \]

     \item Vrai
     \item \SI{1}{\kilo\gram\per\second}

     \begin{align*}
     c_1^2 &= \left(\frac{A_2}{A_1}\right)^2 c^2_2\\
     &= \left(\frac{R_2}{R_1}\right)^4 c^2_2.
     \end{align*}

     Or on a que
     \begin{align*}
     &\Delta p = -\frac{\rho}{2} \Delta c^2 = -\frac{\rho}{2} \left( 1 -
     \left(\frac{R_2}{R_1} \right)^4 \right) c^2_2\\
     \implies &c_2^2 = \frac{-2\Delta p}{\rho
     \left(1-\left(\frac{R_2}{R_1}\right)^4\right)} =
     \SI{4.0159}{\meter\squared\per\second\squared}.
     \end{align*}

     \[
     \dot V = \pi R_2^2 c_2 = \pi \cdot (0.05/4)^2 \cdot \sqrt{4.0159} =
     \SI{e-3}{\meter\cubed\per\second}.
     \]

    \item
    \[
    \mu_{\textnormal{JT}} =  \left(\fpart{T}{p}\right)_H.
    \]
\end{enumerate}
\end{solution}

\end{document}
