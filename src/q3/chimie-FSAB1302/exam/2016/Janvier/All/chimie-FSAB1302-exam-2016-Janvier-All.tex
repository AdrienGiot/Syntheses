\documentclass[fr]{../../../../../../eplexam}

\usepackage{../../../../../../eplchem}
\usepackage{../../../../../../eplunits}

\newcommand{\potstd}[1]{E\std_{\text{éq},\text{#1}}}
\newcommand{\potstdce}{\potstd{cell}}
\newcommand{\potstdca}{\potstd{cathode}}
\newcommand{\potstdan}{\potstd{anode}}

\hypertitle{Chimie et chimie physique}{3}{FSAB}{1302}{2016}{Janvier}
{Louis Devillez}
{Hervé Jeanmart et Joris Proost}

\section{Question 1}
Afin de maximiser le rendement de certaines installations thermiques, on valorise au maximum l'énergie à basse température avant de rejeter le reste à la source froide. C'est le cas des turbines à gaz régénératives dont un équivalent en cycle fermé est composé des transformations suivantes:

\begin{itemize}
	\item 1-2 Compression adiabatique
	\item 2-3 Réchauffement isobare
	\item 3-4 Apport de chaleur isobare
	\item 4-5 Détente adiabatique
	\item 5-6 Refroidissement isobare
	\item 6-1 Échange isobare avec la source froide
\end{itemize}

Le gaz qui parcourt le cycle est assimilé à de l'air (gaz parfait) dont les propriétés thermiques sont telles que $\gamma = 1.3$. Dans ce cycle, la chaleur nécessaire au réchauffement du fluide de 2 à 3 est fournie par le refroidissement du fluide de 5 à 6. Cet échange se fait dans un échangeur parfait. Cela veut dire que $T_6 = T_2$ et qu'il n'y a pas de pertes vers l'extérieur du cycle. Toute la chaleur perdue de 5 à 6 est récupérée de 2 à 3. L'apport de chaleur de la combustion est de ce fait réduit et équivaut à $\SI{1000}{\kilo\joule\per\kilogram}$.

Sur base de ces informations et des valeurs déjà données dans le tableau ci-dessous, on vous demande de
\begin{enumerate}
	\item Compléter le tableau (en justifiant très succinctement)
	\item Calculer le travail net du cycle
	\item Calculer le rendement du cycle.
\end{enumerate}

\begin{center}
	\begin{tabular}{|c|c|c|c|c|}
		\hline
		&p$[\si{\bar}]$&$T[\si{\kelvin}]$&$V[\si{\meter^3}]$&$S-S_1 [\SI{}{\joule\per\kelvin}]$\\
		\hline
		1&1&300&0.002&\\
		\hline
		2&20&&&\\
		\hline
		3&&&&\\
		\hline
		4&&&&\\
		\hline
		5&&&&\\
		\hline
		6&&&&\\
		\hline
	\end{tabular}
\end{center}

\begin{solution}
	Commençons par remplir le tableau avec ce que l'on connaît
	\begin{center}
		\begin{tabular}{|c|c|c|c|c|}
			\hline
			&p$[\si{\bar}]$&$T[\si{\kelvin}]$&$V[\si{\meter^3}]$&$S-S_1 [\SI{}{\joule\per\kelvin}]$\\
			\hline
			1&1&300&0.002&0\\
			\hline
			2&20&$T_2$&&0\\
			\hline
			3&20&&&\\
			\hline
			4&20&&&\\
			\hline
			5&1&&&\\
			\hline
			6&1&$T_2$&&\\
			\hline
		\end{tabular}
	\end{center}
	$$ m = \frac{P V}{R^* T} = 0.0023$$
	
	Étant donné que $\gamma$ nous est donné nous devons recalculer les $C_v$ et $C_p$
	
	$$\gamma = \frac{c_p}{c_v} = \frac{c_v + R^*}{c_v} $$
	$$c_v = \frac{R^*}{\gamma - 1} = \SI{956}{\joule\per\kelvin\per\kilogram} $$
	$$ c_p = c_v + R^* = \SI{1244}{\joule\per\kelvin\per\kilogram} $$
	
	
	$$\Delta S = mc_p\ln(\frac{T_f}{T_i}) - mR^*\ln(\frac{P_f}{P_i}) = 0$$
	$$ T_2 = \exp\left(\frac{R^*\ln(\frac{P_f}{P_i})}{c_p}\right)T_1=599$$
	
	$$Q_{5-6} = -Q{2-3}$$
	$$c_p \Delta T_1 = - c_p \Delta T_2$$
	$$T_3 - T_2 = T_6 - T_5 $$
	$$ T_3 = T_5$$
	$$T_4 = T_3 +  \delta T = T_3 + \frac{Q}{c_p} = T_3 + 803$$
	
	En utilisant la transformation 4-5 qui est adiabatique
	$$mc_p\ln(\frac{T_f}{T_i}) = mR^*\ln(\frac{P_f}{P_i})$$
	$$ T_5 = \exp\left(\frac{R^*\ln(\frac{P_f}{P_i})}{c_p}\right)T_4 = 0.5* T_4$$
	$$2*T_5 = T_5 + 803$$
	$$T_5 = 807$$
	
	on peut ensuite finir le tableau
	
	\begin{center}
		\begin{tabular}{|c|c|c|c|c|}
			\hline
			&p$[\si{\bar}]$&$T[\si{\kelvin}]$&$V[\si{\meter^3}]$&$S-S_1 [\SI{}{\joule\per\kelvin}]$\\
			\hline
			1&1&300&0.002&0\\
			\hline
			2&20&599&0.0002&0\\
			\hline
			3&20&807&0.00027&$S_3$\\
			\hline
			4&20&1611&0.00054&$S_4$\\
			\hline
			5&1&807&0.00538&$S_4$\\
			\hline
			6&1&599&0.00399&$S_6$\\
			\hline
		\end{tabular}
	\end{center}
	$$S_3 = mc_p \ln(\frac{T_f}{T_i}) = 0.86$$
	$$S_4 =S_3 mc_p \ln(\frac{T_f}{T_i}) = 2.86$$
	$$S_6 =S_4 mc_p \ln(\frac{T_f}{T_i}) = 2.00$$
	
	$$W_{tot} = \sum mc_p \Delta T= 0.0023 * 287 (208 + 804 +  - 208 -299) = 1459 $$
	
	$$\eta = \frac{W_{tot}}{Q_C} = \frac{1459}{m*c_p*(T_4-T3)} = 0.628$$
\end{solution}




\end{document}
