\documentclass[fr]{../../../../../../epltest}
\usepackage{../../../../../../eplchem}
\usepackage{../../../../../../eplunits}

\hypertitle{Chimie et chimie physique}{3}{FSAB}{1302}{2016}{Novembre}{All}
{Martin Braquet}
{Hervé Jeanmart et Joris Proost}
\section{Cycle}

Un	système	fermé	évolue	selon	un	cycle	composé	de	trois 
transformations réversibles successives:
	\begin{itemize}
	    \item 1-2       transformation isotherme
	    \item 2-3		transformation	isobare
	    \item 3-1		transformation	adiabatique
	\end{itemize}
Le	gaz	qui	parcourt	le	cycle est	parfait. Il	a des	propriétés	constantes	égales	à	celles	de	l’air	($Mm	=	28,96	\:g/mol$)	à	température	ambiante.		
	
Sur	base	de	la	description	du	cycle,	des	données	du	tableau	ci-dessous,	et	sachant	que	l’apport	de	chaleur	lors	de	la	transformation	$2-3$	est	de	$125	kJ$,	on	vous	demande:	
\begin{itemize}
    \item de	compléter	le	tableau (/6)
    \item de	donner	les	directions	(signes)	et	les	amplitudes	(en	$kJ$)	des	échanges	de	chaleur	lors	des	transformations	1-2	et	3-1 (/2)
    \item de	calculer	le	travail	net	du	cycle	(à	exprimer	en	$kJ$) (/2)		
\end{itemize}	

\begin{center}

		\begin{tabular}{|c|c|c|c|c|}

			\hline

			&p$[\si{\bar}]$&$T[\si{\kelvin}]$&$V[\si{\meter^3}]$&$U-U_1 [kJ]$\\

			\hline

			1&2&300&0,1&0\\

			\hline

			2&&&&\\

			\hline

			3&&&&\\

			\hline


		\end{tabular}

	\end{center}

\begin{solution}

Le nombre de moles parcourant le cycle est 8,018 et $\gamma=\frac{C_p}{C_v}=\frac{7/2R}{5/2R}=1,4$.

Pour la transformation 2-3, la chaleur fournie à pression constante équivaut au $\Delta H$:
 $$d H=d U+pd V+Vd p=\delta Q +\delta W+pd V+Vd p=\delta Q+Vd p=\delta Q$$
 puisque $d p=0$.
 Ainsi
 $$ Q_{p=cst}=\Delta H=C_p\Delta T$$
 $$ \Delta T =T_3-T_2=\frac{Q}{C_p}=\frac{125000}{8,018*7/2R}= 535,71K$$
 On a donc:
 $$T_3=T_2+535,71=835,71K$$
 
 Par la loi de poisson, on calcule les données sur la 3e ligne.
 $$ V_3=V_1\bigg(\frac{T_1}{T_3}\bigg)^{\frac{1}{\gamma-1}}=7,72*10^{-3}\:m^3$$
 $$p_2=p_3=\frac{nRT_3}{V_3}=72,16\:[bar]$$
 $$V_2=\frac{nRT_2}{p_2}=2,77*10^{-3}\:m^3$$
 Passons aux variations d'énergie:
 $$U_2-U_1=0\mbox{  (isotherme)}$$
 $$U_3-U_2=\int^3_2C_v\:\mbox{d}T=C_v(T_3-T_2)=89,29\:[kJ]$$
 
 Voici le tableau complété:

\begin{center}

		\begin{tabular}{|c|c|c|c|c|}

			\hline

			&p$[\si{\bar}]$&$T[\si{\kelvin}]$&$V[\si{\meter^3}]$&$U-U_1 [kJ]$\\

			\hline

			1&2&300&0,1&0\\

			\hline

			2&72,16&300&0,0028&0\\

			\hline

			3&72,16&835,71&0,0077&89,29\\

			\hline


		\end{tabular}

	\end{center}
	
	Pour les échanges de chaleur:
	$$Q_{1-2}=\Delta U-W=-W=\int_1^2p\mbox{d}V=nRT_1\int_1^2\frac{\mbox{d}V}{V}=nRT_1\ln{\frac{V_2}{V_1}}=-71715\:J$$
	$$Q_{3-1}=0 \mbox{   (adiabatique)}$$
	
	On calcule enfin le rendement:
	$$\eta=\frac{|W_{tot}|}{Q_{hot}}=\frac{Q_{tot}}{Q_{2-3}}=\frac{125000-71715}{125000}=42,63\%$$

\end{solution}

\end{document}

