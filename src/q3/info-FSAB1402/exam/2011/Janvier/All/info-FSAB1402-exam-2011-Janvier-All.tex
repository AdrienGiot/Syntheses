\documentclass[fr]{../../../../../../eplexam}
\usepackage{../../../info-FSAB1402-exam}

\usepackage{../../../../../../eplcode}

\DeclareMathOperator{\pgcd}{PGCD}

\hypertitle[']{Informatique}{3}{FSAB}{1402}{2011}{Janvier}
{Philippe Verbist}
{Peter Van Roy}

\lstset{language={Oz},morekeywords={for,do}}


\newcommand{\st}{\mathrm{ST}}
\newcommand{\ce}{\mathrm{CE}}
\newcommand{\mozart}{Mozart}




\newpage


\newpage
\section{Programmation déclarative (5 pts)}

Pour cette question vous allez définir deux fonctions : une fonction \lstinline|Mul2| pour faire le produit
cartésien de deux listes et une fonction \lstinline|MulN| pour faire le produit cartésien de n listes (pour
tout $n\geqslant 0$) :
\begin{itemize}

\item L’appel Cs=\lstinline|{Mul2 As Bs}| prend deux listes, une liste As=$[ a_0 \:a_1 \ldots a_{n-1}]$ et une liste
Bs=$[ b_0\: b_1 \ldots b_{n-1}]$ et renvoie une liste Cs avec la valeur
$$[ a_0\# b_0 \ldots a_0\# b_{m-1}$$
$$a_1\# b_0 \ldots a_1\# b_{m-1}$$
$$\ldots$$
$$a_{n-1}\# b_0 \ldots a_{n-1}\# b_{m-1}]$$
La liste Cs contient \textit{mn} paires de la forme $x\# y$ où l’atome \# permet de faire une paire de
deux valeurs. Par exemple, \lstinline|{Mul2 [a b] [1 2]}| renvoie $[a\#1 \:a\#2\: b\#1\: b\#2]$ .

\item  L’appel Cs=\lstinline|{MulN Ass}| prend une liste de listes \textit{Ass} et fait le produit cartésien de toutes
les listes dans \textit{Ass}. Par exemple, \lstinline|{MulN [As Bs]}| donne le même résultat que \lstinline|{Mul2 As Bs}| , mais \lstinline|MulN| doit aussi marcher correctement avec une liste \textit{Ass} qui contient un nombre
quelconque de listes. Attention à ce que chaque élément de \textit{Cs} soit un tuple de \textit{n} éléments
(pas question d’utiliser \lstinline|Mul2| récursivement : cela donnera des paires imbriquées). Pour
information, l’appel \lstinline|{MakeTuple T N}| crée un tuple avec étiquette T et N éléments qui
sont tous des variables non-liées.

\item  (bonus) Si toutes les fonctions sont écrites avec une belle indentation, il y aura un bonus
de 0,5 points.

\end{itemize}
Faites attention à ce que chaque fonction soit déclarative et récursive terminale. Essayez de les
définir la plus simplement possible et de les écrire avec une belle indentation pour faciliter la
lecture. Attention à la syntaxe ! Toute erreur de syntaxe sera sévèrement pénalisée.

\begin{solution}

\lstinputlisting{2011-01-1.oz}

\end{solution}

\section{Sémantique (5 pts)}

Voci un petit programme:

\lstinputlisting{2011-01-2(a).oz}
Répondez aux questions suivantes:
\begin{enumerate}
	\item Qu'est-ce qui est affiché quand on exécute ce programme ? Attention à bien suivre l’exécution !
	\item Donnez la traduction de ce programme en langage noyau. Attention à donner une traduction complète !
	\item Donnez les environnements contextuels de toutes procédure dans cette traduction.
	\item Donnez quelques pas d'exécution de la machine abstraite pour bien montrer les choses suivantes:
	\subitem La création, la lecture et l'affectation d'une cellule.
	\subitem La définition et l'appel d'une procédure. Qu'est est l'environnement juste avant chaque appel et l'environnement quand on est à l'intérieur de la procédure juste avant l'exécution de son corps ?
\end{enumerate}

\begin{solution}

\begin{enumerate}

\item
Le code affiche 109

\item
\lstinputlisting{2011-01-2(b).oz}

\item
Environnements contextuels:
\begin{itemize}
\item de la fonction \lstinline|F|:$E_{c,F}$ = \lstinline|{C1->c1}| 
\item de la fonction anonyme:$E_{c,anon}$ = 
\lstinline|{A->a, C1->c1, C2->c2prim}|, où \lstinline|c2prim| est la variable en mémoire correspondant à la cellule créée à la ligne 10: \lstinline|{NewCell N8 C2}|
\end{itemize}

\item 
Malheureusement, les questions suivantes n’ont pas encore de solution. Vous êtes encouragés à en soumettre une à l’adresse suivante
\begin{center}
https://github.com/gp2mv3/Syntheses
\end{center}
ou par mail.

\end{enumerate}

\end{solution}

\section{Concepts (5 pts)}

Définissez chacun des concepts suivants avec le plus de précision possible. Pour chaque concept
donnez un exemple concret (code ou algorithme) pour bien illustrer le concept.

\begin{enumerate}

\item Complexité temporelle dans le meilleur cas.

\item Portée lexicale d’une occurrence d’un identificateur.

\item Environnement contextuel d’une procédure.

\item Non-déterminisme.

\item Correspondance des formes (pattern matching).

\item (bonus) Cycle de vie d’un bloc mémoire.

\end{enumerate}

\begin{solution}

\begin{description}

\item[Complexité temporelle dans le meilleur cas]
Soit un problème P(N1, N2, Nn). L'analyse de la complexité temporelle de ce problème consiste à décrire le nombre d'opérations à l'exécution en fonction des paramètres N1 ... Nn.
L'analyse de la complexité temporelle dans le meilleur cas consiste à choisir N1, N2, ... Nn tels que la résolution du problème soit la plus rapide. On obtient ainsi une fonction $\Omega$ définie comme suit:
$$ f(n)=\Omega(g(n)) \Leftrightarrow \exists n_0 > 0, \exists k>0: \forall n>n_0, f(n) \geq k\cdot g(n) $$
où $f(n)$ est la complexité temporelle (la vraie!)
et $\Omega(g(n))$ est la complexité temporelle dans le meilleur cas.

\item[Portée lexicale d'une occurrence d'un identificateur]
Partie d'un programme dans laquelle l'identificateur correspond à la même variable en mémoire.
\begin{lstlisting}
local A=5 in % {A->a1}
	local A=3 in %{A->a2}
		{Browse A}
	end
	{Browse A}
end
\end{lstlisting}
La portée de l'identificateur A défini à la ligne 1 englobe les lignes suivantes: 1, 5-6
La portée de l'identificateur A défini à la ligne 2 englobe les lignes suivantes: 2-4



\item[Environnement contextuel d'une procédure]
L'environnement est une fonction qui permet de faire la correspondance entre des identificateurs et des entités en mémoire. 
L'environnement contextuel d'une procédure est l'environnement qui englobe tous les identificateurs qui sont utilisés dans une procédure mais qui ne sont pas définis dans la définition de celle-ci.

\begin{lstlisting}
local A=5 B=3 in
	proc {FunnyFun C ?Out}
		Out = A+B+C
	end
end
\end{lstlisting}

L'environnement contextuel de FunnyFun est \{A->a, B->b\}.

\item[Non-déterminisme]
Le non-déterminisme apparait, par exemple, lorsqu'on utilise la concurrence avec les cellules. Il s'agit du fait qu'on laisse un certain "choix" à l'ordinateur, et que ce choix influence le résultat. Par exemple:
\begin{lstlisting}
local A = {NewCell 0} in
	thread A:=2 end
	thread A:=5 end
end
\end{lstlisting}

A la fin de l'exécution de ce code, on ne sait pas si la valeur dans la cellule A sera 2 ou 5.


\item[Correspondance des formes (pattern matching)]
Le pattern matching consiste en une succession de clauses.
Une clause est exécutée, si la forme de l'élément dans l'instruction \lstinline|case| correspond à ce qui est compris dans la clause.
La forme correspond si l'étiquette (label) et les arguments correspondent. Si c'est le cas, les identificateurs de la forme sont affectés aux parties correspondantes. Les clauses sont testées dans l'ordre textuel.
La première à correspondre est exécutée et pas les autres.

Exemple:

\begin{lstlisting}
local A = [5 10] B in
	case A of nil then B = 0
	[]H|T then B=H
	[]M then B=matchEverything
	end
end
\end{lstlisting}

Dans cet exemple, B sera lié à 5. A noter que la clause []M permet de matcher n'importe quoi!

\item[Cycle de vie d'un bloc mémoire]

Un bloc mémoire passe successivement par différentes étapes:\\
libre: le bloc n'est pas utilisé\\
actif: le bloc est utilisé par l'exécution d'un programme\\
inactif: lorsque le programme n'a plus besoin du bloc mémoire. Les blocs inactifs peuvent être récupérés par le garbage collector, et redeviennent alors libres.\\
Le passage de libre à actif s'appelle l'allocation et le passage inverse s'appelle la désallocation.

\end{description}

\end{solution}

\end{document}
