\documentclass[fr]{../../../../../../eplexam}
\usepackage{../../../info-FSAB1402-exam}

\usepackage{../../../../../../eplcode}

\DeclareMathOperator{\pgcd}{PGCD}

\hypertitle[']{Informatique}{3}{FSAB}{1402}{2011}{Juin}{All}
{Louis Devillez}
{Peter Van Roy}

\lstset{language={Oz},morekeywords={for,do}}

\newcommand{\st}{\mathrm{ST}}
\newcommand{\ce}{\mathrm{CE}}
\newcommand{\mozart}{Mozart}

\section{Question 1 : Programmation déclarative (5 pts)}
Pour cette question vous devez définir deux fonctions: \lstinline|Claque4| qui renverse les positions des éléments d'une liste quatre par quatre et \lstinline|ClaqueK| qui renverse les positions des éléments d'une liste k par k:
\begin{itemize}
	\item L'appel \lstinline|{Claque4 L}| prend la liste $L=[a_0 a_1 \ldots a_{n-1}]$ et renvoie la liste $[a_3 a_2 a_1 a_0 a_7 a_6 a_5 a_4 \ldots]$.  Par exemple, \lstinline|{Claque4 [vous etes le phenix des hotes de ces bois]}| renvoie \lstinline|[phenix le etes vous ces de hotes des]|. Si n n'est pas un multiple de 4 les éléments supplémentaires à la fin ne se retrouvent pas dans la sortie.
	
	\item L'appel \lstinline|{ClaqueK K L}| prend la liste $L=[a_0 a_1 \ldots a_{n-1}]$ et l'entier K et renvoie la liste $[a_{k-1} a_{k-2} $ $\ldots a_1 a_0 a_{2k-1} a_{2k-2} \ldots a_{k+1} a_k \ldots ]$. Par exemple, \lstinline|{ClaqueK 3 [a ces mots le corbeau ne se sent pas de joie]}| renvoie \lstinline|[mots ces a ne corbeau le pas sent se]|. Vous pouvez supposer que K $\geq 1$. Si n 'est pas un multiple de k, les éléments supplémentaires à la fin ne se retrouvent pas dans la sortie.
\end{itemize}
Vous n'avez pas le droit d'utiliser \lstinline|ClaqueK| pour définir \lstinline|Claque4|. Chaque fonction doit être déclarative et récursive terminale. Essayez de les définir le plus simplement possible et de les écrire avec une belle indentation pour faciliter la lecture. Il y aura un petit bonus si la fonction est bien indentée. Attention à la syntaxe! Toute erreur de syntaxe sera sévèrement pénalisée.	

\begin{solution}
	\lstinputlisting{Juin2011.oz}
\end{solution}
\section{Question 2 : Sémantique (5 pts)}
Voici un petit programme:
\lstinputlisting{Q2.oz}
Répondez aux questions suivantes:
\begin{itemize}
	\item Qu'est-ce qui est affiché quand on exécute ce programme ?
	\item Donnez la traduction de ce programme en langage noyau. Attention à donner une traduction complète !
	\item Donnez les environnements contextuels de toutes procédure dans cette traduction.
	\item Donnez quelques pas d'exécution de la machine abstraite pour bien montrer les choses suivantes:
	\subitem La création, la lecture et l'affectation d'une cellule.
	\subitem La définition et l'appel d'une procédure. Qu'est est l'environnement juste avant chaque appel et l'environnement quand on est à l'intérieur de la procédure juste avant l'exécution de son corps ?
\end{itemize}
\begin{solution}
	\begin{enumerate}
	\item
\begin{lstlisting}
25
6
\end{lstlisting}

 \item
\lstinputlisting{Q2Sol.oz}
		\item Environnement contextuel:
		\subitem F1 :$\{ \}$
		\subitem F2 :$\{ C\rightarrow \xi\}$
		
		\item Machine abstraite (attention il faut utiliser le code donné, ici ce n'est qu'un exemple)
		\begin{enumerate}
			\item
			
			
			$(\{[(\{NewCell\ A\ C\},\{A \rightarrow a,X \rightarrow x,C \rightarrow c\}),\dots ]\},\{a = 1,c,x\},\{\})$
			
			$(\{(X = @C +1,\{A \rightarrow a,C \rightarrow c,X \rightarrow x\})\dots ]\},\{a = 1,c = \xi,x\},\{c:a\})$
			
			$(\{[(C := X,\{A \rightarrow a,C \rightarrow c,X\rightarrow x\})\dots ]\},\{a = 1,c = \xi,x =2\},\{c:a\})$
			
			$(\{[\dots ]\},\{a = 1,c = \xi,x =2\},\{c:x\})$
			
			\item
			$(\{[(F1 = proc\{\$\text{ }X\text{ }R\}R = A + X end,\{A \rightarrow a,F1 \rightarrow f1, B \rightarrow b \}),\dots ]\},\{a = 1,f1\},\{\})$
			
			$(\{[(\{F1\text{ }10\text{ }B\},\{A \rightarrow a,F1 \rightarrow f1,B \rightarrow b\}),\dots ]\},\{a = 1,f1 = (proc\{\$\text{ }X\text{ }R\}R = A + X end, \{A \rightarrow a\}),b\},\{\})$
			
			$(\{[\dots ]\},\{a = 1,f1 = (proc\{\$\text{ }X\text{ }R\}R= A + X end, \{A \rightarrow a\}),b = 11\},\{\})$
			
			
		\end{enumerate}
	\end{enumerate}
\end{solution}
\section{Question 3 : Concepts (5 pts)}
Définissez chacun des concepts suivants avec le plus de précision possible. Pour chaque concept donnez un exemple concret (code, algorithme ou formule) pour bien illustrer le concept.
\begin{itemize}
	\item (1 pt) Le type abstrait
	\item (1 pt) La spécification d'un programme
	\item (1 pt) L'environnement
	\item (2 pts) L'exception. Donnez les deux nouvelles instructions et leur sémantique en expliquant ce qui se passe sur la pile sémantique
	\item(bonus 1 pt) Le cycle de vie d'un bloc mémoire
\end{itemize}

\begin{solution}
	\begin{enumerate}

	\item Le type abstrait: un type abstrait est une manière d’organiser une abstraction de données, où les valeurs et les opérations sont distinctes. Un type est abstrait s’il est défini pour l’ensemble de ses opérations, indépendamment de son implémentation.
	\item La spécification d'un programme: la spécification est une définition du résultat du programme en termes d’entrée.
	\item Environnement: Fonction qui prend un identificateur et renvoie une variable. Il fait correspondre chaque identificateur à une variable en mémoire (et sa valeur). Un même identificateur peut correspondre à différentes variables en différents endroits du programme.
	\item: Exceptions
	
	Une exception est un record lancé par une méthode lorsqu'une erreur intervient. Cet objet contient plusieurs informations sur l'erreur.
	\begin{lstlisting}
	try <s>_1 catch <x> then <s>_2 end
	\end{lstlisting}
	Sémantiquement, on passe de
	\[ ([(1,\ce),\ldots],\sigma) \]
	à
	\[ ([(\verb|<s>_1|,\ce),(\verb|catch <x> then <s>_2|,\ce),\ldots],\sigma). \]
	
	\begin{lstlisting}
	raise <x> end
	\end{lstlisting}
	Sémantiquement, la machine abstraite va dépiler chaque instruction
	de la pile sans même les exécuter jusqu'à arriver à un \lstinline|catch|
	qui correspond à \lstinline|<x>| au niveau du pattern matching.
	
	Par exemple, on passe de
	\[ ([(\verb|raise X end|,\ce_1),(\verb|R=S|,\ce_2),
	(\verb|catch a(M) then skip end|,\ce_3), \]
	\[ (\verb|catch b(M) then skip end|,\ce_4),\ldots],
	\{x=\verb|b(v)|,v=\verb|'bla'|,\ldots\}) \]
	en
	\[ ([(\verb|skip|,\ce_4 + \{\verb|M|\to m\}),\ldots],
	\{x=\verb|b(v)|,v=\verb|'bla'|,m=\verb|'bla'|,\ldots\}). \]
	
	\item Cycle de mémoire:
	Le cycle mémoire est représenté à la figure \ref{fig:ref}
	\subitem Mot dans la pile: Actif $\rightarrow$ Libre $\rightarrow$ Actif $\rightarrow$ Libre$\rightarrow$Actif \dots
	\subitem Mot en mémoire: Actif $\rightarrow$ Inactif $\rightarrow$ Libre $\rightarrow$ Actif $\rightarrow$ Inactif $\rightarrow$ Libre $\rightarrow$ Actif $\rightarrow$ \dots

		\begin{solfig}{ref}{Cycle de mémoire}

			\begin{tikzpicture}
			\draw [thick] (0,0) circle(1);
			\draw [red](0,0)node[]{Actifs};
			\draw [thick] (0,-4) circle(1);
			\draw [yellow](0,-4)node[]{Inactifs};
			\draw [thick] (6,0) circle(1);
			\draw [green](6,0)node[]{Libres};
			\draw [->,line width = 2pt](0,-1) -- (0,-3);
			\draw [->,line width = 2pt](1,-4) -- (6,-4) -- (6,-1);
			\draw [->,line width = 2pt] (1,-0.3) -- (5,-0.3);
			\draw [<-,line width = 2pt] (1,0.3) -- (5,0.3);
			\draw (3,0.7)node[]{Allocation};
			\draw (3,-0.7)node[]{Désallocation};
			\draw (-1.75,-2)node[]{Mot devient inactif};
			\draw (3,-4.7)node[]{Ramassage de miette};
			\draw (-3,0)node[align = left]{Mots utilisés dans pile \\ et dans la mémoire};
			\draw (-3,-4)node[align = left]{Mots en mémoire \\ devenus inacessibles };
			\end{tikzpicture}
		\end{solfig}	
\end{enumerate}
\end{solution}
\end{document}
