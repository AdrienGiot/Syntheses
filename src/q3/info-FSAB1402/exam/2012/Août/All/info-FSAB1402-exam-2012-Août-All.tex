\documentclass[fr]{../../../../../../eplexam}
\usepackage{../../../info-FSAB1402-exam}

\usepackage{../../../../../../eplcode}

\DeclareMathOperator{\pgcd}{PGCD}

\hypertitle[']{Informatique}{3}{FSAB}{1402}{2012}{Août}
{Louis Devillez}
{Peter Van Roy}

\lstset{language={Oz},morekeywords={for,do}}

\newcommand{\st}{\mathrm{ST}}
\newcommand{\ce}{\mathrm{CE}}
\newcommand{\mozart}{Mozart}

\section{Question 1 : Programmation déclarative (5 pts)}
Pour cette question vous devez définir les deux fonctions \lstinline|Courir2| et \lstinline|Courir3|:
\begin{itemize}
	\item La fonction \lstinline|Courir2 L1 L2| prend deux listes d'entiers $L1 = [a_{0} \ldots a_{n-1}]$ et $L2 = [b_{0} \ldots b_{n-1}]$ et renvoie $L3 = [(a_{0} + b_{0}) (a_{1} + b_1) \ldots (a_i +b_i)(a_{i+1 mod n} + b_{i+1 mod m} \ldots]$. Pour calculer les éléments successifs de \lstinline|L3|, on incrémenter les indice de a et b en même temps. Quand un indice arrive à son maximum ($n-1$ pour les a, $m-1$ pour les b) on recommence à 0. La liste \lstinline|L3| se termine quand l'itération rencontre une seconde fois $a_0 + b_0$. par exemple \lstinline|{Courir2 [4 5 4] [78]}| renvoie la liste $[(4+7) (5+8) (4+7) (5+7) (4+8) (4+7)]$, c'est à dire \lstinline|[11 13 11 12 12 12 11]|.
	\item La fonction \lstinline|{Courir3 L1 L2 L3}| fait la même chose mais avec trois listes. C'est-à-dire, quand elle voit arriver $a_0 + b_0 +c_0$ la seconde fois, elle termine son résultat. 
\end{itemize}	
	Faites attention à ce que chaque fonction soit déclarative et récursive terminale (aucune cellule ne sera tolérée !). Essayez de les définir la plus simplement possible et de les écrire avec une belle indentation pour faciliter la lecture. Attention à la syntaxe ! Toute erreur de syntaxe sera sévèrement pénalisée.

\begin{solution}
	\lstinputlisting{Aout2012.oz}
	
	On pouvait aussi imaginer pour \lstinline|Courir3| une fonction comme pour \lstinline|Courir2| donc avec:
	\begin{lstlisting}
fun{FUN L1 L2 L3 A B C}
	case A#B#C of ...
end
	\end{lstlisting}
\end{solution}


\section{Question 2 : Sémantique (5 pts)}
Voici un petit programme:
\lstinputlisting{Q2.oz}
Répondez aux questions suivantes:
\begin{itemize}
	\item Qu'est-ce qui est affiché quand on exécute ce programme ?
	\item Donnez la traduction de ce programme en langage noyau. Attention à donner une traduction complète !
	\item Donnez les environnements contextuels des deux procédures dans cette traduction.
	\item Donnez quelques pas d'exécution de la machine abstraite pour bien montrer les choses suivantes:
	\subitem La création, la lecture et l'affectation d'une cellule.
	\subitem La définition et l'appel d'une procédure. Qu'est est l'environnement juste avant chaque appel et l'environnement quand on est à l'intérieur de la procédure juste avant l'exécution de son corps ?
\end{itemize}
\begin{solution}
	\begin{enumerate}
		\item 
		\begin{lstlisting}
11
21
		\end{lstlisting}
		\item
\lstinputlisting{Q2Sol.oz}
		On aurait pu traduire les opérateurs \lstinline|@| et \lstinline|:=| en language noyaux
		
		\item
		\subitem F1 :$\{ C1\rightarrow \beta, C2 \rightarrow \alpha \}$
		\subitem F2 :$\{ C1\rightarrow \beta, C2 \rightarrow \alpha \}$
		\item Machine abstraite (attention il faut utiliser le code donné, ici ce n'est qu'un exemple)
		\begin{enumerate}
			\item

		
			$(\{[(\{NewCell\ A\ C\},\{A \rightarrow a,X \rightarrow x,C \rightarrow c\}),\dots ]\},\{a = 1,c,x\},\{\})$
			
			$(\{(X = @C +1,\{A \rightarrow a,C \rightarrow c,X \rightarrow x\})\dots ]\},\{a = 1,c = \xi,x\},\{c:a\})$
			
			$(\{[(C := X,\{A \rightarrow a,C \rightarrow c,X\rightarrow x\})\dots ]\},\{a = 1,c = \xi,x =2\},\{c:a\})$
			
			$(\{[\dots ]\},\{a = 1,c = \xi,x =2\},\{c:x\})$
			
			\item
			$(\{[(F1 = proc\{\$\text{ }X\text{ }R\}R = A + X end,\{A \rightarrow a,F1 \rightarrow f1, B \rightarrow b \}),\dots ]\},\{a = 1,f1\},\{\})$
			
			$(\{[(\{F1\text{ }10\text{ }B\},\{A \rightarrow a,F1 \rightarrow f1,B \rightarrow b\}),\dots ]\},\{a = 1,f1 = (proc\{\$\text{ }X\text{ }R\}R = A + X end, \{A \rightarrow a\}),b\},\{\})$
			
			$(\{[\dots ]\},\{a = 1,f1 = (proc\{\$\text{ }X\text{ }R\}R= A + X end, \{A \rightarrow a\}),b = 11\},\{\})$
			
			
		\end{enumerate}
	\end{enumerate}
\end{solution}
\section{Question 3 : Concepts (5 pts)}
Définissez chacun des concepts suivants avec le plus de précision possible. Pour chaque concept donnez un exemple concret (code ou algorithme) pour bien illustrer le concept.
\begin{itemize}
	\item (1pt) Objet
	\item (1pt) Classe
	\item (2pts) Exception. Donnez les deux instructions qui réalisent le concept d'exception et expliquez leur sémantique en montrant ce qui se passe sur la pile sémantique
	\item (1pt) Non déterminisme
	\item (Bonus) (1pt) Problème NP-Complet
\end{itemize}

\begin{solution}

\begin{itemize}
	\item Objet:Objet : un objet est une abstraction de données qui contient à la fois la valeur et le jeu d’opérations. Un objet est une collection de procédures (les « méthodes ») qui ont accès à un état commun (les « attributs »)
	
	exemple d'objet: G
	\begin{lstlisting}
declare
class Dice
	attr value
	meth init() value :=0 end
	meth set(X) value := X end
	meth get(X) @value end
end

G = {New Dice init()}
	\end{lstlisting}
	\item Classe: C’est une structure de données qui définit l’état interne d’un objet, son comportement et les classe dont il hérite. Dans Oz, une classe est un enregistrement qui contient un ensemble de noms d’attributs et un ensemble de méthodes.
	
	Dans l'exemple plus haut Dice est une classe
	
	\item: Exceptions
	
	Une exception est un record lancé par une méthode lorsqu'une erreur intervient. Cet objet contient divers informations sur l'erreur.
	\begin{lstlisting}
try <s>_1 catch <x> then <s>_2 end
	\end{lstlisting}
	Sémantiquement, on passe de
	\[ ([(1,\ce),\ldots],\sigma) \]
	à
	\[ ([(\verb|<s>_1|,\ce),(\verb|catch <x> then <s>_2|,\ce),\ldots],\sigma). \]
	
	\begin{lstlisting}
raise <x> end
	\end{lstlisting}
	Sémantiquement, la machine abstraite va dépiler chaque instruction
	de la pile sans même les exécuter jusqu'à arriver à un \lstinline|catch|
	qui correspond à \lstinline|<x>| au niveau du pattern matching.
	
	Par exemple, on passe de
	\[ ([(\verb|raise X end|,\ce_1),(\verb|R=S|,\ce_2),
	(\verb|catch a(M) then skip end|,\ce_3), \]
	\[ (\verb|catch b(M) then skip end|,\ce_4),\ldots],
	\{x=\verb|b(v)|,v=\verb|'bla'|,\ldots\}) \]
	en
	\[ ([(\verb|skip|,\ce_4 + \{\verb|M|\to m\}),\ldots],
	\{x=\verb|b(v)|,v=\verb|'bla'|,m=\verb|'bla'|,\ldots\}). \]
	
	\item: Non-déterminisme: propriété d'un programme qui peut renvoyer des résultats différents pour des exécutions avec les mêmes paramètres. Le non déterminisme est présent lorsque le système fait des choix lors de l'exécution même lorsque les résultats sont les mêmes.
	
	\begin{lstlisting}
declare
A = {NewCell 0}
thread A := 1 end
thread A := 2 end
{Browse A}
	\end{lstlisting}
	\item NP-complet: Un problème est dans la classe NP si on peut vérifier un candidat solution en temps polynomial. Certains problèmes dans la classe NP ont la propriété, que si on trouve un algorithme efficace pour résoudre le problème, on peut dériver un algorithme efficace pour tous les problèmes NP. On les appelle les problèmes NP-Complet.
	
	
\end{itemize}
\end{solution}
\end{document}
