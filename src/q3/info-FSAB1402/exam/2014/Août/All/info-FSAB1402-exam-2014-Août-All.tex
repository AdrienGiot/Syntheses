\documentclass[fr]{../../../../../../eplexam}
\DeclareMathOperator{\pgcd}{PGCD}
\usepackage{../../../info-FSAB1402-exam}
\usepackage{../../../../../../eplcode}
\hypertitle{Informatique}{3}{FSAB}{1402}{2014}{Août}
{Sebastien Colla \and Martin Braquet}
{Peter Van Roy}

\lstset{language={Oz},morekeywords={for,do}}

\newcommand{\st}{\mathrm{ST}}
\newcommand{\ce}{\mathrm{CE}}
\newcommand{\mozart}{Mozart}

\section{Programmation declarative (5 pts)}

Pour cette question vous allez écrire des fonctions déclaratives pour calculer avec des polynômes.\\
Nous représentons le polynôme $a_0 + a_1x + ...+ a_{n-1}x^{n-1}$ comme une liste $[a_0\quad a_1 ...a_{n-1}]$.
\begin{itemize}
    \item \lstinline|{FoldPoly L X F U}| avec $L=[a_0\quad a_1 ...a_{n-1}]$ renvoie le résultat
$$(a_0* f\:x*(a_1*f \:x *(... (a_{n-1}* f \:x*u)...).$$
Par exemple, si L est une liste d’entiers, \lstinline|{FoldPoly L 1 fun {\$ X Y} X+Y end 0}|
renvoie $$a_0 + a_1 + ...+ a_{n-1}$$ et \lstinline|{FoldPoly} L X fun {\$ X Y} X+Y end 0}| renvoie
$$a_0 + a_1x + ...+ a_{n-1}x^{n-1}.$$
\item \lstinline|{AddPoly L1 L2}| avec $L1=[a_0\quad a_1 ...a_{n-1}]$ et $L2= [b_0\quad b_1 . . . b_{n - 1}]$ renvoie le résultat $[(a_0 +
b_0) (a_1 +b_1) . . . (a_{m - 1} +b_{m - 1}) b_m b_{m +1} . . . b_{n - 1}]$ si $m \leqslant n$ (et de façon analogue si $m \geqslant n$).
\item \lstinline|{MulPolyConst} L C}| avec $L=[a_0\quad a_1 ...a_{n-1}]$  renvoie le résultat \\$[a_0C\quad a_1C ...a_{n-1}C]$.
\item \lstinline|{MulPoly} L1 L2}| avec $L1=[a_0\quad a_1 ...a_{n-1}]$ et $L2= [b_0\quad b_1 . . . b_{n - 1}]$ renvoie la multiplication des polynômes L1 et L2. Ceci peut être réalisé par une fonction récursive qui utilise
\lstinline|{MulPolyConst}| et \lstinline|{AddPoly}|.
\end{itemize}
Pour que votre programme soit lisible, il est impératif que chaque fonction ou procédure récursive
réalise \textit{une seule boucle}. C’est-à-dire que l’itération se fait sur un seul argument (traverser une
liste, incrémenter un entier). Chaque boucle en trop dans une fonction donnera lieu à une pénalité d’un point (sur 5).

\begin{solution}

		\lstinputlisting{Aout-2014-Q1.oz}

\end{solution}

\section{Sémantique (5 pts)}

Voici un petit programme:

\lstinputlisting{Aout-2014-Q2-enonce.oz}

Répondez aux question suivantes:
\begin{enumerate} 

	\item Qu'est-ce qui est affiché quand on exécute ce programme ? 
 
	\item Donnez la traduction de ce programme en langage noyau. Attention à donner une traduction complète ! 
 
	\item Donnez les environnements contextuels des deux procédures dans cette traduction. 
 
	\item Donnez quelques pas d'exécution de la machine abstraite pour bien montrer les choses suivantes: 
 
 \begin{itemize}
 \item La création, la lecture et l'affectation d'une cellule. 

	\item La définition et l'appel d'une procédure. Quel est l'environnement juste avant chaque appel et l'environnement quand on est à l'intérieur de la procédure juste avant l'exécution de son corps ? 
 \end{itemize}
	

\end{enumerate} 

\begin{solution}

\begin{description}

\item[1] $[1 \:2 \:4 \:6]$

\item[2] \lstinputlisting{Aout-2014-Q2.oz}

\item[3]

$$CE_{R1}=\{Newcell\rightarrow newcell,D\rightarrow d\}$$
$$ CE_{MC}=\{Newcell\rightarrow newcell\} $$

\item[4]
Toute l'exécution de la machine abstraite est effectuée. Les états d'exécution à fournir à l'examen sont signalés ci-dessous.
\begin{enumerate}
    \item 
    $$ \Big([(<l1-l36>,\emptyset)],\emptyset,\emptyset\Big) $$
    \item 
    Avant la définition d'une procédure
    $$ \Big([(<l2-l35>,\overbrace{\{Mc\rightarrow mc,MO\rightarrow mo,O1\rightarrow o1,O2\rightarrow o2,I1\rightarrow i1,}^{E_1}$$
    $$\overbrace{RA1\rightarrow ra1,RA2\rightarrow ra2,RA3\rightarrow ra3,
    RA4\rightarrow ra4,}^{E_1}$$
    $$\overbrace{Browse\rightarrow browse,Newcell\rightarrow newcell\}}^{E_1})],$$
    $$\overbrace{\{mc,mo,o1,o2,i1,ra1,ra2,ra3,ra4,browse=(proc\{\$\: X\}...end,}^{\sigma_2}$$
    $$\overbrace{CE_{Browse}),newcell=(proc\{\$ \:X\: Y\}...end,CE_{Newcell})\}}^{\sigma_2},\emptyset\Big) $$
    \item 
    Après la définition d'une procédure
    $$ \Big([(<l27-l35>,E_1)],$$
    $$\sigma_2\cup\{mc=(proc\{\$  \:R1\}<l3-l25>end,CE_{MC}=\{Newcell\rightarrow newcell\})\},\emptyset\Big) $$
    \item 
    Avant l'appel d'une procédure
    $$ \Big([(<l27>,E_1),(<l28-l35>,E_1)],\sigma_3,\emptyset\Big) $$
    \item 
    Après l'appel d'une procédure (l'environnement après l'appel n'est pas $E_1$!)
    $$ \Big([(<l3-l25>,\{Newcell\rightarrow newcell,R1\rightarrow mo\}),
    (<l28-l35>,E_1)],\sigma_3,\emptyset\Big) $$
    \item 
    Avant la création d'une cellule
    $$ \Big([(<l5>,\overbrace{\{Newcell\rightarrow newcell,R1\rightarrow mo,I2\rightarrow i2,D \rightarrow d\}}^{E_{2}}),(<l6-l24>,$$
    $$E_{2}),(<l28-l35>,E_1)],\sigma_3\cup{\{d,i2=0\}},\emptyset\Big) $$
    \item 
    Après la création d'une cellule
    $$\Big([(<l6-l24>,
    E_{2}),(<l28-l35>,E_1)],\sigma_6\cup{\{d=\xi\}},\{d:i2\}\Big) $$
    \item 
    $$\Big([(<l28-l35>,E_1)],$$
    $$\sigma_7\cup{\{mo=(proc\{\$ \:   R2\}<l7-l23>end,CE_{MO}=\{Newcell\rightarrow newcell,D\rightarrow d\})\}},\{d:i2\}\Big) $$
   \item  
    $$\Big([(<l28>,E_1),(<l29-l35>,E_1)],\sigma_8,\{d:i2\}\Big) $$
    \item  
    $$\Big([(<l7-l23>,\{Newcell\rightarrow newcell,R2\rightarrow o1\}),(<l29-l35>,E_1)],\sigma_8,\{d:i2\}\Big) $$
    \item  
    $$\Big([(<l0-l22>,\{Newcell\rightarrow newcell,R2\rightarrow o1,I3\rightarrow i3,C\rightarrow c\}),(<l29-l35>,E_1)],$$
    $$\sigma_8\cup{\{i3=0,c=n\}},\{d:i2,c:i3\}\Big) $$
    \item  
    $$\Big([(<l29-l35>,E_1)],$$
    $$\sigma_{11}\cup{\{o1=(proc\{\$ \:   R3\}<l11-l21> end,CE_{O1}=\{C\rightarrow c,D\rightarrow d\})\}},\{d:i2,c:i3\}\Big) $$
    \item  
    $$\Big([(<l29>,E_1),(<l30-l35>,E_1)],\sigma_{12},\{d:i2,c:i3\}\Big) $$
    \item  
    $$\Big([(<l7-l23>,\{Newcell\rightarrow newcell,R2\rightarrow o2\}),(<l30-l35>,E_1)],\sigma_{12},\{d:i2,c:i3\}\Big) $$
    \item  
    $$\Big([(<l10-l22>,\{Newcell\rightarrow newcell,R2\rightarrow o2,I3\rightarrow i3_2,C\rightarrow c_2\}),(<l30-l35>,E_1)],$$
    $$\sigma_{12}\cup{\{i3_2=0,c_2=\xi_2\}},\{d:i2,c:i3,c_2:i3_2\}\Big) $$
    \item  
    $$\Big([(<l30-l35>,E_1)],$$
    $$\sigma_{15}\cup{\{o2=(proc\{\$ \:R3\}<l11-l21> end,CE_{O1}=\{C\rightarrow c_2,D\rightarrow d\})\}},$$
    $$\{d:i2,c:i3,c_2:i3_2\}\Big) $$
    \item  
    $$\Big([(<l30>,E_1),(<l31-l35>,E_1)],
    \sigma_{16},\{d:i2,c:i3,c_2:i3_2\}\Big) $$
    \item  
    $$\Big([(<l11-l21>,\{C\rightarrow c,D\rightarrow d,R3\rightarrow ra1 \}),(<l31-l35>,E_1)],$$
    $$\sigma_{16},\{d:i2,c:i3,c_2:i3_2\}\Big) $$
    \item  
    Avant la lecture d'une cellule
    $$\Big([(<l12-l22>,\overbrace{\{C\rightarrow c,D\rightarrow d,R3\rightarrow ra1,I4\rightarrow i4,I5\rightarrow i5,I6\rightarrow i6,I7\rightarrow i7,}^{E_4}$$
    $$\overbrace{I8\rightarrow i8,I9\rightarrow i9 \}}^{E_4}),(<l31-l35>,E_1)],
    \sigma_{16}\cup{\{i4,i5,i6,i7,i8,i9\}},$$
    $$\{d:i2,c:i3,c_2:i3_2\}\Big) $$
    \item  
    Après la lecture d'une cellule
    $$\Big([(<l13-l20>,E_4),(<l31-l35>,E_1)],
    \sigma_{19}\cup{\{i5=0\}},\{d:i2,c:i3,c_2:i3_2\}\Big) $$
    \item  
    Avant l'affectation d'une cellule
    $$\Big([(<l15-l20>,E_4),(<l31-l35>,E_1)],
    \sigma_{20}\cup{\{i6=1,i4=1\}},\{d:i2,c:i3,c_2:i3_2\}\Big) $$
    \item  
    Après l'affectation d'une cellule
    $$\Big([(<l16-l20>,E_4),(<l31-l35>,E_1)],
    \sigma_{21},\{d:i4,c:i3,c_2:i3_2\}\Big) $$
    \item  
    $$\Big([(<l31-l35>,E_1)],
    \sigma_{21}\cup{\{i8=0,i9=1,i7=1,ra1=1\}},\{d:i4,c:i7,c_2:i3_2\}\Big) $$
    \item  
    $$\Big([(<l31>,E_1),(<l32-l35>,E_1)],
    \sigma_{23},\{d:i4,c:i7,c_2:i3_2\}\Big) $$
    \item  
    $$\Big([(<l11-l21>,\{C\rightarrow c_2,D\rightarrow d,R3\rightarrow ra2 \}),(<l32-l35>,E_1)],$$
    $$\sigma_{23},\{d:i4,c:i7,c_2:i3_2\}\Big) $$
    \item  
    $$\Big([(<l12-l20>,\overbrace{\{C\rightarrow c_2,D\rightarrow d,R3\rightarrow ra2,I4\rightarrow i4_2,I5\rightarrow i5_2,I6\rightarrow i6_2,I7\rightarrow i7_2,}^{E_5}$$
    $$\overbrace{I8\rightarrow i8_2,I9\rightarrow i9_2 \}}^{E_5}),(<l32-l35>,E_1)],
    \sigma_{23}\cup{\{i4_2,i5_2,i6_2,i7_2,i8_2,i9_2\}},$$
    $$\{d:i4,c:i7,c_2:i3_2\}\Big) $$
    \item  
    $$\Big([(<l19-l20>,E_5),(<l32-l35>,E_1)],
    \sigma_{26}\cup\{i5_2=1,i6_2=1,i4_2=2,,$$
    $$i7_2=2,i8_2=0,i9_2=2\},\{d:i4_2,c:i7,c_2:i3_2\}\Big) $$
    \item  
    $$\Big([(<l32-l35>,E_1)],
    \sigma_{27}\cup{\{ra_2=2\}},\{d:i4_2,c:i7,c_2:i7_2\}\Big) $$
    \item  
    $$\Big([(<l32>,E_1),(<l33-l35>,E_1)],
    \sigma_{28},\{d:i4_2,c:i7,c_2:i7_2\}\Big) $$
    \item  
    $$\Big([(<l11-l21>,\{C\rightarrow c,D\rightarrow d,R3\rightarrow ra3 \}),(<l33-l35>,E_1)],$$
    $$\sigma_{28},\{d:i4_2,c:i7,c_2:i7_2\}\Big) $$
    \item  
    $$\Big([(<l12-l20>,\overbrace{\{C\rightarrow c,D\rightarrow d,R3\rightarrow ra3,I4\rightarrow i4_3,I5\rightarrow i5_3,I6\rightarrow i6_3,I7\rightarrow i7_3,}^{E_6}$$
    $$\overbrace{I8\rightarrow i8_3,I9\rightarrow i9_3 \}}^{E_6}),(<l33-l35>,E_1)],
    \sigma_{28}\cup{\{i4_3,i5_3,i6_3,i7_3,i8_3,i9_3\}},$$
    $$\{d:i4_2,c:i7,c_2:i7_2\}\Big) $$
    \item  
    $$\Big([(<l19-l20>,E_6),(<l33-l35>,E_1)],$$
    $$\sigma_{31}\cup{\{i5_3=2,i6_3=1,i4_3=3,i7_3=4,i8_3=1,i9_3=3\}},\{d:i4_3,c:i7,c_2:i7_2\}\Big) $$
    \item  
    $$\Big([(<l33-l35>,E_1)],
    \sigma_{32}\cup{\{ra3=4\}},\{d:i4_3,c:i7_3,c_2:i7_2\}\Big) $$
    \item  
    $$\Big([(<l33>,E_1),(<l34-l35>,E_1)],
    \sigma_{33},\{d:i4_3,c:i7_3,c_2:i7_2\}\Big) $$
    \item  
    $$\Big([(<l11-l21>,\{C\rightarrow c_2,D\rightarrow d,R3\rightarrow ra4 \}),(<l34-l35>,E_1)],$$
    $$\sigma_{33},\{d:i4_3,c:i7_3,c_2:i7_2\}\Big) $$
    \item  
    $$\Big([(<l12-l20>,\overbrace{\{C\rightarrow c_2,D\rightarrow d,R3\rightarrow ra4,I4\rightarrow i4_4,I5\rightarrow i5_4,I6\rightarrow i6_4			,I7\rightarrow i7_4,}^{E_7}$$
    $$\overbrace{I8\rightarrow i8_4,I9\rightarrow i9_4 \}}^{E_7}),(<l34-l35>,E_1)],$$
    $$\sigma_{33}\cup{\{i4_4,i5_4,i6_4,i7_4,i8_4,i9_4\}},\{d:i4_3,c:i7_3,c_2:i7_2\}\Big) $$
    \item  
    $$\Big([(<l19-l20>,E_7),(<l34-l35>,E_1)],$$
    $$\sigma_{36}\cup{\{i4_4=4,i5_4=3,i6_4=1,i7_4=6,i8_4=2,i9_4=4\}},\{d:i4_4,c:i7_3,c_2:i7_2\}\Big) $$
    \item  
    $$\Big([(<l34-l35>,E_1)],\sigma_{37}\cup{\{ra4=6\}},\{d:i4_4,c:i7_3,c_2:i7_4\}\Big) $$
    \item 
    $$\Big([(<l35>,E_1)],\sigma_{38}\cup{\{i1='|'(RA1 '|'(RA2 '|'(RA3 '|'(RA4\quad nil))))\}},$$
    $$\{d:i4_4,c:i7_3,c_2:i7_4\}\Big) $$
    \item 
    Affiche $i1=[1\quad 2\quad4\quad6]$
    $$
         \Big([],\sigma_{40}=\{i4_4=4,i5_4=3,i6_4=1,i7_4=6,i8_4=2,i9_4=4,ra4=6,$$
        $$i1='|'(RA1 '|'(RA2 '|'(RA3 '|'(RA4\quad nil)))),ra_2=2,i5_3=2,i6_3=1,i4_3=3,i7_3=4,$$
        $$i8_3=1,i9_3=3,ra3=4,i5=0,i6=1,i4=1,i8=0,i9=1,i7=1,ra1=1,i5_2=1,$$
        $$i6_2=1,i4_2=2    ,i7_2=2,i8_2=0,i9_2=2,d=\xi,$$
        $$mo=(proc\{\$ \: 
        R2\}<l7-l23>end,CE_{MO}=\{Newcell\rightarrow newcell,D\rightarrow d\}),i3=0,$$
        $$c=n
        o1=(proc\{\$ \:   R3\}<l11-l21> end,CE_{O1}=\{C\rightarrow c,D\rightarrow d\}),i3_2=0,c_2=\xi_2,$$
        $$o2=(proc\{\$ \:R3\}<l11-l21> end,CE_{O1}=\{C\rightarrow c_2,D\rightarrow d\}),i2=0,$$
        $$browse=(proc\{\$\: X\}...end,CE_{Browse}),
        newcell=(proc\{\$ \:X\:Y\}...end,CE_{Newcell}),$$
        $$mc=(proc\{\$  \:R1\}...end,Newcell\rightarrow newcell)
        \},\{d:i4_4,c:i7_3,c_2:i7_4\}\Big) 
        $$
    
\end{enumerate}
La pile d'instuctions est vide, le programme est terminé.
\end{description}

\end{solution}

\section{Concepts (5 pts)}

Définissez chacun des concepts suivants avec le plus de précision possible. Pour chaque concept donnez un exemple concret (code ou algorithme) pour bien illustrer le concept. 


 
\begin{itemize} 

	\item La portée statique d'une occurence d'un identificateur

	\item Un flot

	\item La complexité temporelle avec la notation $O$
 
	\item  Non déterminisme 

	\item (Bonus) La règle de sémantique pour l'instruction \textit{local in end}
 
\end{itemize} 

\end{document}
