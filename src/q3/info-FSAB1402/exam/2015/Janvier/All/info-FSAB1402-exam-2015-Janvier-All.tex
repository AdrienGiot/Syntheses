\documentclass[fr]{../../../../../../eplexam}
\usepackage{../../../info-FSAB1402-exam}
\usepackage{../../../../../../eplcode}

\hypertitle[']{Informatique}{3}{FSAB}{1402}{2015}{Janvier}{All}
{Martin Braquet \and Sébastien Colla}
{Peter Van Roy}

\newcommand{\ce}{\mathrm{CE}}

\section{Programmation fonctionnelle (3pt)}

Définissez les fonctions suivantes:

\begin{itemize}
\item \lstinline|{Modes L}| avec $L=[a_0\: a_1 \ldots a_{n-1}]$ renvoie une liste M qui contient un sous-ensemble des

éléments de L dans le même ordre que L. Pour tout $a_i\in L$: si $a_{i-1}<a_i>a_{i+1}$ alors $a_i\in M$. Si $a_{i-1}$ ou $a_{i+1}$ n'existent pas, leur condition correspondante est supposée vraie.

\item \lstinline|{ModesFixe L}| renvoie $L_{k-1}$ dans la séquence $L_0\: L_1 \ldots L_{k-1}$ où $L_0=L$, $L_i=$\lstinline|{Modes Li<1}|et $|L_{k-1}|\leqslant 1$.

\end{itemize}

Définissez \lstinline|Modes| et \lstinline|ModesFixe| en fonctions récursives terminales et déclaratives. Une définition

trop compliquée sera pénalisée. Vous êtes fortement encouragés à définir des fonctions auxiliaires

pour simplifier votre définition. Attention à donner une spécification pour chaque fonction

auxiliaire.

\begin{solution}

\lstinputlisting{Janvier-2015-Q1.oz}

\end{solution}

\section{Sémantique (7 pt)}

Voici un petit programme:

\lstinputlisting{Janvier-2015-Q2.oz}

\begin{enumerate}

	\item Qu'est-ce qui est affiché quand on exécute ce programme ?

	\item Donnez la traduction de ce programme en langage noyau. Attention à donner une traduction complète !

	\item Donnez les environnements contextuels de toutes les procédures dans cette traduction.

	\item Donnez un pas d'exécution de la machine abstraite pour montrer la création d'une cellule.

	\item Donnez un pas d'exécution de la machine abstraite pour montrer l'affectation d'une cellule.

  \item Donnez un pas d'exécution de la machine abstraite pour montrer la définition d'une procédure.

	\item Donnez un pas d'exécution de la machine abstraite pour montrer l'appel d'une procédure.

\end{enumerate}
Attention à ne pas faire plus d'un pas d'exécution dans la machine abstraite pour chacun des quatre
cas demandés.

\begin{solution}

\begin{enumerate}

\item Il affiche 2.

\item \lstinputlisting{Janvier-2015-Q2-sol.oz}

\item $$\ce_{M}=\{ C\rightarrow c\}$$
			$$\ce_{R1}=\{ C\rightarrow c, D \rightarrow c\}$$

\item

Malheureusement, les questions suivantes n’ont pas encore de solution. Vous êtes encouragés à en soumettre une à l’adresse suivante
\begin{center}
https://github.com/gp2mv3/Syntheses
\end{center}
ou par mail.

\end{enumerate}

\end{solution}

\section{Concepts des langages de programmation (5pt)}
Définissez chacun des concepts suivants avec précision et si c'est demandé donnez un exemple pour
illustrer le concept:

\begin{enumerate}
\item Environnement contextuel d'une procédure (avec exemple).
\item Problème NP complet (avec exemple).
\item Polymorphisme (avec exemple).
\item Règle sémantique pour l'instruction local (définition formelle).
\item Ordonnancement des fils et l'équité.
\end{enumerate}

\end{document}
