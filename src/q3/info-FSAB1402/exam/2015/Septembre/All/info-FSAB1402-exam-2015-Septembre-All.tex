\documentclass[fr]{../../../../../../eplexam}
\usepackage{../../../info-FSAB1402-exam}
\usepackage{../../../../../../eplcode}

\hypertitle[']{Informatique}{3}{FSAB}{1402}{2015}{Août}{All}
{Martin Braquet \and Luca Derumier}
{Peter Van Roy}

\newcommand{\ce}{\mathrm{CE}}

\section{Programmation fonctionnelle (5 pt)}

Supposez \textit{L} une liste qui contient \textit{n} entiers distincts dans un ordre quelconque:

\begin{itemize}

\item Définissez la fonction \lstinline|{Bosses L}| qui calcule une liste des "bosses" de \textit{L} avec l'algorithme

suivant.  Supposez que L=[$a_1\: a_2 \ldots a_n$].  Pour tout \textit{i} tel que $1 \leqslant i < n$, notez toutes les paires

 ($a_i$, $a_{i+1}$) telles que $a_i > a_{i+1}$ et renvoyer une liste des indices i de ces paires.  Par exemple,

\lstinline|{Bosses [10 30 20]}| renvoie [2] parce que $30>20$ et 30 a l'index 2.

\item Définissez la fonction \lstinline|{Perm L I}| avec $1 \leqslant I < n$ qui prend la liste L=[$a_1\ldots a_i \: a_{i+1} \ldots a_n$] et

qui renvoie la liste [$a_1\ldots a_{i+1} \: a_i \ldots a_n$] (deux éléments sont échangés).  Par exemple,

\lstinline|{Perm [10 30 20] 2}| renvoie [10 20 30].  Le nombre de bosses dans le résultat est changé de 1.

\item Définissez un algorithme \lstinline|{Tri L}| qui renvoie une permutation de \textit{L} qui est triée les éléments

sont en ordre croissant).  Pour cette question il est demandé d'implémenter \lstinline|Tri| avec  \lstinline|Bosses|

et  \lstinline|{Perm|: si \textit{L} contient une paire ($a_i, a_{i+1}$) telle que $a_i > a_{i+1}$, il faut l'inverser avec l'appel

\lstinline|{Perm L I}|.  Il faut continuer jusqu'à ce qu'il n'y ait plus de paires avec cette propriété.

\end{itemize}

Toutes les fonctions doivent êtres récursives terminales et déclaratives. Une définition trop

compliquée sera pénalisée.  Vous êtes encouragé à définir des fonctions auxiliaires si nécessaires

pour simplifier votre définition.  Attention à donner une spécification pour chaque fonction auxiliaire

 (y compris les fonctions locales).

\begin{solution}

\lstinputlisting{Aout-2015-Q1.oz}

\end{solution}

\section{Sémantique (5 pt)}

Voici un petit programme:

\lstinputlisting{Aout-2015-Q2.oz}

\begin{enumerate}

	\item Qu'est-ce qui est affiché quand on exécute ce programme ?  Attention à bien suivre

l'exécution!  Combien de fois est-ce que la fonction \lstinline|AddMe| est appelée?

	\item Donnez la traduction de ce programme en langage noyau. Attention à donner une traduction complète !

	\item Donnez l'environnement contextuel de la procédure \lstinline|AddMe|.

	\item Donnez un pas d'exécution de la machine abstraite pour montrer la création d'une cellule.

	\item Donnez un pas d'exécution de la machine abstraite pour montrer l'affectation d'une cellule.

  \item Donnez un pas d'exécution de la machine abstraite pour montrer la définition d'une procédure.

	\item Donnez un pas d'exécution de la machine abstraite pour montrer l'appel d'une procédure.

\end{enumerate}
Attention à ne pas faire plus d'un pas d'exécution dans la machine abstraite pour chacun des quatre
cas demandés. Pour chaque pas d'exécution, il faut montrer l'état de la machine avant et après le pas.

\begin{solution}

\begin{enumerate}

\item Il affiche 15. La fonction \lstinline|AddMe| est appelée 6 fois.

\item \lstinputlisting{Aout-2015-Q2-sol.oz}

\item $$\ce_{AddMe}=\{ C\rightarrow c, AddMe\rightarrow addme\}$$

\item

Malheureusement, les questions suivantes n’ont pas encore de solution. Vous êtes encouragés à en soumettre une à l’adresse suivante
\begin{center}
https://github.com/gp2mv3/Syntheses
\end{center}
ou par mail.

\end{enumerate}

\end{solution}

\section{Concepts des langages de programmation (5 pt)}
Définissez chacun des concepts suivants avec précision et si c'est demandé donnez un exemple pour
illustrer le concept:

\begin{enumerate}
\item La portée d'une occurrence d'un identificateur  (avec exemple).
\item La complexité temporelle avec la notation grand O  (définition mathématique).
\item La sémantique des exceptions  (effet des instructions try et raise sur la pile sémantique).
\item Le non-déterminisme  (avec exemple).
\item La loi de Moore.
\end{enumerate}

\end{document}
