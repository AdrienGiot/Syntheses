\documentclass[fr]{../../../../../../eplexam}
%\usepackage{../../../info-FSAB1402-exam}
\usepackage{../../../../../../eplcode}
\usepackage{listings}
\usepackage{enumitem}
%\usepackage{microtype}
%\usepackage[htt]{hyphenat}

\lstset{
    escapeinside={\%*}{*)},
    language=Oz,
    tabsize=8,
    breaklines=true,
    breakatwhitespace=true,
}

\hypertitle{info-FSAB1402}{3}{FSAB}{1402}{2017}{Janvier}
{Martin Braquet\and Jean-Martin Vlaeminck}
{Peter Van Roy}


\section{(5 pt)}

Écrivez une fonction \lstinline|{Incorpore L1 L2}| qui implémente les spécifications suivantes.

Si $L1=[a_0, \ldots, a_{m-1}]$ et $L2=[b_0, \ldots, b_{n-1}]$,
la fonction renvoie \lstinline|true| ssi
$\exists i_0<i_1<\ldots<i_{m-1}$ tels que
$b_{i_k}=a_k \quad \forall 0 \le k \le m-1$.
Autrement dit, la liste $L1$ est \og incorporée \fg{} dans $L2$ :
chaque élément de $L1$ est présent dans $L2$, dans le même ordre. Cette fonction peut être écrite en utilisant une seule fonction auxiliaire.

Écrivez ensuite une fonction \lstinline|{Apex LL}| qui renvoie l'apex d'une liste de liste. Si $LL=[L_0, L_1, \ldots, L_{n-1}]$, l'apex existe s'il existe un unique naturel $q$ tel que $0\le q \le \min(LL)-1$, et $\forall k$ tel que $0\le k \le q$, on a \lstinline[mathescape]|{Incorpore $L_k$ $L_{k-1}$}| et $\forall k$ tel que $q<k\le n-1$, on a \lstinline[mathescape]|{Incorpore $L_{k-1}$ $L_k$}|. S'il existe un unique $q$, alors \lstinline|{Apex LL}| renvoie $L_q$. Sinon, la fonction renvoie \lstinline|false|. On peut interpréter cette définition de la manière suivante : chaque liste, en partant d'une liste au centre (l'apex), est incluse dans la liste adjacente la plus éloignée de ce centre.


Dans votre réponse, il faut définir chaque fonction dont vous avez besoin et chaque fonction récursive
doit faire une récursion terminale. Attention aux détails de syntaxe !
\begin{solution}

\lstinputlisting{Janvier-2017-Q1.oz}

\end{solution}

\section{(5 pt)}
Voici un petit programme:

\lstinputlisting{Janvier-2017-Q2.oz}

\begin{enumerate}[label=(\alph*)]

	\item Qu'est-ce qui est affiché quand on exécute ce programme ?

	\item Donnez la traduction de ce programme en langage noyau. Attention à donner une traduction complète !

	\item Donnez les environnements contextuels de toutes les procédures.

	\item Donnez un pas d'exécution de la machine abstraite pour montrer la création d'une cellule.

	\item Donnez un pas d'exécution de la machine abstraite pour montrer l'affectation d'une cellule.

	\item Donnez un pas d'exécution de la machine abstraite pour montrer la lecture d'une cellule.

	\item Donnez un pas d'exécution de la machine abstraite pour montrer la définition d'une procédure.

	\item Donnez un pas d'exécution de la machine abstraite pour montrer l'appel d'une procédure.

\end{enumerate}

\begin{solution}

\begin{enumerate}[label=(\alph*)]

\item Il affiche 2.

\item \lstinputlisting{Janvier-2017-Q2-sol.oz}

\item
Dans ce qui suit, on suppose que $\texttt{R4} \rightarrow r4$ (et donc, $\texttt{R} \rightarrow r4$ aussi), que $\texttt{F} \rightarrow f$ et que $\texttt{P} \rightarrow p$.
\[ \mbox{CE}_{F}=\{ \texttt{NewCell} \rightarrow newcell \} \]
\[ \mbox{CE}_{R4}=\{ \texttt{C} \rightarrow c, \texttt{A} \rightarrow i8 \} \]
\[ \mbox{CE}_{P}= \emptyset \]

\item
Toute l'exécution de la machine abstraite est effectuée. Les états d'exécution à fournir à l'examen sont signalés ci-dessous. Dans ce qui suit, on définit l'opération d'adjonction de deux ensembles $A$ et $B$, notée $A+B$, de la manière suivante :
\begin{itemize}
	\item Pour tout identificateur $\langle x \rangle \in A$ et $\langle x \rangle \in B$ commun aux deux ensembles, $A+B$ contient uniquement le mapping venant de l'ensemble $B$.
	\item Pour tous les autres identificateurs, ils se retrouvent tels quels dans $A+B$.
\end{itemize}
Il s'agit d'une union avec remplacement des anciennes valeurs.

	\begin{enumerate}[label=\arabic*.]
	%\newcommand{\proc}{\text{proc}}
	\newcommand{\procc}[1]{\text{proc \{\$ #1\}}}
	\newcommand{\End}{\text{end}}
	\item
	\[ \bigg( \big[ (<l1-l29>, E_1=\{\texttt{Browse} \rightarrow browse, \texttt{NewCell} \rightarrow newcell \}) \big],\]
	\[ \sigma_1 = \big\{
		browse=(\procc{X}\ldots \End,CE_{Browse}), \]
	\[	newcell=(\procc{X Y}\ldots \End,CE_{NewCell})
	\big\}, \mu_1 = \emptyset \bigg) \]
	État initial

	\item
	\[ \bigg( \big[(<l2-l28>, E_2 = E_1 + \{ \texttt{F} \rightarrow f, \texttt{P} \rightarrow p, \texttt{R4} \rightarrow r4, \texttt{R5} \rightarrow r5, \texttt{I8} \rightarrow i8 \}) \big],
	\sigma_2 = \sigma_1 + \{ f, p, r4, r5, i8 \}, \emptyset \bigg) \]
	Exécution du \lstinline|local|.

	\item \textit{Avant la définition d'une procédure}
	\[ \bigg( \big[(<l2-l15>, E_2), (<l16-l28>, E_2) \big],
	\sigma_2, \emptyset \bigg) \]
	Composition pour séparer les instructions.

	\item \textit{Après la définition d'une procédure}
	\[ \bigg( \big[(<l16-l28>, E_2) \big],
	\sigma_4 = \sigma_2 + \{
		f=(\procc{A R} <l3-l14> \End, CE_{F})
	\}, \emptyset \bigg) \]
	Création de valeur procédurale : affectation à la variable $f$ de la procédure définie par le code des lignes 3 à 14, et capture de la variable \lstinline|NewCell|.

	\item
	\[ \bigg( \big[(<l25-l28>, E_2) \big],
	\sigma_5 = \sigma_4 = \{
		i8=1, p=(\procc{C R3} <l17-l23> \End, CE_{P}=\emptyset)
	\}, \emptyset \bigg) \]
	Composition de deux instructions (lignes 16 à 24 et lignes 25 à 28), et ensuite création de valeur procédurale : affectation à la variable $p$ de la procédure définie lignes 17 à 23, avec aucune capture (environnement contextuel vide). Il y a donc eu deux instructions !

	\item \textit{Avant l'appel d'une procédure}
	\[ \left(
		\begin{bmatrix}
			\big( \texttt{\{F I8 R4\}}, E_2 = \{ \texttt{F} \rightarrow f, \texttt{I8} \rightarrow i8, \texttt{R4} \rightarrow r4, \dots \} \big), \\
			\big( \texttt{\{R4 P R5\} \{Browse R5\}}, E_2 \big) \\
		\end{bmatrix},
	\sigma_5, \emptyset \right) \]
	Après l'affectation, il y a de nouveau eu une composition (pour séparer la ligne 25 des lignes 26 à 28), puis une création de valeur, avec affectation de $i8$ à $1$, et enfin il y a de nouveau eu une composition, avec séparation en lignes 26 et lignes 27 à 28.

	\item Après l'appel d'une procédure
	\[ \bigg( \Big[
		\big( <l3-l14>, E_3 = \{ \texttt{NewCell} \rightarrow newcell, \texttt{A} \rightarrow i8, \texttt{R} \rightarrow r4 \} \big),
		\big( <l27-l28>, E_2 \big) \Big],
	\sigma_5, \emptyset \bigg) \]
	Appel de procédure : les instructions correspondantes, lignes 3 à 14, sont poussées sur la pile. L'environnement $E_3$ est constitué de la manière suivante:
	\begin{itemize}
		\item L'argument \texttt{A} de $f$ correspond à l'identificateur $\texttt{I8} \rightarrow i8=1$ lors de l'appel, donc on ajoute $\texttt{A}\rightarrow i8$ à $E_3$.
		\item L'argument \texttt{R} de $f$ correspond à $\texttt{R4} \rightarrow r4$, donc on ajoute $\texttt{R}\rightarrow r4$.
		\item L'environnement contextuel contenant $\texttt{NewCell} \rightarrow newcell$, on l'ajoute à $E_3$.
		\item Les autres identificateurs de l'environnement $E_2$ ne sont plus utiles pour cette exécution.
	\end{itemize}

	\item \textit{Avant la création d'une cellule}
	\[ \left(
		\begin{bmatrix}
			\big( <l5>, E_4 = E_3 + \{ \texttt{C} \rightarrow c, \texttt{I1} \rightarrow i1 \} \big), \\
			\big( <l6-l13>, E_4 \big), \\
			\big( <l27-l28>, E_2 \big) \\
		\end{bmatrix},
	\sigma_8 = \sigma_5 + \{ i1=0 \}, \emptyset \right) \]
	Pour en arriver là : exécution du \lstinline|local| ligne 3, composition entre les lignes 4 et les lignes 5 à 13, création de valeur pour la variable $i1$, et enfin composition entre la ligne 5 et les lignes 6 à 13 (qui ne sont qu'une instruction chacune).

	\item Après la création d'une cellule
	\[ \bigg( \big[(<l6-l13>, E_4), (<l27-l28>, E_2) \big],
	\sigma_9 = \sigma_8 + \{ c=\xi \}, \mu_2 = \{ c:i1 \} \bigg) \]
	Précision : il s'agit bien de la variable $c$ qui est utilisée dans la mémoire à affectation multiple, et non $\xi$.

	\item
	\[ \left( \begin{bmatrix}
	(<l27>, E_2), \\
	(<l28>, E_2) \\
	\end{bmatrix},
	\sigma_{10} = \sigma_9 + \{ r4=(\procc{B R2} <l7-l12> \End, CE_{R4}) \}, \mu_2 \right) \]
	Création de la valeur procédurale des lignes 7 à 12 dans $r4$, avec capture de l'identificateur \texttt{C}, qui réfère à la variable $c$, dans l'environnement contextuel $CE_{R4}$. On ajoute également une étape de composition d'instruction.

	\item
	\[ \bigg( \big[(<l7-l12>, E_5 = \{ \texttt{A} \rightarrow i8, \texttt{C} \rightarrow c, \texttt{B} \rightarrow p, \texttt{R2} \rightarrow r5 \}), (<l28>, E_2) \big],
	\sigma_{10}, \mu_2 \bigg) \]
	Appel de la procédure $r4$ : l'environnement $E_5$ est obtenu en prenant l'environnement contextuel ($\texttt{C}\rightarrow c$ et $\texttt{A} \rightarrow i8$), l'argument \texttt{B} qui correspond à l'identificateur \texttt{P} référant à $p$, et l'argument \texttt{R2} qui correspond à l'identificateur \texttt{R5} référant à $r5$.

	\item
	\[ \left( \begin{bmatrix}
	(\texttt{\{B C I2\}}, E_6 = E_5 + \{ \texttt{I2} \rightarrow i2, \texttt{I3} \rightarrow i3, \texttt{I4} \rightarrow i4 \}), \\
	(<l9-l11>, E_6), \\
	(<l28>, E_2) \\
	\end{bmatrix},
	\sigma_{12} + \{ i2, i3, i4 \}, \mu_2 \right) \]
	Exécution du local, et composition.

	\item
	\[ \bigg( \big[(<l17-l23>, E_7 = \{ \texttt{C}\rightarrow c, \texttt{R3}\rightarrow i2 \}), (<l9-l11>, E_6), (<l28>, E_2) \big],
	\sigma_{12}, \mu_2 \bigg) \]
	Appel de la procédure $p$ (référée par \texttt{B}).

	\item \textit{Avant la lecture d'une cellule}
	\[ \left( \begin{bmatrix}
	(<l18>, E_8 = E_7 + \{\texttt{I5}\rightarrow i5, \texttt{I6}\rightarrow i6, \texttt{I7}\rightarrow i7 \}), \\
	(<l19-l22>, E_8), \\
	(<l9-l11>, E_6), \\
	(<l28>, E_2) \\
	\end{bmatrix},
	\sigma_{14} = \sigma_{12}+\{ i5, i6, i7, i1=0 \}, \{c:i1\} \right) \]
	Exécution du local, composition.

	\item \textit{Après la lecture d'une cellule}
	\[ \bigg( \big[(<l19-l22>, E_8), (<l9-l11>, E_6), (<l28>, E_2) \big],
	\sigma_{15} = \sigma_{14}+\{ i5=0 \}, \{c:i1\} \bigg) \]
	Les variables $i1$ et $i5$ sont unifiées lors de la lecture : elles prennent donc toutes les deux la valeur $0$.

	\item \textit{Avant l'affectation d'une cellule}
	\[ \bigg( \big[(<l21>, E_8), (<l22>, E_8), (<l9-l11>, E_6), (<l28>, E_2) \big],
	\sigma_{16} = \sigma_{15} + \{ i6=1, i7=1 \}, \{c:i1\} \bigg) \]
	Pour arriver là : composition, ligne 19, composition, ligne 20, et encore composition.

	\item \textit{Après l'affectation d'une cellule}
	\[ \bigg( \big[(<l22>, E_8), (<l9-l11>, E_6), (<l28>, E_2) \big],
	\sigma_{16}, \mu_3=\{c:i7\} \bigg) \]
	Le contenu de la cellule est remplacé. Attention, la cellule stocke bien une variable, et pas une valeur.

	\item
	\[ \bigg( \big[(<l9-l11>, E_6), (<l28>, E_2) \big],
	\sigma_{18} = \sigma_{16} + \{ i2=1 \}, \mu_3 \bigg) \]
	Nouvelle lecture de cellule qui affecte la variable $i2$.

	\item
	\[ \bigg( \big[(<l10>, E_6), (<l11>, E_6), (<l28>, E_2) \big],
	\sigma_{19}=\sigma_{18}+\{ i4=0 \}, \mu_3 \bigg) \]
	Composition. Calcul de $i4$. Composition.

	\item
	\[ \bigg( \big[(<l17-l23>, E_9 = \{\texttt{C}\rightarrow c, \texttt{R3}\rightarrow i3 \}), (<l11>, E_6), (<l28>, E_2) \big],
	\sigma_{19}, \mu_3 = \{c:i7\} \bigg) \]
	Appel de la procédure $p$, mais avec de nouveaux arguments, et donc un nouvel environnement.

	%\item
	%\[ \bigg( \big[(<l21-l22>, \{C\rightarrow c,R3\rightarrow i3,I5\rightarrow i5_b,I6\rightarrow i6_b,I7\rightarrow i7_b \}),
	%(<l11>,E_6),(<l28>,E_2) \big],
	%\sigma_{21}+\{ i5_b=1,i6_b=1,i7_b=2 \}, \{c:i7\} \bigg) \]

	\item
	\[ \bigg( \big[(<l11>, E_6), (<l28>, E_2) \big],
	\sigma_{21} = \sigma_{19}+\{ i3=2, i11=1, i12=1, i13=2 \}, \mu_4=\{c:i13\} \bigg) \]
	On saute toute l'exécution de cette procédure, qui crée de nouvelles variables temporaires en mémoire $i11$ (pour \texttt{I5}), $i12$ (pour \texttt{I6}), $i13$ (pour \texttt{I7}). Au terme de l'exécution, le contenu de la cellule est une variable valant $2$.

	\item
	\[ \bigg( \big[(<l28>, E_2) \big],
	\sigma_{22} = \sigma_{21}+\{ r5=2 \}, \mu_4 \bigg) \]
	La ligne 11 est enfin exécutée et donne $\texttt{R2}\rightarrow r5=0+2=2$.

	\item Affiche $R5\rightarrow r5=2$
	\[ \bigg( \big[ \; \big], \sigma_{22}=\{ i13=2, c=\xi, \dots \}, \mu_4=\{ c:i13 \} \bigg) \]
	Le programme se termine en affichant $2$ dans le Browser.
	\end{enumerate}

La pile d'instructions est vide, le programme est terminé.

\end{enumerate}

\end{solution}

\section{(5 pt)}
Définissez les concepts suivants avec précision. Pour chaque concept sauf NP-complet, donnez un fragment de code pour illustrer.

\begin{itemize}
\item Non-déterminisme
\item Problème NP-complet
\item Modularité
\item Polymorphisme
\item Portée statique d'un identificateur
\end{itemize}

\begin{solution}
Une grande partie des concepts demandés aux examens se trouve dans les définitions des synthèses sur le drive EPL.
\end{solution}

\end{document}
