\documentclass[fr]{../../../eplexercises}

\usepackage{../../../eplcode}

\DeclareMathOperator{\pgcd}{PGCD}

\hypertitle[']{Informatique}{3}{FSAB}{1402}
{Philippe Verbist}
{Peter Van Roy}

\lstset{language={Oz},morekeywords={for,do}}

\newcommand{\st}{\mathrm{ST}}
\newcommand{\ce}{\mathrm{CE}}
\newcommand{\mozart}{Mozart}





\newpage

\skipape
\skipape
\part{Correction des exercices d'APE}
\section{Programmation récursive avec des listes}

\subsection{Le minimum syndical}
\lstinputlisting{3-1.oz}

\subsection{Décomposition}
\lstinputlisting{3-2.oz}

\subsection{Quelle est la plus longue?}
\lstinputlisting{3-3.oz}
Invariant?

\subsection{Concaténer deux listes}
\lstinputlisting{3-4.oz}

\subsection{Pattern matching}
\lstinputlisting{3-5.oz}

\subsection{Sous-séquences dans des listes}
\lstinputlisting{3-6.oz}

\subsection{Multiplions}
\lstinputlisting{3-7.oz}

\subsection{Nombres factoriels}
\lstinputlisting{3-8.oz}

\subsection{Ecriture des listes}
\lstinputlisting{3-9.oz}

\subsection{Traiter une liste}
\lstinputlisting{3-10.oz}


\subsection{Exo sup 1 : Mettons les listes à plat}
\lstinputlisting{3-sup-1.oz}

\subsection{Exo sup 2 : Occurrences d'une sous-liste}
\lstinputlisting{3-sup-2.oz}

\subsection{Exo sup 3 : Représentation binaire d'entiers avec des listes}
\lstinputlisting{3-sup-3.oz}










\newpage
\section{Higher-order programming \& Enregistrements}

\subsection{Enregistrements}
\lstinputlisting{4-1.oz}

\subsection{Listes, tuples, enregistrements}
\lstinputlisting{4-2.oz}

\subsection{Zig zag folding}
\lstinputlisting{4-3.oz}

\subsection{Fibonacci generator}
\lstinputlisting{4-4.oz}











\newpage
\section{Arbres}

\subsection{Promenade arboricole}
\lstinputlisting{5-1.oz}

\subsection{Arbre et programmation d'ordre supérieur}
\lstinputlisting{5-2.oz}

\subsection{Trier une liste avec des arbres}
\lstinputlisting{5-3(a).oz}

\subsection{Construction d'un arbre équilibré}
\lstinputlisting{5-4(a).oz}





\skipape

\newpage
\section{Programmer avec l'état}

\subsection{Accumulateurs et état}
\lstinputlisting{8-1.oz}

\subsection{Une calculatrice}
\lstinputlisting{8-2.oz}

\subsection{Encapsulation de l'état de la pile}
\lstinputlisting{8-3.oz}

\subsection{Mélanger une liste}
\lstinputlisting{8-4.oz}

\subsection{Exo sup 1 : Cellules et listes}
\lstinputlisting{8-sup-1.oz}


\newpage
\section{Programmation orientée objet}
\subsection{Collections}
\lstinputlisting{8bis-1.oz}

\subsection{Des objets pour représenter des expressions}
\lstinputlisting{8bis-2.oz}

\subsection{Elu par cette crapule}
\lstinputlisting{8bis-3.oz}


\newpage
\section{Programmation concurrente et systèmes multi-agents}
\subsection{Expliquez...}
\lstinputlisting{9-1.oz}

Les fils d'exécution sont créés les uns après les autres (comme écrit dans le code source). Cependant, leur exécution ne suit par nécessairement cet ordre.
Ici, les \textit{threads} terminent dans l'ordre: 3 - 4 - 2 - 1. 


\subsection{Analysez...}
\lstinputlisting{9-2(a).oz}
X = 1, Y = 2, Z = 2
\lstinputlisting{9-2(b).oz}
X = 2, Y = 2, Z = 2

\subsection{Producteur/Consommateur}
\lstinputlisting{9-3.oz}
La première version prend environ deux fois moins de temps pour s'exécuter.


\subsection{Filtres}
\lstinputlisting{9-4(a).oz}
\lstinputlisting{9-4(b).oz}

\subsection{Tracking Information}
\lstinputlisting{9-5.oz}



\newpage
\section{Complexité calculatoire}
\subsection{Vrai ou faux?}
\lstinputlisting{10-1.oz}







\end{document}
