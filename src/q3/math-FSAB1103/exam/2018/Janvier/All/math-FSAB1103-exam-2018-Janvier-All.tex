\documentclass[fr]{../../../../../../eplexam}
\usepackage{amsmath,amssymb,amsthm}

\hypertitle{Mathématiques}{3}{FSAB}{1103}{2018}{Janvier}
{Martin Braquet \and Guillaume Prieur \and Maxime Clément}
{Jean-François Remacle, Grégoire Winckelmans et Roland Keunings}

\section{EDP-Q1}
On donne l'équation de transport sous forme conservative :
\[
    \frac{\partial}{\partial x}(cu) + \frac{\partial u}{\partial t} = S
\]
avec $$c = c(x) = \frac{c_0}{\cosh{\left(\frac{x}{L}\right)}}$$ où $c_0$ et $L$ sont des constantes.\\
Les conditions sont :
\begin{itemize} 
    \item Région A : $u(s,0) = Ue^{-s/L_0}$ pour $s \geq 0$, où $U$ et $L_0$ sont des constantes.
    \item Région B : $u(0,\tau) = U$ pour $\tau \geq 0$.
\end{itemize}
Obtenez les équations des caractéristiques pour les régions $A$ et $B$ du quart de plan (i.e. relation entre $s$, $x$ et $x$). Faire une esquisse (propre avec des axes chiffrés !) des courbes caractéristiques qui émanent de $\frac{s}{L} = 0$, $\frac{s}{L} = 1$ et de $\frac{c_0\tau}{L} = 1$. Les deux régions doivent être clairement reconnaissables.\\
On demande ensuite d'exprimer l'EDO à résoudre pour trouver l'évolution de $cu$ le long de chaque caractéristique pour le cas $S$ général, puis de trouver la solution  $u(x,y)$ dans la région $A$ pour le cas particulier $S = c_0\frac{U}{L}$.\\

Aide: $\cosh(0)=1$, $ \cosh(1)\simeq1.54$, $\cosh(2)\simeq3.76$, $\int\cosh(a\eta)\mathrm{d}\eta=\frac{1}{a}\sinh(a\eta)$ et $\int\frac{\mathrm{d}\eta}{\cosh(a\eta)}=\frac{2}{a}\arctan(e^{a\eta})$

\nosolution

\section{EDP-Q2}
On demande de résoudre
\[
    \nabla^2 u = \frac{\partial^2 u}{\partial x^2}+\frac{\partial^2 u}{\partial y^2} = 0
\]
 dans le rectangle tel que $0\leq x\leq L$ et $0\leq y \leq H$. \\
 
 Les conditions limites sont: 
 \[
 u(0,y) = 0 ,\qquad u(L,y) = U\frac{y}{H},\qquad  u(x,0) = 0\qquad \mathrm{et} \qquad\frac{\partial u}{\partial y}(x,H) =\frac{U}{H} f\left(\frac{x}{L}\right).
 \]
 \begin{enumerate}
    \item
  Il est demandé de résoudre cette équation pour le cas $f\left(\frac{x}{L}\right)$ général, en utilisant \textbf{une seule fois la séparation de variables}. (\textbf{Aide:} utiliser le mode 0, $u_0(x,y) = U\frac{x}{L}\frac{y}{H}$, et une superposition). Veillez à écrire clairement les intégrales à devoir finalement effectuer pour obtenir les coefficients du développement en série de la solution.
    \item Résoudre l'équation pour le cas particulier $f(\frac{x}{L}) = \sin(2\pi \frac{x}{L})+\frac{x}{L}$.
\end{enumerate}


\begin{solution}

Le rectangle est représenté comme suit
\begin{center}
    \begin{tikzpicture}
      \draw[->] (0,0) -- (3,0) node[right] {$x$};
      \draw[->] (0,0) -- (0,2.25) node[above] {$y$};
      \draw[-] (2.25,0) -- (2.25,1.5);
      \draw[-] (0,1.5) -- (2.25,1.5);
      \put(-12.5, 20){$1$}
      \put(30, 47.5){$2$}
      \put(70, 20){$3$}
      \put(30, -12.5){$4$}
      \put(60, -12.5){$L$}
      \put(-12.5, 40){$H$}
    \end{tikzpicture}
\end{center}
Les conditions sur les bords sont
\begin{enumerate}
    \item $u(0,y) = 0$
    \item $\frac{\partial u }{\partial y}(x,H) =\frac{U}{H} f\left(\frac{x}{L}\right)$
    \item $u(L,y) = U\frac{y}{H}\frac{x}{L}$
    \item $u(x,0) = 0$
\end{enumerate}

Par séparation de variable, on écrit $u$ sous la forme
\[u(x,y) = X(x)Y(y)\]
On l'introduit dans l'EDP et on obtient
\begin{align*}
    & X'' + Y'' = 0\\
    & \frac{X''}{X} = -\frac{Y''}{Y} = \lambda
\end{align*}
On résout d'abord le mode $\lambda=0$ pour obtenir une solution particulière que nous noterons $u_0(x,y)$.
On obtient l'équation pour $X$
$$X'' = 0$$
qui a pour solution
$$X(x) = Ax+B$$
Par la condition sur 1),
$$X(0) = 0 \Rightarrow B = 0$$
Par la condition sur 3),
$$X(L) = 0 \Rightarrow A = U\frac{y}{H}\frac{1}{L}$$
On a pour solution particulière
$$u_0(x,y) = U\frac{y}{H}\frac{x}{L}$$
Notons maintenant notre solution générale tel que 
$$u(x,y) = u_g(x,y) + u_0(x,y)$$
On résout maintenant le mode $\lambda \neq 0$ pour obtenir cette solution générale.
Au vu des conditions homogènes en $y$, on choisit $\lambda=-k^2$.
On obtient l'équation pour $X$
$$X'' = -k^2X$$
qui a pour solution
$$X(x) = Acos(kx) + Bsin(kx)$$
Redéfinissons la condition 1) pour notre cas général
$$u_g(0,y) = u(0,y) - u_0(0,y) = 0$$
Donc nous avons
$$X(0) = 0 \Rightarrow A = 0$$
Redéfinissons maintenant la condition 3) pour notre cas général
$$u_g(L,y) = u(L,y) - u_0(L,y) = 0$$
Donc nous avons
$$X(L) = 0 \Rightarrow Bsin(kL) = 0 \Rightarrow k_n = \frac{n\pi}{L} \qquad n = 1,2,3, ...$$
On obtient l'équation pour $Y$
$$Y'' = k^2Y$$
qui a pour solution
$$Y(y) = Ccosh(ky) + Dsinh(ky)$$
Par la condition sur 4),
$$Y(0) = 0 \Rightarrow C = 0$$
Notre solution générale est donc donnée par 
$$(u_g)_n(x,y) = \Tilde{B}_nsin(k_nx)sinh(k_ny)$$
avec $\Tilde{B}_n = B_nC$.
Par superposition des solutions, on a 
$$u_g(x,y) = \sum_{n=0}^\infty \Tilde{B}_nsin(k_nx)sinh(k_ny)$$
On peut donc écrire la solution de notre équation en terme de $\Tilde{B}_n$
\begin{align*}
    u(x,y) & = u_0(x,y) + \sum_{n=0}^\infty \Tilde{B}_nsin(k_nx)sinh(k_ny)\\
    & = U\frac{y}{H}\frac{x}{L} + \sum_{n=0}^\infty \Tilde{B}_nsin(k_nx)sinh(k_ny)\\
\end{align*}
Il reste la dernière condition sur le bord 4) pour déterminer $\Tilde{B}_n$.
\begin{align*}
    \frac{\partial u }{\partial y}(x,H) & =\frac{U}{H} f\left(\frac{x}{L}\right)\\
    \sum_{n=0}^\infty \Tilde{E}_nsin(k_nx) & = \frac{U}{H}\left[ f\left(\frac{x}{L}\right) - \frac{x}{L}\right]
\end{align*}
où $\Tilde{E}_n = \Tilde{B}_nk_ncosh(k_nH)$.
Par othogonalité des fonctions propres, on a
$$\sum_{n=0}^\infty \int^L_0sin(k_nx)sin(k_mx) dx = \frac{L}{2}$$
Par conséquent,
$$\Tilde{E}_n = \frac{2}{L}\frac{U}{H} \int^L_0 \left[ f\left(\frac{x}{L}\right) - \frac{x}{L}\right]sin(k_nx) dx$$
La solution finale est donc 
$$u(x,y) = \frac{U}{HL}\left[xy + \sum_{n=0}^\infty 2\frac{L}{n\pi}\frac{\sinh(\frac{n\pi}{L}y)}{\cosh(\frac{n\pi}{L}H)}\sin(\frac{n\pi}{L}x)\int^L_0\left(f\left(\frac{X}{L}\right) - \frac{X}{L}\right)\sin(\frac{n\pi}{L}x)dx \right]$$

\end{solution}

\section{COMPLEXE-Q1}

Soit la fonction complexe
\[
f(z) = g(z) e^{1/z}
\]
avec $g(z)$ une fonction entière.

\begin{enumerate}
    \item Quel est le résidu de la fonction en $z=0$ ?
    \item Donnez les points singuliers de la fonction $f$ et précisez de quel type il s'agit.
\end{enumerate}

\begin{solution}

\begin{enumerate}
\item
Développons la fonction $f(z)$ en série de Laurent
\begin{align*}
    e^Z & = 1 + Z + \frac{Z^2}{2} + \frac{Z^3}{6} + ...\\
    e^{1/z} & = 1 + \frac{1}{z} + \frac{1}{2z^2} + \frac{1}{6z^3} + ...\\
\end{align*}
La fonction $g(z)$ étant entière, nous pouvons, elle-aussi, la développer en séries autour de $z=0$:
\begin{equation*}
    g(z) = g(0) + g'(0)z + g''(0)\frac{z^2}{2} + ...
\end{equation*}
Il suffit à présent de multiplier les deux séries et d'en extraire le coefficient de $\frac{1}{z}$: 
\begin{align*}
    f(z) &= g(z) e^{1/z}\\
    &= (g(0) + g'(0)z + g''(0)\frac{z^2}{2} + ...)(1 + \frac{1}{z} + \frac{1}{2z^2} + \frac{1}{6z^3} + ...)\\
    &= g(0) + g'(0)+ \frac{g''(0)}{4}+...+ (\frac{g(0)}{1.1}+\frac{g'(0)}{1.2}+\frac{g''(0)}{2.6}+...)\frac{1}{z}+...
\end{align*}
D'où on détermine le résidu:
\begin{equation*}
    Res(f,0) = \sum_{n=0}^{\infty}\frac{g^{(n)}(0)}{n!(n+1)!}
\end{equation*}

\item
La fonction g(z) étant entière, le problème se ramène au calcul des points singuliers de la fonction $h(z) = e^{1/z}$. Le seul point singulier est $z=0$. C'est un point singulier essentiel. Pour plus de détails, voir examen août 2017.

\end{enumerate}

\end{solution}

\section{COMPLEXE-Q2}

Résoudre 
\[
\int_0 ^{+\infty} \frac{x^{1/4}}{(x+1)(x+2)} \mathrm{d}x
\]
Veillez à bien justifier toutes les étapes de calcul.
Tous les lemmes de Jordan sont donnés.

\begin{solution}

Pour résoudre cette question, nous considérons la fonction
\begin{equation*}
    f(z) = \frac{z^{1/4}}{(z+1)(z+2)}
\end{equation*}
que nous intégrons sur un contour fermé en keyhole dans le sens anti-horloger. La fonction est multiforme, il faut donc effectuer une coupure avant de pouvoir intégrer. L'unique point de branchement est $z=0$. On décide, par conséquent, de faire une coupure telle que $0\leq arg(\theta)<2\pi$. On observe aisément qu'il y a 2 pôles d'ordre 1 dans le contour; un en $z=-1$ et un en $z=-2$. Nous pouvons d'ores et déjà calculer leur résidu:
\begin{align*}
    Res(f,-1) &= \frac{z^{1/4}}{z+2}\bigg\rvert_{-1}=e^{i\frac{\pi}{4}}=\frac{\sqrt{2}}{2}(1+i)\\
    Res(f, -2) &=\frac{z^{1/4}}{z+1}\bigg\rvert_{-2}= -2^{1/4}e^{i\frac{\pi}{4}}=-2^{-1/4}(1+i)
\end{align*}
Par le théorème des résidus on sait donc que: 
\begin{equation}
    \oint_C f(z) dz = 2i\pi(2^{-1/2}-2^{-1/4})(1+i)=\pi(2^{1/2}-2^{3/4})(i-1)
\end{equation}
Il reste à évaluer l'intégrale sur le keyhole. On peut la décomposer comme suit:
\begin{align*}
    \oint_C f(z)dz =&  \underbrace{\int_\epsilon^R f(z)dz}_\text{I1} +\underbrace{\int_0^{2\pi-\delta}f(Re^{i\theta})Rie^{i\theta} d\theta}_\text{I2}+\underbrace{\int_R^\epsilon f(re^{(2\pi-\delta)i})e^{(2\pi-\delta)i} dr}_\text{I3}\\
    &+ \underbrace{\int_{2\pi-\delta}^0 f(\epsilon e^{i\theta})\epsilon ie^{i\theta} d\theta}_\text{I4} 
\end{align*}

\end{solution}

\end{document}
