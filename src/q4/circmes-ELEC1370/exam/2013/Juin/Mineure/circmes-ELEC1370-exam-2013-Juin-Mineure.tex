\documentclass[en]{../../../../../../eplexam}
\usepackage{../../../../../../eplunits}
\usepackage{circuitikz}
\usepackage{pgfplots}
\usepackage{enumitem}
\pgfplotsset{compat=newest}
\tikzset{meter/.style={draw,thick,circle,fill=white,minimum size =0.75cm,inner sep=0pt}}

\hypertitle{circmes-ELEC1370}{4}{ELEC}{1370}{2013}{Juin}
{Nicolas Verbeek\and Adrien Couplet\and Martin Van Essche\and Guillaume Gilson\and Guillaume Colinet}
{Claude Oestges, Bruno Dehez and Christophe Craeye}

\section{Question Oestges : phaseurs}
Soit le circuit suivant opérant à $\SI{20}{\kilo\hertz}$ avec $V_o=2.46 \angle\ang{126.87}$ V,
\begin{center}
  	\begin{circuitikz} 
  		\draw
		(0,2.5) to[american controlled current source,l=$2 V_x$] (0,0) 
		(2.5,2.5) -- (0,2.5)
		(5,2.5) to [american voltage source,l_=$12\angle\ang{0}$ V] (2.5,2.5) 
		(2.5,0) to[R,l=$\SI{1}{\ohm}$,-*] (2.5,2.5)
		(5,2.5) to[C,l=-j$\SI{1}{\ohm}$,i=$I_C$,v=$V_x$] (5,0)
		(5,2.5) to[european resistor,*-,l=j$X$] (7.5,2.5) to [R,l=$\SI{1}{\ohm}$,v=$V_o$] (7.5,0) -- (5,0)
		(7.5,2.5) to [short,*-o,l=$B$] (8.5,2.5)
		(7.5,0) to [short,*-o,l=$A$] (8.5,0)
		(0,0) -- (5,0);
	\end{circuitikz}
\end{center}
  
On demande de calculer
\begin{enumerate}
    \item Le courant $I_C$ (amplitude et phase) dans la capacité
    \item La valeur de l’impédance $jX$ et la valeur de la capacité ou inductance correspondante.
    \item La tension (amplitude et phase) aux bornes de la source commandée de courant, ainsi que sa valeur efficace.
    \item La puissance active délivrée par la source de tension.
    \item Le dipôle équivalent de Thévenin aux bornes $A-B$
	\item L'impédance de charge qui maximiserait le transfert de puissance (à la charge)
\end{enumerate}

\begin{solution}
\begin{enumerate}
    \item $I_C = 3.48\angle\ang{-98.13}$ A
    \item $X=\SI{1}{\ohm}$
    \item $V_s = 8.57 \angle\ang{-176.7}$ V et $V_\text{eff} = 6.05$ V
    \item $P = \SI{10}{\watt}$
    \item $V_\text{th} = V_o$ et $Z_\text{th} = 0.615 \angle 36.87^\circ$
    \item $Z_\text{L} = Z_\text{th}^{*}$
\end{enumerate}
\end{solution}

\section{Question Oestges : Bode et quadripôles}
On considère le circuit suivant
\begin{center}
	\begin{circuitikz} 
		\draw
  		(2.5,2.5) to [european resistor,l^=$Z_1$,i=$I_i$]  (5,2.5)
 		(5,2.5) to[european resistor,l=$Z_2$] (5,0)
 		(7.5,2.5) to[european resistor,-*,l_=$Z_3$,i=$I_o$] (5,2.5)
  		(7,2.5) to [short,-o] (8,2.5)
 		(5,0) to [short,-o] (8,0)
 		(5,0) to [short,-o] (2,0)
 		(2.5,2.5) to [short,-o] (2,2.5)
 		(2,2.5) to [open,v=$V_i$] (2,0)
		(8,2.5) to [open,v^=$V_o$] (8,0);
	\end{circuitikz}
\end{center}
Le quadripôle de ce circuit est représenté par la matrice $Z$ suivante à la fréquence de $\SI{26.5}{\kilo\hertz}$
\[ \begin{bmatrix} 6-2j & 4-6j \\ 4-6j & 7+2j \end{bmatrix} \]
On demande de
\begin{enumerate}
    \item Déterminer l’expression analytique (en fonction de $Z_1$, $Z_2$ et $Z_3$) de la matrice $Z$ du quadripôle représenté.
    \item Calculer les valeurs complexes de $Z_1$, $Z_2$ et $Z_3$.
    \item Sachant que $Z_1$ et $Z_3$ sont des impédances de type R-L série ($R_1$, $L_1$ et $R_3$, $L_3$) et que $Z_2$ est une impédance de type R-C série ($R_2$, $C_2$) calculer les différents composants.
    \item Réécrire la matrice $Z$ du quadripôle en fonction des différent composants trouvés ci-dessus ainsi que de la pulsation $\omega$.
    \item Tracer le diagramme de Bode du gain $A_\text{vf}$ pour les valeurs calculées des éléments si on branche en sortie du quadripôle une résistance très grande (supposée infinie). De quel type de filtre s’agit-il ?
    \item Calculer l'impédance d’entrée du circuit si on branche en sortie du quadripôle une résistance de $\SI{5}{\ohm}$.
\end{enumerate}

\begin{solution}
\begin{enumerate}
    \item La matrice $Z$ est la suivante
    \[ [Z] = \begin{bmatrix} Z_1 + Z_2 & Z_2 \\ Z_2 & Z_2 + Z_3 \end{bmatrix} \]
    \item $Z_1 = 2+4j$, $Z_2 = 4-6j$ et $Z_3 = 3+8j$
    \item $R_1 = \SI{2}{\ohm}$, $L_1=\SI{24}{\micro\henry}$, $R_3 = \SI{3}{\ohm}$, $L_3 = \SI{48}{\micro\henry}$, $R_3 = \SI{4}{\ohm}$, $C_2 = \SI{1}{\micro\farad}$.
    \item La matrice $Z$ est la suivante
	\[ [Z] = \begin{bmatrix}R_1+R_2+j\omega L_1 + \frac{1}{j\omega C_2} & R_2 + \frac{1}{j\omega C_2} \\ R_2 + \frac{1}{j\omega C_2} & R_2+R_3+j\omega L_3 + \frac{1}{j\omega C_2} \end{bmatrix} \]
    \item Le gain s'écrit
    \[ A_\text{vf} = \frac{j\omega C_2 R_2 + 1}{(j\omega)^2 L_1 C_2 + j\omega C_2 (R_1+R_2) +1} \]
    C'est un filtre passe-bas.
    \item $Z_\text{in} = 8.43 \angle\ang{11.09}$
\end{enumerate}
\end{solution}

\end{document}
