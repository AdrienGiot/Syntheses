\section{Algorithme du simplexe}

\subsection{Algorithme}

Soit un problème d'optimisation linéaire sous la forme
\[
\renewcommand{\arraystretch}{1.5}
\begin{array}{c@{\quad}rcr@{\qquad}l}
	\max\limits_{x \in \Rn} & c^T x &     &      &                       \\
	\textnormal{s.t.}       &   A x & =   & b\,, & A \in \R^{m \times n} \\
	                        &     x & \ge & 0\,. &                       \\
\end{array}
\]

Reconstituons l'algorithme.

\begin{enumerate}
	\item \textbf{Recherche d'un sommet.}

	On a $m$ variables de base (dans $\xb$)
	et $n-m$ variables hors base (dans $\xn$).
	On a un sommet en $\begin{bmatrix} \xb & \xn \end{bmatrix}^T$,
	à condition que $\xn = 0$ et $\xb = B^{-1} b \ge 0$,
	avec $A = \begin{bmatrix} B & N \end{bmatrix}$.
	Cela veut donc dire que la matrice $B$ doit être inversible.
	\item \textbf{Recherche d'un sommet adjacent meilleur.}
	\begin{enumerate}
		\item \textbf{Regarder les sommets adjacents.}

		Un sommet adjacent est un sommet ayant
		une variable de différence dans la base $B$.
		\item Éventuellement, \textbf{choisir un sommet meilleur.}
	\end{enumerate}
	\item \textbf{Arrêt.}

	Si le programme s'arrête, on est à l'optimum.
\end{enumerate}

% TODO : Would've liked to include this but it seems to be a lot more complicated than it looks :(.
%
%\begin{algorithm}
%	\caption{Finding the vertex of the optimum solution to a linear programming problem}
%	\begin{algorithmic}[1]
%		\Require{$B$ is an invertible matrix}
%		\StateX
%		\Function{Simplex}{$A, b, c, x$}
%			\Let{$\xn$}{$0$}
%			\Let{$\xb$}{$B^{-1} b$} \Comment{$x$ is now a vertex}
%			\For{$x_{\textnormal{adj}}$ \gets every vertex adjacent to $x$}
%				\If{$c^T x_{\textnormal{adj}} > c^T x$}
%					\Let{$x$}{$x_{\textnormal{adj}}$}
%				\EndIf
%			\EndFor
%			\State \Return{$x$} \Comment{$x$ is now the vertex of the optimum solution}
%		\EndFunction
%	\end{algorithmic}
%\end{algorithm}

\subsection{Tableau du simplexe}

\begin{myexem}
	Considérons le problème suivant:
	\[
	\renewcommand{\arraystretch}{1.5}
	\begin{array}{c@{\quad}rcrcrcrcr@{\qquad}l}
		\min\limits_{x} &&& x_2 & - & 5 x_3 & + & 5 x_4 &&& \\
		\textnormal{s.t.} & x_1 & + & x_2 & - & 11 x_3 & + & 7 x_4 & = & 10\phantom{\,,} & \\
		&&& x_2 & - & 8 x_3 & + & 4 x_4 & = & 4\phantom{\,,} & \\
		&&&&&&& x_i & \ge & 0\,, & i \in \{1,2,3,4\}\,.
	\end{array}
	\]
	Construisons le tableau du simplexe pour ce problème.
	\[
	\renewcommand{\arraystretch}{1.5}
	\begin{array}{rrrr|r}
		x_1 & x_2 &  x_3 & \multicolumn{1}{r}{x_4} &    \\
		  0 &   1 &   -5 &                       5 &  z \\
		  \hline
		  1 &   1 &  -11 &                       7 & 10 \\
		  0 &   1 &   -8 &                      11 &  4
	\end{array}
	\]
\end{myexem}
