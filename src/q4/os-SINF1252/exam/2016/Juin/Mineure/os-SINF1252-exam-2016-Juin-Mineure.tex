\documentclass[fr, license=none]{../../../../../../eplexam}

\usepackage{../../../../../../eplcode}

\lstset{language={C}}

\hypertitle{Systèmes informatiques}{4}{SINF}{1252}{2016}{Juin}
{Gilles Peiffer}
{Olivier Bonaventure}

% TODO : Partie théorique, vient avant la théorie (where are the questions though...).

Le syllabus est accessible depuis l'URL \url{http://sites.uclouvain.be/SystInfo}.

Les pages de manuel sont accessibles depuis les URLs suivants :
\begin{itemize}
     \item \url{http://sites.uclouvain.be/SystInfo/manpages/man1} (commandes);
     \item \url{http://sites.uclouvain.be/SystInfo/manpages/man2} (appels systèmes);
     \item \url{http://sites.uclouvain.be/SystInfo/manpages/man3} (fonctions librairies).
\end{itemize}

\section{Traduction de code assembleur}

La fonction suivante a été écrite en assembleur.
Traduisez-la en une version équivalente en \clang{}.
Votre fonction doit nécessairement avoir comme nom \texttt{mp}.

\begin{lstlisting}[language={[x86masm]Assembler}, emph={\$,\%,(,),movl,cmpl,addl,subl},emphstyle={\color{blue}\bfseries}]
mp:
    subl $8, %esp
    movl 16(%esp), %edx
    movl 12(%esp), %ecx
    addl %ecx, %ecx
    cmpl %edx,%ecx
    jle mp1
    movl %edx, %eax
    addl $8, %esp
    ret
mp1: movl %ecx, %eax
    addl $8, %esp
    ret

\end{lstlisting}

\begin{solution}

\begin{lstlisting}
int mp(int c, int d)
{
    c += c;
    if (d <= c)
    {
        return d;
    }
    else
    {
        return c;
    }
}

\end{lstlisting}

\end{solution}

\section{\textit{Search and replace}}

Implémentez la fonction \texttt{sr} dont les spécifications sont fournies ci-dessous.

\begin{lstlisting}
#include <sys/uio.h>
#include <unistd.h>
#include <stdlib.h>
#include <stdio.h>
#include <fcntl.h>
#include <sys/stat.h>
#include <string.h>

/*
 * @pre filename!=NULL, src!=NULL, dst!=NULL, len >0
 *      src et dst pointent vers des zones de taille len
 *      le fichier filename a comme taille un multiple entier de len
 * @post modifie le contenu du fichier filename en remplaçant
 *       toutes les occurences des blocs de données src par dst.
 *       Le fichier est supposé découpé en blocs de taille len
 *       aux positions 0, len, 2*len, ...
 *       retourne le nombre de blocs remplacés et -1 en cas d'erreur
 *       Il est interdit d'utiliser mmap et vous devez utiliser malloc
 *       pour allouer la mémoire dont vous avez besoin. Pour accéder au fichier, vous pouvez uniquement utiliser open, read, lseek, write et close.
 */
int sr(char *filename, char *src, char *dst, ssize_t len) {
// insérez votre code ici sans terminer par }

\end{lstlisting}

\begin{solution}

\begin{lstlisting}
int fd = open(filename, O_RDWR);
if (fd < 0)
{
    return -1;
}

char *buf = malloc(len);
if (buf == NULL)
{
    close(fd);
    return -1;
}
int count = 0;
int r = read(fd, buf, len);
if (r < 0)
{
    close(fd);
    free(buf);
    return -1;
}
while (r != 0)
{
    if (strncmp(src, buf, len) == 0)
    {
        lseek(fd, -len, SEEK_CUR);
        if (write(fd, dst, len) < 0)
        {
            close(fd);
            free(buf);
            return -1;
        }
        ++count;
    }
    r = read(fd, buf, len);
    if (r < 0)
    {
        close(fd);
        free(buf);
        return -1;
    }
}
if (close(fd) < 0)
{
    free(buf);
    return -1;
}
free(buf);
return count;

\end{lstlisting}

\end{solution}

\section{Redirection des flux de sortie et d'erreur standards}

Implémentez la fonction \texttt{run\_redir} dont les spécifications sont fournies ci-dessous.
Il est interdit d'utiliser la fonction \lstinline|system| de la librairie standard.

\begin{lstlisting}
#include <unistd.h>
#include <sys/types.h>
#include <sys/wait.h>
#include <sys/stat.h>
#include <fcntl.h>
#include <stdlib.h>
#include <stdio.h>


/*
 * @pre prog!=NULL, argv!=NULL, file!=NULL
 * execute le programme prog avec les arguments argv
 * en redirigeant la sortie standard et l'erreur standard
 * vers le fichier file. Retourne la valeur de retour
 * du programme et -1 en cas d'erreur d'exécution du programme
 * ou d'accès au fichier
 */
int run_redir(char *prog, char* argv[], char *file) {
// insérez ici le code sans terminer par }

\end{lstlisting}

\begin{solution}

\begin{lstlisting}
int status;
pid_t pid = fork();
if (pid < 0)
{
    return -1;
}
if (pid == 0)
{
     // FILS
     int fd = open(file, O_WRONLY|O_CREAT|O_TRUNC, 0600);
     if(fd < 0)
     {
          exit(-1);
     }
     close(1); // On ferme la sortie standard
     close(2); // On ferme la sortie d'erreur
     if (dup2(fd, 1) < 0)
     {
          exit(-1);
     }
     if (dup2(fd, 2) < 0)
     {
          exit(-1);
     }
     char *env[] = {NULL};
     execve(prog, argv, env);
     exit(-1);
}
else
{
     // PERE
     if (waitpid(pid, &status, 0) < 0)
     {
          return -1;
     }
     if (WIFEXITED(status))
     {
          if(WEXITSTATUS(status) == 0)
          {
               return 0;
          }
          return -1;
     }
     return -1;
}

\end{lstlisting}

\end{solution}

\section{\texttt{Arraylist}}

Vous devez modifier une librairie qui implémente une \texttt{ArrayList}
en y ajoutant une fonction. Cette \texttt{ArrayList} s'utilise comme suit

\begin{lstlisting}
int main(void) {
     struct array_list *head = arraylist_init((size_t) 2, (size_t) sizeof(int));
     int first = 5;
     int second = 6;
     int third = 7;
     int tmp;

     int ret;

     if (!head)
             return 0;

     set_element(head, 0, (void *)&first);
     set_element(head, 1, (void *)&second);

     get_element(head, 1, (void *)&tmp);
     // tmp contient 6
     add_tail(head, (void *)&third);
     get_element(head, 2, (void *)&tmp);
     // tmp contient 7
     printf("array-list size: %d element-size %d\n", get_size(head), get_elem_size(head));
     // affiche array-list size: 3 element-size 4
     array_list_destroy(head);
     return 0;
}

\end{lstlisting}

\begin{lstlisting}
#include <errno.h>
#include <stdio.h>
#include <stdlib.h>
#include <string.h>
#include <semaphore.h>

struct array_list {
     char *content;
     size_t size;
     size_t elem_size;
     sem_t mutex;
};

/* Initialize an array-list
 *
 * @return: Pointer to array_list on   success, NULL on error and errno is set.
 */
struct array_list *arraylist_init(size_t n_elem, size_t elem_size)
{
     struct array_list *alist= (struct array_list *)malloc(sizeof(*alist));

     if (!alist)
             return NULL;

     alist->content = malloc(n_elem * elem_size);
     if (!alist->content) {
             free(alist);
             return NULL;
     }

     if (sem_init(&alist->mutex, 0, 1)) {
             free(alist->content);
             free(alist);
             return NULL;
     }

     alist->size = n_elem;
     alist->elem_size = elem_size;

     return alist;
}

/* Set the element 'elem' at position 'index' (indexes start at 0) inside the
 * array-list
 *
 * @return: -1 on error, 0 on success.
 */
int set_element(struct array_list *alist, int index, void *elem)

{
     char *ptr;

     if (index >= alist->size)
             return -1;

     if (sem_wait(&alist->mutex))
             return -1;

     ptr = alist->content + (index * alist->elem_size);
     memcpy(ptr, elem, alist->elem_size);

     if (sem_post(&alist->mutex))
             return -1;

     return 0;
}


/* Get the element at position 'index' of the array-list and copy it into 'elem'.
 * It is exepected that 'elem' has a sufficiently large memory-region to
 * hold the data.
 *
 * @return: -1 on error, 0 on success.
 */
int get_element(struct array_list *alist, int index, void *elem)
// code non fourni

/* Get the current size of the array-list */
size_t get_size(struct array_list *alist)
{
       return alist->size;
}

/* Get the size of the elements of the   array-list */
size_t get_elem_size(struct array_list *alist)
{
       return alist->elem_size;
}

/* Fully destroys the array-list. All elements will be lost */
int array_list_destroy(struct array_list *alist)
// code non fourni

/* Expand the array-list by one element and put 'elem' at this place.
 *
 * @return: -1 on error, 0 on success
 *
 * Vous *devez* utiliser realloc dans cette fonction
 */
int add_tail(struct array_list *alist, void *elem)
// question

\end{lstlisting}

\begin{solution}

\begin{lstlisting}
if (sem_wait(&alist->mutex))
{
    return -1;
}

(alist->size) = (alist->size)+1;
char* tmp = realloc(alist->content, alist->size * alist->elem_size);

if (!tmp)
{ // si erreur de realloc
    sem_post(&alist->mutex);
    return -1;
}
else
{
    alist->content = tmp;
}

char *ptr;

size_t inc = ((alist->size)-1) * (alist->elem_size);
ptr = (alist->content+inc);

memcpy(ptr, elem, alist->elem_size);

if (sem_post(&alist->mutex))
{
     return -1;
}

return 0;

\end{lstlisting}

\end{solution}

\end{document}
