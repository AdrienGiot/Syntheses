\documentclass[fr, license=none]{../../../../../../eplexam}

\usepackage{../../../../../../eplcode}

\lstset{language={C}}

\hypertitle{Systèmes informatiques}{4}{SINF}{1252}{2016}{Septembre}{Mineure}
{Gilles Peiffer}
{Olivier Bonaventure}

% TODO : Partie théorique, vient avant INGInious (where are the questions though...).

Le syllabus est accessible depuis l'URL \url{http://sites.uclouvain.be/SystInfo}.

\textbf{Les pages de manuel sont accessibles depuis les URLs suivants :}
\begin{itemize}
     \item \url{http://sites.uclouvain.be/SystInfo/manpages/man1} (commandes);
     \item \url{http://sites.uclouvain.be/SystInfo/manpages/man2} (appels systèmes);
     \item \url{http://sites.uclouvain.be/SystInfo/manpages/man3} (fonctions librairies).
\end{itemize}

\textbf{Attention:} veuillez utiliser la version \textbf{\html{}} du syllabus.

\section{Manipulation de liste}

Une liste chaînée est implémentée comme suit:

\begin{lstlisting}
/*
 * Un employé d'une entreprise
 */

typedef struct employe {
  char *name;  // nom de l'employé
  unsigned int age; // âge de l'employé
} Employe;
/*
 * Un noeud de la liste
 */
typedef struct node {
  void *data; // pointeur vers la donnée stockée
  struct node *next; // pointeur vers le noeud suivant
} Node;

/*
 * @pre list!=NULL
 * @post initialise la liste chaînée
 */
void initializeList(LinkedList *list) {
  list->head=NULL;
  list->tail=NULL;
}

/*
 * @pre list!=NULL, data!=NULL, data pointe vers Employe
 * @post Crée un noeud contenant un pointeur vers data
 *       et l'ajoute en début de liste
 *       retourne un pointeur vers le noeud créé, NULL en cas d'erreur
 */
Node * addHead(LinkedList *list, void *data) {
  Node *node= (Node *) malloc(sizeof(Node));
  if(node==NULL)
    return NULL;
  node->data=data;
  if(list->head==NULL) {
    list->tail=node;
    node->next=NULL;
  } else {
    node->next=list->head;
  }
  list->head=node;
  return node;
}

/*
 * @pre list!=NULL, age>0
 * @post retire de la liste tous les employés strictement plus jeunes
 *       que age. Retourne le nombre d'employés retirés de la
 *       liste, -1 en cas d'erreur.
 */
int deleteAllYounger(LinkedList *list, int age) {

\end{lstlisting}

\begin{solution}

\begin{lstlisting}
int counter = 0;

while (((struct employe *) (list->head)->data)->age < age && list->head != list->tail)
{
    Node *del = list->head;
    list->head = (list->head)->next;
    free(del);
    counter++;
}

Node *runner = list->head;
Node *tmp = runner;

while (runner != list->tail)
{
    if (((struct employe *) runner->data)->age < age)
    {
        Node *del = runner;
        tmp->next = runner->next;
        runner = runner->next;
        free(del);
        counter++;
    }
    else
    {
        tmp = runner;
        runner = runner->next;
    }
}

if (((struct employe *) runner->data)->age < age)
{
    if (tmp == runner)
    {
        list->head = NULL;
        list->tail = NULL;
    }
    else
    {
        tmp->next = NULL;
        list->tail = tmp;
    }
    free(runner);
    counter++;
}

return counter;

\end{lstlisting}

\end{solution}

\section{Traduction de code assembleur}

La fonction suivante a été écrite en assembleur.
Traduisez-la en une version équivalente en \clang{}.
Votre fonction doit nécessairement avoir comme nom \texttt{f}.

\begin{lstlisting}[language={[x86masm]Assembler}, emph={\$,\%,(,),movl,cmpl,addl,subl},emphstyle={\color{blue}\bfseries}]
f:
  subl $8, %esp
  movl 20(%esp), %edx
  movl 16(%esp), %ecx
  movl 12(%esp), %ebx
  addl %ebx, %ecx
  cmpl %edx, %ecx
  jle m31
  movl %edx, %eax
  addl $8, %esp
  ret
m31:
  movl %ecx, %eax
  addl $8, %esp
  ret

\end{lstlisting}

\begin{solution}

\begin{lstlisting}
int f(int b, int c, int d)
{
    c += b;
    if (d < c)
    {
        return d;
    }
    else
    {
        return c;
    }
}

\end{lstlisting}

\end{solution}

\section{Manipulation de chaines de caractères}

\begin{lstlisting}
/*
 * @pre 0<n<strlen(str), str!=NULL, str se termine par '\0'
 * @post dans un tableau deux pointeurs vers deux chaînes de caractères
 *       terminées par '\0'. La première (indice 0) contient
 *       les caractères en positions 0 à n-1 dans str.
 *       La seconde le reste de la chaîne str.
 *       En cas d'erreur, retourne NULL en ayant libéré correctement
 *       tout mémoire allouée avant l'erreur.
 *
 * Contrainte : la seule fonction de manipulation des chaînes de caractères
 *              que vous pouvez utiliser est strlen. Les autres (e.g.
 *              strcat, strncat, strstr, ...) sont interdites. Vous pouvez
 *              bien entendu utiliser malloc et free.
 */

 char** strsplit(const char *str, int n) {

\end{lstlisting}

\begin{solution}

\begin{lstlisting}
char *a = malloc((n + 1) * sizeof(char));
if (a == NULL)
{
    return NULL;
}
char *b = malloc((strlen(str) - n + 1) * sizeof(char));
if (b == NULL)
{
    free(a);
    return NULL;
}
int i,j;
for (i = 0; i < n; ++i)
{
    a[i] = str[i];
}
a[n] = '\0';
for (j = 0; j < strlen(str) - n + 1; ++j)
{
    b[j] = str[j+n];
}
char** tab = malloc(2 * sizeof(char*));
if (tab == NULL)
{
    free(a);
    free(b);
    return NULL;
}
tab[0] = a;
tab[1] = b;
return tab;

\end{lstlisting}

\end{solution}

\section{Intersection de fichiers}

\begin{lstlisting}
/*
 * @pre *filename!=NULL, size>0
 * @post construit un fichier contenant size nombre complexes contenant les
 *       valeurs 1+i, 1+2i, 1+3i, ... 1+size*i
 *       retourne size en cas de succès et -1 en cas d'erreur
 */
int create(char *filename, int size) {

  int err;
  int fd=open(filename,O_RDWR|O_CREAT,S_IRUSR|S_IWUSR);
  if(fd==-1) {
    return -1;
  }
  for(int i=1; i<=size; i++) {
    complex val;
    val.r=1;
    val.i=i;
    err=write(fd,(void *) &val, sizeof(complex));
    if(err<0) {
       err=close(fd);
       return(-1);
     }
   }
   err=close(fd);
   if(err==-1)
     return err;
   else
     return size;
 }

/*
 * @pre filename1!=NULL, filename2!=NULL
 *      Les deux fichiers contiennent un nombre entier de complexes,
 *      Tous les nombres complexes se trouvant dans un fichier sont
 *      différents.
 *
 * @post retourne le nombre de complexes qui se trouvent à
 *       la fois dans le fichier filename1 et le fichier filename2
 *       -1 si erreur
 *
 * Contrainte: Vous pouvez uniquement utiliser les fonctions open,
 *             read, write, close, mmap et munmap pour accéder aux
 *             fichiers
 */

int count_same(char *filename1, char *filename2) {

\end{lstlisting}

\begin{solution}

\begin{lstlisting}
int fd1 = open(filename1, O_RDONLY|O_CREAT);
if (fd1 < 0)
{
    return -1;
}
int fd2 = open(filename2, O_RDONLY|O_CREAT);
if (fd2 < 0)
{
    close(fd1);
    return -1;
}

int r1, r2;

complex val1, val2;

int matches = 0;

r1 = read(fd1, (void *) &val1, sizeof(complex));
while (r1 != 0)
{
    if (r1 < 0)
    {
        close(fd1);
        close(fd2);
        return -1;
    }
    r2 = read(fd2, (void *) &val2, sizeof(complex));
    while (r2 != 0)
    {
        if (r2 < 0)
        {
            close(fd1);
            close(fd2);
            return -1;
        }
        if (val1.r == val2.r && val1.i == val2.i)
        {
            ++matches;
            break;
        }
        r2 = read(fd2, (void *) &val2, sizeof(complex));
    }
    if (lseek(fd2, 0, SEEK_SET) < 0)
    {
        close(fd1);
        close(fd2);
        return -1;
    }
    r1 = read(fd1, (void *) &val1, sizeof(complex));
}

if (close(fd1) < 0)
{
    close(fd2);
    return -1;
}
if (close(fd2) < 0)
{
    return -1;
}

return matches;

\end{lstlisting}

\end{solution}

\end{document}
