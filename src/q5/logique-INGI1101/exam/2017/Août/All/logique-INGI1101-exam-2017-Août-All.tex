\documentclass[fr]{../../../../../../eplexam}

\hypertitle{logique-INGI1101}{5}{INGI}{1101}{2017}{Août}{All}
{Nicolas Vanvyve}
{Peter Van Roy}

\section{Question 1}
\paragraph{Note :} \textit{Pour la question 1, vous aurez le maximum entre a note de l'interro et celle de la question.}
\subsection*{1.A}
Donnez les règles complètes suivantes pour les preuves manuelles en logique des prédicats : 
\begin{enumerate}
	\item L'élimination de $\exists$.
	\item L'introduction de $\forall$.
\end{enumerate}
Pour chaque règle, donnez ensuite une justification en vous basant sur les interprétations et la structure des preuves manuelles. Essayez de faire cela avec un maximum de précision 

\subsection*{1.B}
Prouvez que l'implication suivante est une tautologie.
$$\exists y \forall x P(x,y) \Rightarrow \forall x \exists y P(x,y)$$ 
Faites attention à donner une preuve rigoureuse  en logique des prédicats, avec une justification pour chaque étapes.
\section{Question 2}
Considérez le programme Prolog suivant qui fait la définition logique du dernier élément d'une liste et qui définit en même temps un programme pour calculer le dernier élément d'une liste : 

\qquad \texttt{add(0,X,X).}

\qquad \texttt{add(s(X),Y,s(Z) :-add(X,Y,Z).}

Ici, un nombre naturel est représenté comme une structure imbriquée, par exemple 2 est représenté comme s(s(0)). On peut lire 0 comme "zéro" et s(X) comme "le successeur de X".

Considérez la requête suivante  

\qquad \texttt{|?-last(s(0),s(0),X)}.

Répondez au questions suivantes:
\begin{enumerate}
	\item Quelle est la forme normale conjonctive de ce programme?
	\item Quelle est la résolvante initiale qui correspond a la requête
	\item En faisant une première résolution avec le programme, expliquez pourquoi la première clause n'est pas résolvable avec la résolvante initiale. Ensuite, donnez la substitution et la nouvelle résolvante obtenue en faisant une résolution avec le deuxième clause.
	\item Faites une deuxième résolution avec le programme pour éviter une erreur.N'oubliez pas le renommage des variables du programme pour éviter une erreur. Quelle est la solution trouvée par le programme?\footnote{Poursuivre l'exécution jusqu'à la fin du programme.}
	\item [Bonus] Est ce que cette requête a d'autre solutions? Si oui, expliquez pourquoi. Si non expliquez pourquoi pas. Raisonnez sur l'algorithme de preuve et ne dites pas simplement que 1+1 n'a qu'une solution! Il est toujours possible que le prédicat add comme nous l'avons défini ne correspond pas exactement au concept d'addition.
	
\end{enumerate}

\section{Question 3}
Définissez le concept d'équilibre structurel d'un graphe. Ensuite , énoncez le théorème d'équilibre structurel et donnez sa preuve.

\section{Question 4}
Quelle est la loi qui gouverne la popularité des pages Web, Pour quantifier cette loi, définissez d'abord une mesure pour la popularité d'une page Web? Ensuite, donnez le modèle d'attachement préférentiel qui explique cette loi. Enfin, expliquez le phénomène de la longue traine et comment cette loi donne lieu à ce phénomène.

\end{document}
