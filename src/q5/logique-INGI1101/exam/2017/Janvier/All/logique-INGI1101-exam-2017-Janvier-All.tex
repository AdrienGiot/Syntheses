\documentclass[fr]{../../../../../../eplexam}

\hypertitle{logique-INGI1101}{5}{INGI}{1101}{2017}{Janvier}{All}
{Nicolas Vanvyve}
{Peter Van Roy}

\section{Base de la logique}
\paragraph{Note :} \textit{Pour la question 1, vous aurez le maximum entre a note de l'interro et celle de la question.}

\subsection{Définissez les trois formes de raisonnement que l'on connaisse : déduction, abduction, induction. Donnez un exemple de chaque forme}
\subsection{Expliquer le méthode scientifique avec les trois formes de raisonnement.}
\subsection{Définissez la sémantique en logique des prédicats. Vous pouvez faire cela en trois étapes :}
\begin{enumerate}
	\item D'abord donnez une grammaire de la syntaxe de la logique des prédicats.
	\item Ensuite, définissez le concept d'interprétation d'une formule logique.
	\item Enfin, définissez le concept de modèle d'un ensemble de formules.
\end{enumerate}

\section{Prolog}
Considérez le programme Prolog suivant qui fait la définition logique du dernier élément d'une liste et qui définit en même temps un programme pour calculer le dernier élément d'une liste : 

\qquad \texttt{last([L|nil],L)}.

\qquad \texttt{last([H|T],L) :-last(T,L)}

avec la requête:

\qquad \texttt{|?-last([1,2],L)}.

Répondez au questions suivantes:
\subsection{Quelle est la forme normale conjonctive de ce programme?}
\subsection{Quelle est la résolvante initiale qui correspond a la requête}
\subsection{En faisant une prmière résolution avec le programme, expliquez pourquoi la première clause n'est pas résolvable avec la résolvante initiale. Ensuite, donnez la substitution et la nouvelle résolvante obtenue en faisant une résolution avec le deuxième clause.}
\subsection{Faites une deuxième résolution avec le programme pour éviter une erreur. Quelle est la solution trouvée par le programme?}
\subsection{[Bonus] Est ce que cette requête a d'autre solutions? Si oui, expliquez pourquoi. Si non expliquez pourquoi pas.}

\section{Pour cette question, vous allez investiguer les réseaux sociaux dans leurs contextes :}
\subsection{Définissez avec précision le concept de réseau d'affiliation social. Une propriété importante des personnes dans un tel réseau est la similitude:  les personnes avec des liens entres eux tendent à se ressembler.}
\subsection{Définissez les mécanisme de sélection et  d'influence sociale, qui peuvent augmenter la similitude entre les noeuds d'un tel réseau d'affiliation social.}
\subsection{Définissez les trois différentes formes de fermeture qui peuvent faire évoluer un réseaux d'affiliation sociale.}
\subsection{En utilisant l'exemple de Wikipedia, expliquez comment on peut comparer les effets de la sélection et de l'influence sociale. Définissez d'abord Wikipedia comme un réseau d'affiliation social et comment la sélection et l'influence sociale se montrent. Ensuite résumez les résultats d'une expérience qui compare les effets des deux mécanismes}

\section{Quelle est la loi qui gouverne la popularité des pages Web, Pour quantifier cette loi, définissez d'abord une mesure pour la popularité d'une page Web? Ensuite, donnez le modèle d'attachement préférentiel qui explique cette loi. Enfin, expliquez le phénomène de la longue traine et comment cette loi donne lieu à ce phénomène.}

\end{document}
