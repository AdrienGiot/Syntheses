\documentclass[fr]{../../../../../../eplexam}
\usepackage{../../../../../../eplcode}

\hypertitle{logique-INGI1101}{5}{INGI}{1101}{2017}{Septembre}{All}
{Nicolas Vanvyve\and Jean-Martin Vlaeminck\and Divers contributeurs\thanks{
Quentin Delmelle (théorème équilibre structurel), Maxime de Streel (diverses remises en page), Romain Pattyn (tentative de résolution de la Q4), Thomas Ponchau (Prolog, définition équilibre structurel).
}}
{Peter Van Roy}
% Delmelle: théorème équilibre structurel
% de Streel: création du document, mises en page
% Romain Pattyn: tentative de résolution de la Q4 mais sans grosse certitude
% Ponchau: réponse Prolog, définition équilibre structurel

\paragraph{Remarques}
Les solutions ont été proposées par de nombreux étudiants
lors de l'année académique 2017--2018, dans un document partagé (ShareLaTeX).
L'exactitude des réponses n'est dès lors pas acquise, mais elles peuvent servir
de point de repère quand à ce qui était demandé pendant l'examen.

\section{Question 1}
\paragraph{Note :} \textit{Pour la question 1, vous aurez le maximum
entre a note de l'interro et celle de la question.}
\begin{enumerate}
\item
Donnez les règles complètes suivantes pour les preuves manuelles en logique des prédicats : 
\begin{enumerate}
	\item L'élimination de $\exists$.
	\item L'introduction de $\forall$.
\end{enumerate}
Pour chaque règle, donnez ensuite une justification en vous basant
sur les interprétations et la structure des preuves manuelles.
Essayez de faire cela avec un maximum de précision.
\item
Prouvez que l'implication suivante est une tautologie.
\[ \exists y \forall x P(x,y) \Rightarrow \forall x \exists y P(x,y) \]
Faites attention à donner une preuve rigoureuse  en logique des prédicats,
avec une justification pour chaque étapes.
\end{enumerate}

\nosolution

\section{Question 2}
Considérez le programme Prolog suivant qui fait
une définition logique de et calcule une addition de deux nombres naturels:

\begin{lstlisting}
add(0,X,X).
add(s(X),Y,s(Z) :-add(X,Y,Z).
\end{lstlisting}

Ici, un nombre naturel est représenté comme une structure imbriquée,
par exemple 2 est représenté comme s(s(0)).
On peut lire 0 comme "zéro" et s(X) comme "le successeur de X".

Considérez la requête suivante:
\begin{lstlisting}
?-add(s(0),s(0),X).
\end{lstlisting}

Répondez au questions suivantes:
\begin{enumerate}
	\item Quelle est la forme normale conjonctive de ce programme?
	\item Quelle est la résolvante initiale qui correspond a la requête
	\item
	En faisant une première résolution avec le programme,
	expliquez pourquoi la première clause n'est pas résoluble avec la résolvante initiale.
	Ensuite, donnez la substitution et la nouvelle résolvante obtenue
	en faisant une résolution avec le deuxième clause.
	\item
	Faites une deuxième résolution avec le programme pour éviter une erreur.
	N'oubliez pas le renommage des variables du programme pour éviter une erreur.
	Quelle est la solution trouvée par le programme?
	\footnote{Poursuivre l'exécution jusqu'à la fin du programme.}
	\item [Bonus] Est ce que cette requête a d'autre solutions?
	Si oui, expliquez pourquoi. Si non expliquez pourquoi pas.
	Raisonnez sur l'algorithme de preuve et ne dites pas simplement que 1+1 n'a qu'une solution!
	Il est toujours possible que le prédicat \lstinline|add|
	comme nous l'avons défini ne correspond pas exactement au concept d'addition.
	
\end{enumerate}

\begin{solution}
\begin{enumerate}
	\item $add(0,x,x) \land (\lnot add(x,y,z) \lor add(s(x),y,s(z)))$
	\item %Je réponds en faisant l'hypothèse qu'il y a une erreur dans l'énoncé : que le but est de trouver add(s(0),s(0),X) à la place de last(s(0),s(0),X)
	La résolvante initiale est la négation de la requête :
	$r = < add(s(0),s(0),x) >$
	(NB : on n'écrit pas les $\lnot$ dans la résolvante)
	\item
	La première clause est \lstinline$add(0,x,x)$.
	Elle n'est pas résoluble avec la résolvante initiale
	car le premier argument de \lstinline$add$ ne peut pas matcher.
	En effet, \lstinline$0$ et \lstinline$s(0)$ ne sont pas égaux, donc le match ne peut pas fonctionner.
	
	La deuxième clause est résoluble.
	La résolvante initiale matche avec \lstinline$add(s(x),y,s(z))$.
	On va renommer le \lstinline$x$ de cette clause en \lstinline$x_1$.
	La substitution à effectuer est donc $\sigma = \{ (x_1, 0), (y, s(0) ) \}$;
	retenons que $x = s(z)$.

	La nouvelle résolvante est $r = <add(x_1,y,z)>\sigma  =  <add(0,s(0),z)>$.
	\item
	La nouvelle résolvant matche avec la première clause $\texttt{add}(0,x_2,x_2)$.
	La substitution à effectuer est $\{(x_2,s(0))\}$ et $z = x_2$, donc $z = s(0)$.
	Le résultat est donc \lstinline$x = s(z) = s(s(0))$. 
	\item Non, cette requête n'a qu'une seule solution.
	En effet, si on essaie de trouver d'autres solutions,
	on va essayer de matcher $<add(0,s(0),z)>$ avec la deuxième clause,
	mais ça ne fonctionne pas car $0$ ne matche pas dans $s(x)$.
	On arrive donc au bout des possibilités, et on doit faire marche arrière.
	Malheureusement, à l'étape précédente, on était déjà à la dernière clause.
	Il n'y a donc plus de choix possible. Nous n'avons donc trouvé qu'un seul résultat.
\end{enumerate}
\end{solution}

\section{Question 3}
Définissez le concept d'équilibre structurel d'un graphe.
Ensuite, énoncez le théorème d'équilibre structurel et donnez sa preuve.

\begin{solution}
Le concept d'équilibre structurel se définit dans un graphe complet
où chaque arête est soit positive, soit négative.
Ce graphe est équilibré si et seulement si pour chaque tuple de n\oe{}ud,
les 3 arêtes reliant ces n\oe{}uds sont soit tous les trois positifs $(+,+,+)$,
soit un positif et deux négatifs $(+,-,-)$.

Théorème : Si un graphe est équilibré, alors 2 cas:
\begin{itemize}
	\item toutes les arêtes sont (+)
	\item on peut partager les n\oe{}uds en 2 ensembles X et Y, tels que
	2 n\oe{}uds d'un même ensemble ne peuvent pas être liées par un (-),
	et chaque membre de X est ennemi de chaque membre de Y.
\end{itemize}

\paragraph{Preuve}
s'il n'y a pas d'arête (-): forcément vrai.
Sinon, on prend un noeud A quelconque: \\
X = amis de A\\
Y = ennemis de A\\
-> démontrer que X et Y satisfont les conditions. Voir cours.
\end{solution}

\section{Question 4}
Quelle est la loi qui gouverne la popularité des pages Web.
Pour quantifier cette loi, définissez d'abord une mesure pour la popularité d'une page Web.
Ensuite, donnez le modèle d'attachement préférentiel qui explique cette loi.
Enfin, expliquez le phénomène de la longue traine et comment cette loi donne lieu à ce phénomène.

\nosolution

\paragraph{Remarque}
En réalité, des réponses ont été proposées, mais elle ne semblent pas très fiables. Elles discutent néanmoins de la loi de puissance, du \og rich get richer\fg{} et de l'attachement préférentiel.

\end{document}
