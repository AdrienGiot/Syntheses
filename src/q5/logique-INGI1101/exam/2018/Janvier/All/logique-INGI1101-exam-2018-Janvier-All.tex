\documentclass[en]{../../../../../../eplexam}
\usepackage{listings}
\hypertitle{logique-INGI1101}{5}{INGI}{1101}{2018}{Janvier}{All}
{Gilles Charlier}
{Peter Van Roy}


\lstset{emph={%  
   {:-}%
     },emphstyle={\color{red}\bfseries\underbar}%
}%

\it{Vous obtiendrez la côte maximale entre la première question de l'examen et la cote de l'interro mi-quadri}
\section{Question 1 (5 points)}
\subsection{Définissez me schéma de la preuve par contradiction et justifiez le en raisonnant sur les interprétations.}
\subsection{L'algorithme de preuve est complet et adéquat, qu'est ce que cela veut dire ?}

\subsection{Définissez la sémantique propositionelle:}
\subsubsection{D'abord, donnez une grammaire de la syntaxe de la logique propositionelle}
\subsubsection{Définissez le concept d'interprétations d'une formule en logique}
\subsubsection{Définissez le concept de modèles d'un ensemble de formules}

\section{Question 2 : Prolog (7 ou 8 points)}
Considérez le programme suivant : 

\begin{lstlisting}[language=Prolog]
    sumList(nil,0)
    sumList([H|T],S):- sumList(T,S1), plus(H,S1,S)
\end{lstlisting}
\it{requête :}
\begin{lstlisting}[language=Prolog]
|?- sumList([5,6],L)
\end{lstlisting}
\subsection{Donnez la formule normale conjonctive (FNC)}
\subsection{Donnez la résolvante initiale}
\subsection{Pourquoi la première clause n'est pas résolvable ?}
\subsection{Donnez la deuxième étape de résolution. N'oulibez pas de remplacer les noms de variables pour éviter d'éventuelles erreurs.}
\subsection{Effectuez les dernières étapes de résolution. Vous pouvez supposer que le prédicat \texttt{plus(x,y,z)} est résolvable si deux des arguments sont des entiers pour satisfaire l'équation \texttt{x+y=z}}

\section{Question 3 : Graphes (7 ou 8 points)}
\subsection{Définissez le concept d'équilibre structurel faible comme une propriété locale du graphe.}
\subsection{Donnez l'énoncé du théorème d'équilibre structurel faible qui fait un lien entre la propriété locale et la propriété globale du graphe.}
\subsection{Donnez la preuve du théorème, faites attention au raisonnement récursif.}


\end{document}
