\documentclass[fr]{../../../../../../eplexam}

\hypertitle{Logique et structures discrètes}{5}{INGI}{1101}{2018}{Septembre}{Majeure}
{François Duchène}
{Peter Van Roy}

\textit{Vous obtiendrez la cote maximale entre la première question de l'examen et la cote de l'interro mi-quadrimestre.}

\section{}

\begin{enumerate}
 \item Décrivez les différents types de raisonnement avec un exemple pour chacun.
 \item Dessinez un schéma représentant les différents types de raisonnement dans la démarche scientifique.
 \item Dans la logique propositionnelle:
    \begin{enumerate}
     \item Donnez la grammaire de la logique des propositions,
     \item Définissez le concept d'interprétation,
     \item Définissez le concept de modèle.
    \end{enumerate}
\end{enumerate}

\section{}
Donnez une preuve de la théorie avec les prédicats suivants. N'oubliez pas que votre preuve est un objet mathématique, par conséquent vous devez être le plus formel possible.
\begin{enumerate}
    \item $\forall x(\textrm{Canard}(X) \Rightarrow \neg\textrm{dansant}(X))$
    \item $\forall x(\textrm{Officier}(X) \Rightarrow \textrm{dansant}(X))$
    \item $\forall x(\textrm{MaVolaille}(X) \Rightarrow \textrm{Canard}(X))$
\end{enumerate}

Prouvez la théorie suivante : $\forall x(\textrm{MaVolaille}(X) \Rightarrow \neg\textrm{officier}(X))$

\section{}
Définissez avec précision l'algorithme de Pagerank\footnote{Le professeur s'attend à un pseudocode plutôt qu'une longue explication en texte.}.

\end{document}
