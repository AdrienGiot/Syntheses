\documentclass[fr]{../../../../../../eplexam}

\hypertitle{}{5}{INGI}{1101}{2020}{Janvier}{Majeure}
{Augustin d'Oultremont}
{Peter Van Roy}

\section{Bases de la Logique}
    \begin{enumerate}
        \item Définissez le schéma de la preuve par contradiction et justifiez-le en raisonnant sur les interprétations.
        \item L'algorithme de preuve pour la logique des prédicats est complet et adéquat. Définissez ces 2 concepts. Par contre, l'algorithme n'est pas décidable, il n'est que semi-décidable. Définissez ce concept.
        \item Vous allez définir la sémantique de la logique des prédicats en 2 étapes. D'abord, définissez une grammaire de la syntaxe de la logique des prédicats.
        \item Ensuite, définissez le concept d'interprétation d'une formule en logique des prédicats.
        \item Enfin, définissez le concept de modèle d'un ensemble de formules.
    \end{enumerate}
    \nosolution

\section{Langage logique Prolog}
    \begin{enumerate}
        \item Donnez le pseudocode de l'algorithme d'exécution de Prolog.
        \item Dans le pseudocode de la partie 1, l'algorithme doit faire des choix pendant son exécution. Un choix fait par l'algorithme peut s'avérer mauvais et il doit gérer cela. Expliquez comment l'algorithme fait la gestion des choix. Il n'est pas nécessaire de donner un pseudocode dans votre explication.
        \item Considérez le programme Prolog suivant qui somme les éléments d'une liste. \texttt{sum([], S, S). sum([X|L], S1, S) :- S2 is X+S1, sum(L, S2, S).} avec la requête \texttt{|? sum([5,6], 0, 8).} Quelle est la forme normale conjonctive (FNC) de ce programme?
        \item Quelle est la résolvante initiale correspondant à cette requête?
        \item Faire 2 résolutions ou renommages.
        \item Faire encore 3 résolutions. Pour la dernière résolution, il faut choisir la première clause. Vous devez alors arriver à une résolution vide. Montrer comment la dernière substitution donne la solution.
    \end{enumerate}
    \nosolution
    
\section{Mécanismes de similitude}
    \begin{enumerate}
        \item Définir le principe de similitude ainsi que 2 mécanismes.
        \item Expliquez la structure de Wikipédia pour montrer comment elle peut nous aider à faire cette comparaison.
        \item Donnez la similitude moyenne entre 2 rédacteurs Wikipédia trouvée par cette étude dans un graphe en fonction d'un paramètre significatif. Définissez ce paramètre. Définissez la formule qui donne la similitude entre 2 rédacteurs.
        \item Expliquez le résultat de cette étude faite avec Wikipédia. Comment peut-on distinguer les 2 mécanismes de similitude dans cette étude?
    \end{enumerate}
    \nosolution

\end{document}
