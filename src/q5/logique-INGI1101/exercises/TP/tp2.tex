\section{TP 2}
%\addcontentsline{toc}{section}{TP 2}


% \section*{Rappel}

% \begin{center}
% \textbf{Liste des équivalences logiques}
% \end{center}

% \textsf{Lois commutatives}
% \begin{enumerate}
% 	\item $p \vee q \Lleftarrow\!\!\!\!\Rrightarrow q \vee p$ \textit{(commutativité de $\vee$)}
% 	\item $p \wedge q \Lleftarrow\!\!\!\!\Rrightarrow q \wedge p$ \textit{(commutativité de $\wedge$)}
% 	\item $p \Leftrightarrow q \Lleftarrow\!\!\!\!\Rrightarrow q \Leftrightarrow q$ \textit{(commutativité de $\Leftrightarrow$)}
% \end{enumerate}

% \textsf{Lois associatives}
% \begin{enumerate}
% 	\item $(p \vee q) \vee r \Lleftarrow\!\!\!\!\Rrightarrow p \vee (q \vee r)$ \textit{(associativité de $\vee$)}
% 	\item $(p \wedge q) \wedge r \Lleftarrow\!\!\!\!\Rrightarrow p \wedge (q \wedge r)$ \textit{(associativité de $\wedge$)}
% \end{enumerate}

% \textsf{Lois distributives}
% \begin{enumerate}
% 	\item $p \wedge (q \vee r) \Lleftarrow\!\!\!\!\Rrightarrow (p \wedge q) \vee (p \wedge r)$ \textit{(distributivité de $\wedge$ sur $\vee$)}
% 	\item $p \vee (q \wedge r) \Lleftarrow\!\!\!\!\Rrightarrow (p \vee q) \wedge (p \vee r)$ \textit{(distributivité de $\vee$ sur $\wedge$)}
% \end{enumerate}

% \textsf{Lois de De Morgan}
% \begin{enumerate}
% 	\item $\neg(p \wedge q) \Lleftarrow\!\!\!\!\Rrightarrow \neg p \vee \neg q$ \textit{(loi 1 de De Morgan)}
% 	\item $\neg(p \vee q) \Lleftarrow\!\!\!\!\Rrightarrow \neg p \wedge \neg q$ \textit{(loi 2 de De Morgan)}
% \end{enumerate}

% \textsf{Loi de la négation}
% \begin{enumerate}
% 	\item $\neg \neg p \Lleftarrow\!\!\!\!\Rrightarrow p$
% \end{enumerate}

% \textsf{Loi du tiers exclu}
% \begin{enumerate}
% 	\item $p \vee \neg p \Lleftarrow\!\!\!\!\Rrightarrow \textbf{true} $
% \end{enumerate}

% \textsf{Loi de la contradiction}
% \begin{enumerate}
% 	\item $p \wedge \neg p \Lleftarrow\!\!\!\!\Rrightarrow \textbf{false}$
% \end{enumerate}

% \textsf{Loi de l'implication}
% \begin{enumerate}
% 	\item $p \Rightarrow q \Lleftarrow\!\!\!\!\Rrightarrow \neg p \vee q$
% \end{enumerate}

% \textsf{Loi du contraposée}
% \begin{enumerate}
% 	\item $p \Rightarrow q \Lleftarrow\!\!\!\!\Rrightarrow \neg q \Rightarrow \neg p$
% \end{enumerate}

% \textsf{Loi de l'équivalence}
% \begin{enumerate}
% 	\item $p \Leftrightarrow q \Lleftarrow\!\!\!\!\Rrightarrow (p \Rightarrow q) \wedge (q \Rightarrow p)$
% \end{enumerate}

% \textsf{Lois de l'idempotence}
% \begin{enumerate}
% 	\item $p \Lleftarrow\!\!\!\!\Rrightarrow p \vee p$ \textit{(idempotence de $\vee$)}
% 	\item $p \Lleftarrow\!\!\!\!\Rrightarrow p \wedge p$ \textit{(idempotence de $\wedge$)}
% \end{enumerate}

% \textsf{Lois de simplification}
% \begin{enumerate}
% 	\item $p \wedge \textbf{true} \Lleftarrow\!\!\!\!\Rrightarrow p$
% 	\item $p \vee \textbf{true} \Lleftarrow\!\!\!\!\Rrightarrow \textbf{true}$
% 	\item $p \wedge \textbf{false} \Lleftarrow\!\!\!\!\Rrightarrow \textbf{false}$
% 	\item $p \vee \textbf{false} \Lleftarrow\!\!\!\!\Rrightarrow p$
% 	\item $p \vee (p \wedge q) \Lleftarrow\!\!\!\!\Rrightarrow p$
% 	\item $p \wedge (p \vee q) \Lleftarrow\!\!\!\!\Rrightarrow p$
% \end{enumerate}

% \begin{center}
% \textbf{Liste des règles d'inférence}
% \end{center}

% \begin{tabular}{c c c c}

% \textsf{Conjonction} & \textsf{Simplification} & \textsf{Addition} & \textsf{Syllogisme disjoint}   \\

% \begin{tabular}{l}
% $p$ \\
% $q$ \\
% \hline
% $p \wedge q$
% \end{tabular}

% &

% \begin{tabular}{l}
% $p \wedge q$ \\
% \hline
% $p$
% \end{tabular}

% &

% \begin{tabular}{l}
% $p$ \\
% \hline
% $p \vee q$
% \end{tabular}

% &

% \begin{tabular}{l}
% $p \vee q$ \\
% $\neg p$ \\
% \hline
% $q$
% \end{tabular} \\

% \textsf{Modus ponens} & \textsf{Modus tollens} & \textsf{Contradiction} & \textsf{Double négation} \\

% \begin{tabular}{l}
% $p \Rightarrow q$ \\
% $p$ \\
% \hline
% $q$
% \end{tabular}

% &

% \begin{tabular}{l}
% $p \Rightarrow q$ \\
% $\neg q$ \\
% \hline
% $\neg p$
% \end{tabular}

% &

% \begin{tabular}{l}
% $p$ \\
% $\neg p$ \\
% \hline
% $q$
% \end{tabular}

% &
% \begin{tabular}{l}
% $\neg \neg p$ \\
% \hline
% $p$
% \end{tabular} \\

% \textsf{Transitivité} & \textsf{Lois de l'équivalence} & \textsf{Théorème de la déduction} & \textsf{Réduction à l'absurde} \\

% \begin{tabular}{l}
% $p \Leftrightarrow q$ \\
% $q \Leftrightarrow r$ \\
% \hline
% $p \Leftrightarrow r$
% \end{tabular}

% &

% \begin{tabular}{l}
% $p \Leftrightarrow q$ \\
% \hline
% $p \Rightarrow q$ \\
% $q \Rightarrow p$
% \end{tabular}

% &

% \begin{tabular}{l}
% $p, \ldots, r, \boxed{s} \vdash t$ \\
% \hline
% $p, \ldots, r \vdash s \Rightarrow t$
% \end{tabular}

% &

% \begin{tabular}{l}
% $p, \ldots, q, \boxed{r} \vdash s$ \\
% $p, \ldots, q, \boxed{r} \vdash \neg s$ \\
% \hline
% $p, \ldots, q \vdash \neg r$
% \end{tabular}

% \end{tabular}

% \newpage
% \section*{Exercices}

\subsection*{Exercice 1}
Démontrez les équivalences logiques suivantes.

\begin{enumerate}
	\item $p \wedge (q \wedge r)  \Lleftarrow\!\!\!\!\Rrightarrow (p \wedge q) \wedge r$
	\item $p \Rightarrow (q \Rightarrow r) \Lleftarrow\!\!\!\!\Rrightarrow (p \Rightarrow q) \Rightarrow (p \Rightarrow r)$
	\item $p \wedge (p \Rightarrow q) \Rightarrow q \Lleftarrow\!\!\!\!\Rrightarrow \textbf{true}$
	\item $(p \vee q) \wedge (\neg p \vee q) \Lleftarrow\!\!\!\!\Rrightarrow q$
	\item $(p \vee q) \vee (\neg p \wedge \neg q) \Lleftarrow\!\!\!\!\Rrightarrow \textbf{true}$
% 	\item $(p \vee q) \wedge (\neg p \wedge \neg q) \Lleftarrow\!\!\!\!\Rrightarrow \textbf{false}$
	\item $p \vee (q \wedge r) \Lleftarrow\!\!\!\!\Rrightarrow \neg (\neg (p \vee q) \vee \neg (p \vee r))$
	\item $(p \vee q) \wedge \neg (p \wedge q) \Lleftarrow\!\!\!\!\Rrightarrow (p \wedge \neg q) \vee (\neg p \wedge q)$
	\item $p \wedge q \Lleftarrow\!\!\!\!\Rrightarrow (p \vee q) \wedge (p \Leftrightarrow q)$
\end{enumerate}

    \subsubsection*{Solution}
    Notons d'abord que toutes les preuves suivantes peuvent aussi être réalisées grâce aux table de vérités.
\begin{enumerate}
	\item
    \begin{flalign*}
    p \land (q \land r) &\Lleftarrow\!\!\!\!\Rrightarrow \lnot \lnot (p \land ( q \land r )) \tag*{Double négation}\\
    &\Lleftarrow\!\!\!\!\Rrightarrow \lnot ( \lnot (q \land r) \lor \lnot p ) \tag*{De Morgan}\\
    & \Lleftarrow\!\!\!\!\Rrightarrow \lnot ((\lnot r \lor \lnot q) \lor \lnot p) \tag*{De Morgan} \\
    & \Lleftarrow\!\!\!\!\Rrightarrow \lnot(( \lnot p \lor \lnot q) \lor \lnot r) \tag*{Associativité}\\
    & \Lleftarrow\!\!\!\!\Rrightarrow \lnot (\lnot p \lor \lnot q) \land \lnot \lnot r \tag*{De Morgan} \\
    & \Lleftarrow\!\!\!\!\Rrightarrow (p \land q) \land r \tag*{De Morgan et double négation}
    \end{flalign*}

	\item
	\begin{flalign*}
    p \Rightarrow (q \Rightarrow r)& \Lleftarrow\!\!\!\!\Rrightarrow p \Rightarrow (\lnot q \lor r) \tag*{Implication} \\
    & \Lleftarrow\!\!\!\!\Rrightarrow \lnot p \lor (\lnot q \lor r) \tag*{Implication} \\
    & \Lleftarrow\!\!\!\!\Rrightarrow (\lnot p \lor \lnot q) \lor r \tag*{Associativité}\\
    & \Lleftarrow\!\!\!\!\Rrightarrow (\text{true} \land (\lnot p \lor \lnot q)) \lor r \tag*{Simplification inverse} \\
    & \Lleftarrow\!\!\!\!\Rrightarrow ((\lnot p \lor p) \land  (\lnot p \lor \lnot q)) \lor r \tag*{Loi du tiers exclus}\\
    & \Lleftarrow\!\!\!\!\Rrightarrow (\lnot p \lor (p \land \lnot q)) \lor r \tag*{Distributivité}\\
    & \Lleftarrow\!\!\!\!\Rrightarrow ((p \lor \lnot q) \lor \lnot p) \lor r \tag*{Associativité} \\
    & \Lleftarrow\!\!\!\!\Rrightarrow (p \land \lnot q) \lor (\lnot p \lor r) \tag*{Associativité} \\
    & \Lleftarrow\!\!\!\!\Rrightarrow \lnot \lnot (p \land \lnot q) \lor (\lnot p \lor r) \tag*{Double négation} \\
    & \Lleftarrow\!\!\!\!\Rrightarrow \lnot (\lnot p \lor q ) \lor ( \lnot p \lor \lnot r) \tag*{De Morgan}\\
    & \Lleftarrow\!\!\!\!\Rrightarrow (p \Rightarrow q) \Rightarrow (p \Rightarrow r) \tag*{Implication}
    \end{flalign*}

	\item
	\begin{flalign*}
    p \land (p \Rightarrow q ) \Rightarrow q & \Lleftarrow\!\!\!\!\Rrightarrow p \land (\lnot p \lor q ) \Rightarrow q \tag*{Implication} \\
    & \Lleftarrow\!\!\!\!\Rrightarrow \lnot (p \land ( \lnot p \lor q )) \lor q \tag*{Implication}\\
    & \Lleftarrow\!\!\!\!\Rrightarrow \lnot p \lor \lnot ( \lnot p \lor q ) \lor q \tag*{De Morgan}\\
    & \Lleftarrow\!\!\!\!\Rrightarrow \lnot p \lor (p \land \lnot q) \lor q \tag*{De Morgan}\\
    & \Lleftarrow\!\!\!\!\Rrightarrow (( \lnot p \lor p ) \land ( \lnot p \lor \lnot q )) \lor q \tag*{Distributivité}\\
    & \Lleftarrow\!\!\!\!\Rrightarrow (\text{true} \land ( \lnot p \lor \lnot q )) \lor q \tag*{Loi du tiers exclu}\\
    & \Lleftarrow\!\!\!\!\Rrightarrow ( \lnot p \lor \lnot q ) \lor q \tag*{Simplification}\\
    & \Lleftarrow\!\!\!\!\Rrightarrow \lnot p \lor (\lnot q \lor q ) \tag*{Associativité}\\
    & \Lleftarrow\!\!\!\!\Rrightarrow \lnot p \lor \text{true} \tag*{Loi du tiers exclu}\\
    & \Lleftarrow\!\!\!\!\Rrightarrow \text{true} \tag*{Simplification}\\
    \end{flalign*}

	\item
    \begin{flalign*}
    (p \lor q) \land (\lnot p \lor q) & \Lleftarrow\!\!\!\!\Rrightarrow (q \lor p) \land (q \lor \lnot p) \tag*{Loi commutative}\\
    & \Lleftarrow\!\!\!\!\Rrightarrow q \lor (p \land \lnot p) \tag*{Distributivité}\\
    & \Lleftarrow\!\!\!\!\Rrightarrow q \lor \text{false} \tag*{Simplification}\\
    & \Lleftarrow\!\!\!\!\Rrightarrow q \tag*{Simplification}
    \end{flalign*}

	\item
    \begin{flalign*}
    (p \vee q) \vee (\neg p \wedge \neg q) & \Lleftarrow\!\!\!\!\Rrightarrow (p \vee q) \vee \neg ( p \vee q) \tag*{De Morgan} \\
    & \Lleftarrow\!\!\!\!\Rrightarrow \text{true} \tag*{Loi du tiers exclu}
    \end{flalign*}

	\item
    \begin{flalign*}
    \lnot ( \lnot ( p \lor q ) \lor \lnot (p \lor r ) ) & \Lleftarrow\!\!\!\!\Rrightarrow (p \lor q ) \land ( p \lor r) \tag*{De Morgan}\\
    & \Lleftarrow\!\!\!\!\Rrightarrow p \lor ( q \land r ) \tag*{Distributivité}
    \end{flalign*}

	\item
    \begin{flalign*}
    (p \lor q) \land \lnot ( p \land q) & \Lleftarrow\!\!\!\!\Rrightarrow \lnot (\lnot (p \lor q ) \lor (p \land q)) \tag*{Double négation et De Morgan}\\
    & \Lleftarrow\!\!\!\!\Rrightarrow \lnot (( \lnot p \land \lnot q) \lor ( p \land q)) \tag*{De Morgan}\\
    & \Lleftarrow\!\!\!\!\Rrightarrow \lnot ((\lnot p \lor p) \land (\lnot p \lor q) \land (\lnot q \lor p) \land (\lnot q \lor q)) \tag*{Distributivité}\\
    & \Lleftarrow\!\!\!\!\Rrightarrow \lnot (\text{true} \land (\lnot p \lor q) \land (\lnot q \lor p) \land \text{true}) \tag*{Simplification}\\
    & \Lleftarrow\!\!\!\!\Rrightarrow \lnot ((\lnot p \lor q) \land (\lnot q \lor p)) \tag*{Simplification}\\
    & \Lleftarrow\!\!\!\!\Rrightarrow \lnot (\lnot p \lor q) \lor \lnot(\lnot q \lor p) \tag*{De Morgan}\\
    & \Lleftarrow\!\!\!\!\Rrightarrow (p \land \lnot q) \lor ( q \land \lnot p) \tag*{De Morgan}
    \end{flalign*}

	\item
    \begin{flalign*}
    (p \lor q) \land (p \Leftrightarrow q) & \Lleftarrow\!\!\!\!\Rrightarrow (p \lor q) \land (\lnot p \lor q) \land (p \lor \lnot q) \tag*{Double implication + loi de l'équivalence}\\
    & \Lleftarrow\!\!\!\!\Rrightarrow (p \land \lnot p \land p) \lor (p \land \lnot p \land \lnot q) \lor (p \land q \land p) \lor (p \land q \land \lnot q)\\
    & \lor (q \land \lnot p \land p)\lor (q \land \lnot p \land \lnot q) \lor (q \land q \land p) \lor (q \land q \land \lnot q) \tag*{Distributivité}\\
    & \Lleftarrow\!\!\!\!\Rrightarrow \text{false} \lor \text{false} \lor (p \land q \land p) \lor \text{false} \lor \text{false} \lor \text{false} \lor (q \land q \land p) \lor \text{false} \tag*{Simplification}\\
    & \Lleftarrow\!\!\!\!\Rrightarrow (p \land q) \lor (q \land p) \tag*{Simplification}\\
    & \Lleftarrow\!\!\!\!\Rrightarrow (p \land q) \tag*{Simplification}
    \end{flalign*}

\end{enumerate}


\subsection*{Exercice 2}
Démontrez, à l'aide d'une table de vérité, la validité des arguments suivants:

\begin{enumerate}
	\item \enter

	\begin{flushleft}
	\begin{tabular}{l}
		$p \vee q$ \\
		$\neg p$ \\
	\hline
	$q$
	\end{tabular}
\end{flushleft}


	\item \enter

	\begin{flushleft}
	\begin{tabular}{l}
		$p$ \\
		\hline
	$p \vee q$
	\end{tabular}

\end{flushleft}

	\item \enter

	\begin{flushleft}
	\begin{tabular}{l}
		$p \Rightarrow q$ \\
		$q \Rightarrow r$ \\
	\hline
	$p \Rightarrow r$
	\end{tabular}

\end{flushleft}

\end{enumerate}

    \subsubsection*{Solution}

    \begin{enumerate}
    	\item \hspace{1em}

    \begin{center}
	\begin{tabular}{cc|ccc}
		$p$ & $q$ & $\lnot p$ & $p \lor q$ & $(\lnot p) \land (p \lor q)$ \\
		\hline
		T&T&F&T&F\\
		T&F&F&T&F\\
		F&\color{red}T&T&T&\color{red}T\\
		F&F&T&F&F\\
	\end{tabular}
    \end{center}

    On remarque que quand $(\lnot p) \land (p \lor q)$ est vrai, $q$ est vrai.

	\item  \hspace{1em}
    \begin{center}
    	\begin{tabular}{cc|c}
    		$p$ & $q$ & $p \lor q$ \\
    		\hline
    		\color{red}T&T&\color{red}T\\
    		\color{red}T&F&\color{red}T\\
    		F&T&T\\
    		F&F&F\\
    	\end{tabular}
    \end{center}

    On remarque que quand $p$ est vrai, $p \lor q$ est vrai.

	\item  \hspace{1em}
    \begin{center}
    	\begin{tabular}{ccc|cccc}
    		$p$ & $q$ & $r$ & $(p \Rightarrow q)$ & $\land$ & $(q \Rightarrow r)$ & $(p \Rightarrow r)$ \\
    		\hline
    		T&T&T&T&\color{red}T&T&\color{red}T\\
    		T&T&F&T&F&F&F\\
    		T&F&T&F&F&T&T\\
    		T&F&F&F&F&T&F\\
    		F&T&T&T&\color{red}T&T&\color{red}T\\
    		F&T&F&T&F&F&T\\
    		F&F&T&T&\color{red}T&T&\color{red}T\\
    		F&F&F&T&\color{red}T&T&\color{red}T\\
    	\end{tabular}
    \end{center}

    On remarque que quand $(p \Rightarrow q) \land (q \Rightarrow r)$ est vrai, $(p \Rightarrow r)$ est vrai.


\end{enumerate}
\subsection*{Exercice 3}
Démontrez que les arguments suivants ne sont pas valides.

\begin{enumerate}
	\item \enter

	\begin{flushleft}
	\begin{tabular}{l}
		$p \vee q$ \\
		$\neg p$ \\
	\hline
	$\neg q$
	\end{tabular}

\end{flushleft}

	\item \enter

	\begin{flushleft}
	\begin{tabular}{l}
		$p \Leftrightarrow q$ \\
		$p \Rightarrow r$ \\
	$r$ \\
	\hline
	$p$
	\end{tabular}

\end{flushleft}


% 	\item \enter
%
% 	\begin{flushleft}
% 	\begin{tabular}{l}
% 		$p \vee q$ \\
% 		$q$ \\
% 	\hline
% 	$p$
% 	\end{tabular}
%
% \end{flushleft}

	\item \enter

	\begin{flushleft}
	\begin{tabular}{l}
		$p \Rightarrow q$ \\
		$q \Rightarrow p$ \\
	\hline
	$p \wedge q$
	\end{tabular}

\end{flushleft}

% 	\item \enter
%
% 	\begin{flushleft}
% 	\begin{tabular}{l}
% 		$p \Rightarrow q$ \\
% 		$q$ \\
% 	\hline
% 	$p$
% 	\end{tabular}
% \end{flushleft}

\end{enumerate}

    \subsubsection*{Solution}
    Il y a deux façons de résoudre cet exercice.
    Nous faisons avec le premier un exemple de ces deux méthodes.
\begin{enumerate}
	\item
    Tout d'abord, l'algorithme de preuve :

    \begin{center}
    \begin{tabular}{|l|l|}
    \hline
    1. $p \lor q$ & Prémisse \\
    2. $\lnot p$ & Prémisse \\
    \hspace{0.5cm} 3. $\lnot q$ & Hypothèse \\
    \hspace{0.5cm} 4. $p$ & Syllogisme disjoint (1, 3) \\
    5. $q$ & Réduction à l'absurde \\
    \hline
    \end{tabular}
    \end{center}

    Ensuite une table de vérité :

    \begin{center}
    	\begin{tabular}{cc|ccc|c}
    		$P$ & $Q$ & $(P \lor Q) $ & $\land$ & $\neg P$ & $\neg Q$ \\
    		\hline
    		T & T & T & F & F & F\\
    		T & F & T & F & F & T\\
    		F & T & T & \color{red}T & T & \color{red}F\\
    		F & F & F & F & T & T\\
    	\end{tabular}
    \end{center}

    On constate que lorsque $(P \lor Q) \land \neg P$ est vrai, $\neg Q$ est faux.

	\item
    Il suffit de trouver une interprétation où c'est faux. Ici on peut prendre :

    \begin{flalign*}
    VAL_{I}(p) &= \text{false}\\
    VAL_{I}(q) &= \text{false}\\
    VAL_{I}(r) &= \text{true}\\
    \end{flalign*}

    Les prémisses sont vraies, pas la conclusion.

	\item
    Il suffit de trouver une interprétation où c'est faux. Ici on peut prendre :

    \begin{flalign*}
    VAL_{I}(p) &= \text{false}\\
    VAL_{I}(q) &= \text{false}\\
    \end{flalign*}

    Les prémisses sont vraies, pas la conclusion.
\end{enumerate}

\subsection*{Exercice 4}
Pour chaque ensemble de prémisses, démontrez la conclusion qui suit. Faites attention à bien identifier les
lois logiques et les règles d'inférence utilisées.
\begin{enumerate}

% \item Premisses: $A \vee B \vee C, \ \neg A, \ \neg B$ \\
% 			Conclusion: $C$
% \item Premisses: $(p \wedge q) \vee r$ \\
%       Conclusion: $\neg q \Rightarrow r$
\item Premisses: $p \Rightarrow q$, \ $q \Rightarrow r$ \\
      Conclusion: $p \Rightarrow r$
\item Premisses: $p \Rightarrow q$, \ $r \Rightarrow t$, \ $q \vee t \Rightarrow u$, \ $\neg u$ \\
      Conclusion: $\neg p \wedge \neg r$
\item Premisses: $\neg p \Rightarrow (q \Rightarrow r)$, \ $t \vee \neg r \vee u$, \ $p \Rightarrow t$, \ $\neg t$ \\
      Conclusion: $q \Rightarrow u$
% \item Premisses: $A \Rightarrow B, \ C \Rightarrow D, \ (B \vee D) \Rightarrow E, \ \neg E$ \\
% 			Conclusion: $\neg A \wedge \neg C$


\item Premisses: $p \Rightarrow \neg q, \ q \vee r \vee s, \ \neg r \vee s \Rightarrow p, \ \neg r$  \\
			Conclusion: $s$


\item Premisses: $\neg p \Rightarrow (q \Rightarrow r), \ s \vee \neg r \vee t, \ p \Rightarrow s, \ \neg s$ \\
			Conclusion: $q \Rightarrow t$


%   \item Premisses: $p \wedge q$ \\
%        ronclusion: $p \vee q$
%  \item Premisses: $(p \wedge q) \vee r$ \\
%        ronclusion: $r \vee q$
 \item Premisses: $\neg ( \neg p \wedge q), \ \neg (\neg q \vee r)$ \\
       Conclusion: $p$
%  \item Premisses: $p \vee q$ \\
%        ronclusion: $p \vee \neg \neg q$
 \item Premisses: $p \vee q, \ \neg q \vee r$ \\
       Conclusion: $p \vee r$
 \item Premisses: $(p \wedge q) \vee (r \wedge s), \ (q \wedge r) \vee (s \wedge t)$ \\
       Conclusion: $r \vee (p \wedge t)$

\end{enumerate}

    \subsubsection*{Solution}
    \begin{enumerate}

    %%% 4.1 %%%
	\item  \hspace{1em}
    \begin{center}
    \begin{tabular}{|l|l|}
    \hline
    1. $p \Rightarrow q$ & Prémisse \\
    2. $q \Rightarrow r$ & Prémisse \\
    \hspace{0.5cm} 3. $p$ & Hypothèse \\
    \hspace{0.5cm} 4. $q$ & Modus ponens (1, 3) \\
    \hspace{0.5cm} 5. $r$ & Modus ponens (2, 4) \\
    6. $p \Rightarrow r$ & Théorème de déduction (3, 5) \\
    \hline
    \end{tabular}
    \end{center}

    %%% 4.2 %%%
	\item  \hspace{1em}
    \begin{center}
    \begin{tabular}{|l|l|}
    \hline
    1. $p \Rightarrow q$ & Prémisse \\
    2. $r \Rightarrow t$ & Prémisse \\
    3. $q \lor t \Rightarrow u $ & Prémisse \\
    4. $\lnot u$ & Prémisse \\
    5. $\lnot (q \lor t)$ & Modus tollens (3, 4) \\
    6. $\lnot q \land \lnot t$ & De Morgan (5) \\
    7. $\lnot t$ & Simplification (6) \\
    8. $\lnot r$ & Modus tollens (2, 7) \\
    9. $\lnot q$ & Simplification (6) \\
    10. $\lnot p$ & Modus tollens (1, 9) \\
    11. $\lnot r \land \lnot p$ & Conjonction (8, 10) \\
    \hline
    \end{tabular}
    \end{center}

    %%% 4.3 %%%
	\item  \hspace{1em}
    \begin{center}
    \begin{tabular}{|l|l|}
    \hline
    1. $\lnot p \Rightarrow (q \Rightarrow r)$ & Prémisse \\

    2. $t \lor \lnot r \lor u$ & Prémisse \\
    3. $p \Rightarrow t$ & Prémisse \\
    4. $\lnot t$ & Prémisse \\
    5. $\lnot p$ & Modus tollens (3, 4) \\
    6. $q \Rightarrow r$ & Modus ponens (1, 5) \\
    7. $\lnot r \lor u$ & Syllogisme disjoint (2, 4) \\
    \hspace{0.5cm} 8. $q$ & Hypothèse \\
    \hspace{0.5cm} 9. $r$ & Modus ponens (6, 8) \\
    \hspace{0.5cm} 10. $u$ & Syllogisme disjoint (7,9) \\
    11. $q \Rightarrow u$ & Déduction (8, 10) \\
    \hline
    \end{tabular}
    \end{center}

    %%% 4.4 %%%
	\item  \hspace{1em}
    \begin{center}
    \begin{tabular}{|l|l|}
    \hline
    1. $p \Rightarrow \lnot q$ & Prémisse \\
    2. $q \lor r \lor s$ & Prémisse \\
    3. $\lnot r \lor s \Rightarrow p$ & Prémisse \\
    4. $\lnot r$ & Prémisse \\
    5. $q \lor s$ & Syllogisme disjoint (2, 4) \\
    6. $\lnot r \lor s$ & Addition(4) \\
    7. $p$ & Modus ponens (3, 6) \\
    8. $\lnot q$ & Modus ponens (1, 7) \\
    9. $s$ & Syllogisme disjoint (5, 8) \\
    \hline
    \end{tabular}
    \end{center}

    %%% 4.5 %%%
	\item  \hspace{1em}
    \begin{center}
    \begin{tabular}{|l|l|}
    \hline
    1. $\lnot p \Rightarrow (q \Rightarrow r)$ & Prémisse \\
    2. $s \lor \lnot r \lor t$ & Prémisse \\
    3. $p \Rightarrow s$ & Prémisse \\
    4. $\lnot s$ & Prémisse \\
    5. $\lnot p$ & Modus tollens (3, 4) \\
    6. $q \Rightarrow r$ & Modus ponens (1, 5) \\
    7. $\lnot r \lor t$ & Syllogisme disjoint (2, 4) \\
    \hspace{0.5cm} 8. $q$ & Hypothèse \\
    \hspace{0.5cm} 9. $r$ & Modus ponens (6, 8) \\
    \hspace{0.5cm} 10. $t$ & Syllogisme disjoint (7,9) \\
    11. $q \Rightarrow t$ & Déduction (8, 10) \\
    \hline
    \end{tabular}
    \end{center}

    %%% 4.6 %%%
	\item  \hspace{1em}
    \begin{center}
    \begin{tabular}{|l|l|}
    \hline
    1. $\lnot ( \lnot p \land q)$ & Prémisse \\
    2. $\lnot ( \lnot q \lor r)$ & Prémisse \\
    3. $p \lor \lnot q$ & De Morgan (1) \\
    4. $q \land \lnot r$ & De Morgan (2) \\
    5. $q$ & Simplification (4) \\
    6. $p$ & Syllogisme disjoint (3, 5) \\
    \hline
    \end{tabular}
    \end{center}

    %%% 4.7 %%%
	\item  \hspace{1em}
    \begin{center}
    \begin{tabular}{|l|l|}
    \hline
    1. $p \lor q$ & Prémisse \\
    2. $\lnot q \lor r$ & Prémisse \\
    3. $q \Rightarrow r$ & Loi de l'implication (2) \\
    \hspace{0.5cm} 4. $\lnot(p \lor r)$ & Hypothèse \\
    \hspace{0.5cm} 5. $\lnot p \land \lnot r$ & De Morgan (4) \\
    \hspace{0.5cm} 6. $\lnot p$ & Simplification (5) \\
    \hspace{0.5cm} 7. $\lnot r$ & Simplification (5) \\
    \hspace{0.5cm} 8. $q$ & Syllogisme disjoint (1, 6) \\
    \hspace{0.5cm} 9. $\lnot q$ & Syllogisme disjoint (2, 7) \\
    10. $p \lor r$ & Réduction à l'absurde (4)\\
    \hline
    \end{tabular}
    \end{center}

    %%% 4.8 %%%
	\item  \hspace{1em}
    \begin{center}
    \begin{tabular}{|l|l|}
    \hline
    1. $(p \land q) \lor (r \land s)$ & Prémisse \\
    2. $(q \land r) \lor (s \land t)$ & Prémisse \\
    3. $(p \lor r) \land (p \lor s) \land (q \lor s) \land (q \lor r)$ & Distributivité (1) \\
    4. $(q \lor s) \land (q \lor t) \land (r \lor s) \land (r \lor t)$ & Distributivité (2) \\
    5. $p \lor r$ & Simplification (3) \\
    6. $r \lor t$ & Simplification (4) \\
    7. $(p \lor r) \land (r \lor t)$ & Conjonction (5, 6) \\
    8. $r \lor (p \land t)$ & Distributivité (7) \\
    \hline
    \end{tabular}
    \end{center}
\end{enumerate}

\subsection*{Exercice 5}
Pour chaque ensemble de prémisses, démontrez la conclusion qui suit. Faites attention à bien identifier les
lois logiques et les règles d'inférence utilisées.
\begin{enumerate}
\item Premisses: \\
      Conclusion: $p \vee \neg (p \wedge q)$
\item Premisses: \\
      Conclusion: $(p \wedge q) \vee \neg p \vee \neg q$
\item Premisses: \\
      Conclusion: $\neg p \vee \neg (\neg q \wedge (\neg p \vee q))$
\end{enumerate}

    \subsubsection*{Solution}
    \begin{enumerate}

	\item  \hspace{1em}
    \begin{center}
    \begin{tabular}{|l|l|}
    \hline
    \hspace{0.5cm} 1. $\neg p$ & Hypothèse \\
    \hspace{0.5cm} 2. $\neg p \lor \neg q$ & Addition (1) \\
    \hspace{0.5cm} 3. $\neg \neg(\neg p \lor \neg q)$ & Négation (2) \\
    \hspace{0.5cm} 4. $\neg(p \land q)$ & De Morgan (3)\\
    5. $\neg p \Rightarrow \neg(p \land q)$ & Déduction (1, 4) \\
    6. $\neg \neg p \lor \neg(p \land q)$ & Implication (5)\\
    7. $p \lor \neg(p \land q)$ & Double négation (6)\\
    \hline
    \end{tabular}
    \end{center}

	\item  \hspace{1em}
    \begin{center}
    \begin{tabular}{|l|l|}
    \hline
    \hspace{0.5cm} 1. $\neg (p \land q)$ & Hypothèse \\
    \hspace{0.5cm} 2. $\neg p \lor \neg q$ & De Morgan (1) \\
    3. $\neg (p \land q) \Rightarrow (\neg p \lor \neg q)$ & Déduction (1, 2) \\
    4. $(\neg \neg(p \land q)) \lor (\neg p \lor \neg q)$ & Implication (3) \\
    5. $(p \land q) \lor (\neg p \lor \neg q)$ & Double négation (4) \\
    \hline
    \end{tabular}
    \end{center}

	\item  \hspace{1em}
    \begin{center}
    \begin{tabular}{|l|l|}
    \hline
    \hspace{0.5cm} 1. $p$ & Hypothèse \\
    \hspace{0.5cm} 2. $p \lor q$ & Addition (1) \\
    \hspace{0.5cm} 3. $(\neg p \lor q) \Leftrightarrow q$ & ex. 2.1 \\
    \hspace{0.5cm} 4. $((\neg p \lor q) \land \neg q) \Leftrightarrow (q \land \neg q)$ & Mystification \\
    \hspace{0.5cm} 5. $\neg ((\neg p \lor q) \land \neg q)$ & Contradiction \\
    6. $p \Rightarrow \neg ((\neg p \lor q) \land \neg q)$ & Déduction (1, 5) \\
    7. $\neg p \lor \neg ((\neg p \lor q) \land \neg q)$ & Implication (6) \\
    \hline
    \end{tabular}
    \end{center}

    \end{enumerate}
