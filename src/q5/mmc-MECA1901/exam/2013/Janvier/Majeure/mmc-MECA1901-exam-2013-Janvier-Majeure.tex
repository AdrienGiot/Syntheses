\documentclass[en]{../../../../../../eplexam}

\usepackage{../../../mmc-MECA1901-exam}

\hypertitle{Mécanique des milieux continus}{5}{MECA}{1901}{2013}{Janvier}{Majeure}
{Vincent Schellekens\and Antoine de Comité\and Aurélien Pignolet\and Mamadou Segpa\and Philippe Greiner}
{Philippe Chatelain et Issam Doghri}

\section{Théorie}
\begin{enumerate}
\item Enoncer et démontrer le théorème du transport de Reynolds(variante de base). Montrer ensuite comment le principe de conservation de la masse permet d'établir une variante simplifiée de ce théorème.

\begin{solution}
Théorème du transport de Reynolds: \textit{ Soit $\Omega(t)$ une portion quelconque se déformant au cours du temps. On suppose que sa frontière $\partial \Omega (t)$ a pour normale unitaire sortante $n(t)$. Pour une quantité physique arbitraire $\phi(t)$ associée à un point matériel appartenant au volume matériel ou à la frontière on a:}
\begin{equation}
\frac{D}{Dt}I(t)=\int_{\Omega_{0}}\left(\frac{\partial\phi(X,t)}{\partial t}J+\phi(X,t)\frac{\partial J}{\partial t}\right)dV
\end{equation}
Avec $J=\frac{\partial x}{\partial X} \rightarrow \frac{\partial J}{\partial t} = J\nabla.v$
En remplacant dans l'équation précédente on obtient:
\begin{equation}
\frac{D}{Dt}I(t)=\int_{\Omega_{0}}J\left(\frac{\partial\phi(X,t)}{\partial t}+\phi(X,t)\nabla.v\right)dV
\end{equation}
En utilisant la formule de $\frac{D}{Dt}$ et en repassant en coordonnées eulérienne:
\begin{equation}
\frac{D}{Dt}I(t)=\int_{\Omega(t)}\left(\frac{D\phi}{Dt}+\phi\nabla.v\right)dv
\end{equation}
En appliquant la dérivée matérielle:
\begin{equation}
\frac{D}{Dt}I(t)=\int_{\Omega(t)}\left(\frac{\partial\phi}{\partial t}+v.\nabla\phi+\phi\nabla.v\right)dv=\int_{\Omega(t)}\left(\frac{\partial\phi}{\partial t}+\nabla.(\phi v)\right)dv
\end{equation}
Par le théoreme de la divergence on obtient enfin :
\begin{equation}
\frac{D}{Dt}I(t)=\int_{\Omega(t)}\frac{\partial\phi}{\partial t}dv + \oint_{\partial\Omega(t)}\phi v.\hat{n}ds
\end{equation}
En utilisant le théorême de la conservation de la masse et le TTR (avec $\phi = \rho$):
\begin{equation}
\frac{D}{Dt}M(t)=\int_{\Omega(t)}\phi dv=0
\end{equation}
\begin{equation}
\frac{D}{Dt}M(t)=\int_{\Omega(t)}\left(\frac{\partial\rho}{\partial t}+\nabla.(\rho v)\right)dv=0 \rightarrow \frac{\partial\rho}{\partial t}+\nabla.(\rho v)=0
\end{equation}
\end{solution}

\item Enoncer et démontrer l'interprétation géométrique de $D_{33}$ et $D_{21}$ en précisant leur dimension physique et leur unité

\begin{solution}
Voir question précédentes pour interprétation physique.
Unité:
\begin{align*}
\left[D\right]&=\frac{m}{ms}\\
& = s^{-1}
\end{align*}
\end{solution}

\item Définir le concept de ligne d'émission et établir les formules qui permettent d'en faire le calcul en partant d'une représentation lagrangienne du mouvement

\begin{solution}
Une ligne d'émission à travers un point $x^*$ à un instant $\tau$ donné est la courbe constituée de toutes les particules qui sont passées par le point $x^*$.
On peut l'obtenir à partir de la définition lagrangienne en utilisant:
\begin{equation}
x=\chi(\chi^{-1}(x^*,t),\tau)
\end{equation}
\end{solution}

\item Enoncer le principe de conservation du moment de quantité de mouvement et établir sa forme locale. Expliquer quelles en sont les conséquences en termes de contraintes et de directions principales.

\begin{solution}
Le moment de quatité de mouvement est défini comme étant :
\begin{equation}
\textbf{L} = \int_{\Omega}\rho \textbf{x} \wedge \textbf{v} \mathrm{d}V
\end{equation}

La conservation de la quantité de mouvement nous dit que la somme des couples agissant sur le milieu est égal à la dérivée particulaire du moment de quantité de mouvement :

\begin{equation}
\PTDeriv{\textbf{L}}{t} = \sum \textbf{M} = \sum \textbf{M}^{distance} + \sum \textbf{M}^{contact}
\end{equation}

Donc :
\begin{equation}
\PTDeriv{ }{t} \Integr{\Omega}{}{\rho \textbf{x} \wedge \textbf{v}}{V} = \Integr{\Omega}{}{\textbf{x}\wedge \rho \textbf{f}}{V} + \oint_{\partial \Omega} \textbf{x} \wedge \textbf{t} \mathrm{d}s
\end{equation}

Pour établir la forme locale, on applique le Théorème de Transport de Reynolds (l'énoncé pour une quatité massique) sur le membre de gauche. On rappele que $\PTDeriv{\textbf{x}}{t} = \textbf{v}$ et que $\textbf{v} \wedge \textbf{v} = 0$.
\begin{equation}
\PTDeriv{ }{t} \Integr{\Omega}{}{\rho \textbf{x} \wedge \textbf{v}}{V} = \Integr{\Omega}{}{\rho \PTDeriv{ }{t}(\textbf{x} \wedge \textbf{v})}{V} = \Integr{\Omega}{}{\rho \textbf{x} \wedge \PTDeriv{\textbf{v}}{t}}{V}
\end{equation}

On peut aussi calculer le couple associé aux contraintes de contact, en utilisant successivment : l'expression de la force tangentielle en fonction du tenseur des contraintes, la notation indicielle pour le produit vectoriel, Green-Ostrogradski et pour finir la dérivée d'un produit :
\begin{equation}
\oint_{\partial \Omega} \textbf{x} \wedge \textbf{t} \mathrm{d}s = \oint_{\partial \Omega} \textbf{x} \wedge \hat{\textbf{n}} \cdot \uuline{\sigma} \mathrm{d}s = \oint_{\partial \Omega} e_{ijk} x_j n_l  \sigma_{lk} \mathrm{d}s = \Integr{\Omega}{}{\PDeriv{}{x_l} (e_{ijk} x_j \sigma_{lk})}{V}
\end{equation}
\begin{equation}
\oint_{\partial \Omega} \textbf{x} \wedge \textbf{t} \mathrm{d}s = \Integr{\Omega}{}{(e_{ijk}( \PDeriv{x_j}{x_l} \sigma_{lk} + x_j \PDeriv{\sigma_{lk}}{x_l})}{V}
\end{equation}

En remettant tout ensemble, on obtient :
\begin{equation}
\Integr{\Omega}{}{\textbf{x} \wedge \left( \rho \PTDeriv{\textbf{v}}{t} - \rho \textbf{f} - \nabla \cdot \uuline{\sigma} \right)}{V} + \Integr{\Omega}{}{e_{ijk}\sigma_{jk}}{V} = 0
\end{equation}

Par conservation de quantité de mouvement, la première intégrale est nulle. Comme Ceci doit être vrai $\forall \Omega$, on a la forme locale :

\begin{equation}
e_{ijk}\sigma_{jk} = 0 \rightarrow \uuline{\sigma} = \uuline{\sigma}^T
\end{equation}

Le tenseur des contraintes est donc symétrique. Par conséquent, la contrainte dans la direction $\Base{1}$ d'une face de normale $\Base{2}$ est égale à la contrainte dans la direction $\Base{2}$ d'une face de normale $\Base{1}$. Ces contraintes sont nulles (si la direction 1 est différente de la 2) pour les directions principales.
\end{solution}


%Vicnent, 'jai fini je vais dodo k bonne nuit :) ya du paté plus haut je vais réparer
\item Enoncer, démontrer et interpréter le théorème de l'énergie cinétique.

\begin{solution}
Nous cherchons à dériver l'expression du théorème de l'énergie cinétique, pour ce faire partons de l'expression de la conservation de la quantité de mouvement.
\begin{equation*}
\rho \PTDeriv{\bm{v}}{t}=\rho \bm{f}+\nabla \cdot \bm{\sigma}
\end{equation*}
Si on effectue le produit scalaire de cette expression avec le champ de vitesse, on obtient
\begin{equation*}
\rho \PTDeriv{\bm{v}}{t}\cdot \bm{v}=\rho \bm{f}\cdot \bm{v}+(\nabla \cdot \bm{\sigma})\cdot \bm{v}
\end{equation*}
On remarque que, grâce à l'expression de la dérivée d'un produit, on peut identifier l'énergie cinétique massique dans le terme de gauche \dots
\begin{equation*}
\PTDeriv{\bm{v}}{t}\cdot \bm{v}=\frac{1}{2}\PTDeriv{(\bm{v}\cdot \bm{v})}{t}
\end{equation*}
 \dots le premier terme de droite peut, quant à lui, être interprété comme la puissance des forces volumiques.
Réintégrons cette expression sur un volume matériel, on obtient
\begin{equation*}
\int_{\Omega(t)}\rho \frac{1}{2}\PTDeriv{(\bm{v}\cdot \bm{v})}{t}dv=\int_{\Omega(t)}(\rho\bm{f}\cdot \bm{v}+(\nabla \cdot \bm{\sigma})\cdot \bm{v})dv
\end{equation*}
Intéressons-nous aux termes des contraintes, celui-ci peut être décomposé en deux parties
\begin{equation*}
\nabla\cdot (\bm{\sigma}^T\cdot \bm{v})-\bm{\sigma}:(\nabla \bm{v})^T
\end{equation*}

En appliquant le théorème de la divergence, Green-Ostrogradski (pour ceux qui parviennent à le prononcer :D) pour le premier terme, on retrouve la puissance des forces de contact
\begin{equation*}
\int_{\Omega(t)}\nabla\cdot (\bm{\sigma}^T\cdot \bm{v})dv=\oint_{\partial \Omega(t)}\bm{t}\cdot \bm{v}ds
\end{equation*}
Cette ligne est obtenue grâce à la définition des forces de contact.

Revenons au second terme qu'il nous faut décomposer (il s'agit du double produit contracté). Nous pouvons décomposer les deux tenseurs impliqués en partie symétrique et en partie antisymétrique, on a alors
\begin{align*}
\bm{\sigma}=\bm{\sigma}^{\text{sym}}+\bm{\sigma}^{\text{anti-sym}}\\
(\nabla \bm{v})^T=\cal{\bm{D}}+\bm{\text{$\Omega$}}
\end{align*}
Si on regarde ce qu'il advient du double produit contracté suite à cette décomposition, on peut dire que, sur les quatre termes qu'on aurait dû y retrouver, il n'en restera que deux (c'est presque comme à Koh-Lanta :). Seuls resteront ceux qui font intervenir deux tenseurs symétriques ou deux tenseurs antisymétriques. On peut donc réécrire
\begin{equation*}
  \bm{\sigma}:(\nabla \bm{v})^T=\bm{\sigma}^{\text{sym}}:\cal{\bm{D}}+\bm{\sigma}^{\text{anti-sym}}:\bm{\text{$\Omega$}}
\end{equation*}
% J4ai pas fini canard il dit que ca bug ici mais je vois pas pq voila reparé



Cependant on peut en plus laisser tomber le dernier terme car il fait intervenir des puissances qui ne sont non nulles que s'il n'y a pas de couple volumiques. Nous supposerons que nous avons affaire à ce cas-là. Réécrivons donc la formulation finale de la conservation de l'énergie cinétique (le TTR est utilisé ici parce qu'il est trop bonne!)
\begin{equation*}
  \frac{D}{Dt}\int_{\text{$\Omega(t)$}}\rho \frac{\bm{v}\cdot \bm{v}}{2}dv=\int_{\Omega(t)}\rho\bm{f}\cdot \bm{v}dv+\oint_{\partial\Omega(t}\bm{t}\cdot\bm{v}ds-\int_{\Omega(t)}\bm{\sigma}:{\cal\bm{D}}dv
\end{equation*}

On peut interpréter la somme du membre de droite comme ceci
\begin{equation*}
\frac{D}{Dt}K=W^d+W^c-W^i
\end{equation*}
Ces termes représentent respectivement les puissances des forces volumiques, des forces de contact et des efforts internes.
\end{solution}
\end{enumerate}


\end{document}
