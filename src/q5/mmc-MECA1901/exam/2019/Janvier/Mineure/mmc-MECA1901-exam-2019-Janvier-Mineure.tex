\documentclass[fr]{../../../../../../eplexam}

\hypertitle{Mécanique des milieux continus}{5}{MECA}{1901}{2019}{Janvier}{Mineure}
{Nathan Jacques}
{Issam Doghri \and Philippe Chatelain}

\renewcommand{\b}[1]{\mathbf{#1}}
\newcommand{\tend}[1]{\oalign{\mbox{\boldmath$#1$}\crcr\hidewidth$\scriptscriptstyle\sim$\hidewidth}}

\section{Théorie}

\subsection{Théorie ID}
\subsubsection{Vrai ou faux, justifier}
Soient $\b{C} = \b{F}^T \cdot \b{F}$, $\b{D}$ le tenseur
taux de déformation et $\frac{D}{Dt}$ la dérivée particulaire,
\begin{enumerate}
  \item $\frac{D\b{C}}{Dt} = \b{F}^T \cdot \b{D} \cdot \b{F}$
  \item $\frac{D}{Dt}(\text{tr}\b{C}) = \text{tr} \left( \frac{D\b{C}}{Dt} \right)$
  \item $\frac{D}{Dt}(\text{det}\b{C}) = \text{det} \left( \frac{D\b{C}}{Dt} \right)$
\end{enumerate}

\nosolution

\subsubsection{Notations}
On donne l'équation aux dérivées partielles suivante (dite de Cahn-Hilliard et
qui décrit un changement de phase):

$$\frac{\partial c}{\partial t} = f(\b{x},t) - \nabla \cdot \{ -\b{M} \cdot \nabla \left[ g(c) - \nabla \cdot \left( \b{K} \cdot \nabla c \right) \right] \}$$

où $c(\b{x},t)$ désigne la concentration des phases, $f(\b{x},t)$ est un terme
source, $\b{M}(\b{x},t)$ et $\b{K}(\b{x},t)$ sont des tenseurs d'ordre 2
symétriques (modélisant respectivement la mobilité et l'énergie d'interface).

\begin{enumerate}
  \item Réécrire cette équation en utilisant la notation indicielle en coordonnées
  cartésiennes
  \item Réécrire le résultat du point 1 dans le acs suivant: $K_{ij} = k(t)\delta_{ij}$
  et $M_{ij} = m(t)\delta_{ij}$
  \item Réécrire le résultat du point 2 en utilisant uniquement le Laplacien dans
  les dérivées spatiales
\end{enumerate}

\nosolution

\subsection{Théorie PC}
\subsubsection{} En bref, qu'énonce le théorème de Green-Naghdi-Rivlin? Expliquez
\textbf{brièvement} comment le prouver.
\subsubsection{} Comment avons-nous étudié une discontinuité des champs se déplaçant
au sein d'un milieu continu? Appliquer ce traitement à la conservation de la quantité
de mouvement et donner un exemple physique où cette relation peut être utile.

\nosolution

\section{Applications}
\subsection{Solide: Cisaillement simple d'un solide hyper-élastique}
Soient $J = \text{det}\b{F}$, $\b{C} = \b{F}^T \cdot \b{F}$ et $\b{B} = \b{F} \cdot \b{F}^T$, un modèle hyper-élastique (dit "neo-Hookean") est défini comme suit:
$$ \tend{\sigma} = \frac{1}{J} \b{F} \cdot \left\{ \mu (\text{det}\b{C})^{-1/3} \b{1} + \left[ \kappa \left( \text{det}\b{C} - \sqrt{\text{det}\b{C}} \right) - \frac{\mu}{3} (\text{det}\b{C})^{-1/3} \text{tr}\b{C} \right] \b{C}^{-1} \right\} \cdot \b{F}^T $$
où $\kappa >0$ et $\mu >0$ des constantes matérielles.

\begin{enumerate}
  \item Exprimer $\tend{\sigma}$ en fonction de $\b{B}$ (et ses invariants), de $\kappa$ et de $\mu$ uniquement;
  \item Exprimer la trace de $\tend{\sigma}$ en fonction de $\kappa$ et $J$ uniquement;
  \item Montrer que la partie déviatorique de $\tend{\sigma}$ est donnée par: $$\text{dev}(\tend{\sigma}) = \mu f(J) \text{dev}(\b{B})$$
  où $f$ est une fonction de quelconque de $J$
\end{enumerate}

On considère un solide initialement cubique (de côté $L_0$ et occupant l'espace $0\leq X_1,X_2,X_3\leq L_0$) soumis à une transformation dite de cisaillement simple qui est décrite comme suit dans un repère orthonormé fixe $(O,\hat{e}_1,\hat{e}_2,\hat{e}_3)$:
$$x_1 = X_1 + \gamma (t)X_2 ; x_2=X_2 ; x_3 = X_3$$
où $\gamma (t)$ est une fonction continue et dérivable qui vérifie
$$\gamma (0)=0 \quad ; \quad \gamma (t) > 0, \forall t>0$$

\begin{enumerate}
  \item Calculer les matrices représentant $\b{F},\b{B}$ et le tenseur de Green-Lagrange;
  \item Calculer la matrice représentant $\tend{\sigma}$. Commentez.
  \item Que devient la facette initialement en $X_2 = L_0$? Calculer et représenter graphiquement les composantes normales et tangentielles des forces de contact surfaciques qui s'exercent sur la facette déformée. Commentez.
  \item Idem pour la facette initialement en $X_1 = L_0$.
  \item Calculer les contraintes principales
  \item Dessiner les cercles de Mohr et calculer la contrainte de cisaillement maximale
  \item Dans le cas de petites déformations, $\gamma (t)$ est petite et on néglige $(\gamma (t))^2$ ainsi que les termes d'ordre supérieur devant 1. Que deviennent les résultats de 1 à 6 dans ce cas? Quel est le lien avec l'élasticité linéaire et isotrope? Que signifient $\mu$ et $\kappa$?
\end{enumerate}

\nosolution

\subsection{Fluide: écoulement visqueux d'un fluide newtonien autour d'un cylindre}

\nosolution

\end{document}
