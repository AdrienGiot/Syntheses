\documentclass[fr]{../../../../../../eplexam}

\hypertitle{Probabilités et statistiques II}{5}{BIR}{1304}{2011}{Août}
{Florian Thuin\and Paul-Henri Callewaert}
{Patrick Bogaert}

\section{}
  
  Un agriculteur possède un champ carré de longueur $\mu$, dont il désire estimer la surface. Pour cela, il effectue 2 mesures indépendantes $X_{1}$ et $X_{2}$ de la longueur de son champs telles que ces mesures suivent une loi normale d'espérance $\mu$ et de variance $\sigma^{2}$. Pour estimer la surface, plusieurs possibilités s'offrent à lui. Il peut faire la moyenne des mesures et l'élever au carré, ou bien il peut élever les mesures au carré et prendre leur moyenne. Une troisième solution revient à multiplier les 2 mesures. Les trois estimateurs correspondants sont les suivants :
  
  \[
   \hat{\theta_{1}} = \frac{(X_{1} + X_{2})^{2}}{4}
  \]

  \[
   \hat{\theta_{2}} = \frac{(X_{1}^{2} + X_{2}^{2})}{2}
  \]
  
  \[
   \hat{\theta_{3}} = X_{1} * X_{2}
  \] \bigskip

\begin{solution}
  Calcul du biais du paramètre $\hat{\theta_{1}}$ : 
  
  \[
   E[\hat{\theta_{1}}] = E[\frac{(X_{1} + X_{2})^{2}}{4}] 
  \]
  \[
   = \frac{1}{4} * E[(X_{1} + X_{2})^{2}]
  \]
  \[
   = \frac{1}{4} * E[X_{1}^{2} + 2 X_{1} X_{2} + X_{2}^{2}]
  \]
  \[
   = \frac{1}{4} * (E[X_{1}^{2}] + E[2 X_{1} X_{2}] + E[X_{2}^{2}])
  \]
  \[
   = \frac{1}{4} * (E[X_{1}^{2}] - E^{2}[X_{1}] + E^{2}[X_{1}] + E[2 X_{1} X_{2}] + E[X_{2}^{2}] - E^{2}[X_{2}] + E^{2}[X_{2}])
  \]
  \[
   = \frac{1}{4} * (\sigma^{2} + E^{2}[X_{1}] + E[2 X_{1} X_{2}] + \sigma^{2} + E^{2}[X_{2}])
  \]
  \[
   = \frac{1}{4} * (\sigma^{2} + \mu^{2} + E[2 X_{1} X_{2}] + \sigma^{2} + \mu^{2})
  \]
  
  Puisque les 2 mesures sont indépendantes :
  \[
   = \frac{1}{4} * (\sigma^{2} + \mu^{2} + 2* E[X_{1}] * E[X_{2}] + \sigma^{2} + \mu^{2})
  \]
  \[
   = \frac{1}{4} * (\sigma^{2} + \mu^{2} + 2* \mu * \mu + \sigma^{2} + \mu^{2})
  \]
  \[
   = \frac{1}{4} * (\sigma^{2} + \mu^{2} + 2* \mu^{2} + \sigma^{2} + \mu^{2})
  \]
  \[
   = \mu^{2} + \frac{\sigma^{2}}{2}
  \]
  
  Le biais est donc de $\frac{\sigma^{2}}{2}$

  
  \bigskip
  
  Calcul du biais du paramètre $\hat{\theta_{2}}$ :

   \[
   E[\hat{\theta_{2}}] = E[\frac{X_{1}^{2} + X_{2}^{2}}{2}] 
  \]
  \[
   = \frac{1}{2} * E[X_{1}^{2} + X_{2}^{2}]
  \]
  \[
   = \frac{1}{2} * (E[X_{1}^{2}] + E[X_{2}^{2}])
  \]
  \[
   = \frac{1}{2} * (E[X_{1}^{2}] - E^{2}[X_{1}] + E^{2}[X_{1}] + E[X_{2}^{2}] - E^{2}[X_{2}] + E^{2}[X_{2}])
  \]
  \[
   = \frac{1}{2} * (\sigma^{2} + E^{2}[X_{1}] + \sigma^{2} + E^{2}[X_{2}])
  \]
  \[
   = \frac{1}{2} * (\sigma^{2} + \mu^{2} + \sigma^{2} + \mu^{2})
  \]
  \[
   = \frac{1}{2} * (2 *\sigma^{2} + 2 * \mu^{2})
  \]
  \[
   = \sigma^{2} + \mu^{2}
  \]
  
  Le biais est donc $\sigma^{2}$
  \bigskip
  
  Calcul du biais du paramètre $\hat{\theta_{1}}$ :
  
  \[
   E[\hat{\theta_{1}}] = E[X_{1} X_{2}]
  \]
  \[
   = E[X_{1}] * E[X_{2}] + Cov[X_{1} X_{2}]
  \]
  
  Puisque les deux observations sont indépendantes, la corrélation vaut zéro.
  \[
   = \mu^{2}
  \]
  
  Cet estimateur est sans biais.
  
\end{solution}

\section{}
  
  Un chercheur étudie l'effet d'un raccourcisseur de paille sur les céréales. Un raccourcisseur de paille est une molécule, fort utilisée en agriculture, qui agit sur la croissance de la plante afin de limiter son développement en hauteur. Dans une expérience, il étudie la différence entre la hauteur des plantes réparties entre 2 groupes. Le premier groupe n'est pas traité et sert de témoin. Le second groupe est traité à l'aide du produit. A la fin de la période de croissance, le chercheur a mesuré la hauteur (en mètres) de quelques plantes choisies au hasard dans chaque groupe. Voici ses résultats : \bigskip
  
  \begin{tabular}{l|cccccccc}
      Plante n\textdegree & 1 & 2 & 3 & 4 & 5 & 6 & 7 & 8 \\
      \hline
      Témoin & 1.37 & 1.01 & 1.36 & 1.48 & 1.34 & 1.46 & 1.21 & 1.43 \\
      Traités & 1.21 & 1.00 & 1.14 & 1.33 & 1.07 & 1.10 & 1.04 & 1.28 \\
  \end{tabular} \bigskip
  
  \begin{enumerate}
   \item Donnez un intervalle de confiance à la hauteur moyenne d'une plante non traitée.
   \item La variance de la hauteur peut-elle être considérée comme identique entre les plantes témoins et les plantes traitées ?
   \item Le traitement induit-il une diminution de la taille des plantes ? Justifiez.
  \end{enumerate}
  
  Pour chacun des tests et intervalles de confiance, considérez le niveau de confiance 1 - $\alpha$ = 0.95.
 
\begin{solution}
  \paragraph{1} Pour la hauteur moyenne dans un groupe, on est dans le cas où on cherche une moyenne avec une variance inconnue.
  
  \[
   \mu \in [\overline{X} \pm t^{(n-1)}_{1-\alpha/2} * \sqrt{\frac{s^{2}}{n}}]
  \]
  
  \[
   \overline{X} = \frac{1}{n} \sum_{t=1}^{N} X_{i}
  \]
  \[
   = \frac{1}{8} * 10,66 = 1,3325
  \]
  
  \[
   s^{2} = \frac{1}{(n-1)} * \sum_{t=1}^{N} (X_{i} - \overline{X})^{2}
  \]
  \[
   s^{2} = \frac{1}{7} * 0,16875 = 0,024
  \]
  \[
   t^{7}_{0,975} = 2,365
  \]
  \[
   \mu \in [1,3325 \pm 2,365 * \sqrt{\frac{0,024}{8}}]
  \]
  \[
   \mu \in [1,203;1,462]
  \]

  \paragraph{2} Pour que la variance de la hauteur des plantes traitées soit considérée identique à celle des plantes non-traitées, il faut que, $H_{0} : \frac{\sigma_{2}^{2}}{\sigma_{1}^{2}} = \gamma_{0}$ où $\gamma_{0} = 1$ dans notre cas.
  
  Considérons le problème de test comme étant,
	  
  \[
   H_{0} \equiv \frac{\sigma_{2}^{2}}{\sigma_{1}^{2}} = 1
   \]
   \[
   H_{1} \equiv \frac{\sigma_{2}^{2}}{\sigma_{1}^{2}} \neq 1
  \]
  
  On calcule que,
  
	\[
	n_{1} = n_{2} = 8, \overline{X}_{1} = 1,3325, \overline{X}_{2} = 1,14625
	\]
	\[
	S^{2}_{1} = \frac{1}{(n-1)} * \sum_{t=1}^{N} (X_{i} - \overline{X})^{2} = \frac{1}{7}*0,16875 = 0,0241
	\]
	\[
	S^{2}_{2} = \frac{1}{(n-1)} * \sum_{t=1}^{N} (X_{i} - \overline{X})^{2} = \frac{1}{7}*0,0313171 = 0,00447388
	\]
	
	On calcule la valeur observée de la statistique :
	
	\[
	F_{obs} = \frac{S^{2}_{1}}{S^{2}_{2}} = \frac{0,0241}{0,00447388} = 5,38788
	\]
	
	Nous remarquons que $S^{2}_{1}>S^{2}_{2}$. Nous allons donc chercher, pour $\alpha=0,05$, le quantile d'ordre $1-\alpha$ dans la distribution de Fisher-Snedecor de paramètre (7, 7),
	\[
	F_{0,95}(7,7) = 3,787 < 5,38788
	\]\\
	Il y a donc rejet de l'hypothèse de $H_{0}$ pour $\alpha=0,05$. Les variances ne sont donc pas considérées comme égales.
	\newline
	Imaginons maintenant que $\alpha = 0,01$, nous obtenons,
	\[
	F_{0,99}(7,7) = 6,993 > 5,38788.
	\]\\
	Dans ce dernier cas, les variances sont considérées comme étant égales et les hypothèses du test sont respectées.

	\paragraph{3} Les variances ne sont pas considérées comme égales et l'hypothèse est rejetée. On ne peut donc pas affirmer que le traitement induit une diminution de la taille des plantes.
	
\end{solution} 
  
\section{}
  Un scientifique fait des recherches sur l'échauffement des cellules lorsqu'elles sont soumises à des ondes électromagnétiques. Il a soumis des cellules à des ondes de fréquences croissantes et a obtenu les résultats suivants : \bigskip
  
  \begin{tabular}{lcccccccc}
   Echauffement (mK) & 0.06 & 0.05 & 0.06 & 0.36 & 0.36 & 0.54 & 0.37 & 0.61 \\
   \hline
   Fréquence (GHz) & 0.3 & 0.5 & 0.7 & 0.9 & 1.1 & 1.3 & 1.5 & 2 \\
  \end{tabular} \bigskip
  
  Un modèle linéaire sous forme de droite (pente + intercept) est envisagé pour décrire l'effet de la fréquence sur l'échauffement. \bigskip
  
  \begin{enumerate}
   \item Ecrivez ce modèle sous forme matricielle et estimez-en les paramètres.
   \item Calculer un intervalle de confiance pour la pente.
   \item D'après le modèle utilisé, la fréquence a-t-elle un impact sur l'échauffement des cellules ?
   \item Durant l'expérience, la puissance de l'onde a également été mesurée. En considérant une extension du modèle précédent sous forme d'un plan, le paramètre supplémentaire étant associé à la puissance, la SCR de ce nouveau modèle est de 0.05. Ce modèle est-il meilleur que le précedent ?
  \end{enumerate}

\begin{solution}
  On sait que la chaleur (variable Y) est dépendante de la fréquence (variable X). On sait également que le modèle est linéaire.
  
  \paragraph{1} Le modèle peut être écrit sous forme matricielle
  
  \[
   E[T] = E[\sum_{i=1}^{n} a_{i} X_{i}] = \sum_{i=1}^{n} a_{i} * \underbrace{E[X_{i}]}_{\mu} = \underbrace{\sum_{i=1}^{n} a_{i}}_{1} *\mu = \mu
  \]

	\[
	\begin{bmatrix}
    n       & \sum_{i=1}^{n} X_{i} \\
    \sum_{i=1}^{n} X_{i} & \sum_{i=1}^{n} X_{i}^{2} \\
\end{bmatrix}
* \begin{bmatrix}
    \alpha \\
    \beta \\
\end{bmatrix}
=
\begin{bmatrix}
    \sum_{i=1}^{n} Y_{i} \\
    \sum_{i=1}^{n} X_{i} Y_{i} \\
\end{bmatrix}
\]

\[
	\begin{bmatrix}
    8       & 8,3 \\
    8,3 & 10,79 \\
\end{bmatrix}
* \begin{bmatrix}
    \alpha \\
    \beta \\
\end{bmatrix}
=
\begin{bmatrix}
    2,41 \\
    3,282 \\
\end{bmatrix}
\]

	On caclule, $\overline{X} = 1,0375$, $\overline{Y} = 0,30125$, $\sum_{i=1}^{8} (X_{i}-\overline{X})^{2} = 2,17875$ et $\sum_{i=1}^{8} (X_{i}-\overline{X})(Y_{i}-\overline{Y}) = 0,76559$
	Nous pouvons manitenant calculer,
	\[
	\widehat{\beta} = \frac{\sum_{i=1}^{8} (X_{i}-\overline{X})(Y_{i}-\overline{Y})}{\sum_{i=1}^{8} (X_{i}-\overline{X})^{2})} = \frac{0,76559}{2,17875} = 0,35139
	\]
	\[
	\widehat{\alpha} = \overline{Y}-\widehat{\beta}*\overline{X} = -0,06332
	\]
	\paragraph{2} Nous calculons que
	\[
	S^{2} = \frac{\sum_{i=1}^{8} (Y_{i}-\overline{Y})^{2}}{n-2} = 0,0572
	\]
	Dans la table de distribution de Student de n-2 degrés de liberté, nous trouvons que $t_{0,95}=1,943$.
	Et l'intervalle de confiance pour la pente ($\beta$) vaut 
	\[
	\begin{bmatrix}
    \widehat{\beta}\pm t_{0,95;n-2}\sqrt{\frac{S^{2}}{\sum_{i=1}^{n} (X_{i}-\overline{X})^{2}}}
	\end{bmatrix}
	\]
	Donc,
	\[
	\begin{bmatrix}
    0,0366, 0,6662
	\end{bmatrix}
	\]
	
	\paragraph{3} Testons l'hypothèse nulle que "la fréquence a un impact sur l'échauffement des cellules".\\
	On s'intéresse au problème de test défini par :\\
	\[
	H_{0}: \beta=0
	H_{1}: \beta \neq 0
	\]
	
	La statistique de test est $T = \frac{\widehat{\beta}}{\sqrt{\frac{S^{2}}{\sum_{i=1}^{n} (X_{i}-\overline{X})^{2}}}}$
	avec laquelle nous pouvons calculer la valeur observée de cette statistique valant $T_{obs}=2,1687$
	
	Or, celle-ci est supérieur à $t_{0,95;6} = 1,943$ que nous avions trouvé précédemment.\\
	Nous pouvons donc conclure qu'il y a un rejet de l'hypothèse nulle pour 0,05. L'échauffement des cellules n'est donc pas lié à la fréquence.
	
	\paragraph{4}
	!!! Ceci est une ébauche de la sous-question 4 !!!
	
	On s'intéresse au problème de test défini par :\\
	\[
	H_{0}: \beta_{1} \neq \beta_{2} ou \alpha{1} \neq \alpha{1} 
	H_{1}: \beta_{1} = \beta_{2} et \alpha{1} = \alpha{1} 
	\]
	
	La statistique de test pour éprouver $H_{0}$ est\\
	\[
	F=\frac{(SCR_{0}-SCR)/2}{SCR/(N-4)}
	\]
	
	On sait que $SCR_{2} = 0,05$\\
	On calcule que 
	\[
		SCR_{1} = S_{1}^{2}*(n_{1}-2)
					  = 0,3432\\
	\]\\
	\[				  
		SRC = 0,5+0,3432 = 0,3932\\
		SRC_{0} = 2,139
	\]\\
	
	\[
	F_{obs}=26,6399
	\]\\
	\[
	F_{0,95}(2,12)=3,885\\
	\]
	
	On rejette donc l'hypothèse nulle au niveau significatif 0,05. On peut donc dire que le second modèle est meilleur que le premier.
	
\end{solution}

\end{document}
