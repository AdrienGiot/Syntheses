\documentclass[fr]{../../../../../../eplexam}

\usepackage{../../../../../../eplunits}

\hypertitle{Probabilité et statistiques}{5}{FSAB}{1105}{2017}{Janvier}{All}
{Martin Braquet \and Etudiants Bac 3 de 2016}
{Anouar El Ghouch et Rainer von Sachs}

\section{(4 points)}

\begin{enumerate}
 \item (1 point)
  Vol Bruxelles - Montreal en 7h30, départ à 4h30 (heure de Montréal). 
  La durée du vol suit une distribution normale de moyenne $\mu$=7h30  et d'écart-type $\sigma= \SI{30}{min}$. 
  Pour quelle heure faut-il réserver le taxi pour n'avoir que 5\% de chance d'attérir après que le taxi soit arrivé à l'aéroport?
  
  \item (1.5 point)
  On veut téléphoner à la réception d'une agence de voyage. On sait que la probabilité que quelqu'un décroche 
  au deuxième appel est de 0.16. On sait aussi que la probabilité qu'ils décrochent est plus grande que la probabilité 
  qu'ils ne décrochent pas. On cherche alors à savoir qu'elle est la probabilité qu'on doive appeler plus de 2 fois 
  avant qu'on nous décroche.
  
  \item (1.5 point)
  Dans une entreprise, on a un certain nombre de machines dont la maintenance est partagée entre deux entreprises. 
  L'entreprise A garantit un bon fonctionnement de 80\% des machines dont elle s'occupe (pendant un an) et l'entreprise 
  B garantit un bon fonctionnement de 90\% des machines dont elle s'occupe (pendant un an). Cette différence se paie, 
  et c'est pourquoi l'entreprise A a 70\% des machines sous sa responsabilité et l'entreprise B seulement 30\%. 
  A la fin de l'année, on a 7 machines qui ont rendu l'âme. Quelle est la probabilité que plus de 2 de ces 7 machines étaient 
  celles réparées par l'entreprise B?
  
\end{enumerate}

\begin{solution}
 \begin{enumerate}
  \item L'écart-type doit être converti en heures: $\sigma = \SI{0.5}{h}$. Soit $X$, la durée du vol, on a $$Z=\frac{x - \mu}{\sigma} \sim N(0,1)$$
  On trouve $z_\alpha=(x_\alpha-\mu)/\sigma=1.645$ et donc $x_\alpha=\mu + 1.645 \sigma = 7h30 + 1.645 \cdot 0.5 = 8h19$. L'heure d'arrivée est $4h30+x_\alpha=12h49$.
  \item On remarque que $X$, le numéro de l'appel qui est décroché, suit une distribution géométrique de probabilité (qu'on nous décroche) $p$. 
  On utilise les données de $P(X = 2) = 0.16$ pour trouver $p$. On a donc 
  $P(X = x) = (1-p)^{x-1}p \Rightarrow P(X = 2) = 0.16 = (1-p) p$. 
  On trouve 2 solutions (0.2 et 0.8), et on rejette 0.2 car la probabilité qu'ils décrochent est plus grande que la probabilité qu'ils ne décrochent pas.
  On peut alors trouver $$P(X > 2 ) = 1-P(X=1)-P(X=2)=1-0.8-0.16=0.04 $$
  
  \item La probabilité qu'une machine cassée provienne de B vaut
  \[
    p=\frac{\mbox{nombre de machines cass\'ees par B}}{\mbox{nombre total de machines cass\'ees}}
    =\frac{(1-0.9)\cdot 0.3M}{(1-0.9)\cdot 0.3M+(1-0.8)\cdot 0.7M}=\frac{3}{17}
  \]
  avec $M$ le nombre de machines.
  Ensuite, on pose $X$ le nombre de machines cassées par B au sein des 7 machines cassées, on 
  remarque ainsi que $X$ suit une distribution binomiale: $X\sim \mathrm{Bin}(n=7,p=3/17)$.
  
  On trouve ainsi
  \[
    P(X>2)=1-P(X=0)-P(X=1)-P(X=2)=1-(1-p)^7+C_1^7\; p (1-p)^6+C_2^7\; p^2(1-p)^5
    =0.11
  \]
  
 \end{enumerate}
 \end{solution}

\section{(4 points)}

Pierre, Paul et Jacques sont des gamers invétérés. Ils jouent à un jeu vidéo qui
consiste à rester en vie le plus longtemps possible. Le temps de vie maximal est de 2h. 

$f(y)$ est la fonction de densité du temps $y$ pendant lequel on est resté en vie (en heures)
\[
  f(y)=\left\{ \begin{array}{cl}
      y/2 & \mbox{si } 0 \leqslant y \leqslant 2 \\
      0 & \text{sinon}
     \end{array}
     \right.
\]

\begin{enumerate}
\item (1 point) Quelle est l'espérance de gain sachant que pour chaque minute où on survit, on gagne 0.3 EUR, et qu'il faut payer 20 EUR pour participer.
\item (1 point) Sachant que Pierre ne peut que gagner ou perdre, il veut estimer la probabilité de gagne avec une certitude de 2\% et donc un intervalle de confiance de 98\%. Trouver le nombre de parties qu'il doit faire pour construire cette intervalle.
\item (1 point) Ils jouent à trois en même temps et leurs parties sont indépendantes. Quelle est la probabilité que le moins bon d'entre eux meure avant 30 minutes de jeu?
\item (1 point) Jacques a joué 80 parties sur l'année. Quelle est la probabilité qu'en moyenne il survive plus de 1h15?
\end{enumerate}

\begin{solution}
\begin{enumerate}
 \item 
    $$\mathrm{Gain} = 0.3 \cdot 60 \cdot Y -20 $$
    $$E[\mathrm{Gain}] = 18 E[Y] -20=18\int_0^2 \frac{y^2}{2} \mathrm{d}y-20=24-20=4$$
 \item Le résultat du jeu pour Pierre suit une distribution de Bernouilli: Be($p$).
	Par le TCL, on a 
	\[
	  \sqrt{\frac{n}{p(1-p)}}(\hat{p}_n-p)\sim \mathcal{N}(0,1)
	\]
	et ainsi on veut que
	\[
 	  P\left(\sqrt{\frac{n}{p(1-p)}}|\hat{p}_n-p|<z_{0.01}\right)=0.98
	\]
	avec $|\hat{p}_n-p|=0.02$ pour que la différence de probabilité soit au maximum de 2\%.
	On cherche donc 
	\[
 	  \sqrt{\frac{n}{p(1-p)}}0.02=2.33\Rightarrow n=3394
	\]
	en prenant $p=0.5$ dans le pire cas.
 \item $$P(\mathrm{min}(Y_1,Y_2,Y_3) < 0.5) = 1 - P(\mathrm{min}(Y_1,Y_2,Y_3) > 0.5) $$ $$= 1 - P(Y>0.5)^3 = 1 - \left(\int_{0.5}^2 f(y) \, \mathrm dy \right)^3= 0.176$$
 \item $$P(\bar{Y}>1.25)=P\left(Z>\frac{1.25-4/3}{\sqrt{2/9}}\sqrt{80}\right)=P(Z>-1.58)=0.9429$$
avec $\sigma^2=E[X^2]-E[X]^2=2/9$ et avec 1.25 qui vient de 1h15 et 4/3 qui est l'espérance.
\end{enumerate}

\end{solution}


\section{(5 points)}

Soit la fonction de densité suivante:

\begin{equation*}
    f(x) = \frac{3x}{\theta}e^{-3x^{2}/(2\theta)}
\end{equation*}
définie sur l'intervalle $0<x<\infty$.

Aide : $$\int_0^{+\infty} x^{2n+1}e^{-x^2/c}dx = \frac{n!}{2}c^{n+1}$$

On demande de:

\begin{enumerate}
    \item (1 point) Calculer $\hat{\theta}$ l'estimateur de maximum de vraisemblance de $\theta$.
    \item (1 point) Montrer que $X^{2} \sim Expo(\beta)$ et donner $\beta$ en fonction de $\theta$.
    
    \item (1.5 points) Calculer le MSE de $\hat{\theta}$, est-ce qu'il est consistent?
    \item (1.5 points) Donner l'estimateur de maximum de vraisemblance de la médiane.
\end{enumerate}

\begin{solution}

\begin{enumerate}
\item On calcule $L(\theta)$ puis $LL(\theta) = \ln(L(\theta))$, on dérive l'expression par rapport à $\theta$ et on égale à 0 pour trouver la valeur de $\theta$ qui maximise la fonction de vraisemblance:
\begin{equation*}
    \hat{\theta} = \frac{3}{2}\frac{\sum x_{i}^{2}}{n}=\frac{3}{2}\overline{x^2}
\end{equation*}

\item 
    Soit $Y=X^2$,
      \begin{align*}
      F_Y(y)&=P(Y<y)=P(X^2<y)=P(-\sqrt y<X<\sqrt y)=\int_{-\sqrt y}^{\sqrt y}f_X(x)\:\mathrm d x \\
       &= \int_{0}^{\sqrt y} \frac{3x}{\theta}e^{-3x^{2}/(2\theta)} \:\mathrm d x=
       \int_{0}^{\frac{3y}{2\theta}} e^{-u} \:\mathrm d u=1-e^{-3y/(2\theta)}
      \end{align*}
     en posant $U=3X^2/(2\theta)$.
     On trouve ainsi que
     \[
	f_Y(y)=F_Y'(y)=\frac{3}{2\theta}e^{-3y/(2\theta)}\qquad \mbox{si } 0\leqslant y \leqslant \infty
     \]
     et $\beta=\frac{2\theta}{3}$.


\item \begin{equation*}
    E(\hat{\theta}) = \theta 
\end{equation*}

\begin{equation*}
    MSE(\hat{\theta})=\mathrm{Var}(\hat{\theta}) = \mathrm{Var}\left( \frac{3}{2n}\sum X_{i}^{2}   \right) = \frac{9}{4n^2}\mathrm{Var}\left( \sum X_{i}^{2} \right) = \frac{9}{4n}\mathrm{Var}(X^2)= \frac{9}{4n}\mathrm{Var}(Y) = \frac{9}{4n}\beta^2=\frac{\theta^2}{n}
\end{equation*}
Consistant car $E(\hat{\theta}) = \theta $ et $\lim_{n \Rightarrow \infty} MSE = 0$
\item 
$$F(m) = 1/2$$
$$F(x) = \int_{0}^x f(x) \, \mathrm dx =1 -e^{\frac{-3x^2}{2\theta}}$$
Donc on obtient :
$$m = \sqrt{\frac{2\theta}{3}\ln{2}}$$
Comme $\hat{\theta}$ est consistent on peut remplacer $\theta$ dans cette solution par $\hat{\theta}$ et on obtient un estimateur de $m$:
\[
  MLE(m)=\sqrt{\frac{\ln(2)}{n}\sum x_{i}^{2}}
\]

\end{enumerate}

\end{solution}


\section{(3 points)}

Fusée Cap canaveral : Une fusée a une durée de vie de maximum 20 minutes. Sur le trajet, à deux reprises elle va enclencher une réserve de fuel. $X$ et $Y$ sont les temps respectifs de l'allumage des réserves de fuel.

$X$: temps d'allumage de la réserve de fuel 1 (en dizaines de minutes)

$Y$: temps d'allumage de la réserve de fuel 2 (en dizaines de minutes)

$$f(x,y) = \frac{3x}{4}\qquad \mbox{si } 0\le X \le Y \le 2$$
%
\begin{enumerate}
\item On demande après combien de minutes on va allumer la première réserve en moyenne.
\item On demande la probabilité qu'on commence à utiliser les deux réserves de fuel avant 10 minutes
\item On demande de calculer après combien de temps en moyenne la deuxième réserve sera enclenchée si on sait que la première a été enclenchée à un quart du temps de vie maximum ?
\item On demande la probabilité que la première réserve de fuel ait été enclenchée dans les 5 minutes avant l'enclenchement de la deuxième réserve sachant que la deuxième réserve a été enclenchée à la quinzième minute.
\end{enumerate}

\begin{solution}
\begin{enumerate}
\item $$E(X)=\int_0^2\int_x^2 \frac{3x^2}{4} \:\mathrm d x \:\mathrm d y = 1 = 10\: \mathrm{minutes} $$
\item $$P(X<1,Y<1)=\int_0^1\int_x^1 \frac{3x}{4} \:\mathrm d x \:\mathrm d y = 0.125 $$
 \item $E(Y|X=1/2)=\int_{1/2}^2 y f(y|x)\dif{y} = \int_{1/2}^2 y \frac{f(x,y)}{f_X(x)}\dif{y} = 12$ min $30$ où $f_X(x) = \int_x^2 f(x,y)\dif{y}$
\item $P(1<X<3/2|Y=3/2)=0.55556$
\end{enumerate}
\end{solution}


\section{(4 points)} 

Une usine dispose de deux machines robotiques A et B. Leur temps d'assemblage est donné par les variables aléatoires X et Y respectivement (X indépendant de Y) normalement distribuées. On dispose de deux échantillons (iid) de données sur une petite partie de la production des deux machines : $x_1,...,x_8$ et $y_1,...y_{12}$.
Les données sont les suivantes
\begin{itemize}
\item Machine A: $n_a=8$, $\sum_{i=1}^8 x_i = 460.8$, $\sum_{i=1}^8 x_i^2 = 26570.92$
\item Machine B: $n_b=12$, $\sum_{i=1}^{12} y_i = 762$, $\sum_{i=1}^{12} y_i^2 = 48415.16$
\end{itemize}


\begin{enumerate}
\item (1 point) Construisez un intervalle de confiance contenant la moyenne de A à 95\% de sûreté. Justifiez et détaillez.
\item (2 points) On propose de tester l'hypothèse suivante: $H_0: \mu_b - \mu_a = 4$ contre $H_1 : \mu_b - \mu_a > 4$. Donnez la région de rejet. Est-ce raisonnable de rejeter $H_0$ si $\alpha = 0.05$ ? Sur base des tables, quelle est la p-valeur de ce test si elle est calculable, et justifiez.
\item (1 point) Quelles hypothèses avez-vous posées pour justifier le raisonnement en b) ? Sont elles vérifiées ici?
\end{enumerate}

\begin{solution}

\begin{enumerate}
\item $$IC=\left[\bar{x} \pm t_{ \alpha /2, n_a-1} * \frac{S_{n_a}}{\sqrt{n_a}}\right]=[55,90; 59,3]$$
\item On rejette si 
\[
  t=\frac{\bar{x}_b-\bar{x}_a-4}{S_p\sqrt{1/n_a+1/n_b}}>t_{n_a+n_b-2,\alpha}
\]
Si $\alpha=0.05$, 
$$t=2,23  > t_{18; 0,05}=1,734$$
On rejette $H_{0}$.
La p-valeur est la valeur $p=P(T>t=2.23)$ où $T \sim t_{n_a+n_b-2}$. Cette valeur n'est pas dans les tables néanmoins l'on trouve $P(T>2.101)=0.025$ donc on peut conclure que $p<0.025$ (et $p>0.01$ par le même raisonnement). L'on pourrait également le montrer graphiquement.
\item Les hypothèses sont : échantillons indépendants et l'égalité des variances.
\end{enumerate}
 
\end{solution}

\end{document}
