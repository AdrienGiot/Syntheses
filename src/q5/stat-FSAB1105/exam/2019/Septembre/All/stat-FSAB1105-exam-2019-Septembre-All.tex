\documentclass[fr]{../../../../../../eplexam}

\hypertitle{}{5}{FSAB}{1105}{2019}{Septembre}{All}
{François Froment}
{Anouar El Ghouch}

\section{(2 points)}
An electrical system is composed of two stations operating independently. 
When at least one station fails, the system stops working. Furthermore, 
we know that the failure time of a station (in hundreds of days) is a random variable characterized by the density function $f(y) = \frac{3}{(1+y)^4}$ if $y > 0$ and $f(y) = 0$ otherwise.
\begin{enumerate}
	\item What is the median failure time of a station ?
	\item Waht is the probability that the system is operational for more than 100 days ?
\end{enumerate}

\begin{solution}
\begin{enumerate}
	\item The median failure time is the time h (in 100 days) where $P(Y=h)=0.5$ or $F_Y(h)=0.5$.
	\[F_y(y) = \int_0^y \frac{3}{(1+x)^4} dx = [\frac{-1}{(1+x)^3}]_0^y = \frac{-1}{(1+y)^3}+1\]
	\[F_y(h) = 0.5 \implies 1 - \frac{1}{(1+h)^3} = 0.5 \implies (1+h)^3 = 2 \implies h=0.259\]
	The median failure time is 26 days.
	\item The probability is $P(Y_1>1,Y_2>1)$, $Y_1$ and $Y_2$ are iid so $P(Y_1>a,Y_2>b)=P(Y_1>a).P(Y_2>b)$ and $P(Y_1>1)=P(Y_2>1)$.
	\[P(Y>1) = \int_1^{\inf} \frac{3}{(1+y)^4} dy = [\frac{-1}{(1+y)^3}]_1^{\inf} = 0 + \frac{1}{8}\]
	\[P(Y_1>1,Y_2>1) = \frac{1}{64}\]
\end{enumerate}
\end{solution}

\section{(4 points)}
A cloud computing server receives two types of requests for performing calculations.
The requests are independent and their type is denoted $R = \{1, 2\}$.
Over 24 hours, the server receives on average 4 requests of type $R=1$ and 8 requests of type $R=2$.
The number of type 1 and type 2 requests are respectively denoted by $N_1$ and $N_2$.
\begin{enumerate}
	\item Which statistical distributions would you choose for $N_1$ and $N_2$ ?
	Provide the values of parameters defining these distributions.
	\item Let us denote by $N = N_1+N_2$, the total number of computation requests. Starting from the distributions chosen for $N_1$ and $N_2$,
	find the probability mass function $P(N=n)$ ?
	(a proof is needed, hint : think about the mgf)
	\item You have been informed that the server recieved \textbf{at least} k requests of type $R=1$
	this day, what is the probability that the servr treats $N_1 = n_1 \geq k$
	requests of type 1 over this day ? Compute this probability for $n_i = 6$ and $k = 2$
	\item Over one year (365 days), calculate the expected number of computation requests of type 1,
	received by the server.
	Find next the variance of the number of type 1 computation requests received by the server over a one-year period.
	Finally, calculate the 90\% two-sided confidence interval for the number of type 1 requests,
	received during a one-year period. 
\end{enumerate}

\nosolution

\section{(2 points)}
The public transport service (STIB) in Brussels aims to understand the extreme delays (more than 5 minutes) in its metro stations.
We are interested in metro station Merode where line 1 and line 5 meet.
On a given day, we know that there is aprobability of 0.4 that at least one of the two lines experience an extreme delay and a probability of 0.1 that both lines experience an extreme delay.
Moreover, we know that there is a probability of 0.3 that line 1 experiences an extreme delay.
How many days do we need to wait on average befor we observe an extreme delay occurring only for line 5 ?

\nosolution

\section{(6 points)}
A social survey claims that the impact of professional experience on salaries is log-linear.
More specifically, the following model is assumed : $log(Y_i) = \beta_0 + \beta_1 X_i + \epsilon_i$
where $Y_i$ measures the monthly salary in EUR of individual i and $X_i$ measures the professional experience in years of individual i.
Furthermore i = 1,...,n and n = 107. You face the following results
\begin{table}[!h]
\centering
\begin{tabular}{|c|c|c|}
	\hline
	$\frac{1}{n}\sum_i X_i = 20.851$ & $\frac{1}{n}\sum_i Y_i = 1995.708$ & $\hat{\beta_1} = 0.029$ \\ 
	\hline
	$\prod_i X_i^{\frac{1}{n}} = 19.305$ & $\prod_i Y_i^{\frac{1}{n}} = 1828.008$ & \\
	\hline
\end{tabular}
\end{table}

\begin{enumerate}
	\item Carefully write down the assumptions on $\epsilon_i$ underlying the model.
	\item Estimate $\beta_0$ and answer the following question : what is the expected salary of an individual with 10 years of professional experience ?
	\item Concerning once again an individual with 10 years of experience,
	you are given the following two intervals (computed at the same level $\alpha$)
	\[Interval_1 = [6.589; 7.803]\]
	\[Interval_2 = [7.091; 7.301]\]
	One is a confidence interval for the individual's expected log-salary;
	the other is a prediction interval of the individual's log-salary.
	Can you tell which is which ? Explain.
\end{enumerate}
On the same sample of observations, we want to determine whether men and women have on average the same level of professional experience.
You face the following results
\begin{table}[!h]
\centering
\begin{tabular}{|c|c|c|}
\hline
 & Men & Women \\
\hline
Number of observations & 61 & 46 \\
\hline
Sample mean & 21.738 & 19.676 \\
\hline
Sample variance & 41.013 & 63.941 \\
\hline
\end{tabular}
\end{table}
Remark : the sample variance is the bias-corrected version (i.e. the one where we divide by the number of observation - 1)
\begin{enumerate}[resume]
	\item Undertake an ANOVA to test whether average experience is different in the two groups.
	Use a significance level of $\alpha = 0.05$
	\item Using the same significance level, test the same hypothesis using a two-sample t-test.
	Is your decission the same ? Could it be expected ? 
\end{enumerate}

\nosolution

\section{(2 points)}
A primary shoolteacher wishes to know the proportion of left-handed pupils in her school.
Knowing that 75\% of the pupils are right-handed in her classeroom, 
she constructs a 95\% confidence interval for the proportion of left-handers and she obtains [0.12, 0.38].
What can be said about n, the number of pupils in the class ?

\nosolution

\end{document}
