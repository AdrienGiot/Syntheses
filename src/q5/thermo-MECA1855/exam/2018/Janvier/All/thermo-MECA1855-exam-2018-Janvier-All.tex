\documentclass[fr]{../../../../../../eplexam}
\usepackage{tikz,siunitx}

\hypertitle{thermo-MECA1855}{5}{MECA}{1855}{2018}{Janvier}
{Martin Braquet \and Léa Paulus \and Lorenzo Barbuscia}
{Miltiadis Papalexandris et Yann Bartosiewicz}

Examen en 2 parties: 1h30 de théorie sans formulaire, et 2h d'exercices avec formulaire.

\section{Questions théoriques de Papalexandris}

\begin{enumerate}
    \item Pentes de transformations isochore et isobare dans un (T,S), laquelle est la plus grande, justifier.
    \item Répartition de température dans un condenseur co-courant.
    \item Isotherme de van der Walls, zone instable et comment réconcilier avec ce qui est observé (construction de Maxwell).
\end{enumerate}

\section{Questions théoriques de Bartosiewicz}

\begin{enumerate}
    \item Pourquoi on peut dire que la vapeur d'eau dans l'air humide est un gaz parfait?
    \item Comparaison du travail entre une compression isotherme et isentropique, justifier pourquoi l'isotherme demande moins de travail, justifier graphiquement en plus.
    \item Expliquer le fonctionnement d'un brumisateur, et quelles sont les conditions pour que cela fonctionne.
\end{enumerate}

\section{Exercice de Papalexandris}
Cycle frigo avec R22 (tables et diagramme fournis).
Chambre froide à garder à -10°.
Fuide frigorigène surchauffé de 5° en fin d'évaporation, écart de 5° avec la chambre froide une fois qu'il est surchauffé.
Fluide sous refroidi de 5° en sortie de condenseur, différence de 5° avec le fluide de refroidissement (air ambiant).
Compression : rendement isentropique de 0.9 et rendement mécanique de 0.9.
\begin{itemize}
    \item Printemps : air ambiant à 10°.
    \begin{itemize}
        \item Tracer le cycle dans un diagramme ($\ln(p),h$).
        \item Calculer tous les états ($p,t,h,s,x$) du cycle.
        \item Calculer COP.
    \end{itemize}
    \item Eté : air ambiant à 20°.
    \begin{itemize}
        \item Calculer les nouveaux états du cycle.
        \item Calculer le nouveau COP et comparer avec l'ancien.
    \end{itemize}
\end{itemize}

    
\section{Exercice de Bartosiewicz}
Il y avait 2 pages d'énoncé, nous les résumons en un schéma.

Un canon à neige est alimenté par un réservoir situé en aval. Une pompe sert à alimenter le canon depuis le réservoir. Un ventilateur souffle de l'air sec, qui se mélange avec l'eau avant d'être éjecté sur les pistes. L'eau parcourt une longueur $L_1=\SI{1000}{\meter}$ entre le réservoir et la pompe, et une longueur  $L_2=\SI{5000}{\meter}$ entre la pompe et le canon. Les pressions de saturation de l'eau sont données pour $\SI{-10}{\degreeCelsius}<T<\SI{10}{\degreeCelsius}$.

\begin{itemize}
    \item La pression en entrée du canon est \SI{1500000}{\pascal}.
    \item La pression du réservoir est
    $p_{1400} = \SI{104100}{\pascal}$.
    \item Le rayon du tuyau est \SI{6}{\centi\meter}.
    \item La rugosité est $\varepsilon = \SI{0.24}{\milli\meter}$.
    \item La surface sur les pistes à enneiger a une longueur de \SI{100}{\meter} et une largeur de \SI{50}{\meter}.
    \item La masse volumique de la neige vaut $\rho_{neige}=\SI{400}{\kilo\gram\per\cubic\meter}$.
    \item La température ambiante à 1600 m vaut \SI{-3}{\degreeCelsius} et $\phi_{1600} = 0.3$.
    \item La température ambiante à 1400 m vaut \SI{2}{\degreeCelsius}.
    \item La neige est idéale si sa température est inférieure à \SI{-6}{\degreeCelsius}. 
    \item Le débit d'air sec entrant dans le ventilateur est de \SI{0.4}{\cubic\meter\per\hour}.
    \item 
    \item 
\end{itemize}
Hypothèses: 
\begin{itemize}
    \item Tous les rendements sont unitaires.
    \item Il n'y a pas de perte de charge singulière.
    \item L'air est considéré comme un gaz parfait et l'eau comme un fluide incompressible.
\end{itemize}

\begin{center}
\begin{tikzpicture}[scale=4]
    \draw (2.3, 1.2) -- (2.5, 1) -- (2.7, 1.2);
    \node at (2.5, 1.2) {Canon};
    \draw (0, 1) -- (0, 0) -- (0.05, 0);
    \draw (0.2, 0) circle (0.15);
    \node at (0.2, 0.) {Pompe};
    \draw (0.35, 0) -- (1, 0);
    \draw (0, 1) -- (2.5, 1);
    \node[above] at (1.25, 1) {$\SI{4800}{\meter}$};
    \node[right] at (0, 0.5) {$\SI{200}{\meter}$};
    \node[above] at (0.7, 0) {$\SI{1}{\kilo\meter}$};
    \draw (1, -0.2) rectangle (1.8, 0.2);
    \node at (1.4, 0) {Réservoir};
    \node[above] at (1.05, 0) {1};
    \node at (1.05,0) {.};
    \node[above] at (0.39, 0.05) {2};
    \draw (0.39, 0.05) -- (0.39,-0.05);
    \node[above] at (0.03, 0.05) {3};
    \draw (0.03, 0.05) -- (0.03,-0.05);
    \node[below] at (2.5,0.95) {4};
    \draw (2.5, 1.05) -- (2.5,0.95);
\end{tikzpicture}
\end{center}

La caractéristique à la pompe est:
\[ \Delta p = -9131.7 \:\dot{m}^2 + 478292.21 \:\dot{m} + 4500000 \]
où $\dot{m}$ est le débit massique.

\begin{enumerate}
    \item Calculer le débit massique en tenant compte des pertes de charge et de la caractéristique de la pompe.
    \item Que vaut la puissance dissipée ($P_{fl}$) et quelle est la différence de température associée à cette puissance dissipée? Sommes-nous inférieur à \SI{2}{\degreeCelsius} pour la température d'admission du canon?
    \item Placer le point de condition ambiante à $\SI{1600}{\meter}$ sur le diagramme et déduisez la température humide correspondante, la neige est telle de bonne qualité ?
    \item Calculer le débit de consommation d'eau et le débit de production de neige en [\si[per-mode=symbol]{\cubic\meter\per\hour}]. Ainsi que, le rapport $(\dot{Q_n}/\dot{Q_e}) $.
    \item Si l'épaisseur de neige à ajouter sur les pistes est de \SI{30}{cm}, donner le volume de neige à produire. En déduire la durée d'enneigement des pistes.
    \item Donner la puissance du ventilateur ainsi que le diamètre du conduit dans le ventilateur.
    \item Calculer la puissance de la pompe et déterminer si un problème de NSPH va survenir.
\end{enumerate}

\begin{solution}

\begin{enumerate}
\item
Soient $p_1 = \SI{104100}{\pascal}$ la pression en sortie du réservoir, $p_4 = \SI{1500000}{\pascal}$ la pression au canon, $p_2$ la pression en entrée de la pompe (i.e. côté réservoir), $p_3$ la pression en sortie de la pompe (i.e. côté canon).

On a
\begin{align*}
  w_m & = \int v d p + \Delta k + g \Delta z + w_f\\
  0 & = \frac{\Delta p}{\rho} + \Delta k + g \Delta z + w_f\\
  0 & = \Delta p + \rho \Delta k + \rho g \Delta z + \rho w_f
\end{align*}

Ce qui donne pour les deux tronçons:
\begin{align*}
  0 & = p_4 - p_3 + \rho g h + \rho w_{f,34}\\
  0 & = p_2 - p_1 + \rho \frac{c^2 - 0}{2} + \rho w_{f,12}
\end{align*}

En additionnant les deux équations, on obtient:
\begin{equation}
  \label{eq:meq}
  0 = p_4 - p_1 - \Delta p_{23} + \rho \frac{c^2}{2} + \rho g h + \rho ( w_{f,12}+w_{f,34})
\end{equation}

On a $\epsilon/D = 0.002$ et donc $\lambda = 0.0235$.
Soit $c$ la vitesse du fluide, on a $\rho c A = \dot{m}$ et donc
\[ w_f = \lambda \frac{L}{D} \frac{c^2}{2} = \lambda \frac{L}{D} \frac{\dot{m}^2}{2 \rho^2 A^2} = 0.0235 \frac{6000}{0.12} \frac{1}{2 \cdot 1000^2 (0.06^2\pi)^2} \dot{m} = 4.59307 \:\dot{m}^2. \]

L'équation~\eqref{eq:meq} devient donc
\begin{align*}
  0 & = p_4 - p_1 - \Delta p_{23} + \frac{1000}{2}\left( \frac{\dot{m}}{1000\cdot \pi \cdot 0.06^2} \right)^2  + \rho g h + 4593.07 \:\dot{m}^2\\
  0 & = 1500000 - 104100 + 9131.7\: \dot{m}^2 - 478292.21 \: \dot{m} - 4500000 + 3.9 \:\dot{m}^2 + 1000 \cdot 9.91 \cdot 200 + 4593.07\:\dot{m}^2\\
  0 & = 13728.67 \: \dot{m}^2 - 478292.21 \:\dot{m} - 1142100.
\end{align*}
Cette équation a pour discriminant \num{2.914815e11}, et on trouve $\dot{m} = \SI[per-mode=symbol]{37.08}{\kilo\gram\per\second}$.
\item 
\begin{align*}
P_{fl}=\dot{m}w_{f}= 4.6908 \dot{m}^3=221.969 [kW]\\
\Delta t=\frac{w_{f}}{c_{p,l}}=1.47 [K]
\end{align*}
Nous constatons que $t_{e,4}=2.47$
\end{enumerate}

\end{solution}

\end{document}