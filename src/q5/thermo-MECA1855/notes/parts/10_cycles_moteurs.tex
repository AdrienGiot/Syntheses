\section{Questions sur le chapitre ``Cycles thermodynamiques moteurs''}
\subsection{À quoi sert le sous-refroidissement d'un cycle vapeur ?}
Pour la même raison que pour un cycle frigorifique. Le sous-refroidissement permet de s'assurer que la condensation est complète. Pour plus de détails voir question \ref{ssec:sous-refroidissement}.

\subsection{Quel paramètre est naturellement imposé en sortie de turbine dans le cycle vapeur (Rankine) et comment ? Quel est son ordre de grandeur ?}
La pression de sortie est imposée par le condenseur. En effet, la condensation est un processus isobare. Autrement dit, la pression en sortie de la trubine va être la pression de l'eau en entrée de la pompe, environ \SI{5.0}{\kilo\pascal}. On peut aussi noter que le titre en sortie de la turbine ne doitpas être trop faible pour ne pas abîmer les ailettes de la turbine. Idéalement, $x > 0.88$. 
