\documentclass[fr]{../../../../../../eplexam}
\usepackage{../../../../../../eplcode}
\lstset{language={Java},morekeywords={for, do}}

\hypertitle{Calculabilité et Complexité}{6}{INGI}{1123}{2017}{Juin}{Majeure}
{Maxime de Streel}
{Yves Deville}

\paragraph{Discussion link}
\url{http://www.forum-epl.be/viewtopic.php?t=13094}

\section{}
Soit l'ensemble $A = \{i |P_i$ termine avec deux valeurs disctinctes pour au moins deux entrées disctinctes\}.
\begin{enumerate}
  \item Énoncer précisément le théorème de Rice.
  \item Utilisé ce théorème pour justifier la non-récursivité de $A$
\end{enumerate}

\begin{solution}
  \begin{enumerate}
    \item 
    Si A est récursif et A $\neq \emptyset, A \neq \mathbb{N}$.

Alors $\exists i \in A \text{ et } \exists j \in \mathbb{N} \backslash A$ tel que $\phi_i = \phi_j$

Si A est récursif et qu'il n'est ni vide, ni ne contient l'ensemble des programmes existant, alors il existe deux programmes dans A et $\bar{A}$ tel qu'ils calculent la même fonction.

    \item 
    \begin{align}
    		P_i & \text{ tel que }\exists x_1,x_2 | x_1 \neq x_2 :: P_i(x_1) = y_1 \wedge P_i(x_2) = y_2 \wedge y_1 \neq y_2
    \end{align}
    Le Th. de Rice contraposé s'énonce de la manière suivante:
    \begin{equation}
    \text{Si } \forall i \in A \text{ et } \forall j \in \bar{A} \text{ on a } \phi_i \neq \phi_j
    \end{equation}
    Alors A est non-récursif ou A = $\emptyset$ ou A = $\mathbb{N}$
    
	\begin{enumerate}
		\item Soit A l'ensemble des programmes $P_i$. A $\subset \mathbb{N}$ et $A \neq \{\emptyset\}$ et $\bar{A} = \mathbb{N} \backslash A$
		\item Supposons A récursif
		\item On construit le programme P$_k$ 
			\begin{lstlisting}[frame=single]
while true do;
			\end{lstlisting}
			On a donc $\forall x : \phi_k(x) = \bot$ ceci implique donc que $k \in \bar{A}$
		\item Soit un $m$ quelconque tel que $m \in A$
		\item Par hypothèse $\forall i \in A, \forall j \in \bar{A} : \phi_i \neq \phi_j.$ Donc $\phi_m \neq \phi_k$.
		\item Soit le programme P(z) pour un x et n fixé
		\begin{lstlisting}[frame=single, mathescape=true]
P$_n$(x);
P$_m$(x);
		\end{lstlisting}
		Le numéro de ce programme est $d$. Que vaut $\phi_d$ ?
		$$\phi_d(z) = \left \{
   \begin{array}{l l l}
\phi_k  & \text{ si } P_n(x) \text{ se termine pas } &(d \in \bar{A}) \\
\phi_m  & \text{ si } P_n(x) \text{ se termine } &(d \in A)
   \end{array}
   \right .$$
   \item On peut donc écrire un programme qui décide HALT
   \begin{lstlisting}[mathescape=true, frame=single]
halt(n,x) $\equiv$ construire P(z) = P$_n$(x); P$_m$(z)
d $\leftarrow$ numero de P(z);
if d in A then print 1;
else print 0;
\end{lstlisting}
\item contradiction car HALT n'est pas récursif
\item Conclusion: A n'est pas récursif
	\end{enumerate}	    
  \end{enumerate}
\end{solution}

\section{}
\begin{mcqs}
  \mcq{Il existe un ensemble infini de chaine finies de caractères (A-Z) qui n'est pas énumérable.}{0}
  {S'il y a $k$ symboles, la bijection avec les entiers est immédiate si on considère la représentation de ces entiers en base k.}
  \mcq{Un sous-ensemble infini d'un ensemble récursivement énumérable est récursivement énumérable.}{0}
  {Exemple: le complément de K qui est bien un sous-ensemble de $\mathbb{N}$}
  \mcq{Il existe un langage de programmation (non trivial) tel que HALT est calculable.}{1}
  {Mini-Java}
  \mcq{La propriété Universelle (U) dit que l'interpréteur de D est calculable.}{0}
  {La propriété U dit que l'interpréteur de D est D-calculable. Il se pourrait que D soit un formalisme plus puissant dans lequel son interpréteur n'est pas calculable (ex: contient un oracle pour la fonction halt).}
  \mcq{La propriété S-m-n indique qu’il existe une fonction calculable qui permet de transformer une fonction en une autre fonction équivalente, mais avec moins d’arguments.}{0}
  {Le Th. S-m-n nous dit que $\forall m,n \geq 0$, il existe une fonction totale calculable $S^m_n : \mathbb{N}^{m+1} \rightarrow \mathbb{N}$, il faut donc que ce soit une fonction totale calculable.}
  \mcq{Le théorème du point fixe est une conséquence du théorème de Rice.}{0}
  {L'inverse est vrai}
  \mcq{SAT $\leq_a$ B et B $\in$ P $\Rightarrow$ SAT $\in$ P.}{0}
  {On a besoin de la réduction polynomiale et pas de la réduction algorithmique (cette dernière sert uniquement à la calculabilité, alors que la réduction polynomiale dit combien de fois on utilise l'algorithme !)}
\end{mcqs}

\section{}
Montrer que si le domaine d'une fonction est fini, cette fonction est calculable.

\begin{solution}
  En effet, si le domaine d'une fonction est fini il suffit de gérer tous les cas possible et de hardcoder l'image dans le programme avec un case sur l'input.
	\begin{lstlisting}[frame=single] 
    switch (x in dom(f)) {
    	case x1: print 1; 
    	case x2: print 2;
    	...
    	default: print 0;
   }
  \end{lstlisting}
\end{solution}

\section{}
\begin{enumerate}
  \item Définir la réduction algorithmique.
  \item Définir la réduction polynomiale.
  \item Montrer que $A \leq_p B \Rightarrow A \leq_a B$.
\end{enumerate}

\begin{solution}
  \begin{enumerate}
    \item Un ensemble A est algorithmiquement réductible à un ensemble B (A $\leq_a$ B) si en supposant B récursif, A est récursif.
    \item Un ensemble A est polynomialement réductible à un ensemble B (A $\leq_p$) B s'il existe une fonction totale calculable $f$ de complexité temporelle polynomiale telle que
      \begin{math}
        a \in A \Longleftrightarrow f(a) \in B
      \end{math}
    \item Supposons que $A \leq_p B$. Dans ce cas, si on suppose que $B$ est récursif, alors pour un élément $a$, il est possible de déterminer si $a \in A$ en effectuant $f(a) \in B$; la fonction $f$ existe (réduction polynomiale) et le test est décidable ($B$ est supposé récursif). Dès lors, si $A \leq_p B$ et $B$ est récursif, $A$ est récursif. Et donc, $A \leq_a B$ par définition de la réduction algorithmique.
  \end{enumerate}
\end{solution}

\end{document}
