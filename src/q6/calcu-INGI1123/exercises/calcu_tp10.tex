\subsection{} % Exercise 1.
Give the complexity of the following algorithms.
\begin{enumerate}
\item
\begin{minted}{julia}
function sum(n)
	sum = n*n*n
	for i = 0:n
		for j = 0:i
			sum += 1
		end
	end
end
\end{minted}
\item
\begin{minted}{julia}
function fun(n)
	if n > 1
		return fun(n/2) + fun(n/2)
	else
		return n
	end
end
\end{minted}
\end{enumerate}

\begin{solution}
\begin{enumerate}
	\item This function has to do two multiplications at the start,
	and one addition
	for every pass of the inner loop.
	It satisfies the following recurrence relation:
	\begin{align*}
		T(0) &= 2\,,\\
		T(n) &= T(n-1) + n + 1\,,\quad \forall n > 0\\
		&= T(n-2) + n + n + 1\\
		&= T(n-3) + n - 1 + n + n + 1\\
		&= T(n-k) + \sum_{i = 0}^{k-2} (n - i) + n + 1\\
		&= T(n-k) + (k-1)n - (k-2)(k-3)/2 + n + 1\,.
	\end{align*}
	We have a base case for \(T(0)\),
	hence we want \(n-k = 0\), or \(k = n\).
	Substituting this in the expression above,
	we obtain
	\(T(n) = 2 + n^2 - n - n^2/2 + 5n/2 - 3 + n + 1 =
	\frac{1}{2} n^2 + \frac{5}{2} n \in \Theta(n^2)\).
	\item This function's complexity satisfies the following recursion:
	\begin{align*}
		T(0) &= 1\,,\\
		T(1) &= 1\,,\\
		T(n) &= 2 T(n/2) + 1\,, \quad \forall n > 1\\
		&= 2 (2 T(n/4) + 1) + 1\\
		&= 2 (2 (2 T(n/8) + 1) + 1) + 1\\
		&= 2^k T(n/2^k) + 2^k - 1\,.
	\end{align*}
	Since we have a base case for \(T(1)\),
	we choose \(k\) such that \(n/2^k = 1\).
	This means that \(k = \lg n\).
	Plugging this in the expression above,
	we find that \(T(n) = 2n-1 \in \Theta(n)\).
\end{enumerate}
\end{solution}

\subsection{} % Exercise 2.
Algorithmic versus functional reduction.
\begin{enumerate}
	\item Prove that for every set \(A, A \ared \bar{A}\).
	\item Prove that for all sets \(A, B\) if \(A \fred B\)
	and \(B\) is recursively enumerable,
	then \(A\) is recursively enumerable.
	\item Show an example for which the relation \(A \fred \bar{A}\)
	does not hold.
	\item Show an example for which the following relation does not hold:
	if \(A \ared B\) and \(B\) is recursively enumerable,
	then \(A\) is recursively enumerable.
\end{enumerate}

\begin{solution}
	\begin{enumerate}
		\item By definition, \(A \ared \bar{A}\)
		means that if \(\bar{A}\) is recursive, then so is \(A\).
		To see why this is true,
		consider the characteristic function of \(\bar{A}\):
		\[
		f_{\bar{A}}(x) = \begin{cases} 1\,, & \textnormal{if } x\in \bar{A}\,,\\ 0\,, & \textnormal{otherwise.}\end{cases}
		\]
		For a recursive set, this function is computable.
		We can then construct the function
		\[
		f_A(x) = 1 - f_{\bar{A}}(x)\,,
		\]
		which is also computable,
		and decides \(A\).
		This proves that \(A \ared \bar{A}\).
		\item If \(B\) is recursively enumerable,
		that means we have a computable partial function
		\[
		f_B(x) = \begin{cases} 1\,, & \textnormal{if } x\in B\,,\\ \bot\,, & \textnormal{otherwise.}\end{cases}
		\]
		Now,
		\(A \fred B \implies \exists f \textnormal{ total and computable such that } a \in A \iff f(a) \in B\).
		This means we can find a computable partial function \(f_A(x)\)
		for \(A\) as follows:
		\[
		f_A(x) = f_B(f(x))\,.
		\]
		This function is computable because it is the composition
		of \(f\) and \(f_B\), two computable functions.
		\item Take for example \(A = \bar{K}\).
		Assume we have \(\bar{K} \fred K\).
		We know that \(K\) is recursively enumerable,
		hence we can apply the result of the previous question
		to show that \(\bar{K}\) is recursively enumerable.
		However, we know this is not the case,
		since \(\bar{K}\) is co-recursively enumerable,
		hence \(\bar{K} \centernot\fred K\).
		\item Take for example \(A = \bar{K}\).
		We know that \(\bar{K} \ared K\),
		and that \(K\) is recursively enumerable,
		however \(\bar{K}\) is not recursively enumerable
		(it is co-recursively enumerable).
	\end{enumerate}
\end{solution}

\subsection{} % Exercise 3.
Prove the following relations.
\begin{enumerate}
	\item Does \(A \ared B\) imply \(A \fred B\)?
	\item Does \(A \fred B\) imply \(A \ared B\)?
\end{enumerate}

\begin{solution}
	\begin{enumerate}
		\item Assume this is true.
		Consider \(\bar{K} \ared K\),
		which is true by the proof of Exercise~10.2.1.
		Then it would imply \(\bar{K} \fred K\).
		In Exercise~10.2.3,
		we showed that \(\bar{K} \centernot\fred K\).
		This shows that \(A \ared B \centernot\implies A \fred B\).
		\item Let \(A \fred B\).
		If \(B\) is recursive,
		then we have a function \(f_B(x)\) such that
		\[
		f_B(x) = \begin{cases} 1\,, & \textnormal{if } x\in B\,,\\ 0\,, & \textnormal{otherwise.}\end{cases}
		\]
		We also know that we have a total computable function \(f\)
		such that \(a \in A \iff f(a) \in B\),
		by virtue of the functional reducibility.
		Hence we can construct \(f_A(x)\) as follows:
		\[
		f_A(x) = f_B(f(x))\,.
		\]
		This function is total computable,
		hence it decides \(A\).
		This proves by construction that
		\(A \fred B \implies A \ared B\).
	\end{enumerate}
\end{solution}

\subsection{} % Exercise 4.
Let \(re = \{A \mid A \textnormal{ is recursively enumerable}\}\).
\begin{enumerate}
	\item State the conditions to prove that HALT is \(re\)-complete
	with respect to \(\ared\).
	\item Prove the conditions.
\end{enumerate}

\begin{solution}
	\begin{enumerate}
		\item We have to prove two conditions:
		\begin{itemize}
			\item \(\mathrm{HALT} \in re\);
			\item \(\forall B \in re, B \ared \mathrm{HALT}\).
		\end{itemize}
		\item Proving that HALT is recursively enumerable is simple:
		write the following program,
		which returns \(1\) if a given pair is in HALT,
		and does not halt otherwise:
\begin{minted}{julia}
function P_HALT(n, x)
	P_n(x)
	return 1
end
\end{minted}

		To prove that HALT is at least as hard to decide
		as any other recursively enumerable set
		is straightforward too.
		Assume HALT is recursive.
		This means that the halt function is total and computable:
		\[
		\halt(n, x) = \begin{cases} 1\,, & \textnormal{if } x\in \HALT\,,\\ 0\,, & \textnormal{otherwise.}\end{cases}
		\]
		Consider any set \(B \in re\).
		This means that this set has a partial function \(f_B(x)\)
		such that
		\[
		f_B(x) = \begin{cases} 1\,, & \textnormal{if } x\in B\,,\\ \bot\,, & \textnormal{otherwise.}\end{cases}
		\]
		Assume \(P_b(x)\) is the program that computes \(f_B(x)\).
		We can then use \(\halt(n, x)\) to make \(f_B(x)\) total,
		hence proving \(B\) is recursive:
		\(f_B(x) = \halt(b, x)\).
		This shows that if we assume \(\HALT\) is recursive,
		then so is \(B\), for all \(B \in re\).
		This is also written \(B \ared \HALT\).

		Together, these proofs show that \(\HALT\) is \(re\)-complete.
	\end{enumerate}
\end{solution}

\subsection{} % Exercise 5.
Let \(G\) be a non-directed graph of size \(\ge 3\).
A Hamiltonian path (resp. cycle) is a path (resp. cycle)
visiting each node of \(G\) exactly once.
Show that the two following problems can be functionally reduced to each other:
\begin{enumerate}
	\item \(G\) admits a Hamiltonian cycle.
	\item \(G\) admits a Hamiltonian path.
\end{enumerate}

\begin{solution}
	Define the following sets
	\begin{align*}
	HP &= \{G : G \textnormal{ admits a Hamiltonian path}\}\,,\\
	HC &= \{G : G \textnormal{ admits a Hamiltonian cycle}\}\,.\\
	\end{align*}
	\begin{enumerate}
		\item Here, we want to show that \(HP \fred HC\).
		This means that we want to show that
		\(G \in HP \iff \exists f
		\textnormal{ total and computable such that } f(G) \in HC\),
		that is, show that if we can determine
		whether a graph \(G\) contains a Hamiltonian cycle,
		then we can also determine
		whether \(G\) contains a Hamiltonian path,
		by running the decision algorithm
		on a modified version of the graph, \(f(G)\).

		We use the following function \(f\):
		let \(f\) be the function that adds one vertex to the graph,
		say \(v\), and that adds edges between \(v\)
		and all other vertices.
		If \(G\) contains a Hamiltonian path,
		then \(f(G)\) contains a Hamiltonian cycle.
		To see why, simply take the Hamiltonian path,
		which we assume starts at \(u\) and ends at \(w\),
		and add \(wvu\) to the path.
		This will be a Hamiltonian cycle.
		\item Now, we want to show that \(HC \fred HP\).
		This means we want to show that
		\(G \in HC \iff \exists f
		\textnormal{ total and computable such that } f(G) \in HP\),
		that is, show that if we can determine
		whether a graph \(G\) contains a Hamiltonian path,
		then we can also determine
		whether \(G\) contains a Hamiltonian cycle,
		by running the decision algorithm
		on a modified version of the graph, \(f(G)\).

		We use the following function \(f\):
		pick a random vertex of the graph, say \(v\).
		Add a vertex \(u\), which only has an edge to \(v\),
		and a vertex \(n\), which has edges to all neighbours of \(v\).
		Finally, add a vertex \(w\) which only has an edge to \(n\).
		If the graph \(f(G)\) obtained
		by applying these manipulations
		to \(G\) contains a Hamiltonian path,
		then \(G\) contains a Hamiltonian cycle.
		To see why, the following construction proves
		that if one finds a Hamiltonian path in \(f(G)\),
		then \(G\) must have a Hamiltonian cycle.
		when we try to find a Hamiltonian path on this graph.
		The path has to start (or end) at \(u\),
		since otherwise there is no way we can ever reach this vertex
		without getting stuck there (we cannot leave,
		since then we would have to visit \(v\) a second time
		which is not allowed).
		This also means that \(v\) is the second vertex of the path.
		The path also has to end (or start) at \(w\),
		for the same reason as above.
		This means that right before arriving at \(n\),
		the path was at a neighbour of \(v\), say \(a\)
		(since those are the only vertices with edges to \(n\)).
		If we now cut out \(uv\) and \(anw\),
		the remaining path visits all vertices of \(G\).
		The final step is to notice that we have not yet used
		the edge from \(a\) to \(v\),
		meaning we can append \(av\) to the path
		in order to obtain a Hamiltonian cycle.
	\end{enumerate}
\end{solution}
