\subsection{} % Exercise 1.
Assume you have a polynomial time algorithm to decide
the satisfiability of a propositional formula.
Describe how to use this algorithm to find satisfying values for the variables
in polynomial time.
What kind of reduction is it?
(\(\ared, \fred\) or \(\pred\)?)

\begin{solution}
	If we have a polynomial time algorithm to decide the satisfiability,
	then we have the following characteristic function:
	\[
	f_{\textnormal{SAT}}(F) = \begin{cases} 1\,, & \textnormal{if } F \textnormal{ is satisfiable,}\\
	0\,, & \textnormal{otherwise,}
	\end{cases}
	\]
	which executes in polynomial time.
	Assume that this function is computed by program \(P_\textnormal{SAT}\).

	Now, we give a linear reduction function to find satisfying values
	in polynomial time.
	Consider the following program:
\begin{minted}[escapeinside=||]{julia}
function satsolve(F)
	if P_SAT(F) == 0
		return "not satisfiable"
	else
		assign = |$\emptyset$|
		for x |$\in$| vars(F)
			F.set(x, true) # This forces the test to set x = true.
			if P_SAT(F)
				assign = assign |$\cup$| {(x, true)}
			else
				assign = assign |$\cup$| {(x, false)}
				F.set(x, false)
			end
		end
		return assign
	end
end
\end{minted}
	This kind of reduction is thus a polynomial reduction,
	since it is more restrictive than simply algorithmic or functional
	(the algorithm needs to be a polynomial function).
\end{solution}

\subsection{} % Exercise 2.
Which of the following can we infer
from the fact that the Hamiltonian cycle (HC) problem
is \(\mathcal{NP}\)-complete,
if we assume that \(\mathcal{P} \ne \mathcal{NP}\)?
What would your answer be if \(\mathcal{P} = \mathcal{NP}\)?
Justify.
\begin{enumerate}
	\item There does not exist an algorithm
	that solves the HC problem in polynomial time.
	\item There exists an algorithm that solves the HC problem
	in polynomial time, but no one has been able to find it yet.
	\item The HC problem is not in \(\mathcal{P}\).
	\item The problem of sorting an array can be reduced
	in polynomial time to the HC problem.
	\item All non-deterministic algorithms for the HC problem
	run in polynomial time.
\end{enumerate}

\begin{solution}
\begin{enumerate}
	\item This depends on whether \(\mathcal{P} = \mathcal{NP}\).
	If \(\mathcal{P} = \mathcal{NP}\), then a polynomial algorithm exists.
	If not, then there cannot be a polynomial algorithm.
	\item Again, this depends on whether \(\mathcal{P} = \mathcal{NP}\).
	If yes, then there is a polynomial algorithm,
	and seen as we don't know that algorithm yet, the statement is true.
	If no, then there is no polynomial algorithm
	to be found for the HC problem.
	\item If \(\mathcal{P} = \mathcal{NP}\), this statement is false.
	If not, it is true.
	\item This statement can be implied from \(\mathcal{NP}\)-completeness
	of the Hamiltonian cycle problem,
	since this property precisely means that any problem in \(\mathcal{NP}\)
	(which we know is true for the sorting problem),
	this problem is polynomially reducible to HC.
	We can write this as \(\mathrm{SORT} \pred HC\).
	\item This statement is false.
	One simply has to construct an algorithm
	which does an exponential number of operations in each branch.
\end{enumerate}
\end{solution}

\subsection{} % Exercise 3.
The Traveling Salesman (TS) problem is the problem of finding a cycle
in a graph with stricly positive weights that touches every vertex
such that the sum of the weights on the cycle do not exceed a given bound.
The Hamiltonian Cycle (HC) problem is the problem of finding,
in a graph, a cycle that touches every vertex eactly once.
Prove that TS is \(\mathcal{NP}\)-complete by using the fact
that HC is \(\mathcal{NP}\)-complete.

\begin{solution}
	We wish to show that one can polynomially reduce
	the HC problem to an instance of the TS problem for some cost, that is,
	\(HC \pred TS(c)\),
	and that \(TS(c) \in \mathcal{NP}\).

	First, we show that \(TS(c) \in \mathcal{NP}\).
	Consider the following nondeterministic program
	which solves \(TS(c)\) in polynomial time:
\begin{minted}[escapeinside=||]{julia}
function ND_TS(G, c)
	cost = 0
	cycle = |$\emptyset$|
	visited = |$\emptyset$|
	for i |$\in$| 1:|G.V|
		edge = choose(|G.E|-1)
		cycle = cycle |$\cup$| {edge} # Add the edge to the cycle.
		visited = visited |$\cup$| {{edge.1, edge.2}} # Multiset of vertices.
		cost = cost + edge.weight
	end
	if visited contains every v |$\in$| G.V exactly twice && cost |$\le$| c
		return YES
	else
		return NO
	end
end
\end{minted}

	Next, we must find a polynomial function \(f\)
	such that \(G \in HC \iff f(G) \in TS(c)\),
	for a given cost \(c\).
	Let \(f\) be the function which assigns a unit weight
	to every edge of the graph.
	We can determine if \(G \in HC\)
	by seeing if \(f(G) \in TS(\abs{V(G)})\).
	This function takes linear time
	in the number of edges of the graph,
	and is hence polynomial.
	This means that \(HC \pred TS(\abs{V(G)})\),
	which concludes the proof.
\end{solution}

\subsection{} % Exercise 4.
Let \(X\) and \(Y\) be two decision problems such that \(X \pred Y\).
True or false?
\begin{enumerate}
	\item If \(X\) is \(\mathcal{NP}\)-complete, then so is \(Y\).
	\item If \(Y\) is \(\mathcal{NP}\)-complete
	and \(X\) is in \(\mathcal{NP}\),
	then \(X\) is \(\mathcal{NP}\)-complete.
	\item If \(X\) is \(\mathcal{NP}\)-complete
	and \(Y\) is in \(\mathcal{NP}\),
	then \(Y\) is \(\mathcal{NP}\)-complete.
	\item If \(X\) is in \(\mathcal{P}\),
	then \(Y\) is in \(\mathcal{P}\).
	\item If \(Y\) is in \(\mathcal{P}\),
	then \(X\) is in \(\mathcal{P}\).
\end{enumerate}

\begin{solution}
	\begin{enumerate}
		\item False.
		We can only say that \(Y\) is \(\mathcal{NP}\)-hard.
		For \(Y\) to be \(\mathcal{NP}\)-complete,
		we need to know that \(Y\) is \(\mathcal{NP}\)-hard
		and that \(Y \in \mathcal{NP}\).
		\item False.
		Consider \(X = \mathrm{SORT}\),
		and \(Y = HC\) for a counterexample
		(since SORT is polynomial).
		\item True.
		Since \(Y\) is at least as hard
		as any problem in \(\mathcal{NP}\)
		by virtue of the polynomial reduction,
		and since \(Y \in \mathcal{NP}\),
		\(Y\) is itself \(\mathcal{NP}\)-complete.
		\item False.
		Consider the following counterexample:
		\(X = \mathrm{SORT}\), \(Y = HC\).
		\item True.
		Since \(X\) is polynomially reducible to \(Y\),
		if there is a polynomial algorithm for \(Y\),
		then there is also one for \(X\)
		(the composition of \(f\) which is polynomial by the reduction
		and the algorithm that solves \(Y\)).
	\end{enumerate}
\end{solution}

\subsection{} % Exercise 5.
Provided TS is \(\mathcal{NP}\)-complete, prove HC is \(\mathcal{NP}\)-complete.

\begin{solution}
We first need to prove that \(HC \in \mathcal{NP}\).
This is straightforward: simply guess a permutation and check:
\begin{minted}[escapeinside=||]{julia}
function ND_HC(G)
	cycle = |$\emptyset$|
	visited = |$\emptyset$|
	for i |$\in$| 1:|G.V|
		edge = choose(|G.E|-1)
		cycle = cycle |$\cup$| {edge} # Add the edge to the cycle.
		visited = visited |$\cup$| {{edge.1, edge.2}} # Multiset of vertices.
	end
	if visited contains every v |$\in$| G.V exactly twice
		return YES
	else
		return NO
	end
end
\end{minted}

To show that HC is \(\mathcal{NP}\)-complete,
we show that \(TS(c) \pred HC\), for some cost \(c\).
This means that there exists a total computable polynomial function \(f\)
such that \(G \in TS(c) \iff f(G) \in HC\).
% TODO https://www8.cs.umu.se/kurser/TDBA77/VT06/algorithms/LEC/LECTUR16/NODE22.HTM
\textcolor{red}{TODO}
\end{solution}
