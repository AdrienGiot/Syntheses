\subsection{} % Exercise 1
Let \(X\) and \(Y\) be two recursive sets.
Which of the following operations are preserving recursivity:
union, intersection and complement?
Prove your answer.
Same question for \(X\) and \(Y\) being recursively enumerable sets.

\begin{solution}
	For union, one can simply define the characteristic function
	\(f_{X \cup Y} \colon \N \to \{0, 1\}\) as
	\begin{equation}
		f_{X \cup Y}(x) = f_X(x) + f_Y(y) - f_X(x)f_Y(x)\,.
	\end{equation}
	Similarly, for intersection,
	\begin{equation}
		f_{X \cap Y}(x) = f_X(x)f_Y(x)\,.
	\end{equation}
	Finally, for complement,
	\begin{equation}
		f_{\overline{X}}(x) = 1 - f_X(x)\,.
	\end{equation}
	Since \(X\) and \(Y\) are recursive sets, \(f_X\) and \(f_Y\)
	are total computable functions,
	hence so are \(f_{X \cup Y}\),
	\(f_{X \cap Y}\) and \(f_{\overline{X}}\).
	This means \(X \cup Y\), \(X \cap Y\) and \(\overline{X}\)
	are recursive sets as well.

	If we instead take \(X\) and \(Y\) to merely be recursively enumerable,
	we arrive at a slightly different conclusion:
	for union, we define the characteristic function as
	\begin{equation}
		f_{X \cup Y}(x) =
		\begin{cases}
			1\,, & \textnormal{if } f_X(x) = 1 \textnormal{ or } f_Y(x) = 1\,,\\
			0\,, & \textnormal{otherwise.}
		\end{cases}
	\end{equation}
	Again, we only depend on \(f_X(x)\) and \(f_Y(x)\) telling us
	whether \(x \in X\) or \(x \in Y\).
	This information is available, as the characteristic function
	for a recursively enumerable set provides it.
	Similarly, for intersection,
	\begin{equation}
		f_{X \cap Y}(x) =
		\begin{cases}
			1\,, & \textnormal{if } f_X(x) = 1 \textnormal{ and } f_Y(x) = 1\,,\\
			0\,, & \textnormal{otherwise.}
		\end{cases}
	\end{equation}
	By much the same reasoning,
	we again find that recursive enumerability is conserved, as with union.
	However, for the complement, we give a counterexample:
	the set \(K = \{n : P_n(n) \textnormal{ halts}\}\)
	is recursively enumerable, while \(\overline{K}\) is not.
\end{solution}

\subsection{} % Exercise 2
Show that all monotonically decreasing total functions
\(f \colon \N \to \N\) are computable.

\begin{solution}
Since the function is monotonically decreasing,
there must exist \(n_0\) such that
for all \(n \ge n_0\), \(f(n) = f(n_0) \ge 0\),
that is, the function has a nonnegative lower bound.
Similarly, for all \(n \ge 0\), \(f(n) \le f(0) < +\infty\).
We know that \(n_0\) is finite, hence we could technically ``hard-code''
the different values of the function,
giving a proof of its computability, as such:
\begin{minted}{julia}
# f: array of size n0 + 1 with hard-coded values from 0 to n0.
function compute_monotonically_decreasing(n)
	return f[min(n, n0)]
end
\end{minted}
This only works because there are finitely many values to be hard-coded.
\end{solution}

\subsection{} % Exercise 3
Let \(X \subseteq \N\) and \(f(n)\) a total function defined by
\begin{equation}
	f(n) = \# \{x \in X : x < n\}\,,
\end{equation}
where \(\# A\) is the cardinal of \(A\).
Show that \(X\) is recursive if and only if \(f\) is computable.

\begin{solution}
We start by showing that if \(X\) is recursive, then \(f\) must be computable.
For this, we write a short program, using the fact
that the characteristic function of a recursive set is total computable.
We assume that \mintinline{julia}{f_X}
computes the characteristic function of \(X\).
\begin{minted}[escapeinside=||]{julia}
function f(n)
	cardinality = 0
	for i |$\in$| 1:n
		cardinality += f_X(i)
	end
	return cardinality
end
\end{minted}

Next we must show that if the function is computable,
then the set must be recursive.
We can again write a simple program for this, using the function defined above:
\begin{minted}{julia}
function f_X(n)
	if n == 0
		return f(1)
	else
		return f(n+1) - f(n)
	end
end
\end{minted}

These programs prove the required implications.
\end{solution}

\subsection{} % Exercise 4
Let \(X \subseteq \N\).
Show that the following propositions are equivalent
(show it at least for (a), (b) and (c)):
\begin{enumerate}
	\item \(X\) is recursively enumerable.
	\item There is a computable partial function \(f\)
	such that \(X = \dom f\).
	\item There is a \java{} program that enumerates \(X\).
	\item \(X = \emptyset\) or there is a total computable function \(f\)
	such that \(X = \image f\).
\end{enumerate}

\begin{solution}
We first show (a) implies (b).
\begin{proof}
	One can simply take the characteristic function of \(X\),
	which is by definition computable and partial,
	with \(X = \dom f\).
\end{proof}

Next, we show (b) implies (c).
\begin{proof}
We can assume without loss of generality
that \(f\) is the characteristic function.
We can then write the following program,
using the \mintinline{julia}{@distributed for} macro
which launches a for loop in multiple threads,
and assumes \mintinline{julia}{f_X}
computes the characteristic function of \(X\).
\begin{minted}[escapeinside=||]{julia}
function enumerate_X()
	if isempty(X)
		return
	else
		@distributed for i |$\in$| 1:Inf
			f_X(i) == 1 ? continue : println(i)
		end
	end
end
\end{minted}
To prove we can launch this potentially countably infinite number of threads,
we can follow this execution scheme:
\[
\begin{array}{c|cccc}
	& f_X(1) & f_X(2) & f_X(3) & \cdots \\
	\hline
	1 & f_X(1)_1 & f_X(2)_1 & f_X(3)_1 & \cdots\\
	2 & f_X(1)_2 & f_X(2)_2 & f_X(3)_2 & \cdots\\
	3 & f_X(1)_3 & f_X(2)_3 & f_X(3)_3 & \cdots\\
	\vdots & \vdots & \vdots & \vdots & \ddots
\end{array}
\]
The parenthesized number denotes the value of \mintinline{julia}{i},
while the subscript denotes which operation of
the computation of \(f_X(i)\) is being done.
We can zig-zag through this table as such:
\((f_X(1)_1 \to f_X(2)_1 \to f_X(1)_2 \to f_X(1)_3 \to \cdots\).
This way, all threads will eventually finish.
\end{proof}

Now, we show (c) implies (d).
\begin{proof}
Call the enumeration program \(P_d\),
and define \(\phi_d\) as the (computable) function it computes.
Since the program enumerates \(X\),
the return values are simply the elements of \(X\).
This means that \(\image \phi_d = X\).
\end{proof}

Finally, we show that (d) implies (a).
\begin{proof}
If \(X\) is empty, it is trivially recursively enumerable.
If a total computable function \(f\) such that \(X = \image f\) exists,
then one has the following characteristic function for \(X\):
\[
f_X(x) = \begin{cases}
1\,, & \textnormal{if } \exists i \in \N \textnormal{ such that } f(i) = x\,,\\
\bot\,, & \textnormal{ otherwise.}
\end{cases}
\]
The second possibility, \(f_X(x) = \bot\),
is a consequence of having to try all naturals as possible inputs for \(f\)
in case \(x \notin X\).
\end{proof}
\end{solution}

\subsection{} % Exercise 5
True or false?
\begin{enumerate}
	\item If the domain of a function is finite,
	then the function is computable.
	\item If a function is computable, then its domain is recursive.
	\item \(X\) is recursive if and only if
	both \(X\) and \(\overline{X}\) are recursively enumerable.
	\item \(X\) is recursive if and only if \(\overline{X}\) is recursive.
	\item A function whose table can be defined in a finite way
	is necessarily computable.
\end{enumerate}

\begin{solution}
\begin{enumerate}
	\item True.
	One must simply hard-code the finite number of values
	and return the requested one when the function is called,
	in a similar manner to what was used in Exercise~2.2.
	\item True.
	If this were false, then one could use this function to decide the set,
	which would be absurd.
	\item True.
	If \(X\) is recursive, then so is \(\overline{X}\).
	This proves the implication in the first direction.
	If both sets are recursively enumerable,
	then one can combine their partial characteristic functions
	in order to obtain a total characteristic function for \(X\)
	(or for \(\overline{X}\)).
	Having this total characteristic function implies recursivity.
	\item True.
	As shown in Exercise~2.1,
	the complement operation preserves recursivity.
	\item True
	(assuming ``finite way'' means
	``as a finite set of input-output pairs'').
	One should simply hard-code all the possible cases
	in the function definitions,
	in a similar manner to what was used in Exercise~2.2.
\end{enumerate}
\end{solution}

\subsection{} % Exercise 6
Construct a monotonically increasing total function that is not computable.
(Hint: use HALT.)


\begin{solution}
We can construct the following function:
\[
f(n) = \sum_{i = 0}^n \halt(i, i)\,.
\]
This function is total because \(\mathrm{halt}\) is total,
and is increasing because \(\mathrm{halt}\) is always non negative,
but is not computable.
To see why, one simply has to observe that if it were computable,
then one could decide \(K\) by doing \(f_K(n) = f(n) - f(n-1)\),
where \(f_K(n)\) would be the total computable characteristic function of \(K\).

Alternatively, a slightly more complicated function
is the busy beaver function,
also called the Radó Sigma,
which is not only monotonically increasing,
but grows faster than any computable function!
\end{solution}
