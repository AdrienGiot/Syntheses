\subsection{} % Exercise 1.
Let \(A = \{i \mid \exists x, y : \phi_i(x) = \phi_i(y) \ne \bot, x \ne y\}\)
be the set of programs that halt while returning the same value
for at least two different inputs.
Using Rice's theorem, show that \(A\) is not recursive.

\begin{solution}
	We start by showing that \(A\) is not trivial:
	to see that it is not empty, simply consider all constant functions,
	which are in \(A\);
	to see that it is not the set of all naturals,
	consider the identity function,
	which is not in \(A\).

	We define \(\bar{A}\) as
	\(\bar{A} = \{j \mid \nexists x, y :
	x \ne y, \phi_j(x) = \phi_j(y) \ne \bot\}\).
	Equivalently,
	\(\bar{A} = \{j \mid \forall x, y :
	x \ne y \implies \phi_j(x) \ne \phi_j(y)
	\lor \phi_j(x) = \bot \lor \phi_j(y) = \bot\}\).
	This implies that
	\(\forall i \in A, \forall j \in \bar{A} \exists x, y :
	x \ne y \land (\phi_i(x) \ne \phi_j(x) \lor \phi_i(y) \ne \phi_j(y))\).
	This basically says that
	\(\forall i \in A, \forall j \in \bar{A}, \phi_i \ne \phi_j\).
	One can now apply Rice's theorem,
	using the fact that \(A\) is not trivial
	and the predicate above
	to conclude that \(A\) is not decidable.
\end{solution}

\subsection{} % Exercise 2.
Explain whether the following sets are recursive
as a consequence of Rice's theorem.
Justify your answers.
\begin{enumerate}
	\item Deciding whether a given program computes a given function \(f\).
	Does this computability result depend on \(f\)?
	\item Deciding whether a given program produces \(0\) as output
	for at least one input.
	\item Deciding whether a given program terminates
	after less than \(1000\) instruction for input \(0\).
	\item Deciding whether a given program terminates
	for at least two inputs.
\end{enumerate}

\begin{solution}
\begin{enumerate}
	\item If \(f\) is not computable,
	then no program can compute it (by definition);
	this means that the set is equal to the empty set,
	which is trivially recursive.

	If \(f\) is computable,
	then we want to find if
	\(A = \{i : \phi_i = f\}\)
	is a recursive set.
	We define \(\bar{A} = \{j : \phi_j \ne f\}\).
	If \(f\) is computable, then there exists
	at least program which computes \(f\),
	hence \(A\) is not empty.
	It is also not the set of natural numbers,
	since if \(f = \bot\), then there are functions which are not in \(A\)
	(for example, constant functions)
	and if \(f \ne \bot\), then \(\phi_n = f+1 \ne f\)
	is also computable, yet not in \(A\).
	We have that
	\(\forall i \in A, \forall j \in \bar{A}, \phi_i = f \ne \phi_j\),
	which also means
	\(\forall i \in A, \forall j \in \bar{A}, \phi_i \ne \phi_j\).
	This allows us to use Rice's theorem,
	which together with the nontriviality of \(A\)
	guarantees that it is not recursive.
	\item We define \(A = \{i \mid \exists x : \phi_i(x) = 0\}\).
	This means that \(\bar{A} = \{j \mid \forall x : \phi_j(x) \ne 0\}\).
	More specifically,
	\(\forall i \in A, \forall j \in \bar{A}, \exists x:
	\phi_i(x) = 0 \ne \phi_j(x)\),
	meaning that
	\(\forall i \in A, \forall j \in \bar{A}, \phi_i \ne \phi_j\).
	This means that by Rice's theorem,
	either \(A\) is not recursive, or it is trivial.
	Take \(f_1 = 0\) and \(f_2 = 1\), for all inputs as examples.
	This shows that \(A\) is neither empty nor equal to \(\N\),
	and hence is not trivial.
	This allows us to conclude that \(A\) is not recursive,
	by Rice's theorem.
	\item To show that this set is recursive,
	we simply show that its characteristic function is computable.
	To show this, we write a program which computes it.
	Simply let the program execute until it
	either reaches \(1000\) instructions
	or until it finishes, whichever happens first,
	returning \(0\) in the first case and \(1\) otherwise.
	One can trivially see this is possible,
	and that this computes the characteristic function for \(A\).
	The computability of its characteristic function guarantees
	that \(A\) is a recursive set.
	\item We define \(A = \{i \mid \exists x, y:
	\phi_i(x) \ne \bot \land \phi_i(y) \ne \bot\}\).
	This set is not empty (consider constant functions which are in it)
	but it is not equal to \(\N\) (consider the \(\bot\) function,
	which is not in the set).
	Reciprocally, we can define \(\bar{A} = \{j \mid \forall x, y:
	\phi_j(x) = \bot \lor \phi_j(y) = \bot\}\).
	This means that \(\forall i \in A, \forall j \in \bar{A}, \exists x, y:
	\phi_i(x) \ne \bot \phi_j(x) \lor \phi_i(y) \ne \bot = \phi_j(y)\),
	or equivalently,
	\(\forall i \in A, \forall j \in \bar{A}, \phi_i \ne \phi_j\).
	Using Rice's theorem as well as the nontriviality of \(A\),
	we can conclude that \(A\) is not a recursive set.
\end{enumerate}
\end{solution}

\subsection{} % Exercise 3.
Let \(A \subseteq \N, A \ne \N\) et \(A \ne \emptyset\).
If it exists \(i \in A\) and \(j \in \N \setminus A\)
such that \(\phi_i = \phi_j\),
can we say that \(A\) is recursive? Prove your answer.

\begin{solution}
	No, we cannot say that.
	To see why, simply consider
	the following counterexample:
	\(A = \{i :
	P_i \textnormal{ halts after an odd number of steps for all inputs}\}\).
	This set is not trivial, yet it is not recursive
	(if it were, then HALT would be recursive too, by reduction:
	one would be able to use the characteristic function of \(A\)
	in order to decide the set \(S_1\) of Exercise~3.1.1,
	which is not decidable by reduction to HALT).
	However, let \(i \in A\).
	One can easily construct a program \(P_j\)
	which computes \(\phi_i\) too,
	while adding one instruction which does nothing at the end
	to make sure that \(j \notin A\).
	Both of these programs compute the same function,
	with one being in the set and the other not being in it,
	however the set is not recursive, which contradicts the statement.
\end{solution}

\subsection{} % Exercise 4.
Let \(A \subseteq \N\) not recursive (\(A \ne \N\) and \(A \ne \emptyset\)).
Can we say that \(\forall i \in A\) and \(\forall j \in \N \setminus A\),
\(\phi_i \ne \phi_j\)? Prove your answer.

\begin{solution}
	Again, this is not true.
	Consider the same set as in the previous exercise,
	\(A = \{i :
	P_i\) halts after an odd number of steps for all inputs\(\}\).
	This set is not recursive by a reduction argument,
	yet it is easy to construct a program that does
	one ``useless'' instruction yet still computes the same result.
\end{solution}

\subsection{} % Exercise 5.
True or false?
\begin{enumerate}
	\item There exists a \java{} program that decides whether
	the function computed by a given program has a finite domain.
	\item There will never exist
	an optimal and safe \java{} garbage collector.
	\item Let \(A = \{i\mid P_i(x) \textnormal{ halts for some input }x\}\).
	Is \(A\) recursive?
	\item Let \(A = \{i\mid P_i(x) \textnormal{ halts for some input }x\}\).
	Is \(A\) recursively enumerable?
\end{enumerate}
\begin{solution}
\begin{enumerate}
\item False.
By Rice's theorem; define
\(A = \{i \mid \phi_i \textnormal{ has a finite domain}\}\).
Clearly, \(A\) is not trivial.
We also have
\(\bar{A} = \{j \mid
\phi_j \textnormal{ has an infinite domain}\}\).
Clearly, functions with different domains cannot be the same.
This means that
\(\forall i \in A, \forall j \in \bar{A}, \phi_i \ne \phi_j\).
Since \(A\) is not trivial, this leads us to conclude
that \(A\) must be non recursive.
If \(A\) is not recursive, then its characteristic function
is not computable, and thus there can be no such program.
\item True.
Determining whether an object can be garbage-collected
is an undecidable problem,
since one cannot know what the lifetime of an object will be:
by Rice's theorem, non trivial program properties
are undecidable.
\item False.
By reduction to HALT:
assume that \(A\) is decidable.
We then have a computable characteristic function for \(A\):
\[
f_A(n) =
\begin{cases}
1\,, & \textnormal{if } P_n(x) \textnormal{ halts for some }x\,,\\
0\,, & \textnormal{otherwise.}
\end{cases}
\]
We can then write a program which we assume to have Gödel number \(d\):
\begin{minted}{julia}
function P_d(x)
	P_n(k)
end
\end{minted}
Here, we let \(n\) and \(k\) be fixed outside of the function.
Now, we have \(\halt(n, k) = f_A(d)\),
which is also computable and a characteristic function for HALT.
Clearly, this is wrong, because HALT is not decidable;
we conclude that our initial assumption is wrong: \(A\) is not decidable.
\item True.
We can easily give the partial characteristic function for this set,
\[
f_{A}(n) =
\begin{cases}
1\,, & \textnormal{if } P_n(x) \textnormal{ halts for some } x\,,\\
\bot\,, & \textnormal{otherwise.}
\end{cases}
\]
For this, we simply have to launch a thread for every possible value of \(x\).
Once one of them halts, we return \(1\).
If none of the threads halt, we compute \(\bot\).
\end{enumerate}
\end{solution}

\subsection{} % Exercise 6.
Let \(A = \{i \mid P_i(x) \textnormal{ halts for every input } x\}\).
\begin{enumerate}
	\item Is \(A\) recursive? Why?
	\item Is \(A\) recursively enumerable? Why?
\end{enumerate}

\begin{solution}
\begin{enumerate}
\item No, it is not.
Assume it is decidable.
We then have the following
total and computable characteristic function for \(A\):
\[
f_A(n) =
\begin{cases}
1\,, & \textnormal{if } P_n(x) \textnormal{ halts for every input } x\,,\\
0\,, & \textnormal{otherwise.}
\end{cases}
\]
We can then write a program which we assume to have Gödel number \(d\):
\begin{minted}{julia}
function P_d(x)
	P_n(k)
end
\end{minted}
Here, we let \(n\) and \(k\) be fixed outside of the function.
Now, we have \(\halt(n, k) = f_A(d)\),
which is also total computable and a characteristic function for HALT.
Clearly, this is wrong, because HALT is not decidable;
we conclude that our initial assumption is wrong: \(A\) is not decidable.
\item No, it is not.
We start by showing that
\(\overline{\HALT} = \{(n, x) : P_n(x) \textnormal{ does not halt}\}\)
is not recursively enumerable.
This is simple: if it were, then both \(\HALT\) and \(\overline{\HALT}\)
would be recursively enumerable, meaning they would both be recursive.
We know this is not the case,
hence \(\overline{\HALT}\) is not recursively enumerable.
Next, we show that \(\overline{\HALT}\) is reducible to \(A\).
For this, suppose that \(A\) is decidable.
This means we have a program \(P_A(d)\)
which tells us whether \(d \in A\), that is,
whether \(P_d\) halts for every input.
We build a program which we assume to have Gödel numbering \(q\):
\begin{minted}{julia}
function P_q(x)
	Exec x instructions of P_n(k)
	if P_n(k) didn't halt
		return 1
	else
		while true
		end
	end
end
\end{minted}
Now, we write the following program,
which determines whether a given program is in \(\overline{\HALT}\):
\begin{minted}[escapeinside=||]{julia}
function P_|$\overline{\HALT}$|(n, k)
	Build P_q(x) # q depends on n and k.
	return P_A(q)
end
\end{minted}
In this function, \(P_A(q)\) uses the program
that computes the characteristic function of \(A\).
This shows that \(\overline{\HALT} \le_\textnormal{a} A\),
that is, that since \(\overline{\HALT}\) is not recursively enumerable,
which we proved above,
neither is \(A\).
\(A\) is \emph{at least} as hard to decide as \(\overline{\HALT}\).

This problem is called the universal halting problem,
and it is known to be harder than the existential halting problem.
Mathematically, the set \(A\) is \href{https://en.wikipedia.org/wiki/Arithmetical_hierarchy}{\(\Pi_2^0\)-complete},
whereas \(\overline{\HALT}\) is \(\Pi_1^0\)-complete.
Since \(\Pi_1^0 \subsetneq \Pi_2^0\),
this makes \(A\) \emph{strictly} harder to decide than \(\overline{\HALT}\).
\end{enumerate}
\end{solution}
