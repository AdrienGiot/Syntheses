\subsection{} % Exercise 1.
The \(S\) property is defined as follows:
\[
\forall k \exists S \textnormal{ total and computable}: \phi_k(x, y) = \phi_{S(x)}(y)\,.
\]
Prove that the \(S\) property is a particular case of \(s_{mn}\)
(i.e. prove that \(s_{mn}\) implies \(S\) for \(m = n = 1\))

\begin{solution}
\begin{proof}
	We recall that the \(s_{mn}\) property is defined as
	\[
	\forall m,n > 0 \exists \smn \colon \N^{m+1} \to \N \forall k : \phi_{k}^{(n+m)}(x_1,\ldots, x_n, x_{n+1}, \ldots, x_{n+m})
	= \phi_{\smn(k, x_{n+1}, \ldots, x_{n+m})}^{(n)} (x_1, \ldots,x_n)\,,
	\]
	where \(\smn\) is total computable.
	Let's state this property, but this time for \(m = n = 1\)
	(and removing some irrelevant parts):
	\[
	\exists s_1^1 \forall k : \phi_k(x_1, x_2)
	= \phi_{s_1^1(k, x_2)}(x_1)\,.
	\]
	We rename some of the variables to obtain
	\[
	\exists S''' \forall k : \phi_k(x, y)
	= \phi_{S'''(k, y)}(x)\,.
	\]
	Next, observe that if \(S'''\) is computable,
	then \(S'''(k, y) = \phi_{S'''}(k, y)\).
	Using the fact that
	\(\forall k \exists k': \phi_k(x, y) = \phi_{k'}(y, x)\),
	we find that
	\(\phi_{S'''}(k, y) = \phi_{S''}(y, k)\),
	for some total computable function \(S''\).
	Now, we apply \(s_{mn}\) to this
	to find a total computable function \(S'\)
	which satisfies
	\(\phi_{S''}(y, k) = \phi_{S'(S'', k)}(y)\).
	We rename this last function as
	\(\phi_{k'}(y)\), where \(k' = S'(S'', k)\),
	which can again be renamed as \(S\) (i.e. \(S = \phi_{k'}(y)\)),
	meaning that
	\(\forall k \exists S \textnormal{ total and computable}:
	\phi_k(x, y) = \phi_{S(k, y)}(x)\).
\end{proof}
\end{solution}

\subsection{} % Exercise 2.
Using the fixed point theorem, show that there exists a program \(P_n\)
such that \(P_n\) terminates only for input \(n\).
(Hint: use the function \(g(n, x) = 1\) if \(x = n\), \(\bot\) otherwise
together with the \(S\) property.)

\begin{solution}
	We can show that \(g\) is computable,
	which implies that
	\(\exists d : \phi_d(n, x) = g(n, x)\).
	By the \(S\) property,
	this means that \(\exists s: \phi_d(n, x) = \phi_{s(n)}(x)\),
	where \(s\) is total and computable.
	By the fixed point theorem,
	we then have \(\exists n : \phi_n = \phi_{s(n)}\),
	and that \(\exists n: \phi_n(x) = g(n, x)\)
	This means that
	\[
	\exists n: \phi_n(x) = \phi_d(n, x) = g(n, x) = \begin{cases}1\,, & \textnormal{if } x = n\,,\\ \bot\,, & \textnormal{otherwise.}\end{cases}
	\]
	This function thus only halts when its input
	is equal to its Gödel number,
	and its existence proves the statement.
\end{solution}

\subsection{} % Exercise 3.
Using the fixed point theorem, show that there exists a program \(P_n\)
that always outputs \(n\) (i.e. that prints its source code).

\begin{solution}
\begin{proof}
	Define a function \(g(n, x) = n\).
	This function is computable,
	hence we can say that \(g = \phi_d\) for some \(f\).
	By the \(S\) property,
	\(\exists s : \phi_d(n, x) = \phi_{s(n)}(x)\),
	with \(s\) total and computable.
	By the fixed point theorem,
	we have that \(\exists n : \phi_n = \phi_{s(n)}\).
	This means that \(\exists n: \phi_n(x) = \phi_{s(n)}(x) = g(n, x)\).
	We can reduce that to \(\exists n :\phi_n = n\),
	hence \(\exists P_n\) such that \(P_n\) always outputs \(n\).
\end{proof}
\end{solution}

\subsection{} % Exercise 4.
Prove Rice's theorem using the fixed point theorem.
(Hint: define the function \(f(x) = i\) if \(x\in A\), \(j\) if \(x\in\bar{A}\),
with \(i \in \bar{A}\) and \(j \in A\).)

\begin{solution}
\begin{proof}
	We rewrite the definition of \(f\):
	\[
	f(x) = \begin{cases} c \in \bar{A}\,, & \textnormal{ if } x \in A\,,\\
	d \in A\,, & \textnormal{ if } x \in \bar{A}\,. \end{cases}
	\]
	We want to prove Rice's theorem:
	\begin{mytheo}[Rice's theorem]
		If \(A\) is recursive and \(A \ne \N\), \(A \ne \emptyset\),
		then \(\exists i \in A, \exists j \in \bar{A}:
		\phi_i = \phi_j\).
	\end{mytheo}
	Since \(A\) is recursive,
	\(f\) is a computable function,
	and since \(A\) is not trivial,
	one can find \(c, d\) which satisfy \(c \in \bar{A}, d \in A\).
	By the fixed point theorem, one then has
	\(\exists n : \phi_n = \phi_{f(n)}\).
	We now distinguish between two possible cases:
	\begin{itemize}
		\item If \(n \in A\), then \(f(n) = c \in \bar{A}\),
		with \(\phi_n = \phi_{f(n)}\).
		We can define \(i \coloneqq n, j \coloneqq f(n)\).
		\item If \(n \in \bar{A}\), then \(f(n) = d \in A\),
		with \(\phi_n = \phi_{f(n)}\).
		We can define \(i \coloneqq f(n), j \coloneqq n\).
	\end{itemize}
	Either way, we obtain that recursivity and nontriviality of \(A\)
	imply the existence of such functions, proving Rice's theorem.
\end{proof}
\end{solution}

\subsection{} % Exercise 5.
Prove \(K\) not recursive using the fixed point theorem.

\begin{solution}
\begin{proof}
We recall the definition of \(K\):
\(K = \{i : \phi_i(i) \textnormal{ halts}\}\).
Assume \(K\) is recursive.
We can then define a function
\[
f(x) = \begin{cases} c \in \bar{K}\,, & \textnormal{ if } x \in K\,,\\
d \in K\,, & \textnormal{ if } x \in \bar{K}\,. \end{cases}
\]
Since \(K\) is assumed recursive,
\(f\) is a computable function,
and since \(K\) is not trivial,
one can find \(c, d\) which satisfy \(c \in \bar{K}, d \in K\).
For example, take
\begin{minted}{julia}
function P_c(x)
	while true
	end
end

function P_d(x)
	return x
end
\end{minted}
By the fixed point theorem, one then has
\(\exists n : \phi_n = \phi_{f(n)}\).
We now distinguish between two possible cases:
\begin{itemize}
	\item If \(n \in K\), then \(f(n) = c \in \bar{A}\),
	with \(\phi_n = \phi_{f(n)}\).
	This means that \(\phi_n = \phi_c\),
	which is a contradiction as we know \(\phi_n(n)\) halts
	while \(\phi_c(n)\) does not.
	\item If \(n \in \bar{A}\), then \(f(n) = d \in A\),
	with \(\phi_n = \phi_{f(n)}\).
	This means that \(\phi_n = \phi_d\),
	which is a contradiction as we know \(\phi_n(n)\) does not halt,
	while \(\phi_d(n)\) does.
\end{itemize}
\end{proof}
\end{solution}

\subsection{} % Exercise 6.
Show that, for any computable total function \(f\),
there exist an infinity of \(k\)'s such that \(\phi_k = \phi_{f(k)}\).
(Hint: show that if it were not the case,
we could find a computable total function
that would not satisfy the fixed point theorem).

\begin{solution}
\begin{proof}
	We show that for every \(K \in \N\), there is \(k > K\)
	such that \(\phi_k = \phi_{f(k)}\).

	Take \(n \in \N\) satisfying \(\phi_n \ne \phi_k\) for all \(k \le K\).
	Such \(n\) exists since there are countably many
	distinct computable functions.

	For such \(n\), define \(g\) as
	\[
	g(k) = \begin{cases} n\,, & \textnormal{if } k \le K\,,\\
	f(k)\,, & \textnormal{otherwise.}\end{cases}
	\]
	\(g\) is total and computable,
	and by the fixed point theorem there exists \(k\)
	such that \(\phi_k = \phi_{g(k)}\).
	But if \(k \le K\), then this means that \(\phi_k = \phi_n\),
	which we stated is not possible
	(\(n\) is defined in a way that this cannot be true).
	This means that necessarily \(k > K\).

	Using the fact that \(K\) can be arbitrarily large,
	this gives a way of proving the existence
	of an infinite number of fixed points.
\end{proof}
\end{solution}
