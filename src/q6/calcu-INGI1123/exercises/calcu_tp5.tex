\documentclass[a4paper,11pt,onecolumn]{article}

\usepackage[french]{babel}
\usepackage[utf8]{inputenc}
\usepackage{pict2e}
\setlength{\unitlength}{1mm}
\usepackage[margin=2cm]{geometry}
\usepackage{amsfonts}
\usepackage{float}

\title{Calculabilité\\Travaux Pratique 5}
\date{}

\begin{document}

\maketitle

\section*{Exercice 1}

\begin{eqnarray}
	P^D_S(f) 	&\equiv& Let \:\: assignedSet \:\: Set\\
				&& Population \:\: assignSet \:\: with \:\: \{T,F\} \:\: combinaisons \\
				&& oven \:\: vars(F) \\
				&& for \:\: a \:\: in \:\: assignSet \\
				&& if \:\: f(a) == TRUE \:\: return \:\: 1 \\
				&& return 0
\end{eqnarray}
A la ligne (1), on definis un ensemble portant le nom \textit{assignSet}, on definis le contenu de cet ensemble à la ligne (2).


\begin{eqnarray}
	P^{ND}_S(f) &\equiv& let \:\: values[\#Vars(F)]  \\
				&& for \:\: v \:\: in \:\: Vars(F); \\
				&& values[V] = choose(1) \\
				&& return \:\: F(valuers) == TRUE
\end{eqnarray}

A la ligne (7), on crée un tableau de variables, à la ligne (9) on choisit une interprétation au hasard et on regarde le résultat à la ligne (10).

On peut conclure que S est non deterministe recursif. Pour un algorithme de complexité Non-D $O(f(n)) \rightarrow$ D $O(2^{f(n)})$

\section*{Exercice 2}
\begin{enumerate}
	\item[(a)] $L \subset {a,b}^*$, consistant de tout les mots qui ont une longueur divisible par 3.

	\textbf{Diagramme d'automate:}
	\begin{figure}[H]
	\centering
	\begin{picture}(70,25)
		\put(15,15){\circle{4}}
		\put(15,15){\circle{10}}
		\put(35,15){\circle{10}}
		\put(55,15){\circle{10}}
		\put( 3,15){\vector(1,0){7}}
		\put(20,15){\vector(1,0){10}}
		\put(40,15){\vector(1,0){10}}
		\put(55,10){\line  (0,-1){5}}
		\put(55,5){\line  (-1,0){40}}
		\put(15,5){\vector(0,1){5}}
		\put(22,16){$a,b$}
		\put(42,16){$a,b$}
		\put(32,6){$a,b$}
	\end{picture}	
	\end{figure}
	\textbf{Expression régulière:}
	$((a+b).(a+b).(a+b))^*$	avec
	\begin{itemize}
		\item[$\rightarrow$] $*$ Répétition infinie
		\item[$\rightarrow$] $.$ Concaténation
		\item[$\rightarrow$] $+$ Or
		\item[$\rightarrow$] $( )$ Fini
	\end{itemize}
	\textbf{Grammaire:} 
	\begin{itemize} 
		\item Symbole terminal: $a,b$
		\item Symbole non-terminal: $S$ (état initial)
		\begin{eqnarray*}
			S &\rightarrow& \varepsilon \\
			S &\rightarrow& aA\\
			S &\rightarrow& bA\\
			A &\rightarrow& aB\\
			A &\rightarrow& bB\\
			B &\rightarrow& aS\\
			B &\rightarrow& bS
		\end{eqnarray*}
	\end{itemize}
	\item[(b)] $L \subset \{(,)\}^*$ consistant de parenthèse bien balancée

	\textbf{Diagramme d'automate:}
	\begin{figure}[H]
	\centering
	\begin{picture}(70,45)
		\put(5,15){\vector(1,0){5}}
		\put(15,15){\circle{10}}
		\put(11,14){FAIL}
		\put(35,30){\circle{10}}
		\put(31,29){FAIL}
		\put(20,20){\vector(20,15){9}}
		\put(22,15){\vector(1,0){26}}
		\put(55,15){\circle{10}}
		\put(55,15){\circle{4}}
		\put(19,12){\line(1,0){6}}
		\put(25,12){\line(0,-1){7}}
		\put(25,5){\line(-1,0){10}}
		\put(15,5){\vector(0,1){5}}
		\put(35,35){\line(0,1){5}}
		\put(35,40){\line(1,0){10}}
		\put(45,40){\line(0,-1){10}}
		\put(45,30){\vector(-1,0){5}}
		\put(29,10){$(,Z/(Z$}
		\put(29,5){$(,(/(($}
		\put(29,0){$),(/\varepsilon$}
		\put(14,25){),Z/Z}
	\end{picture}	
	\end{figure}
	\textbf{Manipulations de la pile}
	\begin{itemize}
		\item $),Z/Z$
		\item $(,Z/(Z$
		\item $(,(/(($
		\item $),(/\varepsilon$
	\end{itemize}
	\textbf{Grammaire hors contexte:}
	\begin{eqnarray*}
		S &\rightarrow& \varepsilon \\
		S &\rightarrow& (S)S
	\end{eqnarray*}
	\textbf{Grammaire régulière:} Symboles terminaux à droite des symboles non terminaux
	\item[(c)] $L = \{a^nb^ma^n|n\in\mathbb{N},m\in\mathbb{N}_0\}$

	\textbf{Diagramme d'automate:}
	\begin{figure}[H]
	\centering
	\begin{picture}(105,40)
		\put(5,20){\vector(1,0){5}}
		\put(20,20){\vector(1,0){15}}
		\put(45,20){\vector(1,0){15}}
		\put(70,20){\vector(1,0){15}}
		\put(15,20){\circle{10}}
		\put(40,20){\circle{10}}
		\put(65,20){\circle{10}}
		\put(90,20){\circle{10}}
		\put(90,20){\circle{4}}
		\put(90,05){\vector(0,1){10}}
		\put(90,05){\line(-1,0){75}}
		\put(15,05){\line(0,1){10}}
		%
		\put(11,23){\line(0,1){7}}
		\put(11,30){\line(1,0){8}}
		\put(19,30){\vector(0,-1){7}}
		\put(9,35){$a,Z/aZ$}
		\put(9,31){$a,a/aa$}
		%
		\put(36,23){\line(0,1){7}}
		\put(36,30){\line(1,0){8}}
		\put(44,30){\vector(0,-1){7}}
		\put(35,35){$b,Z/Z$}
		\put(35,31){$b,a/a$}
		%
		\put(61,23){\line(0,1){7}}
		\put(61,30){\line(1,0){8}}
		\put(69,30){\vector(0,-1){7}}
		\put(60,31){$a,a/\varepsilon$}
		%
		\put(48,6){$b,Z/Z$}
		\put(23,21){$b,a/a$}
		\put(48,21){$a,a/\varepsilon$}
		\put(72,21){$\varepsilon,Z/Z$}
	\end{picture}	
	\end{figure}
	\textbf{Gramaire hors contexte:}
	\begin{eqnarray*}
		S &\rightarrow& aSa \\
		S &\rightarrow& bB\\
		B &\rightarrow& bB\\
		B &\rightarrow& \varepsilon
	\end{eqnarray*}
\end{enumerate}

\section*{Exercice 3}

\begin{itemize}
	\item[(a)] $\overline{L}$: inverse accepting/non-acception state of L
	\item[(b)] $L\cdot M$:
		\begin{itemize}
			\item $\varepsilon$-transitions from accepting state of $L$ to initial state of $M$
			\item make acceptiong state of $L \rightarrow$ non-accepting
		\end{itemize}
		\begin{figure}[H]
		\centering
		\begin{picture}(70,40)
			\put(14,32){$L$}
			\put(53,32){$M$}
			\put(15,20){\circle{20}}
			\put(55,20){\circle{20}}
			\put(35,20){\oval[5](70,40)}
			\put(20,20){\vector(1,0){30}}
			\put(35,22){$\varepsilon$}
			\put(20,20){\circle{2}}
			\put(50,20){\circle{2}}
			\put(20,20){\circle{4}}
			\put(50,20){\circle{4}}
			\put(51,24){$\mu$}
			\put(31,5){$L\cdot M$}
			%\put(60,25){\oval{1,2}}
		\end{picture}	
		\caption{Exercice 3.B}
		\end{figure}
	\item[(c)] $L \cup M$: New init state with $\varepsilon$-transition to init state of $L$ and $M$
		\begin{figure}[H]
		\centering
		\begin{picture}(60,60)
			\put(10,30){\circle{5}}
			\put(45,50){\circle{10}}
			\put(45,10){\circle{10}}
			\put(17,34){\vector(35,20){20}}
			\put(17,26){\vector(35,-20){20}}
			\put(43,49){\circle{2}}
			\put(43,11){\circle{2}}
			\put(43,49){\circle{4}}
			\put(43,11){\circle{4}}
			\put(44,40){$L$}
			\put(44,17){$M$}
		\end{picture}	
		\caption{Exercice 3.C}
		\end{figure}
\end{itemize}

\section*{Exercice 4}
\begin{itemize}
	\item[(a)] Faux, Hoare-Allison
	\item[(b)] Faux
	\item[(c)] Faux, $\mathbb{N}$ et $K \subseteq \mathbb{N}$
	\item[(d)] Vrai
	\item[(e)] Faux, $NDR \Leftrightarrow R$
\end{itemize}

\end{document}