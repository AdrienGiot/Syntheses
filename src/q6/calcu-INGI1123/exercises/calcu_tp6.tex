\subsection{} % Exercise 1.
Let \(S\) be the set of all satisfiable boolean formulas
from propositional logic.
Show that \(S\) is ND-recursive.
Is \(S\) also recursive?

\begin{solution}
One can write an ND-program which decides \(S\):
\begin{minted}[escapeinside=||]{julia}
function P_S(F)
	index = 1
	for v |$\in$| vars(F)
		v = choose(1)
		if v == 1
			values[index] = true
		else
			values[index] = false
		end
		index += 1
	end
	return F(values) # 1 if true, 0 if false.
end
\end{minted}
The existence of this program proves that \(S\) is ND-recursive.
Since any ND-program can be translated to a regular program,
ND-recursive is equivalent to recursive.
\(S\) is thus also recursive.
Note however that while the ND-program has \(\bigoh(f(x))\) complexity,
its deterministic counterpart has \(\bigoh(2^{f(x)})\) complexity.
\end{solution}

\subsection{} % Exercise 2.
For the following languages, choose the suitable automaton formalism
(finite or pushdown automaton),
write its diagram or its transition table,
and write the corresponding grammar
(regular grammar, context-free grammar).
For regular languages, write a corresponding regular expression.
\begin{enumerate}
	\item \(L \subset \{a, b\}^*\),
	consisting of all non-empty words with a length divisible by \(3\).
	\item \(L \subset \{(,)\}^*\),
	consisting of well-balanced parentheses,
	\item \(L = \{a^n b^m a^n \mid n, m \in \N, m \ge 1\}\).
\end{enumerate}

\begin{solution}
	\begin{enumerate}
		\item We use the finite automaton formalism.
		The state diagram for is given in \figuref{6.2.a}.
		\begin{figure}[H]
			\centering
			\begin{tikzpicture}[shorten >=1pt,node distance=4cm,on grid,auto]
			\node[state,initial] (q) {$\ell = 0$};
			\node[state] (q_0) [right=of q]  {$\ell \equiv 1 \mod 3$};
			\node[state] (q_1) [above right=of q_0] {$\ell \equiv 2 \mod 3$};
			\node[state, accepting] (q_2) [below right=of q_0,align=center] {$\ell \equiv 0 \mod 3$ \\ $\ell \ne 0$};
			\path[->]
			(q) edge node {$a, b$} (q_0)
			(q_0) edge  node {$a, b$} (q_1)
			(q_1) edge  node  {$a, b$} (q_2)
			(q_2) edge  node {$a, b$} (q_0);
			\end{tikzpicture}
			\caption{State diagram for \(L \subset \{a, b\}^*\),
			consisting of all non-empty words
			with a length divisible by \(3\).
			The length is of a word is indicated by \(\ell\).}
			\label{fig:6.2.a}
		\end{figure}

		The corresponding regular grammar
		is defined as follows:
		\begin{align*}
			\Sigma &= \{a, b\}\,,\\
			S &\to \varepsilon\,,\\
			S &\to aA\,,\\
			S &\to bA\,,\\
			A &\to aB\,,\\
			A &\to bB\,,\\
			B &\to aS\,,\\
			B &\to bS\,.
		\end{align*}

		The regular expression defining this language is
		\( L = ((a + b) \cdot (a + b) \cdot (a + b))*\).
		\item We use the pushdown automaton formalism.
		The state diagram is given in \figuref{6.2.b}.
		\begin{figure}[H]
		\centering
		\begin{tikzpicture}[shorten >=1pt,node distance=2cm,on grid,auto]
		\node[state,initial,accepting] (bal) {bal};
		\node[state] (fail) [above right=of q]  {fail};
		\node[state] (not bal) [below right=of q] {not bal};
		\path[->]
		(bal) edge node {$), Z/Z$} (fail)
		(bal) edge[bend left] node {$(, Z/(Z$} (not bal)
		(not bal) edge[bend left] node {$\varepsilon, Z/Z$} (bal)
		(not bal) edge[loop right] node[align=left] {$(, (/(($ \\ $), (/\varepsilon$} (not bal);
		\end{tikzpicture}
		\caption{State diagram for \(L \subset \{(,)\}^*\),
		consisting of well-balanced parentheses.}
		\label{fig:6.2.b}
		\end{figure}

		The context-free grammar for this language is
		\begin{align*}
			\Sigma &= \{(, )\}\,,\\
			S &\to (S)S\,,\\
			S &\to \varepsilon\,.
		\end{align*}
		\item We use the pushdown automaton formalism.
		The state diagram is given in \figuref{6.2.c}.
		\begin{figure}[H]
		\centering
		\begin{tikzpicture}[shorten >=1pt,node distance=4cm,on grid,auto]
		\node[state,initial] (q_0) {$q_0$};
		\node[state] (q_1) [above right=of q_0]  {$q_1$};
		\node[state] (q_2) [right=of q_1] {$q_2$};
		\node[state] (q_3) [below right=of q_2] {$q_3$};
		\node[state,accepting] (q_4) [below left=of q_3] {$q_4$};
		\node[state] (q_5) [left=of q_4] {$q_5$};
		\path[->]
		(q_0) edge[bend left] node {$b, a/a$} (q_1)
		(q_0) edge[loop right] node[align=left] {$a, Z/aZ$ \\ $a, a/aa$} (q_0)
		(q_0) edge[bend right] node {$b, Z/Z$} (q_5)
		(q_1) edge[loop above] node {$b, a/a$} (q_1)
		(q_1) edge[bend left] node {$a, a/\varepsilon$} (q_2)
		(q_2) edge[loop above] node {$a, a/\varepsilon$} (q_2)
		(q_2) edge[bend left] node {$b, a/a$} (q_3)
		(q_2) edge[bend right] node {$\varepsilon, Z/Z$} (q_4)
		(q_4) edge[bend right] node[align=left] {$a, Z/Z$ \\ $b, Z/Z$} (q_3)
		(q_5) edge[bend right] node {$\varepsilon, Z/Z$} (q_4)
		(q_5) edge[loop below] node {$b, Z/Z$} (q_5);
		\end{tikzpicture}
		\caption{State diagram for
		\(L = \{a^n b^m a^n \mid n, m \in \N, m \ge 1\}\).}
		\label{fig:6.2.c}
		\end{figure}

		The context-free grammar for this language is
		\begin{align*}
			\Sigma &= \{a, b\}\,,\\
			S &\to aSa\,,\\
			S &\to bB\,,\\
			B &\to bB\,,\\
			B &\to \varepsilon\,.
		\end{align*}
	\end{enumerate}
\end{solution}

\subsection{} % Exercise 3.
Suppose we have two languages \(M\) and \(L\) accepted by some finite automata
and their corresponding diagrams.
Show by construction that the following languages are also accepted
by a finite automaton:
\begin{enumerate}
	\item \(\bar{L}\);
	\item \(L \cdot M\);
	\item \(L \cup M\).
\end{enumerate}

\begin{solution}
\begin{enumerate}
	\item In order to construct a finite automaton that accepts \(\bar{L}\),
	one need simply make all accepting states non accepting,
	and make all non accepting states accepting.
	\item Start by linking the accepting states of \(L\)
	to the initial state of \(M\) by an empty transition.
	Next, make the accepting states of \(L\) not accepting.
	The initial state of \(M\) is also no longer an initial state,
	but simply a normal state.

	Note that there is no need to know when a word of \(L\) ends;
	one can introduce nondeterminism.
	\item Create a new initial state which links
	to the initial states of \(L\) and \(M\)
	(which are no longer initial states).
	Using nondeterminism, this automaton accepts \(L \cup M\).
\end{enumerate}
\end{solution}

\subsection{} % Exercise 4.
True or false?
\begin{enumerate}
	\item The interpreter of finite automata
	can be represented by a finite automaton.
	\item If a language can be decided by a pushdown automaton,
	then this language can also be decided by a finite automaton.
	\item All subsets of a language accepted by
	a finite automaton are recursive.
	\item All regular languages can be modeled by a pushdown automaton.
	\item There exist ND-recursive sets that are not recursive.
\end{enumerate}

\begin{solution}
\begin{enumerate}
	\item False.
	By the Hoare--Allison theorem, this is not possible.
	Finite automata can only compute total functions,
	which in turn means it cannot compute all total functions
	(with its interpreter being one of the uncomputable ones).
	\item False.
	For example, \(L = \{a^n b^n \mid n \in \N\}^*\)
	can only be decided by a pushdown automaton.
	\item False.
	While a finite automaton can define the set of natural numbers,
	not all subsets of \(\N\) are recursive
	(for example, \(\HALT \subsetneq \N\) but \(\HALT\) is not recursive).
	\item True.
	All regular languages can be modeled by a finite automaton,
	which in turn can be modeled by a pushdown automaton,
	as this model is more powerful.
	\item False.
	ND-recursivity is equivalent to recursivity,
	since any ND-program which ND-decides the set
	can be made deterministic at the cost of complexity.
\end{enumerate}
\end{solution}
