\begin{mcqs}
  \mcq{Si la complexité temporelle d'un algorithme est $\bigoh(n^{2})$, alors elle est aussi $\bigoh(n^{3})$}{1}
  {$n^2 \in \bigoh(n^3)$ et $\bigoh$ est transitif.}
  \mcq{Si la complexité spatiale d'un algorithme est $\bigoh(n^{3})$, alors elle ne peut pas être $\bigoh(n^{2})$}{0}
  {$\bigoh(n^{3})$ est une borne supérieure mais ça peut être moins!}
  \mcq{Un algorithme de complexité $\bigoh(n^{2})$ est toujours plus rapide qu'un algorithme de complexité $\bigoh(n^{3})$}{0}
  {Déjà c'est juste des bornes, peut être que le deuxième est $\bigoh(n)$.
  Aussi c'est une complexité de pire cas, sur certains input, le deuxième est peut être plus rapide mais si dans le pire cas il est plus lent.
  Il y a également la constante, peut être que devant le $n^3$ il y a une petite constant, sur des petites inputs le deuxième est alors quand même plus rapide dans le pire cas.}
  \mcq{Un problème qui peut être résolu par un algorithme (de complexité) polynomial est pratiquement faisable.}{1}
  {Pratiquement faisable = Polynomial}
  \mcq{Un problème qui peut être résolu par un algorithme (de complexité) exponentiel est pratiquement infaisable}{0}
  {Il peut très bien être réalisable à la fois par un algorithme exponentiel ET un algorithme polynomial.}
  \mcq{Si un problème est intrinsèquement complexe en MT alors il est aussi intrinsèquement complexe pour le langage Java.}{1}
  {Le choix du modèle de calculabilité impacte la complexité temporelle. Néanmoins, la différence de complexité est tout au plus polynomiale entre tous les modèles de calcul déterministes. Or, la composition de fonctions polynomiales reste polynomiale!}
  \mcq{Un ensemble est a-réductible (algorithmiquement réductible) à son complément.}{1}
  {Dire que A est a-réductible à B signifie que l'algorithme pouvant décider B nous permet de décider A. Ici, décider le complément permet bien évidemment de décider l'ensemble initial, il suffit d'inverser la réponse.}
  \mcq{Tout ensemble récursivement énumérable est a-réductible à HALT.}{1}
  {Halt est le ``plus difficile''. Pour tout ensemble récursivement énumérable $A$, il existe un programme $P$ qui renvoie 1 si $x$ est dans $A$. Pour décider si $x \in A$, on demande à HALT si $P$ termine avec l'input $x$. Si oui, alors on renvoit $P(x)$, sinon on renvoit que $x$ n'appartient pas à $A$.}
  \mcq{Un ensemble est $f$-réductible (fonctionnellement réductible) à son complément.}{0}
  {La plupart du temps, il n'est pas possible de trouver une fonction qui transforme une instance d'un ensemble en une instance de son complément. Par exemple ensemble PAIR est $f$-réductible à IMPAIR. Il suffit de rajouter 1. Mais ce n'est pas nécessairement toujours possible...
  De plus si $A \leq_f B$ et $B$ est récursivement énumérable alors $A$ est récursivement énumérable.
  Si un ensemble est $f$-réductible à son complément ça voudrait donc dire que si $A$ est récursivement énumérable son complément l'est aussi.
  Seulement si un ensemble et son complément sont récursivement énumérables alors il est récursif.
  Ça voudrait donc dire que tout ensemble récursivement énumérable est récursif et donc HALT est récursif, absurde tout ça non ?}
  \mcq{Si $A$ peut être décidé par un algorithme polynomial et si $B$ est $f$-réductible à $A$, alors $B$ peut être décidé par un algorithme polynomial.}{0}
  {Parce que ce que la $f$-réductibilité veut dire : on peut décider $B$ en donnant $f(x)$ (on veut décider si $x$ est dans $B$) à $A$.
  Mais calculer $f(x)$ peut être avoir une complexité exponentielle.}
\end{mcqs}
