\begin{mcqs}
  \mcq{Si $A$ est dans $DTIME(n^{2})$, alors $A$ est dans $DSPACE(n^{2})$}{1}
  {Il faut au moins $n^2$ instructions pour accéder à $n^2$ cases.}
  \mcq{Si $A$ est dans $NTIME(n^{2})$, alors $A$ est dans $DTIME(n^{2})$}{0}
  {}
  \mcq{Si $A \in NTIME(f)$ alors $A\in DTIME(f)$}{0}
  {}
  \mcq{Si $A \in NTIME(n^{2})$ alors $A \in NP$}{1}
  {$n^2$ est polynomial donc c'est non-déterministe polynomial.}
  \mcq{S'il existe un algorithme Java de complexité temporelle $\bigoh(n^{2})$ décidant l'ensemble $A$, alors il existe une machine de Turing de complexité temporelle $\bigoh(n^{2})$ décidant l'ensemble $A$.}{0}
  {Pas de transfert de complexité entre formalismes. On peut seulement dire que la MT va résoudre le problème en un temps polynomial. On ne peut pas assurer $\bigoh(n^{2})$}
  \mcq{Le choix d'un modèle de calculabilité n'influence pas la classe $DTIME(n^{2})$}{0}
  {$n^2$ en Java ne veut pas dire $n^2$ en MT, juste polynomial.}
  \mcq{Si $A \leq_{p} B$ alors $A \leq_{a} B$.}{1}
  {Car la réduction polynomiale, en plus d’avoir une réduction algorithmique, on a aussi une information sur la complexité
  et même si on ne sait pas si $P$ est strictement inclus dans $NP$ on connait des algorithm exponentiels non-polynomials.}
  \mcq{La classe P est strictement inclue dans la classe EXPTIME.}{1}
  {Aucun problème polynomial n'est plus dur qu'un problème exponentiel.}
  \mcq{Tout problème calculable est au moins dans exptime.}{0}
  {Il existe pire comme la fonction d'Ackerman}
  \mcq{Un problème NP-complet peut toujours être décidé par un programme non déterministe de complexité polynomiale.}{1}
  {C'est la définition de NP}
  \mcq{Pour déterminer si un problème A est NP-complet, il suffit de déterminer que A est polynomialement réductible un problème NP complet connu (EG, sat).}{0}
  {Il faut être dans NP, on ne dit pas qu'il est dans NP. De plus, il faut être plus difficile qu'un problème qui est NP-complet (et pas réductible à un problème NP-complet). Il y a donc une double erreur ici!}
  \mcq{Le problème du voyageur de commerce est NP-complet.}{1}
  {Le problème du cycle Hamiltonien est réductible au TSP~\cite[Theorem~34.14]{cormen2009algorithm} et Richard M. Karp a montré en 1972 que le problème du cycle Hamiltonien est NP-complet~\cite[Theorem~34.13]{cormen2009algorithm}.}
\end{mcqs}
