\begin{mcqs}
  \mcq{Un problème NP-complet est intrinsèquement complexe.}{2}
  {ON EN SAIT RIEN (intrinsèquement complexe = qui n'a pas d'algorithme polynomial) VRAI SI $P \neq NP$ et FAUX si $P=NP$.}
  \mcq{Le choix d’un modèle de calculabilité n’influence pas les classes P et NP.}{1}
  {Ça n'influence pas les classes (indépendante des langages).}
  \mcq{Le problème de la programmation linéaire est NP-complet.}{0}
  {La programmation linéaire est dans P (ensemble d'inéquations linéaire avec une fonction objectif linéaire qu'on veut optimiser).
  C'est résolu en temps polynomial par l'interior point method inventée par Yurii Nesterov (UCL) et Arkadi Nemirovski (Georgia Tech).}
  \mcq{Un problème de décision dans P, alors le problème consistant à calculer une solution est également dans P.}{1}
  {Pourquoi la complexité ferait des choses qui ne servent à rien ? Si on se focalise sur un problème de décision, c'est pcq c'est plus facile à traiter d'un point de vue théorique mais dans la pratique ça ne change rien! Donner la réponse si on sait qu'elle existe ça ne change rien.}
  \mcq{Un problème intrinsèquement complexe est dans EXPTIME.}{0}
  {Il y a des complexités pire qu'exponentielle.}
  \mcq{Un problème NP-complet peut être résolu par un algorithme non déterministe de complexité temporelle polynomiale.}{1}
  {La question qui met dehors à un examen oral.}
  \mcq{Un problème NP-complet peut être résolu par un algorithme non déterministe de complexité spaciale polynomiale.}{1}
  {La complexité spatiale est toujours bornée par la complexité temporelle.}
  \mcq{Si SAT${}\in{}$P, alors P$\subseteq$NP.}{1}
  {P est dans NP de toute façon mais si SAT $\in$ P alors P${}={}$NP.}
  \mcq{Si $SAT \in P$, alors P${}={}$NP.}{1}
  {SAT est NP-complet, tout problème NP est polynomialement réductible à lui.}
  \mcq{Si SAT $\leq_{a}$ A et $A \in P$, alors SAT $\in$ P.}{0}
  {On a besoin de la réduction polynomiale et pas la réduction algorithmique (cette dernière sert uniquement à la calculabilité, alors que la réduction polynomiale dit combien de fois on utilise l'algorithme !)}
\end{mcqs}
