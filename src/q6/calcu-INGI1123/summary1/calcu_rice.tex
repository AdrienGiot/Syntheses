\documentclass[a4paper]{article}

\usepackage[french]{babel}
\usepackage[utf8]{inputenc}
\usepackage{amssymb}
\usepackage{graphicx}

% Title
\title{Calculabilité - Démonstration du Théorème de Rice}
\author{}
\date{\today}

\begin{document}

\maketitle

\section{Théorème de Rice}

\textsc{\textbf{Si}} $\forall i \in A, \forall j \in \overline{A} : \phi_i \neq \phi_j$ et $A \neq \varnothing$ et $A \neq \mathbb{N}$.

\textsc{\textbf{Alors}} $A$ non récursif.

Supposons A récursif et $A \neq \varnothing$ et $A \neq \mathbb{N}$, alors \textsc{HALT} récursif.

%\includegraphics[scale=1]{img.png}

\begin{enumerate}
	\item $P_k \equiv$ while \textsc{true} do; \\ avec $k \in \overline{A}$
	\item $A \neq \varnothing$
	\item $\phi_k \neq \phi_m$
	\item Construire un programme qui décide \textsc{HALT} \\
	$HALT(n,x)\equiv$ 
	\begin{itemize}
		\item Construire un programme $P(z) \equiv P_n(x), P_m(z)$
		\item $d \leftarrow n^o P(z)$ (le numero du programme)
		\item if $d \in A$ then $print(1)$ else $print(0)$
	\end{itemize}
\end{enumerate}

Si $P_n(x) = \perp$, $P(z) = \perp$ donc il fait la même chose que $P_k$

Si $P_n \neq \perp$, $P(z) = P_m(z)$.
donc $P(z)$ est soit à \textbf{gauche} soit à \textbf{droite} donc on peut dire si il boucle ou pas (contradiction).

\section{Théorème S-m-n}

Ce théorème est la pierre angulaire de tout langage informatique. Il nous dit qu'un programme avec m+n paramètre peut être réduit à un programme avec m paramètre et n valeurs (des constantes donc).

$S_1^1 : \exists S_1^1$ total calculable tel que 

$\forall P$ programme $P(x,y) \equiv [S_1^1(P, y)](x)$.

$S : \forall P$ programme. $\exists S$ total calculable tel que

$P(x,y) \equiv [S(y)](x)$

La premiere proposition implique la deuxieme, mais les deux propositions nes sont pas équivalentes.

\end{document}
