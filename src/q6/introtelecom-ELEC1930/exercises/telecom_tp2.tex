\documentclass[a4paper, 11pt, onecolumn]{article}

\usepackage[utf8]{inputenc}
\usepackage{picture}
\usepackage{framed}
\usepackage[margin=2cm]{geometry}
\usepackage{float}

\title{Introduction aux télécommunications\\TP2}

\begin{document}

\maketitle
\section{Exercice 1}
    \textit{Un signal $f(t)$ est modulé par une modulation d'amplitude DSB (Double side band) autour d'une fréquence porteuse $f_C=900kHZ$}
	\begin{itemize}
		\item \textit{Donnez l'expresssion du signal modulé en fonction de $f(t)$}
		\begin{framed}
			$s(t) = A_C m(t) cos(2\pi 900000t)$
		\end{framed}
	\end{itemize}

\section{Exercice 2}
	\begin{itemize}
		\item \textit{Un modulateur FM réalise une déviation maximale de 30 kHZ, et présente une sensibilité de $4kHz/V$. Déterminez l'amplitude maximale du signal modulant en entrée.}
		\begin{framed}
			$$Amp\ max \frac{30kHz}{4kHz/V} = 7.5V$$
		\end{framed}
		\item \textit{Un signal modulant de 10V produit une déviation de fréquence de 75kHz à la sortie d'un modulateur FM. Déterminez la sensibilité de celui-ci et a déviation de fréquence provoquée par un signal modulant de 2V. Si le signal modulant à une amplitude maximale de 10V et une fréquence maximale de 15 kHz, calculez la bande occupée par le signal modulé ainsi que l'indice de modulation.}
		\begin{framed}
		$$\frac{75kHz}{10V}=7.5kHz/V=sensibilite$$
		$$\Delta f_2 = 2V*7.5kHz/V = 15kHz$$
		On calcule $\beta$ avec la formule $(10)$
		\begin{eqnarray*}
			\beta &=& \frac{\Delta f}{W}\\
			&=& \frac{75kHz}{15kHz} = 5
		\end{eqnarray*}
		On calcule $B_T$ avec la formule $(11)$
		\begin{eqnarray*}
			B_T &\simeq& 2\Delta f(1+\frac{1}{\beta})\\
			&\simeq& 150kHz (1+\frac{1}{5})\\
			&\simeq& 180kHz
		\end{eqnarray*}
		\item \textit{Comparez la bande nécessaire si le même signal était modulé en AM (DSB)}
		Par la formule $(4)$ on a 
		\begin{eqnarray*}
			B_T&=&2W\\
			&=&30kHz
		\end{eqnarray*}
		FM a une bande passante 6 fois supérieure à la bande passante AM
		\end{framed}	
	\end{itemize}

\section{Exercice 3}
	\textit{On considère une modulation numérique linéaire classique utilisant comme filtre de mise en forme, un filtre de forme triangulaire tel que représenté à la figure 3. Si on envoie la séquence de symbole suivante:}
	\begin{framed}
	$$I(0) = 1+j$$
	$$I(1) = j$$
	$$I(2) = -1$$
	\end{framed}
	\textit{Représentez les signaux $I$ et $Q$ avant la modulation autour de la porteuse. Ces signaux soufriront-ils d'interférence entre symboles? Si oui évaluez l'interférence dont souffre le symbole du temps $t=0 : I(0)$}
	\begin{framed}
	\begin{eqnarray*}
		I_Q(0) &\rightarrow& X_a(0)\\
		X_Q(0) &=& I_q(0) + \frac{1}{2}I_Q(1)\\
		X_Q(1) &=& I_0(1) + \frac{1}{2}I_Q(0)\\
		X_Q(2) &=& \frac{1}{2} I_Q(1)+I_Q(2)
	\end{eqnarray*}
	$\frac{1}{2}I_Q(1)$ sont les interférences
	
	
	\begin{eqnarray*}
		X_I(O) &=& I_I(0)\\
		X_I(1) &=& I_I(1)\\
		&=& 0\\
		X_I(2) &=& I_I(2)
	\end{eqnarray*}
	\end{framed}

	\begin{figure}[H]
	\setlength{\unitlength}{0.6mm}
	\begin{picture}(125,85)
		\put(20,30){\line(0,1){2}} \put(11,22){$-2T$}
		\put(40,30){\line(0,1){2}} \put(32,22){$-T$}
		\put(80,30){\line(0,1){2}} \put(78,22){$T$}
		\put(100,30){\line(0,1){2}} \put(96,22){$2T$}
		\put(120,30){\line(0,1){2}} \put(116,22){$3T$}
		\put(0,31){\vector(1,0){125}} \put(128,29){$t$}
		\put(60,0){\vector(0,1){80}} \put(65,80){$X_Q(t)$}
		\put(20,31){\line(1,1){40}}
		\put(60,71){\line(1,-1){40}}
		\put(40,31){\line(1,1){40}}
		\put(80,71){\line(1,-1){40}}
	\end{picture}
	\caption{Interference Q}
	\end{figure}

	\setlength{\unitlength}{0.6mm}
	\begin{figure}[H]
	\begin{picture}(145,90)
		\put(20,39){\line(0,1){2}} \put(11,31){$-2T$}
		\put(40,39){\line(0,1){2}} \put(32,31){$-T$}
		\put(80,39){\line(0,1){2}} \put(78,31){$T$}
		\put(100,39){\line(0,1){2}} \put(96,31){$2T$}
		\put(120,39){\line(0,1){2}} \put(116,31){$3T$}
		\put(140,39){\line(0,1){2}} \put(136,31){$4T$}
		\put(0,40){\vector(1,0){145}} \put(148,38){$t$}
		\put(60,0){\vector(0,1){80}} \put(65,89){$X_I(t)$}
		\put(20,40){\line(1,1){40}}
		\put(60,80){\line(1,-1){40}}
		\put(60,40){\line(1,-1){40}}
		\put(100,0){\line(1,1){40}}
	\end{picture}	
	\caption{Interference I}
	\end{figure}
\section{Exercice 4}
	\begin{itemize}
		\item \textit{On veut transmettre un signal vidéo numérisé au rythme de 30 images de 180 lignes par secondes, dans un format 16/9 et avec une résolution identique horizontalement et verticalement. Le signal est échantillonné puis quantifié en 256 niveaux. Calculez le débit binaire nécessaire . Le signal échantillonné est transmis au moyen de la modulation numérique considérée ci-desssus, au sein d'une bande passante disponible de 5MHz. Quelle est la fréquence symbole? Quelle taille  de constellation doit-être utilisée}
		
		\begin{framed}
		Soient $N_P$ le nombre pixels par ligne, $N_M $ le nombre de pixels par image et $N_T$ le nombre de pixels par secondes.
		$$N_P = 180 *\frac{16}{9} = 320 \frac{pixels}{lignes}$$
		$$N_M = N_P*180 = 57600 \frac{pixels}{images}$$
		$$N_T = N_M * 30 = 1.728 Mpixels$$
		On va multiplier les millions de pixels par 8 bits pour avoir le ratio binaire.
		$$Ratio\_binaire = 1.728 * 10^6 * 8 = 13.824 Mbits/s$$
		On va diviser la quantité de données à transmettre par la bande passante qui est mise à notre disposition et nous obtiendrons la fréquence symbole.
		$$f_s = \frac{13.824 * 10^6 Mbits/s}{5 * 10^6 Hz} = 2.7648$$
		\end{framed}
		\item \textit{On suppose que pour le système considéré, le rapport signal à bruit est tel que l'on peut se permettre d'utiliser une constellation QAM-16, tout en obtenant une probabilité d'erreur suffisament basse pour l'application vidéo. Calulez quel est le plus haut niveau de quantification admissible.}
		\begin{framed}
			\textsc{Rappel:}
			\begin{itemize}
				\item Constellation à 4 symboles ($2^2$ bits)
				\item Bande passante, $B_\alpha = \frac{1+\alpha}{T}$, avec T la fréquence d'échantillonage et $\alpha$ le roll-off.
			\end{itemize}
			Ici nous avons:
			$$T = \frac{1+\alpha}{5*10^6} = \frac{1+0.2}{5*10^6} = 240nsec$$
			Et avec $R_S$ le taux de symbole
			$$R_S = \frac{1}{240nsec} = 4.17Msymbols/sec$$
			$$bits/symbols=R_B/R_S=3.31bits$$
			Il va donc falloir une constellation de taille $2^4=16$. Plus la constellation est grande, plus l'effet du bruit est important.
	
		\end{framed}
	
	\end{itemize}
\end{document}
