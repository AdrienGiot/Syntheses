\subsection{Exercice 1}
\textit{On envoi, dans canal en bande de base $h(t)$, un signal
\begin{eqnarray}x(t)=rect\left(\frac{t-T/2}{T}\right)\end{eqnarray}
produits la sortie
\begin{eqnarray}
y(t)=
\left\lbrace
\begin{array}{lll}
	A(1-e^{-\frac{\alpha t}{T}})\;\;\;\;\;\;\;\;\;\;\;\;\;\;\;\;\;0\leq t < T\\
	A(1-e^{-\alpha})e^{-\frac{\beta(t-T)}{T}}\;\;\;\;\;\;\;\;\;t \geq T\\
	0\;\;\;\;\;\;\;\;\;\;\;\;\;\;\;\;\;\;\;\;\;\;\;\;\;\;\;\;\;\;\;\;\;\;\;\;\;t < 0
\end{array}
\right.
\end{eqnarray}
Dessinez le signal de sortie $y(t)$ quand le même cannal est nourri par le signal PAM
\begin{eqnarray}
	x(t)=\sum_{n=0}^{N-1}a_{n}rect\left(\frac{t-T/2-nT}{T}\right),
\end{eqnarray}
avec $N=3$, $a_0=1$, et $a_1=a_2=-1$. Le signal $y(t)$ est ensuite échantilonné à $t=nT+T/2$. Déterminez l'expression de l'échantilonnage $y(3T/2)$ et l'ISI (International Symbol Interference) associé.}
\begin{framed}
	\[ Y_X(t)=\sum_{n=0}^2a_{n}h(t-nT) \]
	\[ h(t) = F\left(rect\left(\frac{t-T/2-nT}{T}\right)\right) \]
Casual filter: le filtre suit la cause. Cela veut dire qu'en t=0, il n'y a pas d'interférence pour le 1er symbole.
\begin{itemize}
	\item $1^{er}$ symbole: aucune interférence
	\item $2^{eme}$ symbole: interférence du $1^{er}$
	\item $3^{eme}$ symbole: interférence du $1^{er}$ et du $2^{eme}$
\end{itemize}
	\[ Y\left(\frac{3}{2}T\right) = a_0h\left(\frac{3}{2}T\right) +  a_1\left(\frac{3}{2}T-T\right)+0 \]
On remarque la présence du $+ 0$ qui signifie que le $3^{eme}$ symbole n'interfere pas.
\end{framed}
