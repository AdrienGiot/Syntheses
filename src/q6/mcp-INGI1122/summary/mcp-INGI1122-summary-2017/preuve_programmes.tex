\section{Preuves de programmes}

Pour prouver un programme, il faut le décomposer en \textbf{chemins simples} et utiliser des \textbf{assertions inductives}, càd qui sont conservées sur chaque chemin, et des \textbf{ensembles bien-fondés}, càd que les variants diminuent sur chaque chemin de boucle.
\subsection{Chemins simples}
Une \textbf{position} du programme est un point avant, après ou entre des instructions successives du programme. \textit{Dénotation:} @label\\

Un \textbf{chemin} est une séquence exécutable d'instructions du programme entre 2 positions.\\

Un \textbf{chemin simple} est un chemin qui ne passe pas deux fois par la même instruction, càd par des positions différentes sauf éventuellement quand la fin correspond au début dans le cas d'un cycle.\\

Un ensemble de \textbf{points de coupe} est un ensemble de positions du programme tel que \textbf{toutes les boucles contiennent au moins un point de coupe}. Les chemins entre les points de coupe sont des chemins simples.\\ \vspace{5mm}\\

$\bullet$ L'instruction \textbf{assume}: les conditions dans les instructions du programme deviennent des assomptions dans les chemins simples.\\
@L1 if C \{S1\} else \{S2\} @L2\\
$.\quad \rightarrow$ @L1 assume C; S1 @L2\\
$.\quad \quad$ @L1 assume $\neg$C; S2 @L2\\
@L1 while C \{S\} @L2\\
$.\quad \rightarrow$ @L1 assume C; S @L1\\
$.\quad \quad$ @L1 assume $\neg$C; @L2\\ 

\textbf{assume C} $\equiv$ le reste de l'exécution ne s'exécute que si C est vrai.\\
Règle de l'assume avant: [P] assume C; [P $\wedge$ C]\\
Règle de l'assume arrière: [C $\Rightarrow$ Q] assume C; [Q]\\
On assume que C est vrai $\equiv$ on ignore les exécutions où C est faux.\\ \vspace{5mm}\\

$\bullet$ L'instruction \textbf{assert}: assert C $\equiv$ en cette position, C est vrai\\
Règle de l'assert en version avant : \textbf{Si [P] $\Rightarrow$ C alors [P] assert C; [P $\wedge$ C]}\\
Règle de l'assert en version arrière : \textbf{[C $\wedge$ Q] assert C; [Q]}\\
On affirme que C est vrai, on doit donc le prouver sur toutes les exécutions.\\ \vspace{5mm}\\

Un chemin simple est une séquence d'instructions suivantes:\\
$\bullet$ Affectation V:=E;\\
$\bullet$ Assomption assume C;\\
$\bullet$ Assertion assert C;
\vfill
\subsection{Calcul WP}
La \textbf{plus faible pré-condition} de S (un chemin simple) pour Q : wp(S,Q).\\
Q est vrai après le chemin simple S ssi wp(S,Q) est vrai avant S $\equiv$ \textbf{[P] S [Q] ssi P $\Rightarrow$ wp(S,Q)}\\
Pre $\Rightarrow$ wp(S,Post)
\begin{align*}
	wp(V:=E;, Q) &= Q[V:=E]\\
	wp(assume C;, Q) &= C \Rightarrow Q\\
	wp(assert C;, Q) &= C \wedge Q\\
	wp(S1 S2, Q) &= wp(S1, wp(S2,Q))
\end{align*}
\subsection{Méthodes des assertions inductives}
Pour prouver la \textbf{ correction partielle} de $[P] S [Q]$, il faut choisir un ensemble de \textbf{points de coupe} $L_1,...,L_n$ au début, à la fin du programme et dans les boucles. Ensuite, il faut associer une \textbf{assertion}(pré,post,invariants) $P_i$ à chaque point de coupe $L_i$. Pour chaque \textbf{chemin simple} @$L_i S$ @$L_j$, il faut \textbf{prouver} $[P_i] S [P_j]$\\

Si $[P_i] S_{ij}[P_j]$ est  valide pour tout \textbf{chemin simple} @$L_i S_{ij}$ @$L_j$,\\
alors  $[P_i] S_{ij}...S_{kl} [P_l]$ est valide pour tout \textbf{chemin}  @$L_i S_{ij}...S_{kl}$ @$L_l$.\\
Preuve par induction simple sure le nombre de chemins simples de @$L_i$ à  @$L_l$.\\
En particulier, $[Pre] S [Post]$ est valide pour le programme entier @$Pre S$ @$Post$.
\subsubsection{Assertions inductives et invariantes}
Pour un programme S:\\
Si $[P_i]S_{ij}[P_j]$ est valide pour tout @$L_iS_{ij}$@$L_j$ de $S$, on dit que les assertions $P_i$ sont \textbf{inductives} sur $S$.\\
Si pour toute exécution, si $Pre$ est vrai initialement en @$Pre$, alors $P_i$ est vrai en @$L_i$, on dit que les assertions $P_i$ sont \textbf{invariantes} sur $S$.\\

Si les assertions sont inductives, alors elles sont invariantes mais pas nécessairement vice versa ! On voudrait vérifier que les $P_i$ sont invariants mais c'est impossible. On devrait considérer toutes les exécutions possibles ! On doit trouver des $P_i$ inductifs, et donc aussi invariants, et on peut le prouver via la méthode des invariants inductifs.
\subsection{Méthodes des ensembles bien-fondés}
Pour prouver la \textbf{correction totale} de $[P] S[Q]$ il faut appliquer la méthode des assertions inductives (cfr. ci dessus) et choisir un \textbf{sous-ensemble des points de coupe} de $L'_1,...L'_m$, tel que chaque boucle contient au moins un $L'_i$. Il faut ensuite associer un \textbf{variant} $V_i$ à chaque point de coupe $L'_i$ et pour chaque \textbf{chemin simple} @$L'_i S$@$L'_j$, prouver $[P_i \wedge V_i = v_0] S [V_j < v_0]$.\\
Un variant est une \textbf{expression} dont la valeur est dans un \textbf{domaine bien-fondé} pour la relation considérée, càd sans chaîne infinie.\\

Si $[P_i \wedge V_i = v_0] S [V_j < v_0]$ est  valide pour tout \textbf{chemin simple} @$L'_i S_{ij}$ @$L'_j$,\\
alors $[P_i \wedge V_i = v_0] S_{ij}...S_{kl} [V_l < v_0]$ est valide pour tout \textbf{chemin}  @$L'_i S_{ij}...S_{kl}$ @$L'_l$.\\
Preuve par induction simple sure le nombre de chemins.\\
En particulier, $[P_i \wedge V_i = v_0] S_{ij}...S_{ki} [V_i < v_0]$ est valide pour tout cycle @$L_i S_{ij}...S_{ki}$ @$L_i$. Donc il ne peut pas y avoir d'exécution infinie qui boucle sur @$L_i$, et donc le programme se termine.
