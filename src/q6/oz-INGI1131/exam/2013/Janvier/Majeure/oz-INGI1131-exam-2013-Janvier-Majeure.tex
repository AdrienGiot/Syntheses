\documentclass[en]{../../../../../../eplexam}

\usepackage{../../../../../../eplcode}

\hypertitle{Programming Languages Concepts}{6}{INGI}{1131}{2013}{Janvier}
{Legat Beno\^it\and Aloïs Paulus}
{Peter Van Roy}

\lstset{language={Oz},morekeywords={for,do,lazy}}

\section{Pascal's triangle as a lazy stream}
\begin{itemize}
  \item Define the function \lstinline|LiftToStream2|.
    The call \lstinline|Zs={LiftToStream2 F Xs Ys}| takes a
    two-argument function \lstinline|F| and creates a lazy agent with input
    streams \lstinline|Xs| and \lstinline|Ys| and output stream \lstinline|Zs|.
    Elements of \lstinline|Zs| are calculated according to the formula
    $z_i = f(x_i,y_i)$ where $x_i$, $y_i$, and $z_i$ are elements of
    \lstinline|Xs|, \lstinline|Ys|, and \lstinline|Zs|, respectively,
    and $f$ is the function \lstinline|F|.
    Note that \lstinline|LiftToStream2| is not lazy but the agent it creates is lazy.
  \item Define the analogous function \lstinline|LiftToStream1| that takes a one argument
    function and creates a lazy agent with one input and one output stream.
  \item Now define three functions, \lstinline|AddList|, \lstinline|ShiftLeft|,
    and \lstinline|ShiftRight|, with the following semantics.
    \lstinline|C={AddList A B}| takes two lists \lstinline|A| and \lstinline|B|
    with the same number of integer elements and returns a new list \lstinline|C|
    that contains the pairwise sums of these elements.
    \lstinline|B={ShiftLeft A}| takes a list \lstinline|A| and returns a new list
    \lstinline|B| that has all elements of \lstinline|A| followed by a new element \lstinline|0| (zero).
    \lstinline|B={ShiftRight A}| takes a list \lstinline|A| and returns a new list
    \lstinline|B| that starts with \lstinline|0| (zero) and is followed by all elements of \lstinline|A|.
   % B:{shiftRighr A} takes a list a and returns a new
  \item With these three functions, you can generate the rows of Pascal's triangle recursively. Given
    a row \lstinline|R| represented as a list, the next row is calculated by
    \lstinline|{AddList {ShiftLeft R} {ShiftRight R}}|.
    Starting with the first row, which is \lstinline|[1]|,
    you can calculate successive rows.
    Using this approach together with \lstinline|LiftToStream1| and \lstinline|LiftToStream2|,
    define a program that calculates a lazy stream containing as elements the successive rows of
    Pascal's triangle.
    Hint: the program has a similar structure to the lazy solution of the Hamming problem.
\end{itemize}
All code written in this question should be declarative.

\begin{solution}
  \lstinputlisting{q1.oz}
  \lstinputlisting{q1_alernative.oz}
\end{solution}

\section{A peer-to-peer dictionary built with agents}
For this question you will implement a peer-to-peer dictionary with neighbor connectivity. Consi-
der a pool of n agents, where each agent is identified by a unique integer. Each agent has refe-
rences to some other agents (at least one). This system should implement the following protocol :
\begin{itemize}
  \item Send the message put \lstinline|(K V)| to any agent, where \lstinline|K| is a key (any atom) and \lstinline|V| is any value.
    This message will be routed to agent number \lstinline|I|, where the number \lstinline|I| is calculated as a
    deterministic function of \lstinline|K|. At that agent, the pair \lstinline|(K,V)| will be stored in a local dictionary.
  \item Send the message \lstinline|get(K X)| to any agent, where \lstinline|X| is an unbound variable. This will route
    to the agent containing the key \lstinline|K|, look up the value stored there, and bind \lstinline|X| to the value.
    If there is no value corresponding to \lstinline|K|. the special value none is returned.
\end{itemize}
In your answer: the agents should be active objects or port objects. You should give the state
diagram of the agents. You are free to define the routing algorithm and the deterministic function
in any way you choose, as long as you explain how they work and give some justification for
them. You should set up the agent system with the connectivity that is required by your routing
algorithm. A bonus will be given if the agent system is robust against failures, i.e., if it continues
to work if any agent fails.

\begin{solution}
  \lstinputlisting{q2.oz}
\end{solution}

\section{Building a transaction protocol}
The answer to this question is a step-by-step construction of a practical transaction protocol.

\begin{itemize}
  \item Assume that we have a naive transaction manager that gives a lock to any transaction
    that requests it, if the lock is not yet taken, and that lets a transaction release any of
    its locks whenever it wants. If another transaction has the lock, then the transaction
    manager simply asks the requesting transaction to wait. Give a scenario to show why this
    transaction manager is not serializable. Why is this undesirable ? In your answer, define
    the concept of serializability. Fix the transaction manager so that it is guaranteed to be
    serializable.
  \item What is a cascading abort ? Why is it undesirable and how can it be avoided ? Give a
    scenario to show how it can occur. Define the concepts needed to avoid it. Extend the
    transaction manager of the previous part to avoid cascading aborts.
  \item What is a deadlock ? Why is it undesirable ? Give a formal definition using the wait-for
    graph including a formal definition of the wait-for graph. What are the two kinds of nodes
    in the wait-for graph ? Give an example of deadlock in the real world (not in the world
    of transactions). Explain what is a wait-for graph in your real-world example and explain
    how the existence of deadlocks is handled there. Are they prevented or detected, and how ?
  \item Extend the transaction manager of the previous part to perform deadlock avoidance using
    transaction priorities. What are the advantages and disadvantages of this approach ?
\end{itemize}

\end{document}
