\documentclass[en]{../../../../../../eplexam}

\usepackage{../../../../../../eplcode}

\hypertitle{Programming Languages Concepts}{6}{INGI}{1131}{2014}{Juin}
{Legat Beno\^it}
{Peter Van Roy}

\lstset{language={Oz},morekeywords={for,do,lazy}}

\section{Feedback addition}
For this question, do the following:
\begin{itemize}
  \item In the figure above, the rectangles are the declarative concurrent agents that read and write
    streams, and the triangle build a stream with a given first element and rest.
    Prod is a producer of successive integers starting at 0.
    Add takes two streams of integers and returns a stream whose
    elements are pairwise sum of the input elements.
    All agents are lazy.
    Write the code corresponding to this figue.
  \item For the streams
    A = [$a_0$ $a_1$ $\cdots$],
    B = [$b_0$ $b_1$ $\cdots$],
    C = [$c_0$ $c_1$ $\cdots$],
    write the equations for all $a_i$, $b_i$, and $c_i$.
  \item Solve the equations of the previous point for all $a_i$,
    $b_i$ and $c_i$.
  \item Write a procedure \lstinline|{Touch C N}| that takes
    \lstinline|C| as input and causes \lstinline|N| of \lstinline|C|
    to be calculated.
\end{itemize}
Make sure that all agents are declarative and tail-recursive.
\begin{solution}
  \begin{itemize}
    \item
      \lstinputlisting{q1-1.oz}

    \item
      \begin{align*}
        b_{i+1} &= c_i & b_0 &= 0\\
        a_i &= i\\
        c_i &= b_i + a_i
      \end{align*}
    \item
      therefore
      \[ b_{i+1} = b_{i} + i \]
      so, since $b_0 = 0$
      \[ b_{i+1} = 1 + 2 + ... + i = \frac{i (i+1)}{2} \]
      In conclusion
      \begin{align*}
        c_i & = \frac{i (i+1)}{2}\\
        b_i & = \frac{i (i-1)}{2} & b_0 & = 0\\
        a_i & = i
      \end{align*}
\item
  \lstinputlisting{q1-4.oz}
  \end{itemize}
\end{solution}

\section{Changing color}
For this question, you will write a multi-agent system.
\begin{itemize}
  \item The system consists of three agents, A, B, and C.
    Each agent stores a color as its internal state.
    Possible colors are red, blue and green.
  \item The inital state of each agent is red.
  \item The agents are arranged in a ring, A, B, and C,
    in that order, where each agent knows the next agent in the ring.
  \item Assume an agent has state \lstinline|C|.
    Whent an agent receives the message \lstinline|packet(P)|,
    then it changes its internal state to \lstinline|D| and
    send the message \lstinline|packet(E)| to the next agent
    where \lstinline|E| is any color different from \lstinline|C| and \lstinline|D|.
  \item When the system starts up, the message \lstinline|packet(blue)| is sent to agent A.
  \item Write the complete code of this system, using either port objects or active objects,
    including the definition\footnote{Oral clarification: he means the code, not a definition of the semantics in english.}
    of the port object or active objects abstraction (only define the one you have chosen).
  \item Give the complete state diagram of each agent, using the formalism introduced in the course
    (which we used for the lift control system).
\end{itemize}
The solution to this question will be graded on correctness and
elegance\footnote{Oral clarification: he means a clean, readable sheet.}.

\begin{solution}
  \lstinputlisting{q2.oz}
\end{solution}

\section{Monitors}
\begin{itemize}
  \item Define the monitor abstraction: its operations and their semantics.
    Be as precise as possible in your explanation of the operations semantics.
  \item Explain how to use monitor abstraction in a program.
    Give the basic programming pattern for monitors and explain how it works.
  \item Explain how to use the programming pattern in a simple example using a bounded buffer.
    Define the two basic operations of the bounded buffer,
    \lstinline|put| and \lstinline|get|, and show how to implement them with a monitor.
  \item Explain precisely what happens in the following three scenarios.
    What happens when attempting to \lstinline|put| an element when the buffer is full?
    What happens when attempting to \lstinline|get| an element when the buffer is empty?
    What happens when there are two waiting \lstinline|put| operations and a single \lstinline|get| is done?
\end{itemize}

\end{document}
