\documentclass[en]{../../../../../../epltest}

\usepackage{../../../../../../eplcode}
\lstset{language=Oz}

\hypertitle{Computer Language Concepts}{6}{INGI}{1131}{2019}{Mars}{All}
{Gilles Peiffer}
{Peter Van Roy}

\section{Question 1}
\begin{itemize}
	\item Give the semantics of \lstinline|{Bind X V}|.
	\item Give the semantics of \lstinline|{Wait X}|.
	\item Give the semantics of \lstinline|{WaitNeeded X}|.
	\item Express \lstinline|C = A + B| in dataflow.
	\item Express \lstinline|fun lazy {$ X} X*X end| in dataflow.
\end{itemize}
\begin{solution}
\begin{itemize}
	\item \nosubsolution
	\item \nosubsolution
	\item \nosubsolution
	\item \nosubsolution
	\item \nosubsolution
\end{itemize}
\end{solution}

\section{Question 2}
Give the complete execution tree for the following program.
\begin{lstlisting}
declare A B C in
A = 1
thread B = A + C end
C = 2
\end{lstlisting}

\nosolution

\section{Question 3}
Define the following concepts and operations.
\begin{itemize}
	\item Give an example of an agent
	with one input stream and one output stream.
	\item Given the definition
	$\mathrm{fold}(a\ u\ f) = (\cdots(((u\ f\ a_1)\ f\ a_2)\ f\ a_3) \cdots f\ a_n)$,
	write an implementation of $\mathrm{fold}$.
	Example: $\mathrm{fold}([1 2 3]\ 0\ +) = (((0 + 1) + 2) + 3) = 6$.
	\item Define the fairness property of the scheduler
	and explain the use of time slices.
\end{itemize}

\begin{solution}
	\begin{itemize}
		\item \nosubsolution
		\item \nosubsolution
		\item \nosubsolution
	\end{itemize}
\end{solution}
\end{document}
