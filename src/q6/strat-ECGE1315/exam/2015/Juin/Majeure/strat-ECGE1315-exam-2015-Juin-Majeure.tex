\documentclass[fr]{../../../../../../eplexam}

\hypertitle{Stratégie d'entreprise}{6}{ECGE}{1315}{2015}{Juin}{Majeure}
{Florian Thuin}
{Vincent Meurisse}

\section{Analyse de cas (6 points)}

Lonely Planet

\begin{itemize}
    \item Faire l'analyse PESTEL complète de l'industrie de l'édition
        (3pts)
    \item Expliquer les différentes orientations stratégiques de la
        firme au cours du temps
    \item Imaginer être manager de Lonely Planet et citer les
        opportunités de développement
\end{itemize}

\section{Question sur les TPs (4 points)}

\begin{itemize}
    \item Quels sont les facteurs clés de succès de la distribution des
        articles de sport. Justifier.
    \item Quel a été le changement organisationnel de Benetton ces
        dernières années. Justifier.
    \item Expliquer l'évolution de la stratégie générique de Club Med
    \item Quelle est la logique parentale du groupe Virgin. Expliquer.
\end{itemize}

\section{Partie théorique (10 points)}

\begin{itemize}
    \item Définir : synergie, DAS, stakeholders, intensité
        concurrentielle, partenariat d'impartition (5 pts)
    \item Expliquer la matrice McKinsey et expliquer pourquoi c'est une
        forme de visualisation de l'analyse SWOT. Ensuite, donner un
        exemple de positionnement d'un DAS d'une organisation dans l'une
        des cases de la matrice (5 pts).
\end{itemize}


\end{document}
