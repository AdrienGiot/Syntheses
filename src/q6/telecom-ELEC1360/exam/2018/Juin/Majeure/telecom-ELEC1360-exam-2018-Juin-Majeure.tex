\documentclass[fr]{../../../../../../eplexam}

\hypertitle{Télécommunications MAJ}{6}{ELEC}{1360}{2018}{Juin}
{Martin Braquet}
{Luc Vandendorpe}

\section{}
Soit
\[
    v(t)=Y(t)+I(t)\:\cos(\omega_{cc}t)+Q(t)\:\sin(\omega_{cc}t)
\]
\begin{enumerate}
 \item Trouver $v_a(t)$, le signal analytique de $v(t)$, en expliquant toutes les nouvelles notations utilisées.
 \item Ecrire la transformée de Hilbert de $v(t)$ et expliquer comment vous avez procédé.
\end{enumerate}

\section{}
Exprimer la covariance de l'enveloppe complexe du signal $x(t)$ et la covariance de son signal analytique en fonction des covariances de ses composantes de Rice, en sachant que $x(t)$ est une variable aléatoire faiblement stationnaire à bande étroite.

\section{}
Soit 
\[
 x(t)=x_1(t)\:\cos(\omega_0t)+x_2(t)\:\sin(\omega_0t)
\]
Que vaut ce signal à la sortie d'un discriminateur?

\section{}
On envoie deux signaux de manière équiprobable: $s_0(t)$ et $s_1(t)$. On reçoit $x(t)=s_1(t)+w(t)$ si 1 est envoyé, et $x(t)=s_0(t)+w(t)$ si 0 est envoyé. Les énergies de $s_0(t)$ et $s_1(t)$ valent respectivement $\mathcal{E}_0$ et $\mathcal{E}_1$. $C_{01}$ est la corrélation entre ces deux signaux. 
\begin{enumerate}
 \item Ecrire proprement la probabilité d'erreur.
 \item Expliquer comment se passe la réception , expliquer le filtrage adapté et l'échantillonage. Expliquer la variable de décision qu'il faut employer.
 \item Calculer la probabilité d'erreur.
 \item Comparer cette probabilité avec une corrélation connue (soit pour des signaux orthogonaux ou pour des signaux opposés).
\end{enumerate}

\end{document}
