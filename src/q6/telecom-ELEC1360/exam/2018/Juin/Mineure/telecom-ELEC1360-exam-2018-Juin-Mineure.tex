\documentclass[fr]{../../../../../../eplexam}

\hypertitle{Télécommunications MIN}{6}{ELEC}{1360}{2018}{Juin}
{Martin Braquet}
{Luc Vandendorpe}

\section{}
Soit
\[
    v(t)=Y(t)+I(t)\:\cos(\omega_{cc}t)+Q(t)\:\sin(\omega_{cc}t)
\]
\begin{enumerate}
 \item Trouver $v_a(t)$, le signal analytique de $v(t)$, en expliquant toutes les nouvelles notations utilisées.
 \item Ecrire la transformée de Hilbert de $v(t)$ et expliquer comment vous avez procédé.
\end{enumerate}

\section{}
On envoie deux signaux de manière équiprobable: $s_0(t)=\sin(2\pi f_0t)$ et $s_1(t)=\sin(2\pi f_1t)$. On reçoit $x(t)=s_1(t)+w(t)$ si 1 est envoyé, et $x(t)=s_0(t)+w(t)$ si 0 est envoyé. Les énergies de $s_0(t)$ et $s_1(t)$ valent respectivement $\mathcal{E}_0$ et $\mathcal{E}_1$. $C_{01}$ est la corrélation entre ces deux signaux. 
\begin{enumerate}
 \item Ecrire proprement la probabilité d'erreur.
 \item Expliquer comment se passe la réception , expliquer le filtrage adapté et l'échantillonage. Expliquer la variable de décision qu'il faut employer.
 \item Calculer la probabilité d'erreur.
 \item Calculer les valeurs optimales pour $f_0$ et $f_1$.
\end{enumerate}

\section{}
\nostatement

\end{document}
