\section{}
\subsection{}
\begin{solution}
  The material necessary for this exercise has not been seen this year (2014-2015).
\end{solution}

\subsection{}
\begin{solution}
  \begin{enumerate}
    \item I guess he means ``completeness''.
      In this case, it is rather obvious.
      He has $x$ then it can computes $s \equiv r - cx \pmod{q}$.
    \item From
      \begin{align}
        \label{eq:schnorr1}
        s   & \equiv r - c  x \pmod{q}\\
        \label{eq:schnorr2}
        s^* & \equiv r - c^*x \pmod{q}
      \end{align}
      we have $\eqref{eq:schnorr1} - \eqref{eq:schnorr2}$:
      \[ s - s^* \equiv -(c - c^*)x \pmod{q} \]
      hence
      \[ x \equiv -(s - s^*)(c - c^*)^{-1} \pmod{q}. \]
    \item
      Let's do a reduction: we will show that if there is an
      adversary $P^*$ that can convince $V$, we can build an adversary $\A$ that solve the $\DLog$ problem.

      Let $P^*$ be such that $P^*$ does not know $x$.
      Let's run it.
      When it gives $t$, let's save its state and then give him $e$.
      It will output $s$.
      We will now rewind it to the previous state and give him $e' \neq e$.
      With the $s'$ he outputs, we can find $x$.

      Under the assumption that $\DLog$ is hard, this scheme is therefore sound.
    \item
      Not, what is convincing is the fact that he have found $s$ \emph{after} we have chosen
      $s$.

      Here, he could have just chosen random $c, s$ and picked $t \equiv y^cg^s \pmod{p}$.
    \item
      The verifier has $\langle t,c,s \rangle$ in its transcript.
      We have just seen that it this can be easily computed without knowledge of $x$.
      A simulator can therefore easily output $\langle t,c,s \rangle$ without knowledge of $x$.

      However, it has only been proven that is is zero-knowledge
      if the verifier is ``Honest'' which means that he will pick $c$ at random.
      If he picks $c$ as a function of $t$, it's getting hard to find triple $\langle t,c,s \rangle$
      for the simulator since $c$ is a function of $t$ so we cannot compute $t$ after having chosen
      $c$ and $s$ at random.
      For example, a dishonest verifier could pick $c := \mathcal{H}(t)$ where $\mathcal{H}$ is a hash function.
    \item \emph{I don't understand the Hint ``$m$ is the message to sign'', what message ???}

      We have seen that $c$ is a function of $t$,
      it is not easy to find $\langle t,c,s \rangle$.
      Maybe if we show in the triple that $c$ is a function of $t$,
      we can prove that we know $x$.

      $P$ will randomly pick $r \in_R \mathbb{Z}_q$ and compute $t \equiv g^r \pmod{p}$.
      He will then compute $c = H(y\|t)$ that everybody can recompute to see that $c$ could not have been chosen \emph{a priori}.
      He can now use his knowledge of $x$ to find $s \equiv r - cx \pmod{q}$ and send $\langle t, c, s \rangle$.

      It is sound in the ROM (Random Oracle Model).
      Let's do a reduction to show it.
      If $P^*$ does not know $x$ and is able to convince a verifier,
      we can do the same reduction than 3. but placing ourself in the body of $H$ this time, not of $V$.
      \begin{itemize}
        \item $\A$ starts $P^*$ with $y$ and answers $H(y\|t)$ with random $e$ until $P^*$ does a valid triple $\langle t^*,c^*,s^* \rangle$.
        \item $\A$ starts $P^*$ with $y$ again and answers $H(y\|t')$ with a random $e'$ until $t' = t^*$ and $\langle t',c',s' \rangle$ is valid.
          There will be a probability $\frac{1}{q}$ that $e' = e$ so we are almost sure that we have $e' \neq e$
          and $\A$ we can solve $\DLog$ as explained earlier.
      \end{itemize}
      Sadly, since $P^*$ picks $r$ at random when computing $t$, it is not likely that we will have $t' = t^*$.
      We can as well pick candidates $x_c$ for $x$ randomly and try if $g^{x_c} \equiv y \pmod{p}$ to try to solve $\DLog$,
      $\A$ is not more efficient than a brute force.

      However since we are in thre ROM, we can ``manipulate randomness'' and make $r$ be chosen a little bit more ``often'',
      if you know what I mean.
  \end{enumerate}
\end{solution}
