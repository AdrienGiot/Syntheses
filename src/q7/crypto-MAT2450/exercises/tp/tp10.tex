
\section{}

% From TP4
\subsection{Exercise 1 (Authenticated encryption, or not; August Exam)}

Let $\Pi \define \langle \Gen, \Enc, \Dec\rangle$ be an authenticated encryption
scheme such that $\Enc$ encrypts messages of $n$ bits.
%
Do the following systems provide authenticated encryption?  For those
that do, briefly explain why.  For those that do not, present an
attack that breaks one of the security properties of an authenticated
encryption scheme.

\begin{enumerate}
	\item $\Pi' \define \langle \Gen, \Enc', \Dec'\rangle$ with
	$\Enc'_k(m) = (\Enc_k(m), \Enc_k(m \oplus (0^{n-1}\|1)))$ and
	$\Dec'_k(c_1, c_2) = \Dec_k(c_1)$ if
	$\Dec_k(c_1) \oplus \Dec_k(c_2) = 0^{n-1}\|1$ and $\bot$ otherwise.

	\item $\Pi' \define \langle \Gen, \Enc', \Dec'\rangle$ with
	$\Enc'_k(m) = (\Enc_k(m), \Mac_k(m))$ and $\Dec'_k(c_1, c_2) = \Dec_k(c_1)$

	if $\Vrfy_k(\Dec_k(c_1), c_2)=1$ and $\bot$ otherwise. Here, $\Mac$
	and $\Vrfy$ are deterministic algorithms that are part of a secure
	MAC scheme that is compatible with $\Gen$.
\end{enumerate}

\begin{solution}
	$\Pi \define \langle \text{Gen, Enc, Dec} \rangle$ is an authenticated encryption scheme (AE) if it is CCA-secure and unforgeable.
	\begin{enumerate}
		\item The system $\Pi'$ is not AE because it is \textbf{forgeable} and we can show it with this example. If the adversary A asks for the message $m$ ($m'$ corresponds to the message m with the last bit changed) to the oracle access, he will receive the cipher text $(c_1, c_2)$, where $c_1 = \Enc_k(m)$ and $c_2 = \Enc_k(m\xor 0^{n-1}||1) = \Enc_k(m')$.

		If $\A$ outputs the pair (m',$(c_2, c_1)$), this is a forgery.

		$\Dec_k'(c_2,c_1) = \Dec_k(c_2) = \Dec_k(\Enc_k(m')) = m' \neq \bot $ because $\Dec_k(c_2) \oplus \Dec_k(c_1) = m' \oplus m = m \oplus 0^{n-1}||1 \oplus m = 0^{n-1}||1 $.
		And $m'$ has not been requested before.

		Then we have EncForge$_{A, \Pi'}$(n) = 1  and Pr[EncForge$_{A, \Pi'}$(n)] = 1. $\Pi'$ is then forgeable and it is not an AE.

		With the same technique, an adversary can break the CCA-security of this scheme by querying two different messages $m_1$ and $m_2$, obtaining their encryption, sending \newline
		($m'_1,m'_2$) = ($m_1 \oplus 0^{n-1}||1, m_2 \oplus 0^{n-1}||1$) for the challenge, and compare the encryption of $m'_b$ with the two previously received ciphertexts.

		Note that this doesn't break CPA-security; indeed, the scheme is still CPA-secure.

		\item The sytem $\Pi'$ is not AE because it is not \textbf{CCA-secure} and we can show it because $Mac_k(m)$ does not assure any security (only authentication). So if We use as Mac:
		\[ \Mac_k(m) = m||\Mac'_k(m) \]
		It is a good mac but it is trivial to show that it is not CCA-Secure. $\Pi'$ is then not an AE.

		\strong{Stronger argument}:

		The adversary can send two different messages $m_0$ and $m_1$ to the encryption oracle to get $(\Enc_k(m_0), \Mac_k(m_0))$ and $(\Enc_k(m_1), \Mac_k(m_1))$.

		We then output the same $m_0$ and $m_1$, and receive $(\Enc_k(m_b), \Mac_k(m_b))$.

		As $\Pi$ is an athenticated encryption scheme, we know that $\Enc$ is probabilistic and secure.
		However, $\MAC$ is said to be deterministic, and this causes $\Mac_k(m_b)$ to be the same as one of the $\Mac_k(m_0)$, $\Mac_k(m_1)$ received earlier.
		We can thus just compare the tags, and output the corresponding $b'$.

		The probability of success is $\Pr[b'=b] = \Pr[\Mac_k(m0) \neq \Mac_k(m_1)]=1-\negl(n)$ which is clearly well above what it should be.
	\end{enumerate}
\end{solution}



\subsection{Exercise 2 (Derandomizing signatures)}

Let $S=(\Gen, \Sign, \Vrfy)$ be an EUF-CMA signature scheme defined over $(M, \Sigma)$, where the signing algorithm $\Sign$ is probabilistic.
In particular, algorithm $\Sign$ uses randomness chosen from a space $R$.
We let $\mathsf{S}(sk, m; r)$ denote the execution of algorithm $\mathsf{S}$ with randomness $r$.
Let $F$ be a secure PRF defined over $(K, M, R)$.
Show that the following signature scheme with deterministic signing $S'=(\Gen', \Sign', \Vrfy)$ is EUF-CMA:
\[ \mathsf{G}'(1^n) \define \left\{ (pk, sk) \pick \G(1^n), \qquad k \pick K, \qquad sk' \define (sk, k), \qquad \text{output } (pk, sk') \right\}; \]
\[ \Sign'((sk, k), m) \define \left\{ r \pick F_k(m), \qquad \sigma \pick \mathsf{S}(sk, m; r), \qquad \text{output } \sigma \right\}. \]

\emph{(Hint: Define $S''$ which is like $S'$ byt uses a perfect random function. Make a reduction of the security of $S''$ to the security of $S$, then build a PRF distinguisher based on a adversary against the signature. Finally, compute the link of the advantages of three relevant games.)}


\begin{solution}
	We will prove this in two steps.
	\begin{itemize}
		\item The first step defines $S''$ so that the PRF is replaced by a true random function.
		We will show that in this case, if $S$ is secure, then $S''$ is secure.
		\item The second step proves by reduction that if $S''$ is secure and $F$ is a secure PRF, then $S'$ is secure.
		The reduction proceeds by constructing an adversary $\A_{PRF}$ against the PRF, able to distinguish between the PRF and a random function, given an adversary $\A_{S'}$ against $S'$.
	\end{itemize}

	First, let's define $S''$.
	The only change is that
	\[ \Sign''((sk, k), m) \define \{ r \define f(m), \quad \sigma \define \mathsf{S}''(sk, m), \quad \text{output } \sigma \} \]
	where $\mathsf{S}''(sk, m) = \mathsf{S}(sk, m; f(m))$.

	If we have an adversary against $S''$, we can build an adversary against $S$, simply by relaying oracle calls to $\mathsf{S}_f''(sk, m)$ to our oracle $\mathsf{S}(sk, m; f(m))$.
	As $f$ is a random function, $S$ sees the same randomness as with an $r$, so nothing changes.
	The probabilities are exactly the same:
	\[ \Pr[\Sigforge_{S''}=1] = \Pr[\Sigforge_{S}=1] = \negl(n) \]
	for some $\negl$ negligible.

	Now, let's do the reduction from $F$ to $S'$.
	So, we have an adversary $\A_{S'}$ against $S'$, and we build an adversary against the PRF $\A_F$ as follows:
	\begin{enumerate}
		\item The oracle for the PRF problem $\O_F$ picks $k \pick \K$, kept secret.
		\item $\O_F$ picks $b \pick \bset$, kept secret.
		The oracle defines a challenge function $g=F_k$ if $b=1$, $g=f$ a random function if $b=0$.
		\item $\A_F$ runs $\Gen'$ to create a public key $pk$ and a secret key $sk'=(sk, k)$.
		It discards $k$ as it is not used by $\A_{S'}$ and is defined by $\O_F$.
		It keeps $sk$ secret and sends $pk$ to $\A_{S'}$.
		\item When $\A_{S'}$ asks for $\Sign'((sk, k), m)$, we ask the oracle for $w=g(m)$.
		$w=F_k(m)$ if $b=1$, or $w=f(m)$ with $f$ a random function, if $b=0$.
		We then run $\mathsf{S}(sk, m; r)$ and return the result to $\A_{S'}$.
		\item When $\A_{S'}$ outputs $(m, \sigma)$, its forgery, we outputs $b'=1$ iff $(m, sigma)$ is a valid forgery (we can verify it using $\Vrfy_{pk}$), $b'=0$ otherwise.
	\end{enumerate}
	If $b=0$, we're playing against a true random function, and so we're actually running the scheme $S''$.
	As $\A_{S'}$ is not designed to handle this scheme, its security is the same as the one of $S''$, which is the same as the one of $S$.

	If $b=1$, we're playing the true game for $\A_{S'}$, and so its advantage $\negl'(n)$ is active.

	Then, the difference in probabilities for the distinguishing is:
	\[ |\Pr[\A_F^{F_k}(n)] - \Pr[\A_F^{f}(n)]| = |\negl'(n) - \frac12 \negl(n)| \ge \negl'(n)-\negl(n) \]
	As this difference has to be negligible, as $F$ is a PRF, and $\negl$ is negligible, then $\negl'$ is negligible, and thus the scheme $S'$ is secure.
\end{solution}



\subsection{Exercise 3 (Jan 11 evaluation)}

\copypaste{9}{1}

