\section{}

The TP5 of 2019-2020 reviewed the basics of number theory and group theory for this course. As they are not really useful for the exam and are fairly trivial (once the basics are know, that is), they are skipped in this document.

Below are the exercises of last year pertaining to groups and number theory.



\subsection{Exercise 0 (Group order)}

\copypaste{3}{5}



\subsection{Exercise 3 (Euclidean algorithm for gcd)}

Let $a,b \in \mathbb{Z}$ , $b \neq 0$, consider the following algorithm, presented in Algorithm~\ref{algo:gcd}. ($r=a \% b$ means that $a=qb+r$ where $q$ is the quotient and $r$ is the remainder).

Prove that $x$, the value returned by Algorithm~\ref{algo:gcd}, is $\mathsf{gcd} (a,b)$.

Hint:
\begin{itemize}
	\item Prove that $x$ divides  $ \mathsf{gcd} (a,b)$
	\item Prove that $ \mathsf{gcd} (a,b)$ divides $x$
\end{itemize}

\begin{algorithm}
	\KwIn{$a$, $b$}
	\KwOut{$\mathsf{gcd}(a,b)$}

	\While{ $b\neq 0$}
	{
		$r \leftarrow a\%b$\;

		$a \leftarrow b$\;

		$b \leftarrow r$\;
	}
	\Return($a$)

	\caption{The Euclidean $\mathsf{gcd}$ algorithm.}\label{algo:gcd}
\end{algorithm}


\begin{solution}
	According to the algorithm, we will have as successive value for the different remainder:
	\[(r_2 = r_0 \% r_1, r_3 = r_1 \% r_2, \ r_4 = r_2 \% r_3, \ ... \ , r_n =  r_{n-2} \% r_{n-1})\]
	Where $r_0 = a$, $r_1 = b$ and $r_n$ is the last non null remainder. Then we have the property that :
	\[ gcd(r_{i}, r_{i+1}) = gcd(r_{i+1}, r_{i+2}) \ \forall i : \ 0 \leq i \leq n - 2 \]
	Otherwise if it was not the case, $\exists i < n $ such that $r_i = 0$. But as $r_n$ is the last non null remainder, we prove by contradiction this property.

	As $gcd(r_{n-1}, r_n) = r_n $ because $r_n | r_{n-1}$ (since $r_{n+1} = r_{n-1} \% r_n = 0$), we can conclude that
	\[ gcd(r_0, r_1) = gcd(a, b) = gcd(r_{n-2}, r_{n-1}) = r_n\]
	We have proved the value returned by the algorithm is the $gcd(a,b)$
\end{solution}



\subsection{Exercise 4}

Consider the group $\mathbb{Z}^{\ast}_{17}$.
\begin{enumerate}
	\item Compute $5^{-1}$.
	\item Compute $3^2$, $3^3$ and $3^4$.
	\item Does $3$ generate the group?
	\item Find $\log_{7}(11)$.
\end{enumerate}


\begin{solution}
	Here because p is not too big, it is possible to evaluate "quickly" and "intuitively" the solutions. If it is too hard, there is an algorithm in the slides.
	\begin{enumerate}
		\item Because 35 mod 17 = 1, and $5 \cdot 7 = 35$. \newline Then $5^{-1} = 7$.  ($5 \cdot 7 = 1 \text{ (mod 17)}$)
		\item \begin{itemize}
			\item $3^2 = 9 \text{ (mod 17)}$
			\item $3^3 = 3^2 \cdot 3 = 27 \text{ (mod 17)} = 10 \text{ (mod 17)}$
			\item $3^3 = 3^3 \cdot 3 = 30 \text{ (mod 17)} = 13 \text{ (mod 17)}$
		\end{itemize}
		\item According to \textit{Fermat's little theorem}, if ord(g) = i then if $i|m = |G|$, where G is the commutative group. To see if 3 generate the group, we have to check if $3^i \neq 1$ where $i$ are the divisor of $(p-1) = 16$ (except 16 of course !).
		\begin{itemize}
			\item $3^1$ = 3 mod 17 (trivial)
			\item $3^2$ = 9 mod 17 (evaluated previously)
			\item $3^4$ = 13 mod 17 (evaluated previously)
			\item $3^8$ = $(3^4)^2$ = $(13)^2 \text{ (mod 17)}$ = $(-4)^2 \text{ (mod 17)}$ = 16 mod 17
		\end{itemize}
		We can see that 3 is a generator of the group.

		\textbf{P.S.} : The trick here is to remember the property of the modulo operation here in a $Z^*_p$:
		\[ x = -(p - x) \text{ (mod p)} \]
		It can make a lot of computing easier (it can become a real pain in the ass).
		\item Here we have to find x such that:
		\[ 7^x = 11 \text{ (mod 17)} \]
		After (boring) computations, we have here :
		\begin{itemize}
			\item $7^1$ = 7 mod 17
			\item $7^2$ = 15 mod 17 = -2 mod 17
			\item $7^3$ = -14 mod 17 = 3 mod 17
			\item $7^4$ = 4 mod 17
			\item $7^5$ = 11 mod 17 (Bingo)
		\end{itemize}
		Then $log_7(11)$ = 5
	\end{enumerate}
\end{solution}



\subsection{Exercise 5 (Group order)}

In this exercise we consider the group $\Z_{59}^*$.

\begin{enumerate}
	\item What is the order of $58$?
	\item What are the possible orders of an element of this group?
	\item Find an element of order more than $20$.
\end{enumerate}


\begin{solution}
	The order of $g \in \Z^*_{59}$ is the smallest $i$ where $g^i = 1$
	\begin{enumerate}
		\item ord(58) = 2 because:
		\begin{itemize}
			\item  $58^1$ = 58 mod 59 = -1 mod 59
			\item  $58^2$ = -58 mod 59 = 1 mod 59
		\end{itemize}
		\item According to \textit{Fermat's little theorem}, the possible orders of a group $\mathbb{Z}^*_{p}$ are the divisor of p-1. Then, the possible orders are:
		\begin{itemize}
			\item 1
			\item 2
			\item 29
			\item 58
		\end{itemize}
		\item The best strategy here is to find a number g where
		\[ ord(g) > 2 \]
		(To assure you this is correct, just look at the possible ordrers).  \newline
		2 is a correct candidate.
	\end{enumerate}
\end{solution}
