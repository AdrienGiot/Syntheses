\documentclass[en]{../../../../../../eplexam}

\hypertitle{Modelling and analysis of dynamical systems}{7}{INMA}{2370}{2020}{Janvier}{All}
{Nathan Jacques}
{Denis Dochain}

\section{Question 1}
One of the most popular models in mathematical ecology is the predator-prey model, for which the basic version can be written as follows:

\begin{align}
	\dot{x}_1 &= rx_1 - bx_1x_2 \label{eqn:x1}\\
	\dot{x}_2 &= cx_1x_2 - mx_2 \label{eqn:x2}
\end{align}

where \(x_1\) and \(x_2\) are the concentrations of preys and predators, respectively. \(r, b, c \) and \(m\) are the prey specific growth rate, two yield coefficients, and the predator mortality coefficient, respectively.

\begin{enumerate}
	\item The model can be viewed as a reaction system. How many reactions do you identify? What is the reaction network? Write down the \(C\) matrix and the reaction rate vector \(r(x)\).
	\item What are the equilibrium points of the system?
	\item Consider the following set of parameter values:
	\[r=1,b=1,c=1,m=1\]
	Analyze the stability of the equilibrium points with the first Lyapunov method. What are your conclusions? What can you say about the trajectories of the system in the phase plane?
	\item Let us consider that the growth of the preys are characterized by a logistic model. The system dynamics (\ref{eqn:x1}),(\ref{eqn:x2}) is then modified as follows:
	\begin{align}
		\dot{x}_1 &= rx_1\left( 1-\frac{x_1}{K} \right) - bx_1x_2 \\
		\dot{x}_2 &= cx_1x_2 - mx_2
	\end{align}
	where \(K\) is the carrying capacity.
	Redo point 2 (equilibrium points) and 3 (stability analysis) for the above model and \(K = 100\). Do you get similar results? Explain.
\end{enumerate}

\nosolution

\section{Question 2}
Let us consider the wing of an airplane, with angle \(-\pi < \theta < \pi \) with the wind direction, whose velocity is constant \(v>0\). Let us model the wing subject to a twisting due to:
\begin{itemize}
	\item a torque applied by the wind, of amplitude \(lv \sin \theta \), for a constant \(l >0\) ;
	\item a spring torque \(-k (\theta - \alpha)\) applied by the airplane body (whose axis has a constant angle \(\alpha >0\) with the wind direction) on the wing, with a spring constant \(k >0\) ;
	\item a friction torque \( -m \dot{\theta}\) , for a friction constant \(m>0\).
\end{itemize} 

The law of Newton-Euler for the wing is then written for an inertia moment \(I > 0\):
\[I \ddot{\theta} = -k (\theta - \alpha) + lv \sin \theta - m \dot{\theta} \]

\begin{enumerate}
	\item Write the state space model of the system with the variables \( x = \theta - \alpha \) and \(y = \frac{d}{dt}(\theta - \alpha)\) ;
	\item Let us assume that \(\theta\) and \(\alpha\) are close to 0:
	\begin{enumerate}
		\item From the linearized tangent model of the system, discuss the number and the nature (hyperbolic or not, attractive or not, node or focus) of the equilibria as a function of the input \(v >0\) and of the parameters \(k,l,m\) and \( \alpha \) ;
		\item Draw qualitatively the diagram of the equilibria. What is happening in practice when the velocity is indefinitely increasing from low values? \textit{This phenomenon is known in aeronautics as "divergence".}
	\end{enumerate}
	\item Discuss the number and nature (attractive or repulsive) of the equilibria for the nonlinear model and draw qualitatively the diagram of the equilibria. Compare the diagram of the equilibria with that of the previous case. (Hint: it is not absolutely necessary to have the analytical expression of the equilibrium points to answer this question).
\end{enumerate}


\nosolution

\section{Question 3}
Consider the system
\begin{align*}
	\dot{x} &= -x + x^2y \\
	\dot{y} &= x-y
\end{align*}

\begin{enumerate}
	\item Compute all the equilibria of this system, study their stability, and characterize those that are hyperbolic (attractive or repulsive node or focus, or saddle point);
	\item Write the linearized system around the origin. Ex^press all its trajectories as a function of time and initial state;
	\item For each asymptotically stable equilibrium, give a compact set the interior of which is not empty that is included in the basin of attraction. Assuming that an orbit starts in that set, can you give an upper bound on the time it needs to enter the closed disk centered at the equilibrium and of radius \(\epsilon\) ? Your answer can be valid only for sufficiently small \( \epsilon \) ;
	\item Draw the equilibria and the set given at the preceding subquestion in the state plane.
\end{enumerate}

\nosolution

\end{document}
