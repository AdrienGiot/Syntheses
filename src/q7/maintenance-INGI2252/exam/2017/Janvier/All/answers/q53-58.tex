\section{Give and explain at least 2 design heuristics about the relation between a subclass and its
superclass.}

\begin{description}
\item[Heuristic D.37] \textbf{Derived classes must have knowledge of their base class by  definition, but base classes should not know anything about their derived classes.}

A superclass should not know its subclasses because if base classes have knowledge of their derived classes, then it is implied that if a new derived class is create from the base class, the code of the base class will need modification. This is an undesirable dependency between the abstractions captured in the base and derived class.

\item[Heuristic D.38] \textbf{All data in a base class should be private; do not use protected data.}


Subclasses should not use directly data of superclasses.
There is nothing wrong with defining protected accessor methods. They are allowing derived class implementors access to base class data in an implementation-safe way. The implementors of derived classes have a right to access the data of their base class; however, the users of a class do not have a right to the data of the class they are using.


\end{description}


\section{Discuss the design heuristics which state that “All abstract classes must be base classes”
and “All base classes should be abstract classes”. Do you agree with these heuristics? Under what conditions?}

\begin{description}
\item[Heuristic D.41] \textbf{All abstract classes must be base classes.}

Intuition : why make a method abstract if you won’t concretise it in a subclass?
If a class cannot build objects of itself, then that class must be inherited by some derived class that does know how to build objects. If this is not the case, then the functionality of the base class can never be accessed by any object in the system, and therefore the class is irrelevant in the given domain.

Counter-example : application frameworks

\item[Heuristic D.42] \textbf{All base classes must be abstract classes.}

This heuristic implies that all the roots of an inheritance three should be abstract, while only the leaves should be concrete.
Teacher's point of view \\
Intuition : methods overridden in the subclasses should be abstract in the superclass. \\
Not necessarily true :
they can have a default behaviour or value in the superclass; the subclass may add only new methods.


According to the reference book, if the base class is a concrete class, when adding a new subclass, or the specialization of a subclass, which adds a single additional specific method to the base class, there is a problem. Indeed, we will have to create a derived class just to add a single specific method that will break the design if this happens too often. These classes are not relevant because they add only one new specific method.

It is therefore preferable to have a generalized base class which tends to make the class abstract. It is therefore necessary to find the balance between the concrete basic classes that can break a design if they are too many to have several derived classes not relevant and abstract basic classes.

\end{description}



\section{Several design heuristics are related to the need for high cohesion. Discuss 2 such
heuristics and their relation with cohesion.}

\begin{description}
\item[Heuristic C.27] \textbf{Most of the methods defined on a class should be using most of the data members most of the time.}

If this is not true for a given class, then it is probable that the designer has captured two or more abstraction in one class. A class should capture meaningful abstraction within a domain. In the grossest violation, half of a class's methods will use half of the data members. The class should be split along these lines due to it having too much non communicating behavior. 

\item[Heuristic C.28] \textbf{Classes should not contain more objects than a developer can fit in his or her short-term memory. A favourite value for this number is six (or seven).}

The rationale behind this heuristic is that most the methods defined on a class should be using most of the data members most of the time. Assuming this is true, implementors of a method will need to think about all of the data members while writing the method. If the developer cannot keep all of the data in his or her short-term memory, then items will be omitted and bugs will creep into the code. In short, this heuristic is a complexity metric on a class.
\end{description}

\textbf{Less obvious relationship with coherence}

\begin{description}
\item[Heuristic A.8] \textbf{A class should capture one and only one key abstraction.}

A key abstraction is defined as a main entity within a domain model. Key abstractions often show up as nouns within requirements specifications. Each key abstraction should map to only one class. If it maps to more than one class, then the designer is probably capturing each function as a class. If more than one key abstraction maps to the same class, then the designer is probably creating a centralized system. These classes are often called vague classes and need to be split into two or more classes that capture one abstraction each.


\item[Heuristic A.9] \textbf{Keep related data and behavior in one place.}

A violation of this heuristic will cause a developer to program by convention. That is, to accomplish some atomic system requirement, he or she will need to affect the state of the system in two or more areas. The two areas are actually of the same key abstraction and therefore should have been captured in the same class. The designer should watch for objects that dig data out of other objects via some "get" operation. That type of activity implies that this heuristic is being violated. Consider a user of a stove class trying to preheat an oven for cooking. The user should only send the stove an are\_you\_preheated?() message. The oven can test if the actual temperature has reached the desired temperature, along with any other constraints concerning the preheating of ovens. A user who decide if oven is preheated by asking the oven for its actual temperature, its desired temperature, the status of its gas valve, the status of its pilot light, etc., is violating this heuristic. The oven owns the information of temperature and gas cooking apparatus; it should decide if the object is preheated. It is important to note the need for "get" methods in order to implement the incorrect preheat method.

\item[Heuristic A.10] \textbf{Spin off non related information into another class.}

The developer should look for classes with a subset of methods that operate on a proper subset of the data members. The extreme case is a class where half of the methods work on one half of the data members on the other half of the methods work on the other half of the data members. For a more real-world example, consider a dictionary class. For small dictionaries the best implementation is a property list (a list of words and their definitions), but for larger dictionaries a hash table is better, faster. Both dictionary implementation require the ability to add and find words.

\end{description}



\section{Several design heuristics are related to the need for loose coupling. Discuss 2 such heuristics and their relation with coupling.}

\begin{description}
\item[Heuristic C.22] \textbf{Minimize the number of classes with which another class collaborates.}


This heuristic claims that the list of classes another class uses should be kept to a minimum. In the worst case, an object-oriented design will consist of a collection of primitive, simplistic classes, all of which use each other. The top-level design will be very difficult to comprehend. Notice that our heuristic of avoiding a god class is not violated here. We may have distributed system intelligence very uniformly among the large group of top-level objects. The solution is to look for places in the uses relationship graph where one class communicates with another group of classes and ask, "Can I replace this group of classes with one class containing the group, thereby reducing the number of collaboration?" There will be many such tests of which only a percentage will answer yes. These should be wrapped in containing classes to reduce complexity.
It is important to note that once an object of a class sends a message to an object of another class, a collaboration exists between the two classes.

\item[Heuristic C.24] \textbf{ Minimize the amount of collaboration between a class
and its collaborator, that is, the number of different messages sent.}

It these heuristic are considered equal to the collaboration minimization heuristic (Heuristic C.22), the we would be implying that there is some break-event point between adding more message sends between classes and adding a new collaboration. For example, we might argue that rather than letting a method of the person class use a new alarm clock message, it might be better for the person class to collaborate with some entirely new class. This is never true, because the big hit in complexity for the uses relationship comes at the class level, not the object level. The fact that the person class is dependent on two classes rather than one is the overriding concern. The amount of coupling between the objects of a class is much smaller problem.

\end{description}

\textbf{Less obvious relationship whit loose coupling}

\begin{description}
\item[Heuristic A.2] \textbf{Users of a class must be dependent on its public interface, but a class should not be dependent on its users.}

The rationale behind this heuristic is one of reusability. An alarm clock might be used by a person in a bedroom. The person is obviously dependent on the public interface of the alarm clock. However, the alarm clock should not be dependent on the person. If the alarm clock were dependent on its user, namely the person, then it could not be used to build a timelock safe without attaching a person to the safe. These dependencies are undesirable, since we want to be able to lift the alarm clock out of its domain and deposit it into another domain with no user dependencies. It is best to view the alarm clock as a little machine having no knowledge of its users. It simply performs the behaviors defined in its public interface for whoever sends it a message.

\item[Heuristic A.3] \textbf{Minimize the number of messages in the protocol of a class.}

The exact opposite of this heuristic is the claim is that is a class implementor can think of some operation for a class then someone will want to use it in the future.  Why not implement it ?
The problem with large interfaces is that you can never find what you are looking for. This is a serious problem with reusability in general. Be keeping the interface minimal, we make the system easier to understand and the components easier to reuse.

\item[Heuristic A.7] \textbf{A class should only use operations in the public interface of another class or have nothing to do with that class.}

All other form of coupling allow a class to give away implementation details to other classes, thereby creating implied dependencies between the implementations of the two classes. These implied dependencies always cause future maintenance problems when one of the classes wishes to change its implementation.


\end{description}


\section{Discuss the following design heuristic “Explicit case analysis on the type of an object is usually an error.”}

Or at least bad design : the designer should use polymorphism instead in most of these cases. Indeed, an object should be responsible of deciding how to answer to a message. A client should just send messages and not discriminate messages sent based on receiver type

If a designer creates a method that states, \enquote{If you are of type1, do this; else if you are of type2, do this; else if you are of type3, do this, else if ...,} he or she is making a mistake. The better approach is to have all of the type involved in the explicit case analysis inherit from a common class. This common class defines a pure polymorphic function call \verb#do_this#, and each type can write its own \verb#do_this# method. Now the method the designer creates need only send the \verb#do_this# message to the object in question. The polymorphism mechanism can perform the case analysis implicitly, eliminating the need to modify existing code when a new type is added to the system.


\section{Give a concrete example of and discuss when multiple inheritance would be a valid design
solution.}

Multiple inheritance is the ability for a class to directly inherit from more than one base class. Multiple inheritance is useful for capturing a relationship known as sub-typing for combination. It is used to define a new class that is actually a special type of two classes and those two classes are from different domains.

Consider an example which describes one method for defining a wooden door. We must first run through questions to ensure that our design is correct. These questions which states that if we have multiple inheritance in our design, we should assume we are making a mistake and should prove otherwise. In this example, our wooden door is made completely out of wood, do not worry about steel hinges and brass door knobs.
\begin{itemize}
\item Is a wooden door a special type of door ? yes
\item Is a door part of a wooden door ? No
\item Is a wooden door a special type of wooden object ? Yes
\item Is a wooden object part of a door ? No
\item Is a wooden object a special type of door ? No
\item Is a door a special type of wooden object ? No
\end{itemize}

Since this design satisfies all of our heuristics, it is considered valid multiple inheritance.

