\section{Explain, in your own words, what problem context-oriented programming tries to solve.}

\begin{solution}

\todo[inline]{
\textbf{Cette question est un doublon avec la 85, en 2016-2017 elle est annulée.}

Answer for context-oriented:\newline

Context-oriented programming is an approach to adapt the behavior of software
systems dynamically according to detect context changes in their surrounding
environment. As the technology evolves, devices can collect more and more data
(through sensors), as CPU load, time, internet connectivity,\ldots this is even
more important with smartphones and tablets with accelerometer, geographical
location,\ldots\newline

Systems are not anymore a black box which you give an input to, and receive an
output from: their behavior must adapt if you want to give the user a better
experience (e.g. if the CPU load is already high and/or the battery is already
low, the application can choose to send a CPU-intensive task to a cloud service
instead of running it locally; switch to a vibrator mode for incoming calls if
you are in a quiet place;\ldots).\newline

You cannot accomplish this with old paradigms in a maintainable and reusable
way, it would result in a lot of conditional statements and strategy design
patterns. That's why you need the context-oriented paradigm that will reify
real-life contexts into Context objects and context-oriented systems have
contexts as first-class citizen. A context object contains specific pieces of
software to change the behavior of the program at runtime, they can be activated
(it will take the place of the usual behavior) and deactivated (to let the
previous behavior come back), more complex relations with multiple activation or
multiple contexts exist to provide more complex adaptability to the environment.\newline

Answer for aspect-oriented:\newline

Aspect-oriented programming is an approach to find a reusable and maintainable
way to design object-oriented software even when crosscutting concerns exist.
When you build a software, you focus on the main problems it has to solve and
you try to have a good modularity for those concerns. But often, programmers
have to add code (sometimes the same portion of code) to solve subproblems (memory management, logging, data
validation, languages management,\ldots) inside the code for the main problems,
this is called \textit{scattering} and \textit{tangling}.
Those subproblems are called \textit{crosscutting concerns}, and aspect-oriented
programming provides a way to have them as entities, called \textit{aspects}.
It allows the programmers to put all code related to them inside one place,
such that it becomes more maintainable and reusable.
}

\end{solution}

\section{Explain, in your own words, what a crosscutting concern is, and illustrate it with a concrete example.}

\begin{solution}

A cross-cutting concern is a concern (i.e. something the programmer should care
about) that is spread across different modules. It has a clear purpose
(a \enquote{What?}) and some regular interaction points (a \enquote{Where?})
but as it is not a core functionality of the system, the modularity is often bad
for this concern. An example of that is Logging\footnote{print des infos, des
warnings, des errors, etc.} in the Apache Server Library: in that kind of
software, logging is a subproblem and it is spread across almost all classes
(\textit{scattering}): if you have to maintain or understand the logging it will
be really hard.

\end{solution}

\section{Explain what the tyranny of the dominant decomposition means, and discuss its relation
with aspect-oriented programming.}

\begin{solution}
The tyranny of dominant decomposition means that when you choose one of the
many possible decompositions of the problem to build your software, you will
favor the core functional concerns and it will result in the impossibility for
some subproblems/concerns to be modularised. More concretely, it means that it
is hard/impossible to have a perfect modularity, so when it comes to the choice
of what will be modular, you will prefer the main functionalities of your
software and the price for that is: you have to accept that some other (minor)
functionalities will not be modular.

Tyranny of dominant decomposition is a crosscuting-concern because some concerns cannot be modularised. Aspect-Oriented Programming deals explicitly with crosscutting concerns.
\end{solution}

\section{Explain the notions of tangling and scattering, and illustrate them with a concrete example.
What are the problems with having tangled and scattered code?}
\label{sec:scattering_tangling}

\begin{solution}

\begin{description}
\item[Scattering]: code addressing one concern is spread throughout the entire program (code duplication)
\item[Tangling]: code in one region addresses multiple concerns (significant dependency between systems)
\end{description}

Example :

\begin{center}
\begin{tikzpicture}

% FigureElement class
\umlclass[x=2.5,y=6]{FigureElement}{
}{
moveBy(int,int)\\
refresh()
}

% Point class
\umlclass[x=0,y=0]{Point}{}{
getX()\\
getY()\\
\textcolor{orange!80!black}{setX(int)}\\
\textcolor{orange!80!black}{setY(int)}\\
\textcolor{orange!80!black}{moveBy(int,int)}\\
draw()
}

% Line class
\umlclass[x=5,y=0]{Line}{}{
getP1()\\
getP2()\\
\textcolor{orange!80!black}{setP1(Point)}\\
\textcolor{orange!80!black}{setP2(Point)}\\
\textcolor{orange!80!black}{moveBy(int,int)}\\
draw()
}

% Relations between classes
\umlinherit[geometry=|-|]{Point}{FigureElement}
\umlinherit[geometry=|-|]{Line}{FigureElement}
\umluniassoc[mult2=2]{Line}{Point}

\end{tikzpicture}
\end{center}

All \textcolor{orange!80!black}{methods in orange} end with a call to \verb#Display.update()#. This is scattering (because of the repetition of the call) and tangling (because each method will do what its name tells BUT also an update of the display). \newline

The main problems with this kind of code is:

\begin{itemize}
\item Redundant code
\item Difficult to reason about
\item Functionality is difficult to change (you have to find all the code involved to see what the modification implies)
\end{itemize}

\end{solution}

\section{Explain, in your own words, what an aspect weaver is and how aspect-oriented
programming works.}

\begin{solution}

Aspect-oriented programming is based on a main idea: describe crosscutting concerns as separate, independent entities called \textit{aspects} (modularisation of crosscutting concerns). The base application contains the code of core functionality and \textit{pointcuts} at the places where crosscutting concerns should appear. At compile time, we need to combine the base program and the aspects: code inside the aspects will be \textit{physically woven} into the classes of the base application, this is what \textit{weavers} do. This process is called \textit{weaving} (combination of aspects and classes).

\end{solution}

\section{Explain, in your own words, the following concepts from aspect-oriented programming:
base program, aspect, join point, pointcut and advice. Illustrate with a concrete example.}

\begin{solution}

\begin{description}
\item[Base program] regroups the core functionalities of the object-oriented program, it corresponds to all the classes/concerns that are meaningful for its domain
\item[Aspect] is the modularisation of a crosscutting concern, it corresponds to the representation of a subproblem which was spread across the system
\item[Join point] is a particular point in the base program where an aspect can be woven
\item[Pointcut] is a description of when to execute the join point (a boolean evaluation of some conditions)
\item[Advice] is a mean to specify code to run at a join point (and if it should be executed before or after)
\end{description}

\begin{lstlisting}
pointcut set() : execution(* set*(*)) && (this(Point) || this(Line));

after() : set() {
   Display.update();
}
\end{lstlisting}

The pointcut says \enquote{execute the \textit{advice} set() if there is an
execution of a method whose name starts with \textit{set} and the object is an
instance of Point/Line}. The advice says \enquote{after the pointcut matches (if
the evaluation of join-points returns true), execute the code
\texttt{Display.update()}}. This code is in AspectJ, an other way to write it to
be more complete w.r.t to the example in Question~\ref{sec:scattering_tangling}
is:

\begin{lstlisting}
after():
call(void FigureElement+.set*(..)) ||
call(void FigureElement.moveBy(int,int))
{
 Display.update();
}
\end{lstlisting}

\end{solution}
