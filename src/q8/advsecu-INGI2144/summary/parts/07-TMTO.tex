
\begin{itemize}
    \item TMTO is never better than a brute force
    \item It make sense for (1) repeated attack, (2) lunchtime attack or
        (3) attacker not powerful but can download
        table
        \end{itemize}

\subsection{One-way Function} 

\begin{center}
    Function $h: A \rightarrow B$: 
    \begin{tabular}{l}
        easy to compute \\
        but hard to invert.
    \end{tabular}
\end{center}

One-way function are used to store password. To break password, an exhaustive
search can be used in two different ways.
\begin{table}[ht!]
    \centering
    \begin{tabular}{|c|c|c|}
        \hline
        & \textbf{Online exhaustive search} & \textbf{Precalculated
    exhaustive search} \\
        \hline
        Computation & $ |A|:=N $ & 0 \\
        Storage & 0 & N \\
        Precalculation & 0 & N \\
        \hline
    \end{tabular}
\end{table}

\subsection{Hellman's TMTO}
Hellman table is used to pre-calculation phase to speed up the online attack.

\subsubsection{Pre-calculation}
%A number $M$ of chains is generated from a arbitrary value in A\@:

\begin{eqnarray*}
    \textrm{Hash function } \quad & h &: A \to B \\
    \textrm{Reduction function } \quad & R &: B \to A 
\end{eqnarray*}

\begin{tabular}{m{6cm}m{8cm}}
    \begin{tikzpicture}[node distance=0.6cm]
        \node[draw, circle] (A1) {$S_1$};
        \node[draw, circle] (A2) [below=of A1] {};
        \node[draw, circle] (A3) [below=of A2] {};
        \node[draw, circle] (A4) [below=of A3] {$E_1$};

        \node[draw, circle] (B1) [below right=0.3cm and 3cm of A1] {};
        \node[draw, circle] (B2) [below=of B1] {};
        \node[draw, circle] (B3) [below=of B2] {};

        \node [draw, rectangle, rounded corners=7pt, fit={(A1) (A2) (A3) (A4)}] (F) {};
        \node [draw, rectangle, rounded corners=7pt, fit={(B1) (B2) (B3) }] (FF) {};

        \node [left=0.0cm of F] {\large{A}};
        \node [right=0cm of FF] {\large{B}};

        \draw[->] (A1) edge node[above] {h} (B1);
        \draw[->] (B1) edge node[above] {R} (A2);

        \path[->] (A2) edge (B2)
        (B2) edge (A3)
        (A3) edge (B3)
        (B3) edge (A4);

    \end{tikzpicture}
    &
    \begin{tikzpicture}[node distance=0.6cm]
        \node[draw, circle] (A1) {$S_1$};
        \node[draw, circle] (A2) [right=of A1] {};
        \node[draw, circle] (A3) [right=of A2] {};
        \node[draw, circle] (A4) [right=of A3] {};

        \node[draw, circle] (A5) [right=of A4] {};
        \node[draw, circle] (A6) [right=of A5] {};
        \node[draw, circle] (A7) [right=of A6] {$E_1$};

        \path[bend left, ->] (A1) edge node[above] {h} (A2)
        (A2) edge node[above] {R} (A3)
        (A3) edge node[above] {h} (A4)

        (A5) edge node[above] {h} (A6)
        (A6) edge node[above] {R} (A7);

        \node[draw, circle] (B1) [below=of A1] {$S_2$};
        \node[draw, circle] (B2) [right=of B1] {};
        \node[draw, circle] (B3) [right=of B2] {};
        \node[draw, circle] (B4) [right=of B3] {};

        \node[draw, circle] (B5) [right=of B4] {};
        \node[draw, circle] (B6) [right=of B5] {};
        \node[draw, circle] (B7) [right=of B6] {$E_2$};

        \path[bend left, ->] (B1) edge node[above] {h} (B2)
        (B2) edge node[above] {R} (B3)
        (B3) edge node[above] {h} (B4)

        (B5) edge node[above] {h} (B6)
        (B6) edge node[above] {R} (B7);

        \draw[bend left, dotted, ->] (B4) edge (B5);
        \draw[bend left, dotted, ->] (A4) edge (A5);

    \end{tikzpicture}
    \\
\end{tabular}

Pre-calculation phase to speed up the online attack:
$T \alpha \frac{N^2}{M^2} \quad \begin{cases}
    \textrm{T=time complexity}\\
    \textrm{N=size of problem}\\
    \textrm{M=size of memory}
    \end{cases}$\\
Lenght of the chains is decided and chains can be stopped in either A or B.
$\sqrt{N}$ tables are computed.

\paragraph{Coverage and collisions}

During the generation of the chains, chains may collide between each other, to
avoid that, several table with different reduction function must be used.

\subsubsection{Online attack}

\begin{tabular}{m{5cm}m{5cm}m{5cm}}
    \begin{tikzpicture}[node distance=0.6cm]
        \node[draw, circle] (A1) {$S_1$};
        \node[draw, circle] (A2) [below=of A1] {};
        \node[draw, circle] (A3) [below=of A2] {};
        \node[draw, circle] (A4) [below=of A3] {$E_1$};

        \node[draw, circle] (B1) [below right=0.3cm and 3cm of A1] {};
        \node[draw, circle, fill=blue] (B2) [below=of B1] {};
        \node[draw, circle] (B3) [below=of B2] {};

        \node [draw, rectangle, rounded corners=7pt, fit={(A1) (A2) (A3) (A4)}] (F) {};
        \node [draw, rectangle, rounded corners=7pt, fit={(B1) (B2) (B3) }] (FF) {};

        \draw[->] (A1) edge node[above] {h} (B1);
        \draw[->] (B1) edge node[above] {R} (A2);

        \node [above=0.0cm of F] {\large{A}};
        \node [above=0cm of FF] {\large{B}};

        \path[->] (A2) edge (B2)
        (B2) edge (A3)
        (A3) edge (B3)
        (B3) edge (A4);

    \end{tikzpicture}
    &
    \begin{tikzpicture}[node distance=0.6cm]
        \node[draw, circle] (A1) {$S_1$};
        \node[draw, circle] (A2) [below=of A1] {};
        \node[draw, circle, fill=blue] (A3) [below=of A2] {};
        \node[draw, circle, blue] (A4) [below=of A3] {$E_1$};

        \node[draw, circle] (B1) [below right=0.3cm and 3cm of A1] {};
        \node[draw, circle, fill=blue] (B2) [below=of B1] {};
        \node[draw, circle, fill=blue] (B3) [below=of B2] {};

        \node [draw, rectangle, rounded corners=7pt, fit={(A1) (A2) (A3) (A4)}] (F) {};
        \node [draw, rectangle, rounded corners=7pt, fit={(B1) (B2) (B3) }] (FF) {};

        \node [above=0.0cm of F] {\large{A}};
        \node [above=0cm of FF] {\large{B}};

        \draw[->] (A1) edge node[above] {h} (B1);
        \draw[->] (B1) edge node[above] {R} (A2);

        \path[->] (A2) edge (B2)
        (B2) edge[blue] (A3)
        (A3) edge[blue] (B3)
        (B3) edge[blue] (A4);

    \end{tikzpicture}
    &
    \begin{tikzpicture}[node distance=0.6cm]
        \node[draw, circle, red] (A1) {$S_1$};
        \node[draw, circle, fill=red] (A2) [below=of A1] {};
        \node[draw, circle, fill=blue] (A3) [below=of A2] {};
        \node[draw, circle, blue] (A4) [below=of A3] {$E_1$};

        \node[draw, circle, fill=red] (B1) [below right=0.3cm and 3cm of A1] {};
        \node[draw, circle, fill=blue] (B2) [below=of B1] {};
        \node[draw, circle, fill=blue] (B3) [below=of B2] {};

        \node [draw, rectangle, rounded corners=7pt, fit={(A1) (A2) (A3) (A4)}] (F) {};
        \node [draw, rectangle, rounded corners=7pt, fit={(B1) (B2) (B3) }] (FF) {};

        \draw[->] (A1) edge[red] node[above] {h} (B1);
        \draw[->] (B1) edge[red] node[above] {R} (A2);

        \node [above=0.0cm of F] {\large{A}};
        \node [above=0cm of FF] {\large{B}};

        \path[->] (A2) edge[red] (B2)
        (B2) edge[blue] (A3)
        (A3) edge[blue] (B3)
        (B3) edge[blue] (A4);

    \end{tikzpicture}
\end{tabular}

\paragraph{Idea}
\begin{enumerate}
    \item Take the value $h$ in B and use the reduction function to get in A
    \item Check if the value contained in the last column of the table match.
        \begin{itemize}
            \item If they don't match, use another reduction (table) function and redo the verification.
            \item If the verification fails, redo with another reduction function and so on. 
        \end{itemize}
    \item[$\Rightarrow$] If no match is found then the attack fails.

    \item When there is a match, redo the chains from the beginning and stop when the
current value is equal to $h$. The password is the value just before $h$
\end{enumerate}

\paragraph{False alarms}
Given one output $C \in B$, we compute $Y_1 = R(C)$ and generate a
chain: $Y_1 \rightarrow ... \rightarrow Y_s $
\begin{center}
\begin{tikzpicture}[node distance=0.6cm]
        \node[draw, circle, red] (A1) {$S_j$};
        \node[draw, circle, red] (A2) [right=of A1] {};
        \node[draw, circle, red] (A2b) [right=of A2] {};
        \node[draw, circle, red] (A3) [right=0.3cm of A2b] {$C'$};
        \node[draw, circle] (A4) [right=0.3cm of A3] {};
        \node[draw, circle] (A5) [right=of A4] {};
        \node[draw, circle] (A6) [right=of A5] {};
        \node[draw, circle, blue] (A7) [right=of A6] {};
        \node[draw, circle, blue] (A8) [right=of A7] {};
        \node[draw, circle, blue] (A9) [right=of A8] {$E_j$};

        \node[draw, circle, blue] (B1) [below=2cm of A3] {$C$};
        \node[draw, circle, blue] (B2) [right=0.3cm of B1] {};
        \node[draw, circle, blue] (B3) [below =1.3cm of A5] {};
        \node[draw, circle, blue] (B4) [below =0.4cm of A6] {};

        \node[below=3cm of A1] (tmp) {};
        \node[below=3.25cm of A4] (tmp1) {};
        \node[below=3cm of A9] (tmp2) {};
        \node[above=0.5cm of A4] (tmp3) {};

        \scriptsize
        \draw[<->] (tmp) edge[red] node[below] {time to detect false alarm}(tmp1);
        \draw[<->] (tmp1) edge[blue] node[below] {time to find matching end
        point} (tmp2);

        \path (tmp3) edge[dashed] (A4)
         (A4) edge[dashed] (B2)
         (B2) edge[dashed] (tmp1.center);

        \path[-] (A1) edge[red] (A2)
        (A2) edge[red] (A2b)
        (A2b)edge[red] (A3)
        (A3) edge (A4)
        (A4) edge (A5)
        (A5) edge (A6)
        (A6) edge (A7)
        (A7) edge[blue] (A8)
        (A8) edge[blue] (A9)
        (B1) edge[blue] (B2)
        (B2) edge[blue] (B3)
        (B3) edge[blue] (B4)
        (B4) edge[blue] (A7);
\end{tikzpicture}
\end{center}

A false alarm occur when there is collisions during the online phase

\subsection{Oechslin's Rainbow Tables}

\begin{eqnarray*}
    \textrm{Hash function } \quad & h &: A \to B \\
    \textrm{Arbitraty reduction function } \quad & R_i &: B \to A 
\end{eqnarray*}

\subsubsection{Advantage}
The main difference compared to
Hellman table is that it uses different reduction function per column such that:
\begin{itemize}
    \item If 2 chains collide in different columns, they don't merge.
    \item If 2 chains collide in same column, merge can be detected.
\end{itemize}

\subsubsection{Online attack}
The online procedure is a bit more complex, because we need to take in
account the order of reduction function (location in the chains).
Given one output $C \in B$, we compute $Y_1 = R(C)$ and generate a
chain: 
\begin{center}
    $Y_1$ \frr{$f_{t-s}$} $Y_2$ \frr{$f_{t-s+1}$} ... \frr{$f_t$} $Y_s $
\end{center}


\subsubsection{The success probability of a table is bounded}
\begin{itemize}
    \item
        Given $t$ and a sufficiently large $N$, the expected maximum number of
        chains per perfect rainbow table without merge is:
        $$ m_{\max}(t)\approx\frac{2N}{t+1} $$
    \item
        Given $t$, for any problem of size $N$, the expected maximum probability of
        success of a single perfect rainbow table is:
        $$ P_{\max}(t)\approx 1 - (1-\frac{2}{t+1})^t $$
        which tends toward $ 1 - e^{-2}\approx 86\% $ when $t$ is large.

        %%TODO have we seen that Average cryptanalysis time ?
\end{itemize}

\subsection{Example}
Cracking a 7-char (max) alphanumerical password (NT LM Hash)
on a PC. Size of the problem: $N = 2^{41.7}$.

\begin{center}
\begin{tabular}{|c|c|c|}
    \hline
    & \textbf{Brute force} & \textbf{TMTO} \\
    \hline
    Online attack & $1.78 \times 10^{12}$ &  $4.48 \times 10^7$\\
    Time & 99h & 9s\\
    \hline
    Pre-calculation & 0 & $6.29 \times 10^{14}$\\
    Time & 0 & 1458days \\
    Storage & 0 & 16GB \\
    \hline
\end{tabular}
\end{center}

\subsection{Interleaved TMTOs}
A TMTO treats all possible preimages equally which is not necessarily
true because we can have a \textbf{non-uniform distribution}. For
example, some type passwords are more used than others
(ex:alphanumeric vs special character).

\subsubsection{Concepts}
The input space is partitioned (for example a partition = one type of password).
The idea will be :
\begin{itemize}
	\item To build a TMTO for each partition.
	\item Conduct the search in a interleaved fashion between TMTO.
\end{itemize}

The interleaving order (which represent the memory division between
sub-TMTO's) is computed such that it minimizes average time.

