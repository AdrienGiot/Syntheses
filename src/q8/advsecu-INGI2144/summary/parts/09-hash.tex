

\subsection{Usage}
\begin{tabular}{m{0.5\linewidth}m{0.5\linewidth}}
    \begin{itemize} 
        \item Generate keys and subkeys
        \item Reduce the size of a message to be signed.
        \item Ensure authenticity of a message, when keyed.
    \end{itemize}
    &
    \begin{itemize} 
        \item Password-based authentication.
        \item Check decryption is correct.
        \item Provide a commitment on an information.
    \end{itemize}
\end{tabular}

\subsection{Input size}
\begin{itemize}
    \item In theory, the size is \textbf{unlimited} 
    \item but in practice it is limited
        because the length of the message is concatenated at the end of it and
        follows a predefined format!

    \item[$\Rightarrow$] Internal function with small input size sequentially iterated on
        blocks of messages.
\end{itemize}

\subsection{An insider view}
\begin{center}
    \begin{tikzpicture}
        \node (input){arbitrary length input};
        \node [shape=rectangle, draw, below =of input](preprocessing){data preprocessing};
        \node [shape=rectangle, draw, below =of preprocessing](compression)
        {compression function};
        \node [shape=rectangle, draw, below =of compression](transformation)
        {optional transformation};
        \node [below =of transformation](output){fixed length output};

        \draw [->] (input) -- (preprocessing);
        \draw [->] (preprocessing) -- (compression);
        \draw [->] (compression) edge node (A) {} (transformation);
        \draw [->] (transformation) -- (output);

        \draw[->, red] (A.center) -| (-2.5cm, -2.3cm) -|
        (compression.155);

        \node[red] [left= of compression] {IV};

    \end{tikzpicture}
\end{center}

\subsubsection{Example preprocessing for SHA-1}
\paragraph{Padding}
\begin{itemize}
	\item Suppose that the length of a message m is $l$ bits
	\item Append the bit ''1'' to the end of the message, followed by $s$ zero-bits,
	where $s$ is the smallest non-negative solution to the equation $l + s + 1 = 448 \mod 512$
	\item Append the 64-bit block equal to $l$ in binary representation.
\end{itemize}
For example with a message with 24 bit ($l$)
\begin{center}
\begin{tikzpicture}
	\node[draw,shape=rectangle] at (0,0) (l) {0111 ... 01010011};
	\node[below = 0.1cm of l] (msg) {msg of length $l$ = 24bits}; 
	\node[draw,shape=rectangle,right = of l] (rl) {1};
	\node[draw,shape=rectangle,right = of rl] (padding) {00..00};
	\node[below = 0.1cm of padding] (text) {423 zeros};
	\node[draw,shape=rectangle,right = of padding] (append) {00...0111..01010011};
	\node[below = 0.1cm of append] (atext) {$l$ on 64bits};
\end{tikzpicture}
\end{center}


\subsection{Conclusion}
\begin{center}
    \begin{tabular}{|c|c|c|c|}
        \hline
        Primitive   & Output Length (bits)   & Legacy & Future \\
        \hline
        SHA-2       & 224/256/384/512 & \textcolor{green!50!black}{v}
        & \textcolor{green!50!black}{v} \\
        SHA-3       & TBA             & \textcolor{green!50!black}{v}
        & \textcolor{green!50!black}{v} \\
        Whirlpool   & 512             & \textcolor{green!50!black}{v}
        & \textcolor{green!50!black}{v} \\
        \hline
        RIPEMD-160  & 160             & \textcolor{green!50!black}{v}&
        \textcolor{green!50!black}{v}\\
        SHA-1       & 160             & \textcolor{green!50!black}{v}&
        \textcolor{red!50!black}{x} \\
        \hline
        MD-5        & 128             & \textcolor{red!50!black}{x}&
        \textcolor{red!50!black}{x}\\
        RIPEDMD-128 & 128             & \textcolor{red!50!black}{x}&
        \textcolor{red!50!black}{x}\\
        \hline
    \end{tabular}
\end{center}


