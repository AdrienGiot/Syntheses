\clearpage{}

\section{Explain the goals and principles of postmortem analysis. Describe
the Capability Maturity Model and its five levels. Compare to SPICE and ISO
9000.}

\subsection{Postmortem analysis}

A postmortem analysis is a post-implementation (12 month max) assessment
(based on a questionnaire and requests for evidence to verify answers) of all aspects of the
project (products, processes and resources). It is intended to determine whether
goals were met or not, and to identify areas of improvement for future projects.

\begin{itemize}
        \item Analyze what went wrong for future improvement 
    \end{itemize}

\paragraph{Postmortem analysis process}

\begin{itemize}
    \item \textbf{Project survey} (subjective) : Ask no more than needed, without compromising confidentiality.
    \item \textbf{Objective information} (Objective): Collect relevant data: cost (effort, LOC), schedule, quality.
    \item \textbf{Debriefing meeting} (Group): Allow team members to report problems.
    \item \textbf{Project history day} (Summarize for future) : Identify root causes of key problems, review schedule predictability charts.
    \item \textbf{Publishing the results} (Summarize for now) : Focus on lessons learned.
\end{itemize}

\subsection{Capability Maturity Model}

\begin{itemize}
    \item It's a development model which aims to improve existing
        processes. (process measurements)
\end{itemize}

5 levels of maturity, each with a set of key process area and refers to
the degree of formality (to examine software process as a whole).

Note that these are not 5 discrete rankings, but a
continuous scale.

\begin{figure}[!ht]
    \centering
    \begin{scriptsize}
        \begin{tikzpicture}[node distance=3cm, on grid, auto]
    \node[draw, rectangle] (l1) {Level 1: Initial};
    \node[draw, rectangle, above right =of l1] (l2) {Level 2: Repeatable};
    \node[draw, rectangle, above right =of l2] (l3) {Level 3: Defined};
    \node[draw, rectangle, above right =of l3] (l4) {Level 4: Managed};
    \node[draw, rectangle, above right =of l4] (l5) {Level 5: Optimizing};

    \path[->, line width=2pt] (l1) edge [bend left=45] node {Process discipline} (l2);
    \path[->, line width=2pt] (l1) edge [bend right=45, right] node {Process management} (l2);
    \path[->, line width=2pt] (l2) edge [bend left=45] node {Process definition} (l3);
    \path[->, line width=2pt] (l2) edge [bend right=45, right] node {Engineering management} (l3);
    \path[->, line width=2pt] (l3) edge [bend left=45] node {Process control} (l4);
    \path[->, line width=2pt] (l3) edge [bend right=45, right] node {Quantitative management} (l4);
    \path[->, line width=2pt] (l4) edge [bend left=45] node {Continuous process improvement} (l5);
    \path[->, line width=2pt] (l4) edge [bend right=45, right] node  {Change management} (l5);
\end{tikzpicture}

    \end{scriptsize}
    \caption{Capability maturity model}
\end{figure}

\begin{tabular}{l|m{0.4\textwidth}cm{0.2\textwidth}}
    \bf Level & \bf Description & \bf Measurements & \bf Key process areas \\
    \hline
    1: Initial & Cannot describe development process (ad hoc or chaotic) && None \\
    2: Repeatable & Identified inputs and outputs, constraints (budget, schedule), resources
    (Project management in place) & On project & management activities\\
    3: Defined & Management and engineering activities are documented,
    standardized and integrated & On products & organization\\
    4: Managed & Early projects feedback to later projects & On quality
    & Quantitative and qualitative management \\
    5: Optimizing & Quantitative feedback is incorporated to produce
    continuous process improvement && change management\\
\end{tabular}

\subsection{SPICE (ISO Standard 15504)}

This is an international standard for process assessment. SPICE stands for
Software Process Improvement and Capability dEtermination. It harmonizes
and extends the existing process assessment methods (CMM and its
descendants).

\paragraph{Levels of capability of each process area:}
\begin{enumerate}
    \item \textbf{Not performed}, process is not applied
     and does not achieve its goals.
    \item \textbf{Performed informally}, process is applied and achieve its
    goal.
    \item \textbf{Planned and tracked}, process appliance is planned and
    kept in check. Work product are established and registered.
    \item \textbf{Well-defined}, process is documented so that it ensure its
    capacity to achieve its goals.
    \item \textbf{Quantitatively controlled} process work according to
    goals performance defined.
    \item \textbf{Continuously improved}
\end{enumerate}

Whereas the CMM assesses organization, SPICE assesses processes.

\subsection{ISO 9000}

ISO 9000 is a series of standards that specify actions to be taken when any
system (i.e.\ not necessarily a software system) has quality goals and
constraints. In particular, ISO 9000 applies when a buyer requires a supplier
to demonstrate a given level of expertise in designing and building a product.

\begin{itemize}
    \item \textbf{ISO 9000:} quality management systems and principles.

        (regulate internal quality and to ensure quality of suppliers)

    \item \textbf{ISO 9001:} requirements that organizations must fulfil.

        (It explains what a buyer must do to ensure that the supplier conforms to
        design, development, production, installation and maintenance requirements.)

    \item \textbf{ISO 9000--3:} guidelines for applying ISO 9001 to
        software context
\end{itemize}

