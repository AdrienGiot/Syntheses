\clearpage{}
\section{Define reliability, maintainability and availability, and their
expression as MTBF and MTTR.\@ Discuss the case of a constant failure rate.
Explain the principles of failure prediction models. Explain and discuss
the principles of statistical testing.}

\subsection{RAM attribute (reliability, availability, maintainability)}

\begin{description}

    \item[Reliability] is the probability that a system will perform its
        intended function \textbf{without failure} for a given \textbf{time interval}. 
    $$R(t) = Prob(\text{no failure before t})$$

    \item[Maintainability] is the probability that maintenance of the
        system will retain the system in, or restore it to, a specified
        condition within a given time period. 
        $$M(t) = Prob(\text{restored before t})$$

    \item[Availability] is the probability that the system is operating
        satisfactorily at any time, and it depends on the reliability
        and the maintainability.  
        $$A = Prob(\text{not failed})$$
\end{description}

Hence the study of probability theory is essential for understanding the
reliability, maintainability, and availability of the system.

\begin{itemize}

    \item \textbf{Mean time between failures (MTBF)}: Average time
        between successive failures. Measures reliability.

    \item \textbf{Mean time to failure (MTTF): Idem for non-repairable
        items}

    \item \textbf{Mean time to repair (MTTR)}: Average time to fix a fault.
        Measures maintainability. 

    \item[$\Rightarrow$] Availability: $$A = MTBF / (MTBF + MTTR)$$
        $$MTBF = MTTF + MTTR$$
\end{itemize}

\subsection{Constant failure rate}
The system will fail at constant probability rate of once per hour. 
\begin{align*}
    MTBF &= 3600 s& \\
    \text{Density function}: f(t) &= 1/3600 * \exp( -t / 3600) &(\text{exponential law}) \\
    \text{Distribution function}: F(t) &= 1 - \exp(-t / 3600)& \\
    \text{Reliability}: R (t) &= 1 - F(t) = \exp(-t / 3600)& \\
    \text{Reliability}: R(MTBF) &= \exp(-1) = 37\%& \\
\end{align*}

\subsection{Failure prediction models}
Goal: Given a series of failure times ${t}_{1}, {t}_{2},\ldots, {t}_{n-1}$ \newline
Predict the distribution of future failure times ${t}_{n}, {t}_{n+1},\ldots$ \newline

A good prediction model must include:

\begin{itemize}
    \item \textbf{Prediction model}: probability specification of the stochastic process.
    \item \textbf{Inference procedure}: infer unknown parameter from ${t}_{1}\ldots{t}_{n-1}$.
    \item \textbf{Prediction procedures}: combine model and inference procedure to make predictions about future failure behaviour.
\end{itemize}

\subsection{Principles of statistical testing}

Reliability prediction based on failures occurring during testing \newline
Will this be accurate for typical system usage?

\begin{description}
    \item[Operational profile] probability distribution on inputs (reflecting usage)
    \item[Statistical testing] select tests according to operational profile
        \subitem{} Tests focus on more used parts ($\Rightarrow$ better observable reliability)
        \subitem{} Test reflect usage ($\Rightarrow$ reliability predictions more accurate)
\end{description}

But operational profiles are difficult to define.\newline
Mislead, a small \% of operational profile may account for a large \% of failures.
