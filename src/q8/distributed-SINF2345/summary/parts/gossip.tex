\section{Gossip (epidemic algorithms)}

Gossip is the process by which an information will spread among
entities. Important technique to solve problem in dynamic large scale system
(scalable, simple, robust)

\begin{lstlisting}
push // telling something
pull // asking something
\end{lstlisting}

\subsection{Protocol Characterisitc}
\begin{itemize}
	\item Cyclic/periodic, pair-wise interaction between peers
	\item The amount of information exchanged is of (small)
	bounded size per cycle
	\item The state of each peer is bounded (small)
	\item During interaction between two peers, the state of one changes to
	reflect the state of the other.
	\item Selection of peer is random (full peer set or small set of neighbors)
	\item Reliable communication is not assumed
	\item Protocol cost is negligible
\end{itemize}

\subsection{Protocol Usage}
\begin{itemize}
	\item \textbf{Dissemenitation}: Spread information in a manner that produces
	bounded worst-case loads.
	\item \textbf{Repairing}: Anti-entropy protocols for repairing replicated data,
	which operate by comparing replicas and reconciling differences.
	\item \textbf{Membership}: pick a stable node from a given list then track
    processes \enquote{I've heard from recently} and $push$ them.
	\item \textbf{Aggregates}: Compute a network-wide aggregate. (e.g.\ number of nodes or
	computing average/max/min)
	\item \textbf{Network topology}: Protocols that arrange network topology
    (e.g.\ T-man algorithm for rings)
\end{itemize}

\subsection{Information dissemination}

\begin{itemize}
\item Start with one peer that wants to disseminate some message.
\item Then every peer does the following:

\begin{enumerate}
	\item Buffers every message (information unit) it receives up to a
	certain buffer capacity $b$
	\item Forwards that message a limited number of hops or time steps $t$
	\item Forwards the message each time to $f$ randomly selected set of
	processes (fan-out)
\end{enumerate}
\end{itemize}

\subsubsection{Infect-forever Model}
\begin{itemize}
	\item Fixed population of size $n$ (at round 1, one is infected)
	\item If infected, remains infected forever
	\item $Y_r$ is the number of individuals infected at round $r$
	\item $f$ is the number of individuals that infected ones try to infect
    \item $R$ is the number of round to infect all population
\end{itemize}

\begin{eqnarray*}
Y_r &\approx& \frac{1}{1+n \times e^{-f \times r}}\\
R &=& \log_{f+1}(n)+\frac{1}{f}\log(n)+O(1)\\
\end{eqnarray*}

\subsubsection{Infect-and-die model}

\begin{itemize}
\item Infectious process \enquote{remains infectious} for just one round
\item $\pi$ is the proportion of processes eventually contaminated
\item $R$ the number of round to infect entire system:
\end{itemize}

\begin{eqnarray*}
\pi &=& 1-e^{-\pi \times f}\\
R &=& \frac{\log(n)}{\log(\log(n))}+O(1)\\
\end{eqnarray*}

\subsubsection{Membership}

To ensure scalability, each process has a partial view (random sample of
node).

\begin{itemize}
	\item When a process forwards a message, it
	includes in this message a set of processes it knows.
	\item Hence, the process that receives the message can enhance the
	list of processes it knows by adding new processes.
\end{itemize}

\paragraph{Protocols requirement}

\begin{itemize}
\item \textbf{Uniformity}: all nodes play the same role
\item \textbf{Adaptivity under churn}: the parameters have to be tuned ($t$ and $f$)
\item \textbf{Bootstrapping}: how do nodes enter and leave, how to start
\end{itemize}


\subsubsection{Buffer Management}

Depending on broadcast rate, buffer capacity of processes may be
insufficient to ensure that every message is buffered long enough.

\paragraph{To counter that}
message are classified according to their
age (the number of processes the message went through) $\to$
Old messages are replaced.

\subsection{Small-world network}

There are different ways for a process to choose its infection target:

\begin{description}
	\item[Nearest-neighbor network] Targets are only neighbors (number of
	round is $\bigoh(n^{1/D})$ for a D-dimensional grid)
	\item[Random network] Target is random (number of rounds is
	$\bigoh(\log(n))$)
	\item[Small-world network] In between case (number of rounds
	is $\bigoh(\log(n))$)
\end{description}

\paragraph{Note:} Real world social network tends to be small-world networks.


\subsubsection{Properties}

Small-world network has both nearest neighbor connections as well as
long range connections (Node reachable with a small number of hops).

\paragraph{Properties}
\begin{itemize}
	\item Small average shortest path length (opposed to large path length
	in Neighbor graphs)
	\item A high clustering coefficient (opposed to low CC in Random graphs)
	\begin{description}
	\item[Clustering c\oe{}fficient] between 0 and 1 = Average of

        $$\forall_{p_i \in nodes} \frac{\sum { edge | \quad edge.start \in
            p_i.neighbors \quad AND \quad edge.end \in p_i.neighbors }}
            {p_i.nbrNeighbors (p_i.nbrNeighbors-1) /2)}$$

\end{description}
\end{itemize}

The clustering c\oe{}fficient (CC) measures degree of clustering. Count the
number of edges between neighboring nodes and divide by the maximum
possible for every node. The average of this number is the CC.

\subsection{Gossip Framework (Jelasity-Babaoglu)}
\begin{tabular}{m{0.5\linewidth}m{0.5\linewidth}}
\begin{lstlisting}[caption={Active thread}, mathescape]
do forever
    wait(T time units)
    q = SelectPeer()
    push S to q
    pull $S_q$ from q
    S = Update(S,$S_q$)
\end{lstlisting}
&
\begin{lstlisting}[caption={Passive thread}, mathescape]
do forever
    (p,$S_p$) = pull * from *
    push S to p
    S = Update(S,$S_p$)
\end{lstlisting}
\end{tabular}

To instantiate the framework, define
\begin{itemize}
	\item Local state $S$,
	\item Method \texttt{SelectPeer()} and \texttt{Update()}
	\item Style of interaction : \texttt{push, pull, push-pull}
\end{itemize}

\subsubsection{Aggregation}
The style of interaction is push-pull
\begin{itemize}
	\item $S$ is the current estimate of global aggregate
	\item \texttt{SelectPeer()}: Single random neighbor
	\item \texttt{Update()}: Numerical function defined according to desired global
	aggregate (arihtmetic/geometric mean,max,\ldots)
\end{itemize}

$$ \textrm{Converge factor} = \rho = \frac{E(\sigma^2_{i+1})}{E(\sigma^2_i)} \approx \frac{1}{2 \sqrt{e}} \approx 0.303 $$

\todo[inline]{Network size estimation}

\subsubsection{Topology Management}

Topology gradually appears as a result of a ranking function (can be provided
as an abstract service)

\begin{itemize}
\item Given a set of $N$ node (each node has a \texttt{view} of size $c$, \texttt{profile} is used
to calculate ranking)
\item Given a ranking function
    $$R: R(x, \{ y_1,\ldots,y_m \}) = {\textrm{all orderings of } \{y_1,\ldots,y_m\}} $$
\begin{itemize}
\item[$\to$] $R$ can be defined by a distance function $d(x,y)$
\end{itemize}
\end{itemize}

\paragraph{T-Man algorithm}

T-man is a general topology management service that converges quickly
(logarithmic) based on a gossip framework.

\begin{itemize}
    \item \texttt{InitialView()}: random sample of nodes
    \item \texttt{Merge($v_1, v_2$)}: $v_1 \cup v_2$
    \item \texttt{SelectPeer(v)}: rank current view $v$ according to $R$ and return
    random sample from first half
    \item \texttt{SelectView(b)}: rank $b$ according to $R$ and return first $c$ elements
\end{itemize}

\begin{lstlisting}[caption={Active thread}, mathescape]
view $\leftarrow$ initialView()
do at a random time once in each consecutive interval of T time units
    p $\leftarrow$ selectPeer()
    myDescriptor $\leftarrow$ (myAddress, myProfile)
    buffer $\leftarrow$ merge(view, {myDescriptor})
    send buffer to p
    receive view$_p$ from p
    buffer $\leftarrow$ merge(view$_p$, view)
    view $\leftarrow$ selectView(buffer)
\end{lstlisting}

\begin{lstlisting}[caption={Passive thread}, mathescape]
do forever
    (q, view$_q$) $\leftarrow$ waitMessage()
    myDescriptor $\leftarrow$ (myAddress, myProfile)
    buffer $\leftarrow$ merge(view, {myDescriptor})
    send buffer to q
    buffer $\leftarrow$ merge(view$_q$, view)
    view $\leftarrow$ selectView(buffer)
\end{lstlisting}

\paragraph{Convergence analysis}
There are two phases: \textbf{rapid convergence} phase followed by
\textbf{endgame} phase.

\begin{itemize}
\item After 1 cycle, view is closest $c$ out of $2c$
\item After $i$ cycles, view is closest $c$ out of $2^i c$
\end{itemize}

\subparagraph{Handling the endgame}

Near the end of the convergence, a few nodes are still left behind, there are
two improvements to manage them (depending on the actual topology):

\begin{itemize}
\item \textbf{Balancing}: during rapid convergence, receiving node refuses
contact if there are already too many
\item \textbf{Routing}: during endgame, instead of random selection of a
peer node, select closest one on the topology (with
exponentially decreasing probabilities)
\end{itemize}
