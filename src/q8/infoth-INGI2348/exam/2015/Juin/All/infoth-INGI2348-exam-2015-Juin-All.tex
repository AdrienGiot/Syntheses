\documentclass[en]{../../../../../../eplexam}

\usepackage{enumitem}
\usepackage{../../../../../../eplunits}

\hypertitle{Information theory and coding}{8}{INGI}{2348}{2015}{Juin}{All}
{John de Wasseige \and Beno\^it Legat}
{Benoît Macq, Jérôme Louveaux and Olivier Pereira}

\paragraph{Discussion link}
\url{http://www.forum-epl.be/viewtopic.php?t=13090}

\section{BM-1}
Explain why the cryptographic coding is done after the compression.
\begin{solution}
  For two reasons.
  \begin{itemize}
    \item First because the cryptographic protocols are stronger when the message space is distributed uniformly
    \item and second because the output of the cryptographic does not contains any redundance so we cannot compress anything.
  \end{itemize}
\end{solution}

\section{BM-2}
Encode the sentence
\begin{center}
  miss ping is missing
\end{center}
\nosolution

\section{BM-3}
Give the rate distortion for a variable $X$ with PDF $f_X(x) = 2 - 2x$, $0 \leq x \leq 1$.
\nosolution

\section{JL-1}
A transmission is performed using the Hamming code $\mathbb{H}_3$ of redundancy 3.
\begin{enumerate}
  \item Determine a parity and generating matrix for the code.
    What is the code rate?

    \textbf{Hint}: You can always use an equivalent code by reordering the columns of the parity matrix
    if it makes computations simpler.
  \item Characterize the correcting capability of the code.
    What is the covering radius of the code?
  \item In the following case when is it sure that the error is going to be corrected,
    when is it sure it will \emph{not} be corrected,
    when is it unsure (i.e. it depends on the particular codeword transmitted and error positions):
    \begin{itemize}
      \item 1 bit erro in the block
      \item 2 bit errors in the block
      \item 3 bit errors in the block
    \end{itemize}
  \item Establish the list of coset leaders for this code.
  \item Assume that, after transmission on a binary symmetric channel,
    the received vector is given by
    \[ \underline{y} =
      \begin{pmatrix}
        1 & 0 & 0 & 0 & 1 & 0 & 0
      \end{pmatrix}
    \]
    for the code you have defined in question 1.
    Perform the syndrome decoding for the received vector $\underline{y}$.
\end{enumerate}

\begin{solution}
  \begin{enumerate}
    \item The redundancy is 3 and thus the Hamming code has length $n = 2^3 - 1 = 7$.
    The code rate is $k/n = 4/7$.
    The parity matrix in its canonical form is given by
    \[
      H =
      \begin{bmatrix}
        1 & 1 & 0 & 1 & 1 & 0 & 0 \\
        1 & 0 & 1 & 1 & 0 & 1 & 0 \\
        0 & 1 & 1 & 1 & 0 & 0 & 1 \\
      \end{bmatrix}
    \]
    and the generating matrix in its canonical form
    \[
      G =
      \begin{bmatrix}
        1 & 0 & 0 & 0 & 1 & 1 & 0 \\
        0 & 1 & 0 & 0 & 1 & 0 & 1 \\
        0 & 0 & 1 & 0 & 0 & 1 & 1 \\
        0 & 0 & 0 & 1 & 1 & 1 & 1 \\
      \end{bmatrix}
    \]
    
    \item The correcting capability of the code is 1 and its covering radius
    is $\frac{d-1}{2} = 1$, where $d=3$ is the minimal distance.
    
    \item
      \begin{itemize}
        \item It is \emph{sure} that it will be corrected.
        \item It is \emph{sure} that it will \emph{not} be corrected.
        \item It is \emph{sure} that it will \emph{not} be corrected.
      \end{itemize}
    
    \item The maximum weight of a coset leader is given by its covering radius,
    which is one in this case. The 8 (i.e. $2^{n-k}$) coset leaders are thus the
    7 vectors with one bit at each position and the zero-weight vector.
    
    \item The syndrome is given by
    \[
      s = H y^T = 
      \begin{bmatrix}
        0 & 1 & 0 \\
      \end{bmatrix}
    \]
    which corresponds to the coset leader $[0~0~0~0~0~1~0]$
    and thus the error is in the sixth position, 
    leading to a corrected $\tilde{y} = [1~0~0~0~1~1~0]$.
  \end{enumerate}
\end{solution}

\section{JL-2}
We consider a memoryless channel taking 3 possible inputs,
which are the nonzero elements of the Galois field $\mathbb{F}_4$ of size 4.
The primitive element of $\mathbb{F}_4$ is denoted by $\alpha$ and satisfies $\alpha^2 = \alpha + 1$.
The channel can be used at the rate $10^6$ symbols per second.
For each input $X$, the channel output is charaterized as
\[
  Y =
  \begin{cases}
                 X & \text{with probability } q = 0.89\\
    \alpha \cdot X & \text{with probability } q = 0.11
  \end{cases}
\]
\begin{enumerate}
  \item Write down the channel transition probability matrix
    and compute the capacity of the channel.
  \item What is the maximum bit rate (in bits/s) that can
    be transmitted on the channel, whatever the choice of coding.
\end{enumerate}
A stream of equiprobable and independent information bits,
at \si{1}{Mbits/s}, needs to be transmitted on this channel.
The stream is first encoded into the alphabet of the channel
and then transmitted at the maximum symbol rate of the channel.
\begin{enumerate}[resume]
  \item What is the code rate being used?
  \item Evaluate (apprimately) the required length (block size)
    of the code in order to obtain a probability of error lower than
    $10^{-5}$.
\end{enumerate}

\begin{solution}
  \begin{enumerate}
    \item We have $A = B = \{1, \alpha , \alpha^2\}$.
      The transition matrix is
      \[
        \mathcal{P} =
        \begin{pmatrix}
          0.89 & 0.11   & 0    \\
          0    & 0.89   & 0.11 \\
          0.11 & 0      & 0.89 \\
        \end{pmatrix}
      \]
      which is symmetric so the capacity is reached for
      an uniform input $\Pr[X = 1] = \Pr[X = \alpha] = \Pr[X = \alpha^2] = 1/3$
      and $\Pr[Y = 1] = \Pr[Y = \alpha] = \Pr[Y = \alpha^2] = 1/3$.
      We have $H(Y) = \log_2(3) = 1.585$
      and $H(Y|X) = 0.5$ so
      $C = I(X;Y) = H(Y) - H(Y|X) = 1.085~\si{bit/symbol}$.
    \item The maximum bit rate is given by
    \[ 
      \frac{C}{\tau_C} = \frac{1.085~\si{bit\per symbol}}{10^{-6}~\si{s\per symbol}} 
      = 1.085 \cdot 10^6~\si{bit/s}
    \]
    \item The code rate is $R = \tau_C/\tau_s^* = 1$.
    \item The required length of the code $n$ to obtain $p$ lower than $10^{-5}$
    is given by
    \[
      n \geq \Big\lceil \frac{\log_2(1/p_{\text{max}})}{C-R} \Big\rceil
      = \frac{\log_2(10^5)}{C-R} = 196
    \]
  \end{enumerate}
\end{solution}


\section{OP-1}
Let us consider the following message authentication code
$\mathsf{MAC} \colon \mathcal{M} \times \mathcal{K} \to \mathcal{D}$,
defined as $\mathsf{MAC}_k(m) = k \cdot m$, where
$\mathcal{M} = \{0,1\}^{c \times 1} \setminus \{0\}^{c \times 1}$,
$\mathcal{K} = \{0,1\}^{l \times c}$,
$\mathcal{D} = \{0,1\}^{l \times 1}$
and addition is performed modulo 2.

For instance, if $l=2$ and $c=3$, we could have:
\[ \mathsf{MAC}_k(m) = k \cdot m =
  \begin{pmatrix}
    0 & 1 & 0\\
    1 & 1 & 0
  \end{pmatrix}
  \cdot
  \begin{pmatrix}
    1\\
    0\\
    1
  \end{pmatrix}
  =
  \begin{pmatrix}
    0\\
    1
  \end{pmatrix}.
\]
\begin{enumerate}
  \item What are the optimal values of $Pd_0$ and $Pd_1$
    for this scheme,
    from a security point of view. (Justify.)
  \item Show that $Pd_0 = 1/2^l$
  \item Show that $Pd_2 = 1$
\end{enumerate}

\begin{solution}
  \begin{enumerate}
    \item An adversary that simply guess will have a probability
      of $1/2^l$ of forgery so
      $Pd_0 \geq 1/2^l$ and $Pd_2 \geq 1/2^l$.
    \item Let's evaluate
      \[ \mathop{\mathrm{payoff}}(m,d) = \frac{\|\{\, k \in \mathcal{K} \mid \mathsf{MAC}_k(m) = d \,\}\|}{\|\mathcal{K}\|} \]
      Let's split the numbers $1 \leq n \leq c$ in 3 groups.
      Those for which $m_n = 0$, the last one $n^*$ for which $m_{n^*} = 1$ and the other for which $m_n = 1$.
      Let $x$ be the number of ones in $m$.

      We can see the values of $k$ in the columns of the first group doesn't matter ($2^{l(c-x)}$ possibilities),
      for each of the values of the $x-1$ columns of the third group ($2^{l(x-1)}$ possibilities),
      we can set
      \[ k_{ln^*} = d_l \oplus k_{ln_1} \oplus \cdots \oplus k_{ln_{x-1}} \]
      where $n_1, \ldots, n_{x-1}$ are the members of the third group.
      So we have
      \[ \mathop{\mathrm{payoff}}(m,d) = \frac{2^{l(c-x)} \cdot 2^{l(x-1)}}{2^{lc}} = \frac{1}{2^l} \]
      Therefore $Pd_0 = 1/2^l$.
    \item The adversary observing different $m_1,m_2$ with tags $d_1,d_2$ can forge the tag $d_1 \oplus d_2$
      for the message $m_1 \oplus m_2$ so $Pd_2 \geq 1$.
      Since it is a probability it means $PD-2 = 1$.
  \end{enumerate}
\end{solution}

\end{document}
