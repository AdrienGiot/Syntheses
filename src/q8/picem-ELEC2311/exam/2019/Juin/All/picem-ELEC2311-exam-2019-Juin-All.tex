\documentclass[fr]{../../../../../../eplexam}

\hypertitle{Physique interne des convertisseurs électromécaniques}{8}{ELEC}{2311}{2019}{Juin}{All}
{Etudiants ELME 2019}
{Bruno Dehez}

\section*{Déroulement de la présentation orale}
On pioche un papier avec un numéro et on a 10 minutes pour préparer une synthèse sur le séminaire. Puis on lui décrit globalement le séminaire et il pose des questions un peu plus précises sur certains points pendant votre description.

\section*{Séminaire 1}

\begin{itemize}
    \item Comment trouve-t-on le terme de source de flux pour l'aimant permanent ?
    \item Comment trouve-t-on le terme de source électromotrice pour la bobine ?
    \item Où est représentée la variation des positions ?
    \item Quels sont les termes inconnus et sources ?
    \item Comment trouver la matrice $A$ des réluctances ?
    \item Comment procéder pour mettre à jour les variations de réluctance du circuit ?
    \item Une fois qu'on a tout ($F_{em}$, flux aux n\oe{}uds et aux branches) comment dimensionner la machine ? 
    \item Comment est calculé le couple et où le flux dans la formule se trouve-t-il sur le circuit ?
\end{itemize}

\section*{Séminaire 2}

\begin{itemize}
    \item Comment arrive-t-on aux équations ?
    \item Hypothèses à poser, et qu'est-ce qu'il se passe si on les change (exemple: présence de courants).
    \item Comment expliquer intuitivement la position de couple maximum ? 
\end{itemize}

\section*{Séminaire 3}

\begin{itemize}
    \item 
    \item 
\end{itemize}

\section*{Séminaire 4}

\begin{itemize}
    \item 
    \item 
\end{itemize}

\section*{Séminaire 5}

\begin{itemize}
    \item De quoi sont composés les éléments du maillage ? D'où viennent les sources ?
    \item Dans quel cas apparaît une source de flux/une source d'EMF ? Comment les calcule-t-on ? 
    \item Pourquoi une méthode hybride est plus adaptée dans ce cas-ci qu'une méthode reposant uniquement sur un réseau de réluctances ?
    \item Comment arrive-t-on au modèle analytique ? Sous quelle forme seront les équations ?
    \item Comment relie-t-on le modèle analytique et le modèle avec réseau de réluctances ? 
     \item Citer toutes les conditions et équations qui permettent de déterminer $A$ dans le modèle analytique. En gros, les séries de Fourier, $n_{21} . (B_1-B_2)=0$ et $n_{21} \times (H_1-H_2)=0$
\end{itemize}

\section*{Séminaire 6}

\begin{itemize}
    \item 
    \item 
\end{itemize}

\section*{Séminaire 7}

\begin{itemize}
    \item Expliquer l'origine du couple de la machine (couple électrodynamique).
    \item Expliquer la disposition du moteur, le sens des aimants, le couplage des bobines pour maximiser l'emf, les forces $x$ et $y$.
\end{itemize}

\section*{Séminaire 8}

\begin{itemize}
    \item Comment appliquer une force en $x$ et $y$ sur le rotor? Quel est l'alignement du flux statorique par rapport au flux rotorique ?
    \item Comment contrôler la machine ?
    \item À quelle fréquence fonctionne le notch ?
    \item Comment est stabilisée la machine ? (À travers le courant dans les enroulements)
    \item Les enroulements au stator sont assez particuliers; pourquoi ? (Les dessiner, et expliquer qu'on peut gagner de la place en fonction du pitch des enroulements)
\end{itemize}

\section*{Séminaire 9}

\begin{itemize}
    \item On parle de conversion d'énergie (de la rotation vers une EMF dans la bobine), il y a-t-il des pertes ? À quoi pourraient elles êtres dues ? Il y a des pertes joules dans le bobinage.
    \item On stabilise sur l'axe $Z$. Mais quid de la stabilisation sur l'axe radial ? Ce n'est pas expliqué, mais on peut faire une référence au séminaire 8. Joachim travaille sur des aimants permanents en anneau qui assurerait le centrage de l'axe.
    \item Quelle est la forme de la force de poussée en fonction de la vitesse ? Pourquoi ? Ce n'est pas intuitif et même lui ne sait pas. Quelle est la forme du couple de drag en fonction de la vitesse ? Cela fait penser à quelle courbe ? Couple/vitesse d'une machine asynchrone.
    \item Expliquer le théorème d'Earnshaw dans les grandes lignes\ldots{} Une référence à un autre séminaire m'a semblé suffisante.
    \item Expliquer les différentes difficultés de stabilisation (mécanique/électrique). Si vous connaissez bien les graphes avec les zones ``bleues'', c'est un plus\ldots
    \item Il n'a rien demandé sur les résultats. Il était plutôt intéressé (et satisfait) que j'ai bien compris le fonctionnement ``passif'' de la machine. Une bonne comparaison au séminaire 8 est pertinente. L'approche circuit était pas vraiment nécessaire au final. Retaper les schémas comme ça, ça lui allait\ldots{} 0 équation pour celui-ci\ldots
\end{itemize}

\section*{Séminaire 10}

\begin{itemize}
    \item 
    \item 
\end{itemize}

\section*{Séminaire 11}

\begin{itemize}
    \item Explication de toute la méthodologie des problèmes d'optimisation, des points 1 - 3.
    \item Explication différence mode factoriel/mode DEO.
    \item Explication PSO et NSGA-2.
\end{itemize}

\section*{Séminaire 12}

\begin{itemize}
    \item Quel est le but de la machine ? 
    \item Comment obtenir le rapport de vitesse ainsi que la valeur du paramètre $N_s$ (nombre de barres ferromagnétiques) ? Qu'est ce que signifient physiquement les conditions à imposer pour obtenir ces 2 relations ? 
    \item Est-il possible d'obtenir d'une autre manière la forme du flux dans l'entrefer et le couple? Oui via la méthode analytique de Fourier comme dans le séminaire 2,3 et 4 MAIS il y aurait beaucoup de domaines à définir donc beaucoup de conditions limites à trouver donc pas facile.
    \item Quelle est l'optimisation dont on a parlé dans le séminaire et comment a-t-elle été réalisée (via quelle méthode d'optimisation).
\end{itemize}

\end{document}
