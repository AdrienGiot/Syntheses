\documentclass[en]{../../../../../../eplexam}

\usepackage{../../../../../../eplcode}
\lstset{language={C}}

\hypertitle{Computer System Security}{8}{INGI}{2347}{2019}{Juin}{All}
{INFO and ELEC students 2019}
{Ramin Sadre}

\section{TLS}
\begin{enumerate}
    \item What other than the chain of trust must the client verify in the certificate? 
    \item Explain the TLS key exchange (assume RSA), only the steps after the validation of the certificate. How is the certificate used in this process?
    \item In a society, they use TLS for their private network, you also self-signed the authority certificate (like in the exercises) and then install a root certificate on each computer of the company. What are the security risk? What can you do to solve this problem?
\end{enumerate}

\nosolution

\section{Review C code}
Find 3 vulnerabilities in the code and explain.

\begin{lstlisting}
plural(char *str, char* plu){
    char *buf;
    plu[0] = '\0';
    if(str == 0) return 
    int n = strlen(str);
    if(n == 0) return; 
    buf = malloc(n);
    char last = str[n-1];
    
    if(last == 's') // already a plural
        strcpy(buf, str);
    else if(last == 'h') // plural with h => hash, hashes
        sprintf(buf, "%ses", str);
    else  // normal plural
        sprintf(buf, "%ss", str);
    
    strcpy(plu, buf);
    free(buf);
}
    
report_print(char * str, int n) {
    char item_plural[256];
    plural(str,item_plural);
    printf(blable %d, %s, n, item_plural);
}
\end{lstlisting}

\nosolution

\section{Buffer overflow}
\begin{lstlisting}
char userName[20];
int userAccessRights;
void initUser(char *name) {
    userAccessRights=0;
    strcpy(userName,name);
} 
\end{lstlisting}

The argument name is an input entered by the user. During a buffer overflow attack, the attacker can: change data, inject and execute code or run a function in the system library.
\begin{enumerate}
    \item Explain how Random Canaries would help or not against each of those problems.
    \item Same for ASLR.
    \item Same for DEP.
\end{enumerate}

\nosolution

\section{Firewall}
For a society with IP 1.1.1.0/24 with a stateless network-based FW and with a web server (1.1.1.2), a mail server (1.1.1.3), a telnet server (1.1.1.4) and hosts. The FW only checks traffic that is going inside or outside of the network, it does not check internal traffic.

The boss wants:  
\begin{enumerate}
    \item Mail server (port 25) and web server (port 80 and 443) are accessible from internet and intranet.
    \item The telnet server (port 23) is only accessible from inside.
    \item No employee should be able to connect to other telnet server.
    \item Employees can navigate on the web.
\end{enumerate}

You are asked to:

\begin{enumerate}
    \item Write a minimum set of rule for the stateless FW.
    \item Is it possible to only allow arriving (inbound?) UDP packet with such FW? Explain.
    \item Write a minimum set of rule for the stateful FW.
    \item Sketch a DMZ for the society (no need to write the rules).
\end{enumerate}

\nosolution

\section{Netflows} 

IP of the society 192.12.0.0/24
\begin{center}
    \begin{tabular}{lllllll}
     \textbf{start time} & \textbf{end time} & \textbf{src IP} & \textbf{dst IP} & \textbf{Packets} & \textbf{Data}\\
     13:38:00 & 13:39:01 & 192.12.1.1:4546 & 42.161.1.1:80 & 5 & 300 bytes\\
     13:39:01 & 13:39:06 & 42.161.1.1:80 & 192.12.1.1:4546 & 20 & 3000 bytes\\
     hh:mm:ss & hh:mm+?:ss & 192.12.1.1:4654 & 64.64.87.1:53 & 300 & 30000 bytes\\
     hh:mm:ss & hh:mm+?:ss & 164.64.87.1:53 & 192.12.1.1:4654 & 4200 & 500000 bytes\\
     hh:mm:ss & hh:mm+1:ss & 192.12.1.2:4546 & 42.161.1.2:80 & 5 & 300 bytes\\
     hh:mm:ss & hh:mm+?:ss & 42.161.1.2:80 & 192.12.1.2:4546 & 20 & 3000 bytes
    \end{tabular}
\end{center}

Snippet of IP network traffic.
\begin{enumerate} 
    \item What kind of attack can you observe?
    \item What timeout was set for the netflow?
    \begin{enumerate}
        \item 15 seconds
        \item 30 seconds
        \item 1 minute
        \item 5 minutes
    \end{enumerate}
    \item Which of the following attacks can we detect with netflow? Just say yes or no. If it depends explain why. Think carefully!
    \begin{enumerate}
        \item SQL injection
        \item DDOS SYN flooding
        \item SSH password dictionary attack
        \item DNS cache poisoning variant 2 (the one that is not fixed by the Bailiwick check)
        \item Man-in-the-middle attack in a Diffie-Helmann key exchange
    \end{enumerate}
\end{enumerate}

\nosolution

\section{Cryptography}

\begin{enumerate}
    \item You need to authenticate a communication and have at your disposal two different kinds of hash:
    \begin{itemize}
        \item Alice sends $(m,F(k,m))$
        \item Alice sends $(m,H(k||m))$
    \end{itemize}
    where $F$ is the HMAC using SHA256 and $H$ is SHA256. Which one is the best to the integrity of the message? Explain.
    \item Is $(17,63)$ a valid public key for RSA encryption?
    \item You have an implementation of DES on a C code and you are asked to implement it in Java. But you only have a 3DES implementation in the Java libraries, can you still implement a DES based on the 3DES implementation?
\end{enumerate}

\nosolution

\end{document}
